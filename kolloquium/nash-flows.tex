\subsection{Dynamische Nash-Flüsse}

\begin{frame}{Dynamische Nash-Flüsse}
	\begin{definition}[Nash-Fluss]
		Ein zulässiger dynamischer Fluss $f$ heißt \emph{Nash-Fluss}, falls er nur entlang aktiver Kanten fließt, d. h. falls $f_{vw}^+(\theta) > 0$ für fast alle $\theta\in\R$ bereits $\theta\in l_v(\Theta_{vw})$ impliziert.
	\end{definition}
	\pause\begin{definition}
		Für einen zulässigen Fluss $f$ und einen Zeitpunkt $\theta$ bezeichne
		$x_{vw}^+(\theta)\coloneq  F_{vw}^+(l_v(\theta))$ bzw. $x_{vw}^-(\theta)\coloneq  F^-_{vw}(l_w(\theta))$ für $vw\in E$.
	\end{definition}
	\pause\begin{theorem}
	Ein zulässiger dynamischer Fluss $f$ ist genau dann ein Nash-Fluss,
	wenn $x_e^+(\theta) = x_e^-(\theta)$ für alle $e\in E$ und $\theta\in\R$ gilt.
	Ist dies der Fall, so definiere man $x_e(\theta)\coloneq x_e^+(\theta) = x_e^-(\theta)$ für $e\in E$.
	\end{theorem}
\end{frame}