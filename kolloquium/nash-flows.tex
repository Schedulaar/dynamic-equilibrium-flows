\section{Dynamische Nash-Flüsse}

\begin{frame}
	\begin{definition}
		Für einen zulässigen Fluss $f$ und einen Zeitpunkt $\theta$ bezeichne
		\begin{itemize}[label=\color{darkblue}$\bullet$]
			\item $x_{vw}^+(\theta)\coloneq  F_{vw}^+(l_v(\theta))$ bzw. $x_{vw}^-(\theta)\coloneq  F^-_{vw}(l_w(\theta))$ für $vw\in E$,
			\pause\item $b_v(\theta) \coloneq  \sum_{e\in\delta^+(v)} x_e^+(\theta) - \sum_{e\in\delta^-(v)} x_e^-(\theta)$ für $v\in V$.
		\end{itemize}
	\end{definition}
	
	\pause\begin{definition}\label{def-flow-along-active-edges}
		Man sage, der Fluss $f$ \emph{fließe nur entlang aktiver Kanten}, falls $f_{vw}^+(\theta) > 0$ für fast alle $\theta\in\R$ bereits $\theta\in l_v(\Theta_{vw})$ impliziert.
	\end{definition}

	\pause\begin{definition}
	Man sage, der Fluss $f$ \emph{fließe ohne Überholungsmöglichkeiten}, falls $b_s(\theta) = -b_t(\theta)$ für alle $\theta\in\R$ gilt.
	\end{definition}
\end{frame}

\begin{frame}
	\begin{theorem}[Charakterisierung dynamischer Nash-Flüsse]\label{thm-equivalencies-nash-flow}
		Für einen zulässigen dynamischen Fluss $f$ sind die Aussagen
		\begin{enumerate}[label=(\roman*)]
			\item Der Fluss $f$ fließt nur entlang aktiver Kanten.
			\item Für alle $e\in E$ und $\theta\in\R$ gilt $x_e^+(\theta) = x_e^-(\theta)$.
			\item Der Fluss $f$ fließt ohne Überholungen.
		\end{enumerate}
		äquivalent. Gilt eine davon, so nennt man $f$ einen \emph{dynamischen Nash-Fluss}.
	\end{theorem}
\end{frame}