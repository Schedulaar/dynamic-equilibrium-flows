\subsection{Kürzeste Wege}

\begin{frame}
	\begin{definition}[Kürzeste Wege]
		Für einen Fluss $f$ bezeichne:
		\begin{itemize}[label=\color{darkblue}$\bullet$]
			\item\pause $l^P(\theta)\coloneq T_{e_k}\circ\dots\circ T_{e_1}(\theta)$ die Ankunftszeit am Endknoten eines Pfades $P=(e_1,\dots,e_k)$ zur Startzeit $\theta$ am Startknoten,
			\item\pause $\mathcal{P}_w$ die Menge aller $s$-$w$-Pfade,
			\item\pause $l_w(\theta) \coloneq \min_{P\in\mathcal{P}_w} l^P(\theta)$ die früheste Ankunftszeit bei $w$ zur Startzeit $\theta$.
		\end{itemize}
	\pause Ein Pfad $P\in \mathcal{P}_w$ heißt \emph{kürzester $s$-$w$-Pfad zur Zeit $\theta$}, falls $l^P(\theta)=l_w(\theta)$.
	\end{definition}
	\pause\begin{lemma}[Dreiecksungleichung]
		Für alle Kanten $vw\in E$ gilt in einem zulässigen Fluss $T_{vw}(l_v(\theta))\geq l_w(\theta)$.
	\end{lemma}
\end{frame}

\begin{frame}
	\begin{definition}[Aktivität einer Kante]
		Eine Kante $vw\in E$ ist \emph{aktiv zum Zeitpunkt $\theta$}, falls $T_{vw}(l_v(\theta)) = l_w(\theta)$ gilt; sonst ist sie \emph{inaktiv zum Zeitpunkt $\theta$}.
		
		\pause
		Es sei $\Theta_e$ die Menge der Zeitpunkte, zu denen $e$ aktiv ist.
		Für $\theta\in\R$ sei $G_\theta\coloneq (V, E_\theta)$ der Teilgraph der zur Zeit $\theta$ aktiven Kanten.
	\end{definition}
	\pause\begin{lemma}
		Für einen zulässigen Fluss ist $G_\theta$ ein azyklischer Graph, in dem $s$ jeden Knoten erreichen kann.
	\end{lemma}
	\pause\begin{proposition}
			Für einen zulässigen Fluss $f$ ist $(l_v(\theta))_{v\in V}$ die eindeutige Lösung von
		\[ \tilde{l}_w = \begin{cases}
		\theta, & \text{falls } w=s, \\
		\min\limits_{vw\in \delta^-(w)} T_{vw}(\tilde{l}_v), & \text{sonst}.
		\end{cases} \]
	\end{proposition}
\end{frame}