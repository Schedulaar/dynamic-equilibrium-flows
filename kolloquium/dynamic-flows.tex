\section{Dynamische Flüsse}
\subsection{Grundlegende Definitionen}

\begin{frame}{Grundlegende Definitionen}
	\begin{definition}[Netzwerk]
		Ein \emph{Netzwerk} $(G, s, t, u, \tau)$ ist ein gerichteter Graph mit
		\begin{itemize}[label=\color{darkblue}$\bullet$]
			\item einer Quelle $s\in V$, sodass alle Knoten von $s$ aus erreichbar sind,
			\item einer Senke $t\in V$,
			\item Kantenkapazitäten $u\in\R_+^E$,
			\item Verzögerungszeiten $\tau\in\R_{\geq 0}^E$, sodass Zyklen eine positive Gesamtverzögerung haben.
		\end{itemize}
	\end{definition}
	\pause
	\begin{definition}[Funktionenraum $\mathfrak{F}_0$]
		Der Funktionenraum $\mathfrak{F}_0$ sei die Menge
		\[ \{ g: \R\to\R_{\geq 0} \mid \text{$g$ lokal Lebesgue-integrierbar, $g(t) = 0$ für $t<0$} \} \]
	\end{definition}
\end{frame}


\begin{frame}{Grundlegende Definitionen}
	\begin{definition}[Dynamischer Fluss]
		Ein dynamischer Fluss ist ein Paar $f=(f^+, f^-)$ mit $f^+, f^-\in\mathfrak{F}_0^E$.
		
		Für eine Kante $e\in E$ und einen Zeitpunkt $\theta\in\R$ bezeichnet
		\begin{itemize}[label=$\color{darkblue}\bullet$]
			\pause\item $f_e^+(\theta)$ bzw. $f_e^-(\theta)$ die \emph{Zu- bzw. Abflussrate an $e$ zur Zeit $\theta$},
			\pause\item $F_e^+(\theta) \coloneq \int_0^\theta f^+_e(t) \diff t$ bzw. $F^-_e(\theta)\coloneq\int_0^\theta f^-_e(t)\diff t$ den Zu- bzw. Abfluss an $e$ bis zur Zeit $\theta$,
			\pause\item $z_e(\theta)\coloneq F_e^+(\theta) - F_e^-(\theta + \tau_e)$ bzw. $q_e(\theta)\coloneq z_e(\theta)/u_e$ die Warteschlange bzw. Wartezeit an $e$ zur Zeit $\theta$,
			\pause\item $T_e(\theta)\coloneq\theta + q_e(\theta) + \tau_e$ die Austrittszeit aus $e$ bei Eintrittszeit $\theta$.
		\end{itemize}	
	\end{definition}
\end{frame}


\begin{frame}{Grundlegende Definitionen}
	\begin{definition}[Zulässiger Dynamischer Fluss]
		Ein zulässiger dynamischer Fluss erfüllt folgende Voraussetzungen:
		\begin{enumerate}[label=(F\arabic*)]
			\pause\item\label{def-feasible-flow-capacity} Kapazitätsbedingung: $\forall e\in E, \theta\in\R: f_e^-(\theta)\leq u_e$.
			\pause\item\label{def-feasible-flow-no-negative-flow} Keine Flussentstehung in Kanten:\\ $\forall e\in E, \theta\in\R: F_e^-(\theta + \tau_e) \leq F_e^+(\theta).$
			\pause\item\label{def-feasible-flow-no-flow-at-node} Flusserhaltung in Knoten:
			\[
			\forall\theta\in\R: \sum_{e\in\delta^+(v)}f^+_e(\theta) - \sum_{e\in\delta^-(v)} f_e^-(\theta) \begin{cases}
				\geq 0, \text{ falls $v=s$,}\\
				\leq 0, \text{ falls $v=t$,}\\
				= 0, \text{ sonst.}
			\end{cases}\]
			\pause\item\label{def-feasible-flow-queue-with-capacity} Warteschlangenabbau:\\
			$\forall e\in E, \theta\in\R: z_e(\theta) > 0 \implies f_e^-(\theta + \tau_e) = u_e$.
		\end{enumerate}
	\end{definition}
\end{frame}

\subsection{Eigenschaften zulässiger Flüsse}
\begin{frame}{Eigenschaften zulässiger Flüsse}
	\begin{proposition}\label{prop-feasible-flow}
		Für eine Kante $e\in E$ gilt in einen zulässigen dynamischen Fluss $f$:
		\begin{enumerate}[label=(\roman*)]
			\item\label{prop-feasible-flow-T-mon-inc-cont} Die Funktion $\theta \mapsto \theta + q_e(\theta)$ ist monoton wachsend und stetig.
			\item\label{prop-feasible-flow-positive-queue} Für alle $\theta\in\R$ ist die Warteschlange $z_e$ auf $(\theta, \theta + q_e(\theta))$ positiv.
			\item\label{prop-feasible-flow-det-outflow} Zu jeder Zeit $\theta\in\R$ gilt $F_e^+(\theta) = F_e^-(T_e(\theta))$.
		\end{enumerate}
	\end{proposition}
\end{frame}

