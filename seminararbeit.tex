\newcommand*{\DocType}{scrartcl}
%\renewcommand*\DocType{article} % Uncomment for screen optimization

\newcommand*\ClassList{scrartcl,article}
\documentclass[\DocType, paper=a4,fontsize=11pt,abstracton,headsepline,notitlepage]{generalclass}


\usepackage[utf8]{inputenc}
\usepackage[automark]{scrpage2}	        % Seiten-Stil für scrartcl
% Mathematische Zeichensätze und Umgebungen
\usepackage{amsfonts, amssymb}	        % Definition einer Liste mathematischer Fontbefehle und Symbole
\usepackage[intlimits,sumlimits]{amsmath} % Integral-/Summationsgrenzen über/unter Zeichen
\usepackage{tabto}
\usepackage{mathabx}
% mathematische Verbesserungen
\usepackage{amsthm}	                    % spezielle theorem Stile
\usepackage{aliascnt} 
\usepackage{array}		                % erweiterte Tabellen
% Schriftzeichen, Format
\usepackage{latexsym}		            % Latex-Symbole
\usepackage[english, german, ngerman]{babel} % Mehrsprachenumgebung
% Layout
\usepackage{geometry}                   % Seitenränder
\usepackage{xcolor}                     % Farben
\usepackage{bbm}
% Tabellen und Listen
\usepackage{float}		                % Gleitobjekte 
\usepackage[flushright]{paralist}       % Bessere Behandlung der Auflistungen
\usepackage{datetime}
% Bilder
\usepackage[final]{graphicx}            % Graphiken einbinden
\usepackage{caption}                    % Beschriftungen
\usepackage{subcaption}                 % Beschriftungen für Unterteilung
\usepackage{tikz}
% Interaktive Referenzen, und PDF-Keys
\usepackage{xr-hyper}  
\usepackage[pagebackref,pdftex, plainpages=false]{hyperref} % Rückreferenz im Literaturverzeichnis, Treiber für ps zu pdf ; für direkt nach pdf: pdftex
\usepackage{enumitem}

%%%%%%%%%%%%%%%%%%%%%%%%%%%%%%%%%%%%%%%%%%%%%%%%%%%%%%%%%%%%%%%%%%%%%%%%%%%%%%%%
% Zusammenfassung einiger nützlicher Pakete und Befehle
%-------------------------------------------------------------------------------
% Kopf-Zeilen
%-------------------------------------------------------------------------------

\pagestyle{scrheadings}		     % Kopfzeilen nach scr-Standard		
\ifx\chapter\undefined 		     % falls Kapitel nicht definiert sind
  \automark[subsection]{section} % Kopf- und Fusszeilen setzen
\else				             % Kapitel sind definiert
  \automark[section]{chapter}	 % Kopf- und Fusszeilen setzen
\fi

%-------------------------------------------------------------------------------
%   Maske für Überschrift 
%-------------------------------------------------------------------------------
% Belegung der notwendigen Kommandos für die Titelseite
\newcommand{\autor}{Markl, Michael} 		% bearbeitender Student
\newcommand{\veranstaltung}{Seminar zur Optimierung und Spieltheorie} 	% Titel des ganzen Seminars
\newcommand{\uni}{Institut für Mathematik der Universität Augsburg} % Universit\"at
\newcommand{\matrikelnummer}{1474802}
\newcommand{\lehrstuhl}{Diskrete Mathematik, Optimierung und Operations Research} % Lehrstuhl
\newcommand{\semester}{Sommersemester 2019}	% Winter-/Sommersemester mit Jahr
\newcommand{\datum}{27.06.2019} 			% Datumsangabe
\newcommand{\thema}{Nash Gleichgewichte in Dynamischen Flüssen}  		% Titel der Seminararbeit

\newcommand{\ownline}{\vspace{.7em}\hrule\vspace{.7em}} % horizontale Linie mit Abstand

\newcommand{\seminarkopf}{
	% Befehl zum Erzeugen der Titelseite 
 \textsc{\autor}  \hfill{\datum} \\ 
\textbf{\veranstaltung} \\ 
\uni \\ 
\lehrstuhl \\
\semester \hfill{Matrikelnummer: \matrikelnummer}
\ownline 

\begin{center}
{\LARGE \textbf{\thema}}
\end{center}

\ownline
}			% Befehle und Pakete für Titelseite


\DeclareMathOperator{\e}{ex}
\DeclareMathOperator{\ma}{mate}
\DeclareMathOperator{\Ex}{Ex}

%-------------------------------------------------------------------------------
%   Befehle für Nummerierung der Ergebnisse
%   fortlaufend innerhalb eines Abschnittes
%-------------------------------------------------------------------------------
\theoremstyle{plain}            % normaler Stil
\newtheorem{theorem}{Theorem}
\numberwithin{theorem}{section}
% Lemma
\newaliascnt{lemma}{theorem}
\newtheorem{lemma}[lemma]{Lemma}
\aliascntresetthe{lemma}
% Satz
\newaliascnt{satz}{theorem}
\newtheorem{satz}[satz]{Satz}
\aliascntresetthe{satz}
% Korollar
\newaliascnt{corollary}{theorem}
\newtheorem{corollary}[corollary]{Korollar}
\aliascntresetthe{corollary}
% Proposition
\newaliascnt{proposition}{theorem}
\newtheorem{proposition}[proposition]{Proposition}
\aliascntresetthe{proposition}
%-------------------------------------------------------------------------------
\theoremstyle{definition}	% Definitionsstil
% Definition
\newaliascnt{definition}{theorem}
\newtheorem{definition}[definition]{Definition}
\aliascntresetthe{definition}
% Beispiel
\newaliascnt{example}{theorem}
\newtheorem{example}[example]{Beispiel}
\aliascntresetthe{example}
% Problem
\newaliascnt{problem}{theorem}
\newtheorem{problem}[problem]{Problem}
\aliascntresetthe{problem}
% Algorithmus
\newaliascnt{algorithmus}{theorem}
\newtheorem{algorithmus}[algorithmus]{Algorithmus}
\aliascntresetthe{algorithmus}
%-------------------------------------------------------------------------------
\theoremstyle{remark}		% Bemerkungsstil
% Bemerkung
\newaliascnt{remark}{theorem}
\newtheorem{remark}[remark]{Bemerkung}
\aliascntresetthe{remark}
% Vermutung
\newaliascnt{conjecture}{theorem}
\newtheorem{conjecture}[conjecture]{Vermutung}
\aliascntresetthe{conjecture}
% Notation
\newaliascnt{notation}{theorem}
\newtheorem{notation}[notation]{Notation}
\aliascntresetthe{notation}

%-------------------------------------------------------------------------------
% automatische Referenzen mit interaktiven Text
%-------------------------------------------------------------------------------

% Texte
\renewcommand{\theoremautorefname}{Theorem}
\newcommand{\lemmaautorefname}{Lemma}
\newcommand{\satzautorefname}{Satz}
\newcommand{\korollarautorefname}{Korollar}
\newcommand{\propositionautorefname}{Proposition}

\newcommand{\definitionautorefname}{Definition}
\newcommand{\beispielautorefname}{Beispiel}
\newcommand{\problemautorefname}{Problem}
\newcommand{\algorithmusautorefname}{Algorithmus}

\newcommand{\bemerkungautorefname}{Bemerkung}
\newcommand{\vermutungautorefname}{Vermutung}
\newcommand{\notationautorefname}{Notation}

%-------------------------------------------------------------------------------
% Nummerierung der Gleichungen innerhalb der obersten Ebene
%-------------------------------------------------------------------------------
\ifx\chapter\undefined 			% Kapitel sind definiert
  \numberwithin{equation}{section}	% Gleichungsnummern in Section
\else					% Kapitel sind nicht definiert
  \numberwithin{equation}{chapter}	% Gleichungsnummern in Kapiteln
\fi





\makeatletter
\newcommand{\overleftsmallarrow}{\mathpalette{\overarrowsmall@\leftarrowfill@}}
\newcommand{\overrightsmallarrow}{\mathpalette{\overarrowsmall@\rightarrowfill@}}
\newcommand{\overleftrightsmallarrow}{\mathpalette{\overarrowsmall@\leftrightarrowfill@}}
\newcommand{\overarrowsmall@}[3]{%
	\vbox{%
		\ialign{%
			##\crcr
			#1{\smaller@style{#2}}\crcr
			\noalign{\nointerlineskip\vskip0.4pt}%
			$\m@th\hfil#2#3\hfil$\crcr
		}%
	}%
}
\def\smaller@style#1{%
	\ifx#1\displaystyle\scriptstyle\else
	\ifx#1\textstyle\scriptstyle\else
	\scriptscriptstyle
	\fi
	\fi
}
\makeatother
			% Mathematische Befehle und Pakete

% Literatur-Bibliothek
\bibliographystyle{alphadin}               % deutscher Bibliotheksstil

% Erweiterte Einstellungen zu hyperref

\hypersetup{
        breaklinks=true,        % zu lange Links unterbrechen
        colorlinks=true,        % Färben von Referenzen
        citecolor=black,        % Farbe der Zitate
        linkcolor=black,        % Farbe der Links
        extension=pdf,          % Externe Dokumente können eingebunden werden.
        ngerman,		
	pdfview=FitH,
	pdfstartview=FitH,		
	bookmarksnumbered=true,     % Anzeige der Abschnittsnummern	% pdf-Titel
	pdfauthor={\autor}          % pdf-Autor
}

% Namen für Referenzen 

\newcommand{\ownautorefnames}{
  \renewcommand{\sectionautorefname}{Kapitel}
  \renewcommand{\subsectionautorefname}{Unterkapitel}
  \renewcommand{\subsubsectionautorefname}{\subsectionautorefname}
  \renewcommand{\appendixautorefname}{Anhang}
  \renewcommand{\figureautorefname}{Abbildung}
}

% Rückreferenzentext zur Literatur
\def\bibandname{und}%
\renewcommand*{\backref}[1]{}
\renewcommand*{\backrefalt}[4]{%
\ifcase #1 %
 (Nicht zitiert, also Ergänzungsliteratur.)%
\or
 (Zitiert auf Seite #2.)%
\else
 (Zitiert auf den Seiten #2.)%
\fi
}
\renewcommand{\backreftwosep}{ und~} % seperate 2 pages
\renewcommand{\backreflastsep}{ und~} % seperate last of longer 

			% Befehle und Pakete für Referenzen

\geometry{a4paper, top=30mm, bottom=30mm, left=28mm, right=28mm}

\IfClass{article}{ % Optimize for screen
	\geometry{papersize={160mm,161.5mm},margin=5mm} 
}

\parindent0em

\addtolength{\footskip}{-0.5cm}          % Seitenzahlen höher setzen
\renewcommand{\descriptionlabel}[1]{\hspace{\labelsep}\textit{#1}}

\makeatletter
\newcommand{\customlabel}[2]{%
	\protected@write \@auxout {}{\string \newlabel {#1}{{#2}{\thepage}{#2}{#1}{}} }%
	\hypertarget{#1}{#2}
}
\makeatother
\newcommand{\todo}[1]{{\color{red}#1}}

\numberwithin{figure}{section}	% Abbildungsnummern in Section

\newcommand{\R}{\mathbb{R}}
\newcommand{\Q}{\mathbb{Q}}
\newcommand{\Z}{\mathbb{Z}}
\newcommand{\N}{\mathbb{N}}
\newcommand*\diff{\mathop{}\!\mathrm{d}}
\setlist[enumerate]{topsep=0.5ex,itemsep=0ex,partopsep=0ex,parsep=0.8ex}


\renewcommand{\[}{
	\setlength\abovedisplayskip{1.1ex}
	\setlength{\belowdisplayskip}{1.1ex}
	\setlength{\abovedisplayshortskip}{1.1ex}
	\setlength{\belowdisplayshortskip}{1.1ex}\begin{equation*}}

\renewcommand{\]}{\end{equation*}}

%%%%%%%%%%%%%%%%%%%%%%%%%%%%%%%%%%%%%%%%%%%%%%%%%%%%%%%%%%%%%%%%%%%%%%%%%%%%%%%%
% Start des Dokuments
\begin{document}

\ownautorefnames		% Änderung einiger automatischen Texte von hyperref (wie in referenz.tex definiert)
%\parindent0em 			% kein Einzug nach einer Leerzeile

%%%%%%%%%%%%%%%%%%%%%%%%%%%%%%%%%%%%%%%%%%%%%%%%%%%%%%%%%%%%%%%%%%%%%%%%%%%%%%%%
% Titelseite
\thispagestyle{empty}   % leerer Seitenstil, also keine Seitennummer
\begin{titlepage}
\seminarkopf            % Titelblatt (wie in kopf.tex definiert)
\begin{abstract}
Diese Arbeit gibt eine formale Definition von dynamischen Flüssen mit Warteschlangenmodell und darauf aufbauend eine Charakterisierung von Nash Gleichgewichten in diesem Kontext.
Dabei werden Partikel des Flusses als gleichwertige Spieler betrachtet, die von Zeit zu Zeit kontinuierlich bei einer Quelle entstehen und per Selfish Routing versuchen ihre Ankunftszeit bei einer Senke zu minimieren.
Eine spezielle Klasse von statischen Flüssen, die schmalen Flüsse mit Zurücksetzen, kann verwendet werden, um Nash Flüsse mit Zeithorizont zu erweitern.
\end{abstract}

\end{titlepage}

%%%%%%%%%%%%%%%%%%%%%%%%%%%%%%%%%%%%%%%%%%%%%%%%%%%%%%%%%%%%%%%%%%%%%%%%%%%%%%%%
\thispagestyle{empty}
\tableofcontents        % Inhaltsverzeichnis
%\listoffigures         % Abbildungsverzeichnis (eventuell einfügen)
%\listoftables          % Tabellenverzeichnis (eventuell einfügen)
\setcounter{page}{0}    % Eigentlicher Inhalt beginnt auf Seite 1
\clearpage              % neue Seite für eigentlichen Inhalt

%%%%%%%%%%%%%%%%%%%%%%%%%%%%%%%%%%%%%%%%%%%%%%%%%%%%%%%%%%%%%%%%%%%%%%%%%%%%%%%%
% Eigentlicher Inhalt der Seminararbeit; die einzelnen Teile werden hier (aus Gründen der Übersichtlichkeit) über \input{file} eingebunden

\section{Einführung}\label{introduction}

Routenplanungsspiele spielen in der Analyse und Optimierung von Verkehrs- und Kommunikationsnetzwerken eine große Rolle.
Bei der Modellierung solcher Netzwerke wurden dabei in der Vergangenheit häufig statische Flüsse zu Rate gezogen, welche jedoch nicht abbilden können, wie sich Veränderungen des Flusses im Laufe der Zeit innerhalb des Netzwerks, wie sie beispielsweise in Straßennetzen auftreten, auf das System auswirken.

Im Artikel \glqq Nash Equilibria and the Price of Anarchy for Flows over Time\grqq\ \cite{Koch2011} der Autoren Ronald Koch und Martin Skutella werden daher Routenplanungsspiele auf dynamischen Flüssen (engl. flow over time) betrachtet, die hier dem deterministischen Warte\-schlangen-Modell (engl. deterministic queuing model) entsprechen.
Jedes Partikel im Fluss entspricht dabei einem Spieler, der seine Entscheidungen selbst egoistisch trifft und versucht seine eigene Ankunftszeit zu minimieren.
Dies wird auch Selfish Routing genannt.
Zudem schreibt man den Spielern hier zu, während ihrer Entscheidungsfindung der zu wählenden Route bereits den zukünftigen Flussverlauf und damit künftige Wartezeiten jeder Kante zu kennen.

Diese Arbeit gibt in den Abschnitten~\ref{sec-dynamic-flows} und~\ref{sec-travel-times} eine formale Definition dynamischer Flüsse und des Warteschlangenmodells in diesem Szenario und formuliert eine Charakterisierung von Nash Gleichgewichten im Kontext dynamischer Flüsse in Abschnitt~\ref{sec-nash-flows}.
In Abschnitt~\ref{sec-thin-flows} wird schließlich eine Klasse von statischen Flüssen eingeführt, die verwendet wird, um bereits bestehende dynamische Nash-Flüsse mit Zeithorizont zu erweitern.
\section{Dynamische Flüsse}\label{sec-dynamic-flows}

Zunächst werden einige grundlegende Begriffe eingeführt:

\begin{definition}[Netzwerk]
	Ein \emph{Netzwerk} $(G, u, s, t, \tau)$ ist ein gerichteter, endlicher Graph $G=(V,E)$ mit einer \emph{Quelle} $s\in V$ und einer Senke $t\in V$, sodass alle Knoten von $s$ aus erreichbar sind.
	Jeder Kante $e\in E$ werden eine Kapazität $u_e > 0$ und eine Verzögerungszeit $\tau_e\geq 0$ zugeordnet, sodass alle Zykel $C$ eine positive Gesamtverzögerung $\sum_{e\in C}\tau_e$ haben.
\end{definition}

\begin{definition}
	Der Funktionenraum $\mathfrak{F}_0$ sei die Menge aller Funktionen $g: \R \to \R_{\geq 0}$, die lokal integrierbar bzgl. des Lebesgue-Maßes sind, also $\int_a^b |g(t)| \diff t< \infty$ für beliebige beschränkte Intervalle $(a,b)$ erfüllen, und auf der negativen Achse verschwinden, das heißt, es gilt $g(t)=0$ für $t<0$.
\end{definition}

\begin{definition}[Dynamischer Fluss]
	Ein \emph{dynamischer Fluss $f=(f^+, f^-)$} ist ein Paar zweier über die Kanten $E$ eines Netzwerks indizierten Familien mit $f^+_e,f^-_e\in\mathfrak F_0$ für alle $e\in E$.
	Dabei bezeichnen $f_e^+(\theta)$ bzw. $f_e^-(\theta)$ die \emph{Zu- bzw. Abflussrate an Kante $e\in E$ zum Zeitpunkt $\theta\in\R$}.
	
	Der (kumulative) \emph{Zu- bzw. Abfluss an einer Kante $e$ bis zum Zeitpunkt $\theta$} sei definiert durch $F^+_e(\theta):=\int_0^\theta f^+_e(t) \diff t<\infty$ bzw. $F^-_e(\theta):=\int_0^\theta f^-_e(t) \diff t<\infty$.
	
	Die \emph{(Länge der) Warteschlange $z_e(\theta)$ und die Wartezeit $q_e(\theta)$ an Kante $e\in E$ zum Zeitpunkt $\theta\in\R$} seien gegeben durch $z_e(\theta):= F_e^+(\theta) - F_e^-(\theta + \tau_e)$ und $q_e(\theta) := z_e(\theta) / u_e$.
	
	Man beschreibe die \emph{Austrittszeit $T_e(\theta)$ aus einer Kante $e\in E$ bei Eintrittszeit $\theta$}, zu der ein Partikel eine Kante verlässt, die es zum Zeitpunkt $\theta$ betreten hat, als $T_e(\theta):=\theta + q_e(\theta) + \tau_e$.
\end{definition}

Der Definition der Austrittszeit kann man bereits entnehmen, wie sich Partikel, die an einer Kante $e$ ankommen, verhalten sollen:
Nachdem sie zur Zeit $\theta$ die Kante betreten haben, müssen sie sich zunächst in eine Warteschlange einreihen, welche mit der Kapazität $u_e$ nach dem FIFO-Prinzip abgebaut wird.
Nachdem diese Wartezeit $q_e(\theta)$ vorüber ist, vergeht eine weitere konstante Verzögerungszeit $\tau_e$, bevor sie wieder aus der Kante austreten.
Des Weiteren müssen sich die Partikel bereits sofort bei der Ankunft an einem Knoten entscheiden, in welche Kante sie eintreten wollen, und können nicht an einem Knoten verweilen.

Man kann sich die Kanten also als Transportbänder vorstellen, deren Breite die Transportkapazität bestimmt und deren Länge die Verzögerungszeit darstellt.
Überschreiten die Güter die Kapazität, so bildet sich eine Warteschlange vor dem Band.

Dies führt zu folgender Definition der Zulässigkeit dynamischer Flüsse:

\begin{definition}[Zulässiger dynamischer Fluss]
	Ein dynamischer Fluss $f=(f^+, f^-)$ heißt \emph{zulässig}, falls er die folgenden Eigenschaften erfüllt:
	\begin{enumerate}[label=(F\arabic*)]
		\item\label{def-feasible-flow-capacity} Keine Abflussrate übersteigt die Kapazität, d.h. $\forall e\in E, \theta\in\R: f_e^-(\theta)\leq u_e$.
		\item\label{def-feasible-flow-no-negative-flow} Fluss verlässt eine Kante nur, falls er sie zuvor betreten hat,\\ d.h. $\forall e\in E, \theta\in\R: F_e^+(\theta) \geq F_e^-(\theta + \tau_e).$
		\item\label{def-feasible-flow-no-flow-at-node} Bis auf Quelle und Senke erfüllt jeder Knoten $v$ Flusserhaltung,\\
		d.h. $\forall\theta\in\R: \sum_{e\in\delta^+(v)}f^+_e(\theta) - \sum_{e\in\delta^-(v)} f_e^-(\theta) = 0$.\\
		Für die Senke $t$ muss dieser Wert nicht-positiv und für die Quelle $s$ nicht-negativ sein und heißt für $s$ \emph{der Zufluss $d(\theta)$ in das Netzwerk}.
		\item\label{def-feasible-flow-queue-with-capacity} Warteschlangen werden mit der Kapazität der Kante abgebaut,\\ d.h. $\forall e\in E, \theta\in\R: z_e(\theta) > 0 \implies f_e^-(\theta + \tau_e) = u_e$.
	\end{enumerate}
\end{definition}

Die folgende Proposition beschreibt wichtige Folgerungen über zulässige dynamische Flüsse:

\begin{proposition}\label{prop-feasible-flow}
	Für eine Kante $e\in E$ und einen zulässigen dynamischen Fluss $f$ gilt:
	\begin{enumerate}[label=(\roman*)]
		\item\label{prop-feasible-flow-T-mon-inc-cont} Die Funktion $\theta \mapsto \theta + q_e(\theta)$ ist monoton wachsend und stetig.
		\item\label{prop-feasible-flow-positive-queue} Für alle $\theta\in\R$ ist die Warteschlange $z_e$ auf dem Intervall $(\theta, \theta + q_e(\theta))$ positiv.
		\item\label{prop-feasible-flow-det-outflow} Zu jeder Zeit $\theta\in\R$ gilt $F_e^+(\theta) = F_e^-(T_e(\theta))$.
		\item\label{prop-feasible-flow-queue-delay} Für alle $\theta_1 \leq \theta_2$ mit $\int_{\theta_1}^{\theta_2} f^+_e(t) \diff t = 0$ und $q_e(\theta_2)>0$ gilt $\theta_1 + q_e(\theta_1) = \theta_2 + q_e(\theta_2)$.
	\end{enumerate}
\end{proposition}
\begin{proof}
	In~\ref{prop-feasible-flow-T-mon-inc-cont} folgt die Stetigkeit bereits aus der Stetigkeit von $F_e^+$ und $F_e^-$.
	Um zu zeigen, dass die Funktion monoton wachsend ist, seien $\theta_1 \leq \theta_2$ gegeben.
	Mit der Monotonie von $F_e^+$ und mit $F_e^-(\theta_2 + \tau_e) = F_e^-(\theta_1+\tau_e) + \int_{\theta_1+\tau_e}^{\theta_2+\tau_e} f_e^-(t)\diff t\leq F_e^-(\theta_1 + \tau_e) + (\theta_2 - \theta_1)u_e$ gilt: 
	\[
		\theta_1 + q_e(\theta_1)
		= \theta_1 + \frac{F_e^+(\theta_1) - F_e^-(\theta_1 + \tau_e)}{u_e}
		\leq \theta_1 + \frac{F_e^+(\theta_2) - F_e^-(\theta_1+\tau_e)}{u_e} \leq \theta_2 + q_e(\theta_2).
	\]
	
	Für $\theta'\in (\theta, \theta+q_e(\theta))$ gilt also $\theta' + q_e(\theta') \geq \theta + q_e(\theta)$, womit $q_e(\theta') \geq \theta + q_e(\theta) - \theta' > 0$ gerade Aussage (ii) beweist.
	
	Aussage (iii) folgt dann mit~\ref{def-feasible-flow-queue-with-capacity} und Aussage~(ii), da
	$\int_{\theta}^{\theta + q_e(\theta)}f_e^-(t + \tau_e) \diff t = q_e(\theta)  u_e = z_e(\theta)$ gilt, und damit ist $F_e^-(T_e(\theta)) = F_e^-(\theta+\tau_e) + \int_{\theta+\tau_e}^{\theta+\tau_e+q_e(\theta)}f_e^-(t)\diff t = F_e^+(\theta)$.
	
	Zu Aussage (iv):
	Für alle $\theta'\in [\theta_1, \theta_2]$ gilt $F_e^+(\theta') = F_e^+(\theta_2)$.
	Also ist die Warteschlange $z_e(\theta') = F_e^+(\theta_2) - F_e^-(\theta' + \tau_e) \geq z_e(\theta_2) > 0$ positiv und nach~\ref{def-feasible-flow-queue-with-capacity} gilt $f_e^-(\theta' + \tau_e)=u_e$.
	Die Differenz der Warteschlangen ist
	$z_e(\theta_1)-z_e(\theta_2)=-F^-_e(\theta_1 + \tau_e) + F^-_e(\theta_2 + \tau_e) = (\theta_2 - \theta_1)u_e$, was $q_e(\theta_1) - q_e(\theta_2) = \theta_2 - \theta_1$ impliziert.
\end{proof}
\section{Kürzeste Wege}\label{sec-travel-times}

In diesem Abschnitt wird der Begriff der frühesten Ankunftszeit an einem Knoten eingeführt und erörtert, wann eine Kante $vw$ in einem kürzesten $s$-$w$-Pfad liegt.

\begin{definition}
	Für einen dynamischen Fluss $f$ und einen Pfad $P=(e_1,\dots,e_k)$ definiere $l^P(\theta):=T_{e_k}\circ\dots\circ T_{e_1}(\theta)$ den Zeitpunkt, an dem ein Partikel den Endknoten des Pfads erreicht, falls er den Pfad zum Zeitpunkt $\theta$ betritt.
	
	Für einen Knoten $w\in V$ beschreibe $\mathcal{P}_w$ die Menge aller $s$-$w$-Pfade.
	Dann ist die früheste Ankunft eines Partikels, das zur Zeit $\theta$ bei $s$ startet, gegeben durch $l_w(\theta):=\min_{P\in\mathcal{P}_w}l^P(\theta)$.
	Ein Pfad $P\in \mathcal{P}_w$ heißt \emph{kürzester $s$-$w$-Pfad zur Zeit $\theta$}, falls $l_w(\theta)=l^P(\theta)$.
\end{definition}

\begin{proposition}\label{prop-abs-cont-sur}
	Für einen zulässigen Fluss $f$ sind die Funktionen $F_e^+$ und $F_e^-$ für alle $e\in E$ lokal absolut stetig.
	Die Funktionen $T_e$, $l^P$ sowie $l_v$ sind dabei für alle Kanten $e\in E$, Pfade $P$ in G und Knoten $v\in V$ monoton wachsend, lokal absolut stetig und surjektiv.
\end{proposition}
\begin{proof}
	Nach dem Hauptsatz der Differential- und Integralrechnung für das Lebesgue-Inte\-gral (siehe dazu \cite[Satz 4.14]{Elstrodt2011Abs}) ist $G: [a,b] \to \R, x\mapsto \int_a^x g(t) \diff t$ für eine Lebesgue integrierbare Funktion $g: [a,b] \to \R$ absolut stetig.
	Insbesondere sind also $F_e^+$ sowie $F_e^-$ und damit auch $q_e$ und $T_e$ lokal absolut stetig.
	Als Komposition bzw. punktweises Minimum endlich vieler lokal absolut stetiger Funktionen ist auch $l^P$ bzw. $l_v$ für alle Pfade $P$ und Knoten $v$ lokal absolut stetig.
	Nach Proposition~\ref{prop-feasible-flow}~\ref{prop-feasible-flow-T-mon-inc-cont} ist die Monotonie von $T_e$ bereits gegeben, welche auch die Monotonie von $l^P$ und $l_v$ impliziert.
	Wegen $f_e^+, f_e^-\in\mathfrak{F_0}$ gilt $q_e(\theta)=0$ für $\theta\leq 0$, wodurch auch $\lim_{\theta\to-\infty} T_e(\theta) = - \infty$ folgt.
	Mit $T_e(\theta)\geq \theta$ ergibt sich die Surjektivität von $T_e$.
	Daher sind auch $l^P$ und $l_v$ surjektiv.
\end{proof}

Wie im statischen Szenario von kürzesten Pfaden, gilt auch hier die Dreiecksungleichung: 

\begin{lemma}\label{lemma-dreicksungl}
	Für alle Kanten $vw\in E$ gilt in einem zulässigen Fluss 
	$T_{vw}(l_v(\theta)) \geq l_w(\theta)$.
\end{lemma}
\begin{proof}
	Sei ein kürzester $s$-$v$-Pfad $P$ zum Zeitpunkt $\theta$ gegeben.
	Hängt man an $P$ die Kante $vw$ an, erhält man einen $s$-$w$-Pfad, der zur Eintrittszeit $\theta$ die Ankunftszeit $T_{vw}(l_v(\theta))$ liefert.
	Da $l_w(\theta)$ das Minimum über die Ankunftszeit aller $s$-$w$-Pfade ist, gilt die Behauptung.
\end{proof}

\begin{definition}
	Man bezeichne eine Kante $vw\in E$ als \emph{aktiv zum Zeitpunkt $\theta$}, falls sie auf einem zur Zeit $\theta$ kürzesten $s$-$w$-Pfad liegt, d.h. falls $T_{vw}(l_v(\theta)) = l_w(\theta)$ gilt; sonst nennt man sie \emph{inaktiv zum Zeitpunkt $\theta$}.
	Die Menge $\Theta_e$ ist die abgeschlossene Menge aller Zeitpunkte, zu denen die Kante $e$ aktiv ist.
\end{definition}

Man beachte, dass Teilpfade kürzester Pfade im statischen Sinne wieder kürzeste Pfade sind; im dynamischen Sinne gilt dies nicht unbedingt, jedoch aber in folgendem Teilgraph:

\begin{lemma}\label{lemma-shortest-path-using-active-edges}
	Für einen zulässigen Fluss ist der durch die zur Zeit $\theta$ aktiven Kanten induzierte Teilgraph $G_\theta:=(V, E_\theta)$ zu jeder Zeit ein azyklischer Graph, in dem $s$ jeden Knoten $v\in V$ erreichen kann.
\end{lemma}
\begin{proof}
	Angenommen es existiere ein Zyklus $C=(v_1, \dots, v_n)$ mit $v_1=v_n$ und ausschließlich aktiven Kanten.
	Es ist $l^C(\theta) > \theta$, da für Zyklen eine positive Gesamtverzögerung vorausgesetzt ist.
	Aufgrund der Aktivität aller Kanten in $C$ erzeugt $l_{v_1}(\theta) = l^C(l_{v_1}(\theta)) > l_{v_1}(\theta)$ einen Widerspruch.
	
	Für jeden Knoten $w\neq s$ existiert mindestens eine eingehende aktive Kante -- zum Beispiel die letzte Kante eines kürzesten $s$-$w$-Pfades, welcher wiederum existiert, da $w$ von $s$ aus erreichbar ist.
	Daher ist $w$ von $s$ aus in $G_\theta$ erreichbar.
\end{proof}

\begin{proposition}\label{prop-arrival-times-vector}
	Für einen zulässigen Fluss $f$ ist der Vektor $(l_v(\theta))_{v\in V}$ die eindeutige Lösung des Gleichungssystems
	\[ \tilde{l}_w = \begin{cases}
	\theta, & \text{falls } w=s, \\
	\min\limits_{vw\in \delta^-(w)} T_{vw}(\tilde{l}_v), & \text{sonst}.
	\end{cases} \]
\end{proposition}
\begin{proof}
	Offenbar löst $(l_v(\theta))_{v\in V}$ dieses System, da nach Lemma~\ref{lemma-shortest-path-using-active-edges} jeder Knoten $w\neq s$ eine aktive eingehende Kante hat.
	Für eine Lösung $(\tilde{l}_v)_{v\in V}$ des Gleichungssystems, zeige man $l_w(\theta) = \tilde{l}_w$ für jeden Knoten $w\in V$.
	Dabei ist der Teilgraph $G'=(V, E')$ mit
	\[ E' := \{ vw \in E \mid T_{vw}(\tilde{l}_v ) = \tilde{l}_w \} \]
	ein azyklischer Graph, in dem $s$ jeden Knoten $w\in V$ erreichen kann:
	Zyklen können wegen der positiven Gesamtverzögerung nicht entstehen und jeder Knoten $w\neq s$ hat mindestens eine eingehende Kante $vw$ mit $T_{vw}(\tilde{l}_v) = \tilde{l}_w$.
	Daher ist jede Knotenbewertungen $\tilde{l}_w$ bereits durch einen $s$-$w$-Pfad $P$ in $G'$ festgelegt auf $l^P(\theta)\geq l_w(\theta)$.
	Für einen zur Zeit $\theta$ kürzesten $s$-$w$-Pfad $Q$ gilt außerdem $\tilde{l}_w \leq T^Q(\tilde{l}_s) = T^Q(\theta) = l_w(\theta)$.
\end{proof}

Um $(l_v(\theta))_{v\in V}$ für alle $\theta\in\R$ gleichzeitig zu berechnen, kann der Bellman-Ford-Algorithmus auf den Distanzvektor-Funktionen $(l_v)_{v\in V}$ genutzt werden:
Dazu wird in jeder der $n-1$ Iterationen für jede Kante das punktweise Minimum $l_w := \min\{ l_w, T_{vw}\circ l_v \}$ gebildet.
Sind Operationen auf Funktionen nicht möglich oder zu teuer, so kann $(l_v(\theta))_{v\in V}$ für ein spezielles $\theta\in\R$ mittels Dijkstra-Algorithmus ermittelt werden, wobei man die Kosten einer Kante $vw$ erst bei Scanning von $v$ in der Form $q_{vw}(l_v(\theta)) + \tau_{vw}$ berechnet.
\section{Dynamische Nash-Flüsse}\label{sec-nash-flows}

Dieser Abschnitt dient dazu, Nash Gleichgewichte im Kontext dynamischer Flüsse einzuführen.
Dabei hilft die Anschauung, dass Partikel, die zur Zeit $\theta$ an der Quelle erscheinen, in einem Nash Gleichgewicht möglichst früh, also zum Zeitpunkt $l_t(\theta)$, an der Senke ankommen.
Für die formale Einführung benötigt man weitere Definitionen: 

\begin{definition}
	Für eine Kante $vw\in E$ bezeichne $x_{vw}^+(\theta):= F_{vw}^+(l_v(\theta))$ den Zufluss bis zur frühestmöglichen Ankunftszeit von Partikeln in $v$, die zur Zeit $\theta$ in $s$ starten.\\
	Dagegen bezeichne $x_{vw}^-(\theta):= F^-_{vw}(l_w(\theta))$ den Abfluss bis zur frühestmöglichen Ankunftszeit von Partikeln in $w$, die zur Zeit $\theta$ in $s$ starten.
	
	Für einen Knoten $v\in V$ sei $b_v(\theta):=\sum_{e\in\delta^+(v)} x_e^+(\theta) - \sum_{e\in\delta^-(v)} x_e^-(\theta)$ die Balance des Knoten $v$ zum Zeitpunkt $\theta$.
\end{definition}

\begin{remark}\label{remark-x^-leqx^+}
	In einem zulässigen Fluss ist $x_{vw}^-(\theta) = F_{vw}^-(l_w(\theta)) \leq F_{vw}^-(T_{vw}(l_v(\theta)))=F_{vw}^+(l_v(\theta)) = x_{vw}^+(\theta)$
	nach Proposition~\ref{prop-feasible-flow}~\ref{prop-feasible-flow-det-outflow} und mit der Monotonie von $F_{vw}^-$.
\end{remark}

\begin{lemma}\label{lemma-balance-0}
	Für einen zulässigen dynamischen Fluss $f$ gilt $b_v(\theta)=0$ für alle Knoten $v\in V\setminus\{ s,t \}$ und alle $\theta\in\R$.
\end{lemma}
\begin{proof}
	Unter Benutzung der Voraussetzung~\ref{def-feasible-flow-no-flow-at-node} folgere man für $v\in V\setminus \{ s, t\}, \theta\in\R$:
	\[ \sum_{e\in\delta^-(v)} x_e^-(\theta) = \int_{0}^{l_v(\theta)} \sum_{e\in\delta^-(v)} f_e^-(t) \diff t = \int_{0}^{l_v(\theta)} \sum_{e\in\delta^+(v)} f_e^+(t) \diff t = \sum_{e\in\delta^+(v)}x_e^+(\theta). \]
\vspace{-1.4 \baselineskip}

\end{proof}

\subsection{Charakterisierung dynamischer Nash-Flüsse}

\begin{notation}
	 $M^c:= \R\setminus M$ bezeichne das Komplement von $M\subseteq\R$, $\overline{M}$ den Abschluss.
\end{notation}

\begin{definition}\label{def-flow-along-active-edges}
	Man sage, der Fluss $f$ \emph{fließe nur entlang aktiver Kanten}, falls $f_{vw}^+$ fast überall auf $l_v(\Theta_{vw}^c)$ verschwindet für alle Kanten $vw\in E$.
\end{definition}

\begin{remark}
	Diese Definition weicht von der Definition von Koch und Skutella ab und entspricht derjenigen aus~\cite[Definition 1]{Cominetti2015}:
	Nach \cite[Definition 2]{Koch2011} sagt man, $f$ \emph{sende Fluss nur entlang aktuell kürzester Pfade}, falls $f_{vw}^+\circ l_v$ fast überall auf $\Theta_{vw}^c$ verschwindet für alle Kanten $vw$.

	Entspricht $f$ dieser Definition, so auch Definition~\ref{def-flow-along-active-edges}: 
	Da $l_v$ nach Proposition~\ref{prop-abs-cont-sur} absolut stetig ist, bildet es nach~\cite[Aufgabe 4.9]{Elstrodt2011Abs} Nullmengen wieder auf Nullmengen ab, weshalb folgende Menge eine Nullmenge ist: \[ l_v(\{ \theta \in \Theta_{vw}^c \mid f_{vw}^+ (l_v(\theta)) > 0 \}) = \{ \xi \in l_v(\Theta_{vw}^c) \mid f_{vw}^+ (\xi) > 0 \}. \]
	 
	Koch und Skutella zeigen im Beweis von~\cite[Lemma 1]{Koch2011} die entsprechende Äquivalenz von Lemma~\ref{lemma-only-active-edges} (i) und (iii) -- jedoch in (i) unter Verwendung ihrer Definition -- und
	verwenden bei der Implikation (iii)$\Rightarrow$(i) das Argument, dass für jede Kante $vw\in E$ und alle $\theta\in \Theta_{vw}^c$ eine Umgebung $U$ von $\theta$ existiert, sodass $f_{vw}^+$ fast überall in $l_v(U)$ verschwindet.
	Dies reicht aber nicht aus, um zu zeigen, dass $f_{vw}^+(l_v(\theta))=0$ für fast alle $\theta\in\Theta_{vw}^c$ gilt:
	So kann $f_{vw}^+(l_v(\theta))$ für ein $\theta\in\Theta_{vw}^c$ positiv sein und $l_v$ konstant in einer Umgebung um $\theta$.
	Dann ist $f_{vw}^+ \circ l_v$ in einer Umgebung um $\theta$ positiv, was im Widerspruch zur Forderung ist.
	
	Dies wurde in~\cite[Example 2]{Cominetti2015} ausgenutzt, um einen Beispielfluss anzugeben, der beweist, dass die Forderung von Koch und Skutella sogar echt stärker ist.
\end{remark}

Für eine äquivalente Umschreibung dieser Definition, benötigen wir folgendes Lemma der Maßtheorie:

\begin{lemma}\label{lemma-vanishes-intervals}
	Seien $g: \R \to \R_{\geq 0}$ eine lokal Lebesgue-integrierbare Funktion und $((a_i, b_i))_{i\in I}$ eine Familie offener Intervalle.
	Dann verschwindet $g$ fast überall auf $\Theta:=\bigcup_{i\in I} (a_i, b_i)$ genau dann, wenn es für alle $i\in I$ fast überall auf $(a_i, b_i)$ verschwindet.
\end{lemma}
\begin{proof}
	Verschwindet $g$ fast überall auf $\Theta$, so erst recht auf jedem Intervall $(a_i, b_i)$.
	Für die andere Richtung definiert die Funktion $\mu(A):= \int_A g \diff \lambda$ ein Maß auf den Borelmengen~$\mathfrak{B}$.
	Da jede offene Menge $O\subseteq\R$ $\sigma$-kompakt ist, also eine Darstellung als abzählbare Vereinigung kompakter Mengen -- hier $O=\bigcup_{n\in\N}(\{ x \in\R \mid d(x, O^c) \geq 1/n \} \cap [-n, n] )$ -- besitzt, ist jede offene Menge nach~\cite[1.2 Folgerungen (e)]{Elstrodt2011Top} innen regulär.
	Das heißt, es gilt
	\[ \mu(O)=\sup\{ \mu(K) \mid K\subseteq O \text{ kompakt} \} \]
	für offene Mengen $O\subseteq\R$.
	Für ein kompaktes $K\subseteq \Theta$ existiert eine endliche Teil\-über\-deckung $\bigcup_{i=1}^n (a_i, b_i) \supseteq K$, für die $\mu(K) \leq \sum_{i=1}^{n} \mu((a_i, b_i)) = \sum_{k=1}^{n} \int_{a_i}^{b_i} g(t) \diff t = 0$ gilt.
	Also ist auch $\mu(\Theta)=0$.
\end{proof}

\begin{lemma}\label{lemma-only-active-edges}
	Für einen zulässigen Fluss $f$ sind folgende Aussagen äquivalent:
	\begin{enumerate}[label=(\roman*)]
		\item Der Fluss $f$ fließt nur entlang aktiver Kanten.
		\item Für jede Kante $vw\in E$ und für fast alle $\xi\in\R$ mit	$f_{vw}^+(\xi)>0$ gilt $\xi \in l_v(\Theta_{vw})$.
		\item Für jede Kante $e\in E$ und für alle $\theta\in\R$ gilt $x_e^+(\theta) = x_e^-(\theta)$.
	\end{enumerate}
\end{lemma}
\begin{proof}
	$(i) \Leftrightarrow (ii)$: Bedingung~(ii) gilt genau dann, wenn $f_{vw}^+$ fast überall auf $l_v(\Theta_{vw})^c$ verschwindet.
	Daher genügt es, zu zeigen, dass sich $l_v(\Theta_{vw})^c$ und $l_v(\Theta_{vw}^c)$ nur um eine Nullmenge voneinander unterscheiden.
	Mit der Surjektivität von $l_v$ gilt $l_v(\Theta_{vw})^c\subseteq l_v(\Theta_{vw}^c)$.
	
	Des Weiteren ist $S:=l_v(\Theta_{vw}^c)\setminus l_v(\Theta_{vw})^c = l_v(\Theta_{vw}^c)\cap l_v(\Theta_{vw})\subseteq l_v(\Q)$:
	Für ein $\xi\in S$ gibt es $\theta\in\Theta_{vw}^c$ und $\theta'\in\Theta_{vw}$ mit $l_v(\theta)=\xi=l_v(\theta')$.
	Da $\theta\neq\theta'$ ist, existiert ein $\theta_q\in\Q\cap(\theta,\theta')$.
	Wegen der Monotonie von $l_v$ gilt $l_v(\theta_q)=\xi$, womit $\xi\in l_v(\Q)$ folgt.
	Also unterscheiden sich die beiden Mengen nur um eine abzählbare Menge.
	
	$(i)\Leftrightarrow (iii)$: Sei eine Kante $vw\in E$ gegeben.
	Für ein $\theta\in\R$ bezeichne $\omega_\theta\leq \theta$ den spätesten Startzeitpunkt, sodass man unter Benutzung von $vw$ zum Zeitpunkt $l_w(\theta)$ zu $w$ gelangt:
	\[ \omega_\theta:=\max\{ \omega\leq\theta \mid l_w(\theta) = T_{vw}(l_v(\omega)) \}. \]
	
	Es gilt $\Theta_{vw}^c = \bigcup_{\theta\in\R} (\omega_\theta, \theta)$:
	Für $\theta\in\Theta_{vw}^c$ gilt $T_{vw}(l_v(\theta)) > l_w(\theta)$.
	Aufgrund der Stetigkeit von $T_{vw}\circ l_v$ und von $l_w$ existiert ein $\varepsilon>0$, sodass $T_{vw}(l_v(\theta')) > l_w(\theta+\varepsilon)$ für $\theta'\in[\theta,\theta+\varepsilon]$ gilt.
	Also ist $\theta\in(\omega_{\theta+\varepsilon}, \theta+\varepsilon)$.
	Ist umgekehrt $\theta'\in (\omega_\theta,\theta)$, so ist aufgrund der Monotonie $T_{vw}(l_v(\theta'))\geq T_{vw}(l_v(\omega_\theta)) = l_w(\theta)\geq l_w(\theta')$.
	Die erste Ungleichung kann nicht mit Gleichheit erfüllt sein, da $\omega_\theta$ maximal mit der Eigenschaft $T_{vw}(l_v(\omega)) = l_w(\theta)$ ist, wodurch $\theta'\in\Theta_{vw}^c$ folgt.
	
	Mit $l_v(\Theta_{vw}^c) = \bigcup_{\theta\in\R}(l_v(\omega_\theta),l_v(\theta))$ verschwindet $f_{vw}^+$ nach Lemma~\ref{lemma-vanishes-intervals} genau dann fast überall auf $l_v(\Theta_{vw}^c)$, wenn es für alle $\theta\in\R$ fast überall auf $(l_v(\omega_\theta),l_v(\theta))$ verschwindet.
	Dies ist nach Proposition~\ref{prop-feasible-flow}~\ref{prop-feasible-flow-det-outflow} wiederum äquivalent zu
	$F_{vw}^+(l_v(\theta))-F_{vw}^-(l_w(\theta))=0$ für alle $\theta\in\R$.
\end{proof}

\begin{definition}
	Man sage, ein zulässiger dynamischer Fluss $f$ \emph{fließe ohne Überholungen}, falls $b_s(\theta) = -b_t(\theta)$ für alle $\theta\in\R$.
\end{definition}

Dabei betrachte man folgende Intuition:
 Partikel, die zur Zeit $\theta\in\R$ bei $s$ starten und sich auf einem kürzesten Weg zu $t$ bewegen  -- also zur Zeit $l_t(\theta)$ in $t$ ankommen --, überholen andere Partikel, falls $b_s(\theta) > -b_t(\theta)$.
Falls jedoch $b_s(\theta) < - b_t(\theta)$ gilt, wurde das Partikel bereits von anderen überholt.
Ein Nash-Gleichgewicht sollte diese Eigenschaft daher erfüllen.

\begin{definition}
	Seien ein statischer Fluss $f \in \R^E$ in einem Graphen $G=(V,E)$ mit Kapazitäten $u\in \R_+^E$ und ein Balancevektor $b\in\R^V$ mit $\sum_{v\in V} b_v = 0$ gegeben.
	Der Fluss $f$ heißt \emph{$b$-Fluss}, falls er Flusserhaltung bzgl. $b$ gewährt, d.h. falls alle $v\in V$ die Bedingung $\sum_{e\in\delta^+(v)}f_e - \sum_{e\in\delta^-(v)}f_e = b_v$ erfüllen.
\end{definition}

\newcommand{\newv}{\mathbf{v}}
\begin{lemma}\label{lemma-b-graph}
	Seien ein dynamischer Fluss $f$ in einem Graphen $G=(V,E)$ und ein Zeitpunkt $\theta\in\R$ gegeben.
	Der Graph $H$ entstehe aus $G$, indem man jede Kante $vw\in E$ aus $G$ durch einen neuen Knoten $\newv_{vw}$ und zwei Kanten $v\newv_{vw}$ und $\newv_{vw}w$ ersetze.
	Der statische Fluss $g$ auf $H$ sei definiert durch
	\[ g_{v\newv_{vw}} := x_{vw}^+(\theta) \text{ und } g_{\newv_{vw}w} := x_{vw}^-(\theta) \text{ für alle $vw\in E$} \]
	und die Balance $b$ auf $H$ sei gegeben durch $b_v:= b_v(\theta)$ für $v\in V$ und $b_{\newv_e}:= x_e^-(\theta) - x_e^+(\theta)$ für $e\in E$.
	Dann gelten die folgenden Aussagen:
	
	\begin{enumerate}[label=(\roman*)]
		\item Der Fluss $g$ ist ein statischer $b$-Fluss.
		\item\label{lemma-b-graph-imp} Ist $f$ zulässig, so gilt $\forall e\in E : x_e^+(\theta) = x_e^-(\theta)\iff b_s(\theta) + b_t(\theta) = 0$.
	\end{enumerate}
\end{lemma} 
\begin{proof}
	$(i)$: Um zu zeigen, dass die Summe über die Balanceeinträge verschwindet, erkenne man, dass der Anteil einer Kante $e\in E$ in $\sum_{v\in V} b_v$ gerade $x_e^+(\theta) - x_e^-(\theta)$ ist.
	Damit gilt:
		\[ \sum_{v\in V}b_v + \sum_{e\in E} b_{\newv_e} = \sum_{e\in E}  (x_e^+(\theta) - x_e^-(\theta) + x_e^-(\theta) - x_e^+(\theta)) = 0. \]
		Es bleibt zu zeigen, dass $g$ bezüglich $b$ Flusserhaltung gewährt.
		Für die Knoten der Form $\newv_{vw}$ gilt dies, da $g_{\newv_{vw}w} - g_{v\newv_{vw}} = x_{vw}^-(\theta) - x_{vw}^+(\theta) = b_{\newv_{vw}}$.
		Für $v\in V$ gilt nach Konstruktion
		\[ b_v =
		\sum_{e\in\delta^+_G(v)} x_{e}^+(\theta) - \sum_{e\in\delta^-_G(v)} x_{e}^-(\theta) =
	\sum_{e\in\delta_H^+(v)} g_e - \sum_{e\in\delta^-_H(v)}g_e
		. \]
	
	$(ii)$: Tatsächlich benötigt man aus $(i)$ nur die Eigenschaft, dass die Summe über die Einträge des Balancevektors verschwindet.
	Mit Lemma~\ref{lemma-balance-0} gilt wegen der Zulässigkeit von $f$ sogar $b_s(\theta)+b_t(\theta) + \sum_{e\in E} b_{\newv_e} = 0$.
	
	Angenommen, es gelte $x_e^+(\theta) = x_e^-(\theta)$ für alle $e\in E$.
	Dann sind auch alle $b_{\newv_e} = 0$ und es gilt $b_s(\theta) + b_t(\theta) = 0$.
	Setzt man $b_s(\theta) + b_t(\theta) = 0$ voraus, so ist $\sum_{e\in E} b_{\newv_e} = 0$ und, da $f$ zulässig ist, gilt $x_e^-(\theta)\leq x_e^+(\theta)$ nach Bemerkung~\ref{remark-x^-leqx^+}.
	Daher gilt $b_{\newv_e}\leq 0$ für alle $e\in E$, weshalb bereits alle $b_{\newv_e} = 0$ sein müssen.
\end{proof}

Die Ergebnisse aus Lemma~\ref{lemma-only-active-edges} und Lemma~\ref{lemma-b-graph} werden im folgenden Theorem gesammelt, welches Nash-Gleichgewichte in dynamischen Flüssen charakterisiert:

\begin{theorem}[Charakterisierung dynamischer Nash-Flüsse]\label{thm-equivalencies-nash-flow}
	Für einen zulässigen dynamischen Fluss $f$ sind die folgenden Aussagen äquivalent:
	\begin{enumerate}[label=(\roman*)]
		\item Der Fluss $f$ fließt nur entlang aktiver Kanten.
		\item Für alle Kanten $e\in E$ und zu jeder Zeit $\theta\in\R$ gilt $x_e^+(\theta) = x_e^-(\theta)$.
		\item Der Fluss $f$ fließt ohne Überholungen.
	\end{enumerate}
	Gilt eine dieser Aussagen, so nennt man $f$ einen \emph{dynamischen Nash-Fluss}.
\end{theorem}

\subsection{Eigenschaften dynamischer Nash-Flüsse}

In diesem Abschnitt werden einige Ergebnisse über Nash-Flüsse gesammelt, die in Abschnitt~\ref{sec-thin-flows} benötigt werden.

\begin{remark}\label{remark-s-t-flow}
	In einem Nash-Fluss ist der statische Fluss $x(\theta)$ mit $x_e(\theta):=x_e^+(\theta)=x_e^-(\theta)$ nach Lemma~\ref{lemma-balance-0} für alle $\theta\in\R$  ein statischer $s$-$t$-Fluss.
	Wegen der Monotonie von $x_e$ ist auch $x(\theta_2) - x(\theta_1)$ für $\theta_1 \leq \theta_2$ ein statischer $s$-$t$-Fluss, genauso wie $x'(\theta)$, falls $x_e$ für alle $e\in E$ differenzierbar in $\theta$ ist, da Differenzieren die Flusserhaltung erhält und $x_e$ monoton wachsend ist für alle $e\in E$.
\end{remark}

\begin{lemma}\label{lemma-x-locally-constant}
In einem dynamischen Nash-Fluss ist $x_e$ eingeschränkt auf $\overline{\Theta_e^c}$, also dem Abschluss der Menge der inaktiven Zeitpunkte von $e$, für jede Kante $e\in E$ lokal konstant.
\end{lemma}
\begin{proof}
Da $\Theta_{vw}^c$ eine in $\R$ offene Menge ist, hat sie eine Darstellung als abzählbare Vereinigung paarweise disjunkter offener Intervalle.
Innerhalb eines solchen Intervalls $(\theta_1, \theta_2)$ gilt $x_{vw}(\theta_2) - x_{vw}(\theta_1) = \int_{l_v(\theta_1)}^{l_v(\theta_2)} f_{vw}^+(t) \diff t = 0$, da $f$ nur entlang aktiver Kanten fließt.
Der Rest folgt mit der Monotonie und Stetigkeit von $x_{vw}$.
\end{proof}

\begin{lemma}\label{lemma-nash-flow-waiting-queue-implies-active-edge}
	Seien ein dynamischer Nash-Fluss $f$, eine Kante $vw\in E$ und ein Zeitpunkt $\theta\in\R$ gegeben.
	Gilt eine der folgenden Aussagen, so ist $vw$ zum Zeitpunkt $\theta$ aktiv:
	\begin{enumerate}[label=(\roman*)]
		\item Die Ableitung $x_{vw}'(\theta)$ existiert und es gilt $x_{vw}'(\theta)> 0$.
		\item Die Wartezeit $q_{vw}$ an der Kante $vw$ ist zur Zeit $l_v(\theta)$ positiv.
	\end{enumerate}
	Insbesondere verschwindet nach (iii) die Wartezeit $q_{vw}(l_v(\theta))$ für $\theta\in\overline{\Theta_{vw}^c}$.
\end{lemma}
\begin{proof}
	Zu Aussage (i): Angenommen, $vw$ wäre zum Zeitpunkt $\theta$ nicht aktiv, so würde wegen der Offenheit von $\Theta_{vw}^c$ und Lemma~\ref{lemma-x-locally-constant} die Ableitung $x_{vw}'(\theta)$ verschwinden.
	
	Für Aussage (ii) zeige man $T_{vw}(l_v(\theta)) \leq l_w(\theta)$.
	Sei $\theta_1$ der früheste Zeitpunkt mit $x_{vw}^+(\theta_1)= x_{vw}^+(\theta)$.
	Dieser existiert, da $l_v$ nach Proposition~\ref{prop-abs-cont-sur} surjektiv ist.
	Dann ist $\theta_1\in \Theta_{vw}$ nach Lemma~\ref{lemma-x-locally-constant}.
	Außerdem ist $\theta_1 \leq \theta$ wegen der Monotonie von $F_{vw}^+ \circ l_v$.
	Nach Aussage (i) gilt nun $T_{vw}(l_v(\theta_1)) = l_w(\theta_1)$.
	Nach Proposition~\ref{prop-feasible-flow}~\ref{prop-feasible-flow-queue-delay} ist $T_{vw}(l_v(\theta_1)) = T_{vw}(l_v(\theta))$ und mit der Monotonie von $l_w$ folgt $T_{vw}(l_v(\theta))\leq l_w(\theta)$.
\end{proof}

\begin{proposition}\label{prop-nash-flow-s-t-path-decomposable}
	Für einen dynamischen Nash-Fluss $f$ und zwei Zeitpunkte $\theta_1 \leq \theta_2$ ist der statische $s$-$t$-Fluss $x(\theta_2) - x(\theta_1)$ eine Komposition von $s$-$t$-Wegen.
\end{proposition}
\begin{proof}
	Sei $\theta$ das Infimum aller Zeitpunkte $\xi\geq\theta_1$, zu denen $x(\xi) - x(\theta_1)$ nicht in $s$-$t$-Wege zerlegbar ist.
	Man nehme $\theta \leq \theta_2$ an.
	Da inaktive Kanten zum Zeitpunkt $\theta$ bereits kurz vor $\theta$ und noch kurz nach $\theta$ inaktiv sind, existiert ein Intervall $[\theta - \varepsilon, \theta + \varepsilon]$, in der keine inaktive Kante aktiv wird.
	Außerdem existiert für $\xi_0 := \max \{ \theta_1, \theta - \varepsilon \}$ eine $s$-$t$-Wegezerlegung von $x(\xi_0) - x(\theta_1)$.
	
	Für einen Pfad $P$ und einen statischen Fluss $g$ sei $g^P := \min_{e\in P} g_e$ der Fluss, der auf dem Pfad $P$ fließt.
	Für einen Zyklus $C$ ist $(x(\xi) - x(\xi_0))^C = 0$ für $\xi\in [\xi_0, \theta+\varepsilon]$, da aufgrund der Azyklizität von $G_{\xi_0}$ eine Kante $e\in C$ des Zyklus existiert, die zur Zeit $\xi_0$ und damit in ganz $[\xi_0, \theta+\varepsilon]$ inaktiv ist, wodurch $x_e(\xi_0) = x_e(\xi)$ nach Lemma~\ref{lemma-x-locally-constant} folgt.
	
	Also hat der $s$-$t$-Fluss $x(\xi) - x(\xi_0)$ für $\xi\in [\xi_0, \theta + \varepsilon]$ keinen Zyklus mit positivem Fluss und besitzt daher eine $s$-$t$-Wegezerlegung.
	Addiert man diese zur $s$-$t$-Wegezerlegung von $x(\xi_0) - x(\theta_1)$, so erhält man eine $s$-$t$-Wegezerlegung von $x(\xi) - x(\theta_1)$, was für $\xi > \theta$ einen Widerspruch zur Definition von $\theta$ darstellt.
\end{proof}

\begin{corollary}
	Für einen dynamischen Nash-Fluss $f$ ist der statische $s$-$t$-Fluss $x(\theta)$ zu jeder Zeit $\theta$ eine Komposition von $s$-$t$-Wegen.
\end{corollary}
\begin{proof}
	Nach Proposition~\ref{prop-abs-cont-sur} existiert ein Zeitpunkt $\xi_0$ mit $l_v(\xi_0) \leq 0$ für alle Knoten $v\in V$.
	Für $\theta \leq \xi_0$ ist $x(\theta)$ der Nullfluss und offenbar in $s$-$t$-Wege zerlegbar, da die Funktionen $f_e^+$ und $f_e^-$ links der $y$-Achse verschwinden.
	Sonst ist $x(\theta)=  x(\theta) - x(\xi_0)$ nach Proposition~\ref{prop-nash-flow-s-t-path-decomposable} in $s$-$t$-Wege zerlegbar.
\end{proof}
\section{Erweiterung dynamischer Nash-Flüsse}\label{sec-thin-flows}

Der Abschnitt beginnt mit der Einführung einer neuen Klasse statischer $s$-$t$-Flüsse:

\begin{definition}[Schmaler Fluss mit Zurücksetzen]\label{def-thin-flow}
	Seien ein statischer $s$-$t$-Fluss $x'$ von Wert $F$ in einem Netzwerk $(G, u, s, t)$ mit Versorgungsrate $d$ sowie eine Teilmenge $E_1\subseteq E$ der Kanten gegeben.
	Der Fluss $x'$ heißt \emph{schmaler Fluss mit Zurücksetzen auf $E_1$}, falls eine Knotenbewertung $l'\in\R^V$ existiert, die die folgenden Bedingungen erfüllt:
	\begin{enumerate}[label=(T\arabic*)]
	\item\label{def-thin-flow-source} $l_s' = F/d$,
	\item\label{def-thin-flow-x-zero} $l_w' \leq l_v'$, \tabto{4cm} für $vw\in E \setminus E_1$ mit $x'_{vw}=0$,
	\item\label{def-thin-flow-x-positive} $l_w' = \max(l_v', x'_{vw} / u_{vw} )$,  \tabto{4cm} für $vw\in E\setminus E_1$ mit $x'_{vw} > 0$,
	\item\label{def-thin-flow-resetting-edge} $l_w' = x'_{vw} / u_{vw}$,  \tabto{4cm} für $vw\in E_1$,
	\item\label{def-thin-flow-no-resetting-edge} $l_w' \geq \min_{vw\in \delta^-(w)} l_v'$, \tabto{4cm} falls $\delta^-(w)\cap E_1 = \emptyset$.
	\end{enumerate}
\end{definition}

Dabei nennt man $x_e'/u_e$ die Auslastung einer Kante $e\in E$.
Ist $E_1 = \emptyset$, dann ist die Knotenbeschriftung $l_v'$ gerade die Auslastung $\max_{e\in P} x_e'/u_e$ eines jeden $s$-$v$-Pfades $P$ mit positiven Fluss.
Kanten in $E_1$ setzen dann die Auslastung jedes vorangegangenen Pfades auf ihre eigene zurück.

\begin{remark}\label{remark-thin-flow}
	Hier ist, wie in~\cite[Definition~4]{Cominetti2011}, im Vergleich zu~\cite[Definition~6]{Koch2011} die Bedingung~\ref{def-thin-flow-no-resetting-edge} zusätzlich eingeführt worden.
	Diese ist nötig, um im Beweis von Theorem~\ref{thm-alpha-extension-is-nash-flow} zu zeigen, dass die erweiterten Ankunftszeiten tatsächlich mit den angegebenen übereinstimmen.
	Ohne diese Bedingung würde dies nämlich nicht gelten, wie man in Abbildung~\ref{figure-labels} erkennen kann.
\end{remark}

Das folgende Theorem liefert das Resultat, dass ein dynamischer Nash-Fluss zu jedem Zeitpunkt $\theta\in\R$ einen schmalen Fluss mit Zurücksetzen induziert.
Die Voraussetzung, dass hierbei die entsprechenden Ableitungen existieren, kann man dadurch rechtfertigen, dass nach \cite[Folgerung~4.12~b)]{Elstrodt2011Abs} absolut stetige Funktionen fast überall differenzierbar sind.
Zudem gilt nach \cite[Aufgabe~4.10]{Elstrodt2011Abs} die Substitutionsregel für absolut stetige Funktionen, wodurch $\int_{l_v(\theta_1)}^{l_v(\theta_2)} f_{vw}^+(t) \diff t = \int_{\theta_1}^{\theta_2} f_{vw}^+(l_v(t)) l_v'(t) \diff t$ gefolgert werden kann.
Daher gilt auch $x_{vw}'(\theta) = f_{vw}^+(l_v(\theta))l_v'(\theta)$ bzw. analog $x_{vw}'(\theta) = f_{vw}^-(l_w(\theta)) l_w'(\theta)$ für fast alle $\theta\in\R$.

\begin{theorem}
	Seien ein dynamischer Nash-Fluss $f$ in $(V,E)$ sowie $\theta\in\R$ gegeben.
	Existieren die Ableitungen $x_{vw}'(\theta)$ und $l_v'(\theta)$ mit $x_{vw}'(\theta) = f_{vw}^+(l_v(\theta)) l_v'(\theta)= f_{vw}^-(l_w(\theta))l_w'(\theta)$ für alle Kanten $vw$ und Knoten $v$, so ist der statische Fluss $x'(\theta) \in \R^{E_\theta}$, eingeschränkt auf die zu $\theta$ aktiven Kanten, ein schmaler $d(\theta)$-wertiger Fluss mit Zurücksetzen auf den Kanten mit Warteschlange $E_1:=\{vw\in E \mid q_{vw}(l_v(\theta))>0 \}$ bei Versorgungsrate~$d(\theta)$.
	Als Knotenbewertung dienen dazu die Ableitungen $(l_v'(\theta))_{v\in V}$.
\end{theorem}
\begin{proof}
	Man bemerke zunächst $l_s'(\theta) = 1$.
	Da außerdem der Einfluss in das Netzwerk $d(\theta)=  \sum_{e\in \delta^+(s)} f_e^+(\theta) - \sum_{e\in\delta^-(s)} f_e^-(\theta) = \sum_{e\in \delta^+(s)} x_e'(\theta) - \sum_{e\in\delta^-(s)} x_e'(\theta)$ erfüllt, folgt~\ref{def-thin-flow-source}, weil $x_e'(\theta)$ nach Lemma~\ref{lemma-x-locally-constant} für inaktive Kanten $e$ verschwindet.
	
	Um Bedingung~\ref{def-thin-flow-no-resetting-edge} zu zeigen,
	nehme man an, dass die Warteschlange jeder eingehenden Kante $vw$ eines Knotens $w$ zur Zeit $l_v(\theta)$ leer ist.
	Da für alle $n\in\N$ eine eingehende und zur Zeit $\theta_n := \theta + 1/n$ aktive Kante existiert, gibt es eine Kante $vw$ die zu unendlich vielen $\theta_n$ aktiv ist.
	Betrachtet man diese Teilfolge $(\theta_{n_k})_{k\in\N}$ der aktiven Zeitpunkte von $vw$, ist $vw$ wegen der Stetigkeit auch zur Zeit $\theta$ aktiv und es gilt 
	\[
	l_w'(\theta) = \lim_{k\to\infty} \frac{l_w(\theta_{n_k})- l_w(\theta)}{1/n_k} = \lim_{n\to\infty} \frac{ l_v(\theta_{n_k}) + q_{vw}(l_v(\theta_{n_k})) - l_v(\theta) }{1/n_k} \geq l_v'(\theta).
	\]
	Also ist insbesondere $l_w'(\theta) \geq \min_{vw\in \delta^-(w)\cap E_\theta} l_v'(\theta)$.
	
	Sei nun eine Kante $vw\in E_\theta$, also eine aktive Kante zum Zeitpunkt $\theta$, gegeben. Man prüfe die Bedingungen~\ref{def-thin-flow-x-zero}, \ref{def-thin-flow-x-positive} und \ref{def-thin-flow-resetting-edge} jeweils in den folgenden drei Fällen:
	
	\begin{description}[leftmargin=0cm, topsep=0cm, itemindent=0.5cm]
		\item[1. Fall:] $\exists \varepsilon > 0:\forall \theta'\in (\theta, \theta + \varepsilon ] : q_{vw}(l_v(\theta')) > 0$.
		
		Nach Lemma~\ref{lemma-nash-flow-waiting-queue-implies-active-edge} ist $[\theta,\theta+\varepsilon]\subseteq \Theta_{vw}$.
		Außerdem ist $q_{vw}$ nach Proposition~\ref{prop-feasible-flow}~\ref{prop-feasible-flow-positive-queue} auf dem Intervall $[ l_v(\theta') , l_w(\theta') - \tau_{vw} )$ positiv, also insbesondere auf $( l_w(\theta)-\tau_{vw} , l_w(\theta + \varepsilon) - \tau_{vw} )$.
		Man folgere $x_{vw}(\theta + \varepsilon) - x_{vw}(\theta) = \int_{l_w(\theta)-\tau_{vw}}^{l_w(\theta + \varepsilon)-\tau_{vw}} f_{vw}^-(t+\tau_{vw}) dt
		= u_{vw} (l_w(\theta + \varepsilon) - l_w(\theta))$ mit~\ref{def-feasible-flow-queue-with-capacity}.
		Teilt man diese Gleichung durch $\varepsilon$, so erhält man für $\varepsilon\rightarrow 0$ die Bedingung $x_{vw}'(\theta) = u_{vw} l_w'(\theta)$.
		Ist $vw\in E_1$, so ist also Bedingung~\ref{def-thin-flow-resetting-edge} erfüllt.
		Für Bedingung~\ref{def-thin-flow-x-zero} setze man $x_{vw}'(\theta)=0$ voraus, wodurch $l_w'(\theta)=0$ folgt und mit der Monotonie von $l_v$ gilt $0 \leq l_v'(\theta)$.

		Ist $vw\notin E_1$, ist also die Warteschlange zum Zeitpunkt $l_v(\theta)$ leer, so gilt: $l_w(\theta+\varepsilon) - l_w(\theta) = l_v(\theta + \varepsilon) + q_{vw}(l_v(\theta + \varepsilon)) - l_v(\theta) \geq l_v(\theta + \varepsilon) - l_v(\theta)$.
		Teilt man wieder durch $\varepsilon$, so erhält man für $\varepsilon  \rightarrow 0$ Bedingung~\ref{def-thin-flow-x-positive} mit $l_w'(\theta) \leq l_v'(\theta)$ und dem Resultat des letzten Absatzes.
		
		\item[2. Fall:] $\exists \varepsilon > 0: (\theta, \theta + \varepsilon] \subseteq \Theta_{vw}^c$.
		
		Nach Lemma~\ref{lemma-nash-flow-waiting-queue-implies-active-edge} gilt bereits $vw\notin E_1$ und nach Lemma~\ref{lemma-x-locally-constant} ist $x_{vw}'(\theta) = 0$. 
		Es muss also nur Bedingung~\ref{def-thin-flow-x-zero} geprüft werden:
		Wegen $\theta\in\Theta_{vw}$ gilt $l_w(\theta + \varepsilon) - l_w(\theta) < l_v(\theta + \varepsilon) - l_v(\theta)$.
		Teilt man diese Ungleichung  durch $\varepsilon$, so erhält man für $\varepsilon\rightarrow 0$ die Bedingung $l_w'(\theta)\leq l_v'(\theta)$.
		
		\item[3. Fall:] $\forall \varepsilon>0: \exists \theta_{\varepsilon}\in (\theta, \theta+\varepsilon]: T_{vw}(l_v(\theta_\varepsilon)) = l_w(\theta_\varepsilon)$.
		
		Dies ist die exakte Umkehrung der Bedingung von Fall 2.
		Zusätzlich betrachte man diesen Fall nur, falls Fall 1 nicht eintritt.
		Das heißt, für alle $\theta_\varepsilon$ existiert ein $\theta'\in(\theta, \theta_\varepsilon]$ mit $q_{vw}(l_v(\theta')) = 0$; insbesondere ist $vw$ nicht in $E_1$ enthalten.
		Man wähle $\theta'_\varepsilon:=\max\{ \theta'\in (\theta, \theta_\varepsilon] \mid q_{vw}(l_v(\theta')) = 0 \}$ als das Maximum solcher Zeitpunkte, welches aufgrund der Stetigkeit von $q_{vw}\circ l_u$ existiert.
		Nach Konstruktion ist $q_{vw}\circ l_v$ im Intervall $(\theta_\varepsilon', \theta_\varepsilon)$ positiv und nach Lemma~\ref{lemma-nash-flow-waiting-queue-implies-active-edge} ist die Kante $vw$ in diesem Intervall aktiv.
		Nun impliziert $\theta_\varepsilon'\in \Theta_{vw}$ gerade $\theta_\varepsilon\in\Theta_{vw}$, da $\Theta_{vw}$ ab\-ge\-schlossen ist.
		Daher folgt aus $l_w(\theta_\varepsilon') - l_w(\theta) = l_v(\theta_\varepsilon') - l_v(\theta)$ 		Bedingung~\ref{def-thin-flow-x-zero}, indem man durch $\theta_\varepsilon'-\theta$ teilt und $l_w'(\theta) = l_v'(\theta)$ für $\varepsilon\rightarrow0$ erhält.
		
		Für Bedingung~\ref{def-thin-flow-x-positive} bleibt zu zeigen, dass $x_{vw}'(\theta) /u_{vw}\leq l_w'(\theta)$ gilt.
		Wegen Bedingung~\ref{def-feasible-flow-capacity} ist $x_{vw}(\theta + \varepsilon)-x_{vw}(\theta) = \int_{l_w(\theta)}^{l_w(\theta+\varepsilon)} f_{vw}^-(t) dt\leq (l_w(\theta + \varepsilon) - l_w(\theta)) u_{vw}$ für beliebiges $\varepsilon>0$.
		Durch Teilen mit $\varepsilon u_{vw}$ erhält man das gewünschte Resultat für $\varepsilon\rightarrow 0$.
	\end{description}
\vspace{-1.2 \baselineskip}
\end{proof}

\begin{definition}
	Ein \emph{dynamischer Fluss $f$ mit Zeithorizont $T\geq0$} ist ein Fluss, für dessen Zufluss $d(\theta)= 0$ für $\theta\geq T$ gilt.
\end{definition}

Durch folgende Proposition erkennt man, dass in einem Nash-Fluss mit Zeithorizont $T$ der Zu- bzw. Abfluss ab der frühestmöglichen Ankunftszeit am Start- bzw. Endknoten jeder Kante bei Startzeit $T$ fast überall verschwinden.

\begin{proposition}
Für einen dynamischen Nash-Fluss $f$ mit Zeithorizont $T$ und eine Kante $e\in E$ gilt $x_{e}(\theta) = x_{e}(T)$ für alle $\theta \geq T$.
Insbesondere verschwinden $f_{vw}^+$ ab dem Zeitpunkt $l_v(T)$ und $f_{vw}^-$ ab dem Zeitpunkt $l_w(T)$ fast überall für alle Kanten $vw\in E$.
\end{proposition}
\begin{proof}
	Der statische $s$-$t$-Fluss $x(\theta) -x(T)$ ist nach Proposition~\ref{prop-nash-flow-s-t-path-decomposable} in $s$-$t$-Wege zerlegbar.
	Zudem hat der Fluss $x(T)$ den gleichen Flusswert wie $x(\theta)$, da
	\[ \sum_{e\in \delta^+(s)} x_e(T) - \sum_{e\in\delta^-(s)} x_e(T) = \sum_{e\in \delta^+(s)} x_e(\theta) - \sum_{e\in\delta^-(s)} x_e(\theta)\]
	wegen $d(\xi) = 0$ für $\xi \geq T $ gilt.
	Daher ist $x(\theta)- x(T)$ bereits der Nullfluss, sodass $x(T)$ und $x(\theta)$ bereits auf allen Kanten übereinstimmen, wodurch die Behauptung folgt.
\end{proof}

\begin{definition}[$\alpha$-Erweiterung]
	Seien ein dynamischer Nash-Fluss $f$ mit Zeithorizont $T$ und ein $\tilde{d}$-wertiger schmaler Fluss $x'$ mit Zurücksetzen auf $E_1 := \{ vw\in E \mid q_{vw}(l_v(T)) > 0 \} $ und Knotenbewertung $l'$ im Graphen $G_T$ bei Versorgungsrate $\tilde{d}$ und ein $\alpha > 0$ gegeben.
	
	Ergänzt man die Werte aus $f$, sodass für alle zur Zeit $T$ aktiven Kanten $vw\in E_T$
	\[ \tilde{f}_{vw}^+(\theta):= \frac{x_{vw}'}{l_v'} \text{ für $\theta\in [l_v(T), l_v(T)+\alpha l_v')$ und } \tilde{f}_{vw}^-(\theta):=\frac{x_{vw}'}{l_w'} \text{ für $\theta\in [l_w(T), l_w(T)+\alpha l_w')$} \]
	gelten, so erhält man eine \emph{$\tilde{d}$-wertige $\alpha$-Erweiterung $\tilde{f}$ von $f$}.
	Dabei entspricht der Netzwerkzufluss von $\tilde{f}$ im Intervall $[T, T+\alpha)$ gerade $\tilde{d}$.
\end{definition}

Im nächsten Theorem wird schließlich gezeigt, dass eine solche $\alpha$-Erweiterung wieder einen Nash-Fluss erzeugt.

\begin{notation}
	Im folgenden Theorem und Beweis werden alle zur $\alpha$-Erweiterung $\tilde{f}$ gehörigen Größen wie der kumulative Zufluss $\tilde{F}_e^+$, die Wartezeit $\tilde{q}_e$ etc. mit einer Tilde notiert.
\end{notation}

\begin{theorem}\label{thm-alpha-extension-is-nash-flow}
	Für jede $\alpha$-Erweiterung $\tilde{f}$ eines dynamischen Nash-Flusses $f$ mit Zeithorizont $T$, die für alle Kanten mit positiver Warteschlange zum Zeitpunkt $T$ die Bedingung
	\[ l_w(T) - l_v(T) + \alpha(l_w' - l_v') \geq \tau_{vw} \]
	erfüllt, gelten die folgenden Aussagen:
	\begin{enumerate}[label=(\roman*)]
		\item Für positive $x_{vw}'$ und $\theta\in[T, T+\alpha]$ gilt $l_w(T) + (\theta - T)l_w' \geq l_v(T) + (\theta - T)l_v' + \tau_{vw}$.
		Für $\theta \leq l_v(T)$ gilt $\tilde{f}^-_{vw}(\theta + \tau_{vw})=f^-_{vw}(\theta + \tau_{vw})$ für alle $vw\in E$.
		Insbesondere ist $\tilde{q}_e(\theta) = q_e(\theta)$ und $\tilde{T}_e(\theta) = T_e(\theta)$ für $\theta \leq l_v(T)$.
		\item $\tilde{f}$ ist zulässiger dynamischer Fluss mit Zeithorizont $T+\alpha$ und für $\gamma \in [0, \alpha]$ gilt
		\[ \tilde{F}_{vw}^+(l_v(T) + \gamma l_v') = \tilde{F}_{vw}^-(l_w(T) + \gamma l_w'). \]
		\item Gilt zusätzlich für alle zum Zeitpunkt $T$ inaktiven Kanten die Bedingung \[
		l_w(T) - l_v(T) + \alpha(l_w' -l_v') \leq \tau_{vw},
		\]
		so sind die $\tilde{f}$ zugeordneten frühesten Ankunftszeiten $\tilde{l}_v(\theta)$ für $\theta \leq T+\alpha$ durch
		\[ \tilde{l}_v(\theta) = \begin{cases}
		l_v(\theta), & \text{ falls $\theta < T$,} \\
		l_v(T) + (\theta - T) l_v', & \text{ falls $\theta \in [T, T+\alpha]$.}
		\end{cases}\]
		gegeben und der Fluss $\tilde{f}$ ist ein dynamischer Nash-Fluss.
	\end{enumerate}
\end{theorem}
\begin{proof}
	Zu $(i)$: Ist $vw\in E_1$ mit $l_w'<l_v'$, so gilt
	$l_w(T)-l_v(T) + (\theta - T)(l_w' - l_v') \geq l_w(T)-l_v(T)+\alpha(l_w'- l_v')\geq \tau_{vw}$  mit der Voraussetzung an $\alpha$.
	Sonst gilt $l_w' \geq l_v'$ nach Bedingung~\ref{def-thin-flow-x-positive} und mit $T\in \Theta_{vw}$ folgt $l_w(T)+(\theta-T)l_w'=l_v(T) + q_{vw}(l_v(T))+\tau_{vw}+(\theta - T)l_w' \geq l_v(T) + (\theta-T)l_v'+\tau_{vw}$.
	Daraus folgt mit $\theta + \tau_{vw} \leq l_v(T) + \tau_{vw} \leq l_w(T)$ auch sofort $\tilde{f}_{vw}^-(\theta + \tau_{vw}) = f_{vw}^-(\theta + \tau_{vw})$ für $\theta\leq l_v(T)$.
	Insbesondere gilt $\tilde{q}_{vw}(\theta) = q_{vw}(\theta)$ und sogar $ \tilde{T}_{vw}(\theta)= T_{vw}(\theta)$ für $\theta \leq l_v(T)$.
	
	Um zu zeigen, dass $\tilde{f}$ zulässig ist, zeige man die Eigenschaften \ref{def-feasible-flow-capacity}-\ref{def-feasible-flow-queue-with-capacity}.
	Die Bedingung~\ref{def-feasible-flow-no-flow-at-node} gilt, da $x'$ ein statischer $s$-$t$-Fluss ist und Flusserhaltung in $V\setminus \{ s, t \}$ erfüllt.
	Für die Bedingungen~\ref{def-feasible-flow-capacity}, \ref{def-feasible-flow-no-negative-flow} und \ref{def-feasible-flow-queue-with-capacity} genügt es, Kanten $e\in E_T$ mit $x_e' > 0$ zu prüfen, da sonst $\tilde{f}_e$ mit $f_e$ übereinstimmt und $f$ bereits zulässig ist.
	Man nehme also $x_{vw}' > 0$ und $vw\in E_T$ an.
	Es gilt die Kapazitätsbeschränkung \ref{def-feasible-flow-capacity}, da $l_w'\geq x_{vw}' / u_{vw}$ wegen~\ref{def-thin-flow-x-positive} und~\ref{def-thin-flow-resetting-edge} gilt, wodurch $\tilde{f}^-_{vw}(\theta)=x_{vw}'/l_w'\leq u_{vw}$ für $\theta\in[l_w(T), l_w(T)+\alpha l_w')$ folgt.
	
	Für \ref{def-feasible-flow-no-negative-flow} zeige man $\tilde{F}^+_{vw}(\theta)\geq \tilde{F}_{vw}^-(\theta+\tau_{vw})$ für alle $\theta\in\R$.
	Für $\theta\leq l_v(T)$ gilt dies bereits nach (i).
	Existiert ein $\gamma\in[0, \alpha]$ mit $\theta=l_v(T) + \gamma l_v'$, so gilt
	\[
	\tilde{F}_{vw}^+(l_v(T) + \gamma l_v')=F_{vw}^+(l_v(T))+\gamma x_{vw}' = F_{vw}^-(l_w(T))+ \gamma x_{vw}'= \tilde{F}_{vw}^-(l_w(T)+\gamma l_w').
	\]
	Daraus folgt die Aussage, da $\theta + \tau_{vw}\leq l_w(T) + \gamma l_w'$ nach (i) erfüllt ist.
	Für $\theta > l_v(T)+\alpha l_v'$ gilt $\tilde{F}_{vw}^+(\theta) = \tilde{F}_{vw}^+(l_v(T) + \alpha l_v') = \tilde{F}_{vw}^-(l_w(T) + \alpha l_w') \geq \tilde{F}_{vw}^-(\theta + \tau_{vw})$.
	
	Es bleibt Bedingung \ref{def-feasible-flow-queue-with-capacity} zu prüfen, d.h. Warteschlangen sollen mit der Kantenkapazität abgebaut werden.
	Sei also $\tilde{q}_{vw}(\theta)$ positiv.
	Nach (i) ist $\tilde{f}_{vw}^-(\theta + \tau_{vw}) = f_{vw}^-(\theta + \tau_{vw}) = u_{vw}$ für $\theta\leq l_v(T)$.
	Ist $\theta > l_v(T)$, so unterscheide man, ob die Warteschlange von $vw$ zur Zeit $l_v(T)$ positiv ist:
	Ist dies der Fall, so gilt $l_w' = x_{vw}' / u_{vw}$ nach~\ref{def-thin-flow-resetting-edge}, und ohne Einschränkung gelte $\theta + \tau_{vw} \geq l_w(T)$, denn $f_{vw}^-$ ist nach Proposition~\ref{prop-feasible-flow}~\ref{prop-feasible-flow-positive-queue} und Eigenschaft~\ref{def-feasible-flow-queue-with-capacity} konstant $u_{vw}$ auf $[l_v(T)+\tau_{vw},l_w(T))$.
	Existiert ein $\gamma\in [0, \alpha]$ mit $\theta = l_v(T) + \gamma l_v'$, so gilt $\theta + \tau_{vw} = l_v(T) + \gamma l_v' + \tau_{vw} \leq l_w(T) + \gamma l_w'$ nach (i).
	Damit ist $\tilde{f}^-_{vw}(\theta + \tau_{vw}) = x_{vw}'/l_w' = u_{vw}$.
	Ist  hingegen $\theta > l_v(T) + \alpha l_v'$, so gilt $0<\tilde{F}^+_{vw}(\theta) - \tilde{F}_{vw}^-(\theta +\tau_{vw}) = \alpha x_{vw}' - \min \{ \alpha l_w', \theta + \tau_{vw} - l_w(T) \} x_{vw}' / l_w'$. Insbesondere ist also $\theta + \tau_{vw} < l_w(T) + \alpha l_w'$ und auch hier gilt $\tilde{f}^-_{vw}(\theta + \tau_{vw}) = x_{vw}'/l_w' = u_{vw}$.	
	Nun betrachte man Kanten $vw$, die zum Zeitpunkt $l_v(T)$ keine Warteschlange haben.
	Nach~\ref{def-thin-flow-x-positive} gilt hier $l_w' = \max \{ l_v', x_{vw}' / u_{vw}  \}$.
	Existiert ein $\gamma\in [0, \alpha]$ mit $\theta = l_v(T) + \gamma l_v'$, so ist $\theta+\tau_{vw}$ in $[l_w(T), l_w(T) + \alpha l_w']$ enthalten und es gilt $0<\tilde{F}_{vw}^+(\theta) - \tilde{F}_{vw}^-(\theta + \tau_{vw})= \gamma x_{vw}' - \gamma l_v' x_{vw}'/l_w'$.
	Daher ist $l_w' > l_v'$ und es müssen $l_w' = x_{vw}'/u_{vw}$ und $\tilde{f}_{vw}^-(\theta + \tau_{vw}) = u_{vw}$ gelten.
	Für $\theta > l_v(T) + \alpha l_v'$ gilt $\theta + \tau_{vw} > l_w(T) + \alpha l_v'$ und $0 < \tilde{F}_{vw}^+(\theta) - \tilde{F}_{vw}^-(\theta + \tau_{vw}) = \alpha x_{vw}' - \min \{ \alpha l_w', \theta - l_v(T) \}x_{vw}'/l_w'$.
	Insbesondere ist $\theta - l_v(T) < \alpha l_w'$, was äquivalent zu $\theta + \tau_{vw} < l_w(T) + \alpha l_w'$ ist.
	Nun kann man $l_v' < l_w'$ folgern, was $l_w' = x_{vw}'/u_{vw}$ impliziert.
	Damit gilt $\tilde{f}^-(\theta + \tau_{vw}) = u_{vw}$.

	Um Aussage (iii) zu zeigen, bemerke man, dass $(l_v(\theta))_{v\in V}$ das Gleichungssystem in Proposition~\ref{prop-arrival-times-vector} für $\theta \leq T$ erfüllt, da $\tilde{T}_{vw}(\theta) = T_{vw}(\theta)$ für $\theta\leq T$ nach (i) gilt.
	Für $\theta \in (T, T+\alpha)$ löst $(l_v(T) + (\theta - T)l_v')_{v\in V}$  das System:
	Wegen $l_s' = 1$ gilt $l_s(T) + (\theta - T)l_s' = \theta$.
	Für $w\neq s$ ist $l_w(T) + (\theta - T) l_w'$ eine untere Schranke für zur Zeit $T$ inaktive Kanten $vw$:
	Zunächst zeige man $l_w(T) + (\theta - T)l_w' \leq l_v(T) + (\theta - T)l_v' + \tau_{vw}$:
	Ist $l_w' \leq l_v'$, so gilt dies bereits, da $vw$ zur Zeit $l_v(T)$ keine Warteschlange hat.
	Für $l_w' \geq l_v'$ gilt $l_w(T) - l_v(T) + (\theta - T) (l_w' - l_v') \leq l_w(T) - l_v(T) + \alpha (l_w' - l_v') \leq \tau_{vw}$ mit der zusätzlichen Voraussetzung an~$\alpha$.
	Demnach ist also $l_w(T) + (\theta - T) l_w' \leq l_v(T) + (\theta - T) l_v' + \tau_{vw} \leq \tilde{T}_{vw}(l_v(T) + (\theta - T) l_v')$.
	
	Im nächsten Schritt zeige man $\tilde{T}_{vw}(l_v(T) + (\theta - T)l_v') = l_w(T) + (\theta - T) l_w'$ für Kanten $vw\in E_T$, für die $x_{vw}'$ positiv ist oder deren Warteschlange zur Zeit $l_v(T)$ positiv ist:
	Für $x_{vw}' > 0$ impliziert~(ii) zusammen mit Proposition~\ref{prop-feasible-flow}~\ref{prop-feasible-flow-det-outflow} bereits $\tilde{T}_{vw}(l_v(T) + (\theta-T)l_v') = l_w(T) + (\theta - T) l_w'$.
	Für $vw\in E_1$ mit $x_{vw}'=0$ gilt $l_w' = x_{vw}' / u_{vw} = 0$ nach~\ref{def-thin-flow-resetting-edge}.
	Entsprechend erfüllt die Warteschlange
	\begin{align*}
	\tilde{z}_{vw}(l_v(T) + (\theta - T)l_v') - z_{vw}(l_v(T)) &= (\theta - T) x_{vw}' - \int_{l_v(T) + \tau_{vw}}^{l_v(T) + (\theta - T) l_v' + \tau_{vw}} f_e^-(t) \diff t \\
	&= (\theta - T) (x_{vw}' -l_v' u_{vw}),
	\end{align*}
	da zusätzlich $(\theta-T) l_v' \leq \alpha (l_v' - l_w') \leq q_{vw}(l_v(T))$ gilt.
	Daher ist $\tilde{T}_{vw}(l_v(T) + (\theta - T)l_v') = l_w(T) + (\theta - T) l_v'  + \tilde{q}_{vw}(l_v(T) + (\theta - T)l_v')- q_{vw}(l_v(T)) = l_w(T) + (\theta - T) l_w'$.
	
	Zuletzt betrachte man aktive Kanten $vw\notin E_1$ mit $x_{vw}' = 0$.
	Hier gilt $l_w(T) = l_v(T) + \tau_{vw}$ und $l_w' \leq l_v'$ nach~\ref{def-thin-flow-x-zero}.
	Außerdem ist $z_{vw}(l_v(T) + (\theta - T) l_v') = F_{vw}^+(l_v(T)) - F_{vw}^-(l_w(T)) = 0$, wodurch man $\tilde{T}_{vw}(l_v(T) + (\theta - T) l_v') = l_w(T) + (\theta - T)l_v' \geq l_w(T) + (\theta - T) l_w'$ folgern kann.
	Dabei gilt sogar Gleichheit nach~\ref{def-thin-flow-no-resetting-edge}, falls $w$ keine eingehende Kante mit positiver Warteschlange hat.
	Daher ist $l_w(T)+(\theta - T) l_w'$ nicht nur eine untere Schranke, sondern tatsächlich das Minimum.
	
	Um nun zu erkennen, dass $\tilde{f}$ ein Nash-Fluss ist, zeige man Bedingung~(ii) aus Theorem~\ref{thm-equivalencies-nash-flow}, d.h. $\forall e\in E, \theta\in\R: \tilde{x}_e^+(\theta) = \tilde{x}_e^-(\theta)$.
	Für $\theta \leq T$ gilt $\tilde{x}_e^+(\theta) = x_e^+(\theta) = x_e^-(\theta) = \tilde{x}_e^-(\theta)$ für alle Kanten $e\in E$.
	Des Weiteren gilt $\tilde{x}_e^+(\theta) = \tilde{x}_e^+(T) = \tilde{x}_e^-(T) = \tilde{x}_e^-(\theta)$ für Kanten mit $x_{e}' = 0$ und $\theta > T$.
	Für $\theta$ zwischen $T$ und $T+\alpha$ liefert~(ii) die Behauptung.
	Für $\theta > T+\alpha$ ist schließlich $\tilde{x}_e^+(\theta) = \tilde{x}_e^+(T + \alpha) = \tilde{x}_e^-(T+\alpha) = \tilde{x}_e^-(\theta)$.
\end{proof}

Der Abschnitt wird mit der zu Bemerkung~\ref{remark-thin-flow} gehörigen Abbildung abgeschlossen.
\begin{figure}[h!]
	\centering
	\begin{tikzpicture}
	\node[draw, circle] (S) at (0,2) {$s$};
	\node[draw, circle] (V) at (4,2) {$v$};
	\node[draw, circle] (T) at (2,0) {$t$};
	
	\path [->] (S) edge node[above] {$1/1$} (V);
	\path [->] (S) edge node[left] {$1/1$} (T);
	\path [->] (V) edge node[right] {$10/1$} (T);
	\end{tikzpicture}
	\caption{In der Abbildung ist ein Netzwerk mit Kantenbeschriftung $\tau_e / u_e$ zu erkennen.
		Man betrachte den dynamischen Nullfluss mit Zeithorizont $0$.
		Die Kanten $st$ und $sv$ sind aktiv, wohingegen die Kante $vt$ inaktiv ist.
		Dann ist der Fluss $x'$ mit $x_{sv}'=0$ und $x_{st}'=2$ nach \cite[Definition 6]{Koch2011} ein schmaler Fluss mit Zurücksetzen auf $\emptyset$ und Knotenbewertungen $l_s' = 1$, $l_v' = 0$ und $l_t' = 2$, aber für eine $\alpha$-Erweiterung ist $l_v(\theta) = \theta + 1 > 1 = l_v(0) + l_v' \theta$ für $\theta > 0$.} 
	\label{figure-labels}
\end{figure}
\section{Fazit und Ausblick}

Die Autoren Koch und Skutella haben in ihrer Arbeit eine Einführung in dynamische Nash-Flüsse gegeben.
Cominetti, Correa und Larré zeigen in~\cite{Cominetti2015} die Existenz von Nash-Flüssen unter Verwendung schmaler Flüsse mit Zurücksetzen.
Außerdem analysieren Koch und Skutella den sog. Preis der Anarchie von Nash-Flüssen im Bezug auf die Zielsetzung, den Zufluss der Senke zu jeder Zeit zu maximieren, der in statischen Routenplanungsspielen konstant ist.
Des Weiteren zeigen sie, dass im dynamischen Fall Instanzen von Nash-Flüssen existieren, deren Preis der Anarchie linear mit der Anzahl der Kanten des Netzwerks steigt.

\clearpage          % neue Seite für Literaturverzeichnis

%%%%%%%%%%%%%%%%%%%%%%%%%%%%%%%%%%%%%%%%%%%%%%%%%%%%%%%%%%%%%%%%%%%%%%%%%%%%%%%%
% Literaturverzeichnis
\nocite*  % Nicht zitierte Quellen werden auch ins Literaturverzeichnis aufgenommen
\thispagestyle{empty}
\bibliography{literature}  % Literaturverzeichnis liegt in der Datei literature

%%%%%%%%%%%%%%%%%%%%%%%%%%%%%%%%%%%%%%%%%%%%%%%%%%%%%%%%%%%%%%%%%%%%%%%%%%%%%%%%
%%%%%%%%%%%%%%%%%%%%%%%%%%%%%%%%%%%%%%%%%%%%%%%%%%%%%%%%%%%%%%%%%%%%%%%%%%%%%%%%
% Ende des Dokuments
\end{document}		
