\section{Einführung}\label{introduction}

\newcommand{\R}{\mathbb{R}}

\begin{definition}[strategisches Spiel]
	Ein \emph{strategisches Spiel} $\mathcal{G}$ ist ein Tupel $(P, X, \theta)$, wobei $P$ eine Menge von Spielern, $X$ die Menge der Strategien  und $\theta=(\theta_p)_{p_\in P}$ die Familie der Auszahlungsfunktionen der Spieler ist, wobei $\theta_p$ eine Funktion $X^P\to\R$ für jeden Spieler $p\in P$ ist.
\end{definition}

Man beachte, dass in dieser Definition eines strategischen Spiels die Spieler eine gemeinsame Strategiemenge haben.

\begin{definition}[statisches Flussnetzwerk]
	Ein \emph{statisches Flussnetzwerk} $\mathcal{N}:=(G,s,t,\mu)$ ist ein Graph $G:=(V,E)$ mit Knoten $V$ und Kanten $E$, einer \emph{Quelle} $s\in V$, einer \emph{Senke} $t\in V$ zusammen mit einem \emph{$d$-wertigen statischen Fluss} $\mu: \mathcal{P} \to \R$, wobei $\mathcal{P}$ die Menge aller einfachen $s$-$t$-Pfade in G ist und $d=\sum_{p\in\mathcal{P}}\mu(p)$ gilt.
\end{definition}

\begin{definition}[statisches Routenplanungsspiel]
	Sei ein Graph $G:=(V,E)$ mit Knoten $V$ und Kanten $E$, einer Quelle $s\in V$, einer Senke $t\in V$ und einem Zufluss $d\in\R_+$ gegeben.
	Ein \emph{statisches Routenplanungsspiel} ist ein strategisches Spiel mit  Spielermenge $[0,d]\subseteq\R$, wobei ein Spieler als \emph{Flusspartikel} bezeichnet wird, und mit Strategiemenge $\mathcal{P}$ aller $s$-$t$-Pfade in $G$.
	Die Auszahlungsfunktion $\theta_x$ gibt die Kosten des Partikels $x$ in Abhängigkeit des Flusses an.
	
	Der Fluss $\mu$ heißt Nashfluss, falls $l_P(\mu)=min_{P'\in\mathcal{P}}l_{P'}(\mu)$ für alle $P\in\mathcal{P}$ mit $\mu_P > 0$ gilt.
\end{definition}

Static $\to$ dynamic:
Betrachte Zeitraum $[0,T]$ mit $T\in\R_+$.
$d$ Spieler erscheinen an Quelle $s$ über Zeitraum von $0$ bis $T$

\begin{definition}
	
\end{definition}
