\section{Durchlaufzeiten}

\begin{definition}
	Für einen Fluss $f$ und einen Pfad $P=(e_1,\dots,e_k)$ definiere $l^P(\theta):=T_{e_k}\circ\dots\circ T_{e_1}(\theta)$ den Zeitpunkt, an dem ein Partikel den Endknoten des Pfads erreicht, falls er den Pfad zum Zeitpunkt $\theta$ betritt.
	
	Für einen Knoten $w\in V$ beschreibe $\mathcal{P}_w$ die Menge aller $s$-$w$-Pfade.
	Dann ist die früheste Ankunft eines Partikels, das zum Zeitpunkt $\theta$ bei $s$ startet, gegeben durch $l_w(\theta):=\min_{P\in\mathcal{P}_w}l^P(\theta)$.
	Ein Pfad $P\in \mathcal{P}_w$ heißt \emph{kürzester $s$-$w$-Pfad zum Zeitpunkt $\theta$},  falls $l_w(\theta)=l^P(\theta)$.
\end{definition}

\begin{proposition}
	Die Funktionen $T_e$, $l^P$ und $l_v$ sind für alle Kanten $e\in E$, Pfade $P$ in G und Knoten $v\in V$ monoton wachsend, lokal absolut stetig und surjektiv.
\end{proposition}
\begin{proof}
	\todo{TODO. Auch: Warum brauch ich nochmal absolute Stetigkeit?}
\end{proof}

\todo{$l_v$ wohldefiniert, d.h. minimum existiert, da Pfade mit Kreisen nie das Minimum sind $\implies$ endlich viele}

Für einen Knoten $v \in V$ und einen zulässigen Fluss ist die Funktion $l_v$ als Minimum von Kompositionen stetiger und monoton wachsender Funktionen $T_e$ (Proposition~\ref{prop-feasible-flow}~\ref{prop-feasible-flow-T-mon-inc-cont}) ebenfalls stetig und monoton wachsend.


\begin{lemma}\label{lemma-dreicksungl}
	Für alle Kanten $vw\in E$ gilt in einem zulässigen Fluss 
	$T_{vw}(l_v(\theta)) \geq l_w(\theta)$.
\end{lemma}
\begin{proof}
	Sei ein kürzester $s$-$v$-Pfad $P$ zum Zeitpunkt $\theta$ gegeben.
	Hängt man an $P$ die Kante $vw$ an, erhält man einen $s$-$w$-Pfad, der zur Eintrittszeit $\theta$ die Ankunftszeit $T_{vw}(l_v(\theta))$ liefert.
	Da $l_w(\theta)$ das Minimum über die Ankunftszeit aller $s$-$w$-Pfade ist, gilt die Behauptung.
\end{proof}

\begin{definition}
	Man bezeichne eine Kante $vw\in E$ als \emph{aktiv zum Zeitpunkt $\theta$}, falls $T_{vw}(l_v(\theta)) = l_w(\theta)$ gilt.
	Die Menge $\Theta_{vw}$ bezeichne alle Zeitpunkte, zu denen die Kante $vw$ aktiv ist.
\end{definition}
\begin{notation}
	Das Komplement einer Menge $M$ notiere man als $M^c:= \R\setminus M$.
\end{notation}

\todo{Bla: d.h. falls die Kante in einem kürzesten $s$-$w$-Pfad liegt}.

\begin{lemma}
	Für alle Knoten $v\in V$ ist in einem zulässigen Fluss die Funktion $l_v$ monoton wachsend und stetig.
\end{lemma}
\todo{
	Mittels Belman-Ford-Algorithmus kann man $l_w$ auch berechnen, indem man die Lösung des folgenden Gleichungssystem löst:
	
	$$ l_w(\theta) = \begin{cases}
	\theta & \text{falls } w=s \\
	\min_{vw\in E} T_{vw}(l_v(\theta)) & \text{sonst}
	\end{cases} $$
	
	Warum kann man das? Zykeln?
}