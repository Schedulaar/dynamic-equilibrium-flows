\section{Durchlaufzeiten}

\begin{definition}
	Für einen Fluss $f$ und einen Pfad $P=(e_1,\dots,e_k)$ definiere $l^P(\theta):=T_{e_k}\circ\dots\circ T_{e_1}(\theta)$ den Zeitpunkt, an dem ein Partikel den Endknoten des Pfads erreicht, falls er den Pfad zum Zeitpunkt $\theta$ betritt.
	
	Für einen Knoten $w\in V$ beschreibe $\mathcal{P}_w$ die Menge aller $s$-$w$-Pfade.
	Dann ist die früheste Ankunft eines Partikels, das zum Zeitpunkt $\theta$ bei $s$ startet, gegeben durch $l_w(\theta):=\min_{P\in\mathcal{P}_w}l^P(\theta)$.
	Ein Pfad $P\in \mathcal{P}_w$ heißt \emph{kürzester $s$-$w$-Pfad zum Zeitpunkt $\theta$},  falls $l_w(\theta)=l^P(\theta)$.
\end{definition}

\begin{proposition}
	Die Funktionen $T_e$, $l^P$ und $l_v$ sind für alle Kanten $e\in E$, Pfade $P$ in G und Knoten $v\in V$ monoton wachsend, lokal absolut stetig und surjektiv.
\end{proposition}
\begin{proof}
	\todo{TODO. Auch: Warum brauch ich nochmal absolute Stetigkeit?}
\end{proof}

\todo{$l_v$ wohldefiniert, d.h. minimum existiert, da Pfade mit Kreisen nie das Minimum sind $\implies$ endlich viele}

Für einen Knoten $v \in V$ und einen zulässigen Fluss ist die Funktion $l_v$ als Minimum von Kompositionen stetiger und monoton wachsender Funktionen $T_e$ (Proposition~\ref{prop-feasible-flow}~\ref{prop-feasible-flow-T-mon-inc-cont}) ebenfalls stetig und monoton wachsend.


\begin{lemma}\label{lemma-dreicksungl}
	Für alle Kanten $vw\in E$ gilt in einem zulässigen Fluss 
	$T_{vw}(l_v(\theta)) \geq l_w(\theta)$.
\end{lemma}
\begin{proof}
	Sei ein kürzester $s$-$v$-Pfad $P$ zum Zeitpunkt $\theta$ gegeben.
	Hängt man an $P$ die Kante $vw$ an, erhält man einen $s$-$w$-Pfad, der zur Eintrittszeit $\theta$ die Ankunftszeit $T_{vw}(l_v(\theta))$ liefert.
	Da $l_w(\theta)$ das Minimum über die Ankunftszeit aller $s$-$w$-Pfade ist, gilt die Behauptung.
\end{proof}

\begin{definition}
	Man bezeichne eine Kante $vw\in E$ als \emph{aktiv zum Zeitpunkt $\theta$}, falls $T_{vw}(l_v(\theta)) = l_w(\theta)$ gilt.
	Die Menge $\Theta_{vw}$ bezeichne alle Zeitpunkte, zu denen die Kante $vw$ aktiv ist.
\end{definition}

\begin{lemma}\label{lemma-shortest-path-using-active-edges}
	Zu jeder Zeit $\theta$ ist der durch die aktiven Kanten induzierte Teilgraph $G_\theta:=(V, E_\theta)$ azyklisch jeder Knoten ist von $s$ erreichbar.
\end{lemma}
\begin{proof}
	Angenommen es existiere ein Zyklus $C=(v_1, \dots, v_n)$ mit $v_1=v_n$ und ausschließlich aktiven Kanten.
	Es ist $l^C(\theta) > \theta$, da für Zykeln eine positive Gesamtverzögerung vorausgesetzt ist.
	Augrund der Aktivität von $C$ erzeugt $l_{v_1}(\theta) = T^C(l_{v_1}(\theta)) > l_{v_1}(\theta)$ einen Widerspruch.
	
	Für jeden Knoten $w\neq s$ existiert zumindest eine eingehende aktive Kante -- zum Beispiel die letzte Kante eines kürzesten $s$-$w$-Pfades, welcher existiert, da $w$ von $s$ aus erreichbar ist.
	Folgt man diesen aktiven Rückwärtskanten, so erreicht man schließlich $s$, da keine Zykeln auftreten.
%	Man induziere über die maximale Kantenzahl $k$ kürzester $s$-$w$-Pfade -- diese existiert, da für alle Zykeln eine positive Gesamtverzögerung vorausgesetzt wird.
%	Für $k>0$ seien ein kürzester $s$-$w$-Pfad $P$ mit maximaler Kantenzahl $k$ und seine letzte Kante $vw$ gegeben.
%	Wegen der Monotonie von $T_{vw}$ gilt $T_{vw}(l_v(\theta)) \leq T_{vw}(l^{P-vw}(\theta)) = l_w(\theta)$, wobei $P-vw$ der Pfad $P$ ohne die letzte Kante $vw$ ist.
%	Also ist $vw$ zur Zeit $\theta$ aktiv.
%	Die Erweiterung eines kürzesten $s$-$v$-Pfades $Q$ um $vw$ ist ein kürzester $s$-$w$-Pfad, weshalb die maximale Kantenzahl eines kürzesten $s$-$v$-Pfades kleiner als $k$ ist.
%	Nach Induktionsannahme existiert ein kürzester $s$-$v$-Pfad, der nur aktive Kanten verwendet.
%	Hängt man an diesen Pfad $vw$ an, so erhält man einen kürzesten $s$-$w$-Pfad mit ausschließlich aktiven Kanten.
\end{proof}

\todo{Move notation to first occurrence}
\begin{notation}
	Das Komplement einer Menge $M$ notiere man als $M^c:= \R\setminus M$.
\end{notation}

\todo{Bla: d.h. falls die Kante in einem kürzesten $s$-$w$-Pfad liegt}.

\begin{lemma}
	Für alle Knoten $v\in V$ ist in einem zulässigen Fluss die Funktion $l_v$ monoton wachsend und stetig.
\end{lemma}
\todo{Wo ist der Beweis hin?}

\begin{proposition}
	Der Vektor $(l_v(\theta))_{v\in V}$ lässt sich ermitteln, indem man folgendes Gleichungssystem eindeutig löst:
	$$ \tilde{l}_w = \begin{cases}
	\theta & \text{falls } w=s \\
	\min\limits_{vw\in \delta^-(w)} T_{vw}(\tilde{l}_v) & \text{sonst}
	\end{cases}.$$
\end{proposition}
\begin{proof}
	Offenbar löst $l_v(\theta)$ dieses System nach Lemma~\ref{lemma-shortest-path-using-active-edges}.
	
	Sei $(\tilde{l}_v)_{v\in V}$ eine Lösung dieses Gleichungssystems.
	Man zeige für jeden Knoten $w\in V$ gilt $l_w(\theta) = \tilde{l}_w$.
	Dazu führe man eine Induktion über die maximale Kantenzahl $k$ kürzester $s$-$w$-Pfade durch.
	Solch ein Maximum existiert für jeden Knoten $w$, da für Zyklen eine positive Gesamtverzögerung vorausgesetzt ist.
	Für $k=0$ gilt $l_s(\theta)=\theta = \tilde{l}_s$ für die Quelle $s$.
	Sei für $w\neq s$ ein kürzester $s$-$w$-Pfad $P$ zum Zeitpunkt $\theta$ mit maximaler Kantenzahl $k$ gegeben und sei $vw$ die letzte Kante von $P$.
	Offenbar ist $vw$ aktiv zur Zeit $\theta$.
	Der Pfad $P-vw$, also der Pfad $P$ ohne die letzte Kante $vw$, ist ein kürzester $s$-$v$-Pfad mit maximaler Kantenzahl zur Zeit $\theta$:
	Für einen kürzesten $s$-$v$-Pfad $Q$ ist $l^Q(\theta)\leq l^{P-vw}(\theta)$ und mit der Monotonie von $T_{vw}$ gilt $l^{Q+vw}(\theta)\leq l^P(\theta)$.
	Wegen der Maximalität der Kantenzahl von $P$ ist also auch die Kantenzahl von $P-vw$ maximal.
	Damit gilt nach Induktionsvoraussetzung $l_v(\theta) = \tilde{l}_v$
	und damit gelten $l_w(\theta) = T_{vw}(\tilde{l}_v)$ und $\tilde{l}_w \leq l_w(\theta)$.
\end{proof}
\todo{Belmann-Ford? Warum kann man das? Zykeln?}