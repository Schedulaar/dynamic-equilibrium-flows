\section{A special class of static flows}

\begin{lemma}\label{lemma-no-inflow-until-l}
	Sei $f$ ein Nash-Fluss über Zeit.
	Dann gilt $F_{vw}^+(l_v(0)) = 0$ für jede Kante $vw\in E$.
\end{lemma}
\begin{proof}
	Für alle $\theta\in [0, l_v(0)]$ gilt nach Bedingung~\ref{def-feasible-flow-no-flow-at-node}, dass $f_{vw}^+(\theta) \leq \sum_{e\in\delta^-(v)} f_e^-(\theta)$.
	Also ist $F_{vw}^+(l_v(0)) \leq \sum_{uv\in\delta^-(v)} F_{uv}^-(l_v(0))\leq \sum_{uv\in\delta^-(v)} F_{uv}^+(l_u(0))$ nach Bemerkung~\ref{remark-x^-leqx^+}.
	\todo{TODO this doesn't work yet}
\end{proof}

\begin{lemma}\label{lemma-nash-flow-waiting-queue-implies-active-edge}
	Seien $f$ ein Nash-Fluss über Zeit, $uv\in E$ eine Kante und $\theta\in\R$ ein Zeitpunkt gegeben.
	Gilt eine der folgenden Aussagen, so ist $vw$ zum Zeitpunkt $\theta$ aktiv:
	\begin{enumerate}[label=(\roman*)]
		\item Es existiert ein $\varepsilon>0$ mit $x_{vw}(\theta')<x_{vw}(\theta)$ für alle $\theta'\in(\theta-\varepsilon, \theta)$.
		\item Die Ableitung $x_{vw}'(\theta)$ existiert und es gilt $x_{vw}'(\theta)> 0$.
		\item Die Wartezeit $q_{vw}(\theta)$ an der Kante $vw$ ist zur Zeit $\theta$ positiv.
	\end{enumerate}
\end{lemma}
\begin{proof}
	Zu Aussage (i): Für $\theta'$ gilt $F_{vw}^+(l_v(\theta')) > F_{vw}^+(l_v(\theta))$.
	Also existiert eine Teilmenge $Q_\varepsilon \subseteq (l_v(\theta-\varepsilon), l_v(\theta))=l_v((\theta - \varepsilon, \theta))$ mit positivem Maß und $f_{vw}^+\big|_{Q_\varepsilon} > 0$.
	Da $f$ ein Nash-Fluss ist, fließt $f$ nach Theorem~\ref{thm-equivalencies-nash-flow} nur entlang aktiver Kanten.
	Daher ist $\xi\in l_v(\Theta_{vw})$ für fast alle $\xi\in Q_\varepsilon$; insbesondere existiert ein $\theta_\varepsilon\in (\theta-\varepsilon, \theta)$ mit $\theta_\varepsilon\in\Theta_{vw}$.
	Für $\varepsilon\rightarrow0$ ist also auch $\theta\in\Theta_{vw}$ wegen der Abgeschlossenheit von $\Theta_{vw}$.
	
	Aussage (ii) folgt direkt aus (i) mittels Differenzenquotienten.
	\todo{don't argue with 0}Für Aussage (iii) ist wieder $T_{vw}(l_v(\theta)) \leq l_w(\theta)$ zu zeigen.
	Da $q_{vw}(l_v(\theta))$ positiv ist, gilt auch $F^+_{vw}(l_v(\theta)) > 0$.
	Nach Lemma~\ref{lemma-no-inflow-until-l} ist $F_{vw}(l_v(0))=0$ und wegen der Stetigkeit und Monotonie von $F_{vw}^+\circ l_v$ existiert
	ein frühester Zeitpunkt $\theta_1\in (0, \theta]$ mit $F_{vw}^+(l_v(\theta_1)) = F_{vw}^+(l_v(\theta))$.
	Nach Aussage (i) gilt also $T_{vw}(l_v(\theta_1)) = l_w(\theta_1)$.
	Nach Proposition~\ref{prop-feasible-flow}~\ref{prop-feasible-flow-queue-delay} ist $T_{vw}(l_v(\theta_1)) = T_{vw}(l_v(\theta))$ und mit der Monotonie von $l_w$ folgt $T_{vw}(l_v(\theta))\leq l_w(\theta)$.	
\end{proof}
\todo{Das ist glaub ich die Definition von schwachem Nash-flow}

\begin{definition}[Schmaler Fluss mit Zurücksetzen]\label{def-thin-flow}
	Seien ein statischer $F$-wertiger Fluss $f$ in einem Graph $G=(V,E)$ mit Kantenkapazitäten $u\in\R^E$, eine Quelle $s\in V$ mit Zuflusskapazität $d\in\R$ und eine Senke $t\in V$ sowie eine Teilmenge $E_1\subseteq E$ von Kanten gegeben.
	Der Fluss $f$ heißt \emph{schmaler Fluss mit Zurücksetzen auf $E_1$}, falls eine Knotenbewertung $l\in\R^V$ existiert mit:
	$$\begin{array}{ll}
	(1)~~~	l_s = F/d& \\
	(2)~~~	l_v \leq l_u &\text{für $uv\in E \setminus E_1$ mit $f_{uv}=0$}\\
	(3)~~~	l_v = \max(l_u, f_{uv} / u_{uv} ) &\text{für $uv\in E\setminus E_1$ mit $f_{uv} > 0$}\\
	(4)~~~	l_v = f_{uv} / u_{uv} & \text{für $uv\in E_1$}
	\end{array}$$
\end{definition}

\begin{theorem}
	Seien ein Nash-Fluss über Zeit $f$ auf einem Graphen $G=(V,E)$ gegeben sowie ein Zeitpunkt $\theta$ gegeben.
	Der zum Fluss $f$ und dem Zeitpunkt $\theta$ zugeordnete statische Fluss sei $x(\theta)$ mit $x_{uv}(\theta) := x^+_{uv}(\theta) = x^-_{uv}(\theta)$.\todo{(muss vlt noch gezeigt werden)}
	Man betrachte den Teilgraph $G_\theta = (V, E_\theta)$, der nur zum Zeitpunkt $\theta$ aktive Kanten enthält, d.h. $E_\theta := \{ uv\in E \mid T_{uv}(l_u(\theta)) = l_v(\theta) \}$.
	
	Existieren die Ableitungen $\frac{dx_e}{d\theta}(\theta)$ und $\frac{dl_v}{d\theta}(\theta)$ für alle $e\in E$ und $v\in V$, so ist der statische Fluss $g\in\R^{E_\theta}$ mit $g_e:=\frac{dx_e}{d\theta}(\theta)$ ein schmaler $d$-wertiger Fluss mit Zurücksetzen auf den Kanten mit Warteschlange $E_1:=\{uv\in E \mid q_{uv}(l_u(\theta))>0 \}$ im Graphen $G_\theta$.
	Als Knotenbewertung dienen dazu die Ableitungen $(\frac{dl_v}{d\theta}(\theta))_{v\in V}$.
\end{theorem}
\begin{proof}
	zu zeigen: $\frac{dl_s}{d\theta}(\theta) = F/d$.
	Dabei ist $\frac{dl_s}{d\theta}(\theta) = 1$, da $l_s(\theta) = \theta$. Bleibt zu zeigen $F= d$.
	Der Wert von $g$ ist $\frac{d}{d\theta}\sum_{e\in\delta^+(s)}x_e(\theta) = \frac{d}{d\theta} d\cdot\theta = d$. \todo{Geht auch mit variablen Zufluss}.
	
	Sei nun eine Kante $uv\in E_\theta$, also eine aktive Kante zum Zeitpunkt $\theta$, gegeben. Wir prüfen die restlichen drei Bedingungen jeweils in den folgenden drei Fällen:
	
	\begin{description}
		\item[1. Fall:] $\exists \varepsilon > 0:\forall \theta'\in (\theta, \theta + \varepsilon ] : q_{uv}(l_u(\theta')) > 0$.
		
		Für $\theta'\in(\theta,\theta+\varepsilon]$ ist die Wartezeit $q_{uv}(l_u(\theta'))$ positiv und nach Proposition~\ref{prop-feasible-flow}~\ref{prop-feasible-flow-positive-queue} ist $q_{uv}$ positiv auf dem Intervall $[ l_u(\theta')  , l_u(\theta')+q_{uv}(l_u(\theta')) )$.
		Also ist $q_{uv}$ positiv auf dem Intervall $( l_u(\theta) , l_u(\theta + \varepsilon) + q_{uv}(l_u(\theta + \varepsilon) )
		\subseteq ( l_u(\theta) + q_{uv}(l_u(\theta)) , l_u(\theta + \varepsilon) + q_{uv}(l_u(\theta) + \varepsilon ) )$.
		
		Nach Voraussetzung ist die Kante ${uv}$ nach Lemma~\ref{lemma-nash-flow-waiting-queue-implies-active-edge} auch für $\theta'\in  (\theta, \theta + \varepsilon ]$ aktiv und es gilt $l_v(\theta') = T_{uv}(l_u(\theta')) = l_u(\theta') + q_{uv}(\theta') + \tau_{uv}$.
		Mit Bedingung~\ref{def-feasible-flow-queue-with-capacity} folgere man $x_{uv}(\theta + \varepsilon) - x_{uv}(\theta) = F_{uv}^-(l_v(\theta + \varepsilon)) - F_{uv}^-(l_v(\theta))
		= \int_{l_v(\theta)}^{l_v(\theta + \varepsilon)} f_{uv}^-(t) dt
		= \int_{l_u(\theta) + q_{uv}(\theta)}^{l_u(\theta + \varepsilon) + q_{uv}(\theta + \varepsilon)} f_{uv}^-(t + \tau_{uv}) dt
		= u_{uv} (l_v(\theta + \varepsilon) - l_v(\theta))$.
		Teilt man diese Gleichung durch $\varepsilon$, so erhält man für $\varepsilon\rightarrow 0$ die Bedingung $g_{uv} = \frac{dx_{uv}}{d\theta}(\theta) = u_{uv} \frac{dl_v}{d\theta}(\theta)$.
		Ist $uv\in E_1$, so ist also Bedingung (4) erfüllt.
		Für Bedingung (2) setze man $g_{uv}=\frac{dx_{uv}}{d\theta}(\theta)=0$ voraus.
		Also ist auch $\frac{dl_v}{d\theta}(\theta)=0$ und, da $l_u$ monoton wachsend ist, gilt $0 \leq \frac{dl_u}{d\theta}$.

		Ist $uv\notin E_1$, ist also die Warteschlange zum Zeitpunkt $l_u(\theta)$ leer, so gilt: $l_v(\theta+\varepsilon) - l_v(\theta) = l_u(\theta + \varepsilon) + q_{uv}(l_u(\theta + \varepsilon)) - l_u(\theta) \geq l_u(\theta + \varepsilon) - l_u(\theta)$.
		Teilt man wieder durch $\varepsilon$, so erhält man für $\varepsilon  \rightarrow 0$ Bedingung (3) mit $\frac{dl_v}{d\theta}(\theta) \leq \frac{dl_u}{d\theta}(\theta)$ und dem Resultat des letzten Absatzes.
		
		\item[2. Fall:] $\exists \varepsilon > 0: \forall \theta'\in(\theta, \theta + \varepsilon]: T_{uv}(l_u(\theta'))>l_v(\theta')$.
		
		Die Kante ${uv}$ ist also im Intervall $(\theta, \theta + \varepsilon]$ nicht aktiv.
		Nach Lemma~\ref{lemma-nash-flow-waiting-queue-implies-active-edge} ist $q_{uv} \circ l_u\big|_{(\theta, \theta+\varepsilon]}=0$ und wegen Stetigkeit ist auch $q_{uv}(l_u(\theta))=0$, was $uv\notin E_1$ impliziert.
		Da $f$ nur entlang aktiver Kanten fließt, ist außerdem $f_{uv}^+ \circ l_u \big|_{(\theta, \theta+\varepsilon)} = 0$, und es gilt $x_{uv}(\theta + \varepsilon) - x_{uv}(\theta)=0$.
		Wird durch $\varepsilon$ geteilt, so erhält man $\frac{dx_{uv}}{d\theta}(\theta) = 0$ für $\varepsilon\rightarrow0$.
		Es muss also nur Bedingung (2) geprüft werden:
		
		
		Da $uv$ zum Zeitpunkt $l_u(\theta)$ noch aktiv ist, folgt $l_v(\theta + \varepsilon) - l_v(\theta) < l_u(\theta+\varepsilon) - l_v(\theta) + \tau_{uv} = l_u(\theta + \varepsilon) - l_u(\theta)$.
		Teilt man diese Ungleichung  durch $\varepsilon$, so erhält man für $\varepsilon\rightarrow 0$ die Bedingung $\frac{dl_v}{d\theta}(\theta)\leq\frac{dl_u}{d\theta}(\theta)$.
		
		\item[3. Fall:] $\forall \varepsilon>0: \exists \theta_{\varepsilon}\in (\theta, \theta+\varepsilon]: T_{uv}(l_u(\theta_\varepsilon)) = l_v(\theta_\varepsilon)$.
		
		Dies ist die exakte Umkehrung der Bedingung von Fall 2.
		Zusätzlich betrachte man diesen Fall nur, falls Fall 1 nicht eintritt.
		Das heißt, für alle $\theta_\varepsilon$ existiert ein $\theta'\in(\theta, \theta_\varepsilon]$ mit $q_{uv}(l_u(\theta')) = 0$; insbesondere ist $q_{uv}(l_u(\theta))= 0$, wodurch $uv$ nicht in $E_1$ enthalten ist.
		Man wähle $\theta'_\varepsilon:=\max\{ \theta'\in (\theta, \theta_\varepsilon] \mid q_{uv}(l_u(\theta')) = 0 \}$ als das Maximum solcher Zeitpunkte $\theta'$, welches aufgrund der Stetigkeit von $q_{uv}\circ l_u$ existiert.
		Nach Konstruktion ist $q_{uv}\circ l_u$ im Intervall $(\theta_\varepsilon', \theta_\varepsilon)$ positiv und nach Lemma~\ref{lemma-nash-flow-waiting-queue-implies-active-edge} ist die Kante $uv$ in diesem Intervall aktiv.
		Ist dieses Intervall nicht-leer, so ist $uv$ auch zum Zeitpunkt $\theta_\varepsilon'$ wegen der Stetigkeit von $T_{uv}\circ l_u$ und $l_v$ aktiv.
		Ansonsten ist $\theta_\varepsilon'=\theta_\varepsilon$ trotzdem eine Zeit, zu der $uv$ aktiv ist.
		Also gilt: $l_v(\theta_\varepsilon') - l_v(\theta) = l_u(\theta_\varepsilon') - l_u(\theta)$.
		Bedingung~(2) wird erfüllt, indem man durch $\theta_\varepsilon'-\theta$ teil und $\frac{dl_v}{d\theta}(\theta) = \frac{dl_u}{d\theta}(\theta)$ für $\varepsilon\rightarrow0$ erhält.
		
		Für Bedingung~(3) bleibt zu zeigen, dass $\frac{dx_{uv}}{d\theta}(\theta) /u_{uv}\leq \frac{dl_v}{d\theta}(\theta)$ gilt.
		Wegen Bedingung~\ref{def-feasible-flow-capacity} ist $x_{uv}(\theta + \varepsilon)-x_{uv}(\theta) = \int_{l_v(\theta+\varepsilon)}^{l_v(\theta)} f_{uv}^-(t) dt\leq (l_v(\theta + \varepsilon) - l_v(\theta)) u_{uv}$ für beliebiges $\varepsilon>0$.
		Durch Teilen mit $\varepsilon u_{uv}$ erhält man das gewünschte Resultat für $\varepsilon\rightarrow 0$.
	\end{description}
\end{proof}

\begin{definition}
	Ein \emph{Fluss über Zeit $f$ mit Zeithorizont $T>0$} ist ein Fluss, für dessen Zufluss $d(\theta)= 0$ für $\theta\geq T$ gilt.
\end{definition}

\begin{definition}
	Seien ein Nash-Fluss über Zeit $f$ mit Zeithorizont $T$ sowie dessen induzierter schmaler Fluss mit Zurücksetzen im Graphen $G_T$ zum Zeitpunkt $T$ und ein $\alpha > 0$ gegeben.
	
	\todo{Brauche vermutlich sowas wie $f_{vw}^+(\geq l_v(\theta))=0$}
	
	Man erhalte die \emph{$\alpha$-Erweiterung $\tilde{f}$ von $f$}, indem man die Werte aus $f$ übernehme und für alle $vw\in E$ den Zu- bzw. Abfluss setze auf
	$$\tilde{f}_{vw}^+(\theta):= \frac{x_{vw}'}{l_v'} \text{ für $\theta\in (l_v(T), l_v(T)+\alpha l_v']$ und } \tilde{f}_{vw}^-(\theta):=\frac{x_{vw}'}{l_w'} \text{ für $\theta\in (l_w(T), l_w(T)+\alpha l_w']$.}$$
\end{definition}

\begin{lemma}
	Jede $\alpha$-Erweiterung $\tilde{f}$ eines Nash-Flusses über Zeit $f$ mit Zeithorizont $T$ ist ein zulässiger Fluss über Zeit mit Zeithorizont $T+\alpha$, falls für alle Kanten mit positiver Warteschlange zum Zeitpunkt $T$ gilt:
	$$l_w(T) - l_v(T) + \alpha(l_w' - l_v') \geq \tau_{vw}.$$
	Dann gelten die folgenden Aussagen:
	\begin{enumerate}[label=(\roman*)]
		\item Gilt $x_{vw}' > 0$ für $vw\in E$, so ist $l_w(T) + (\theta - T)l_w' \geq l_v(T) + (\theta - T)l_v' + \tau_{vw}$ für $\theta\in[T, T+\alpha]$
		\item Der Fluss $\tilde{f}$ ist zulässiger Fluss über Zeit.
		\item Die zu $\tilde{f}$ gehörigen frühesten Ankunftszeiten $\tilde{l}_v(\theta)$ sind für $\theta \leq T+\alpha$ gegeben durch:
		$$\tilde{l}_v(\theta) = \begin{cases}
		l_v(\theta) & \text{ für $\theta < T$} \\
		l_v(T) + (\theta - T) l_v' & \text{ für $\theta \in [T, T+\alpha]$}
		\end{cases}$$
		\item Der Fluss $\tilde{f}$ ist Nash-Fluss über Zeit.
	\end{enumerate}
	\todo{ Zusätzliche Voraussetzung $l_w(T) - l_v(T) + \alpha(l_w' - l_v') \leq t_{vw}$, falls $vw$ nicht aktiv zum Zeitpunkt $T$. }
\end{lemma}
\begin{proof}
	Zu $(i)$: Ist $vw\in E_1$ mit $l_w'<l_v'$, so gilt
	$l_w(T)-l_v(T) + (\theta - T)(l_w' - l_v') \geq l_w(T)-l_v(T)+\alpha(l_w'- l_v')\geq \tau_{vw}$  mit der zusätzlichen Voraussetzung an $\alpha$.
	Sonst gilt $l_w' \geq l_v'$ nach $(3)$ und mit $T\in \Theta_{vw}$ folgt $l_w(T)+(\theta-T)l_w'=l_v(T) + q_{vw}(l_v(T))+\tau_{vw}+(\theta - T)l_w' \geq l_v(T) + (\theta-T)l_v'+\tau_{vw}$.
	
	Zu $(ii)$: Um zu zeigen, dass $\tilde{f}$ zulässig ist, zeige man die Eigenschaften \ref{def-feasible-flow-capacity}-\ref{def-feasible-flow-queue-with-capacity}.
	
	Die Kapazitätsbeschränkung \ref{def-feasible-flow-capacity} ist erfüllt, da für $x_{vw}'>0$ gilt $l_w'\geq x_{vw}' / u_{vw}$ nach Definition~\ref{def-thin-flow} und damit ist $\tilde{f}^-_{vw}(\theta)=x_{vw}'/l_w'\leq u_{vw}$ für $\theta\in(l_w(T), l_w(T)+\alpha l_w']$.
	
	Für \ref{def-feasible-flow-no-negative-flow} zeige man $\tilde{F}^+_{vw}(\theta)\geq \tilde{F}_{vw}^-(\theta+\tau_{vw})$ für alle $vw\in E$ und $\theta\in\R$.
	Ist $x_{vw}'=0$, so ist $\tilde{F}_{vw}^+(\theta) = F_{vw}^+(\min\{ l_v(T), \theta \} )$ und $\tilde{F}_{vw}^+(\theta + \tau_{vw}) = F_{vw}^+(\min\{ l_w(T), \theta +\tau_{vw}\})$.
	Mit $F_{vw}^+(l_v(T)) = F_{vw}^-(l_w(T))$ und $F_{vw}^+(\theta)\geq F_{vw}^-(\theta+\tau_{vw})$ folgt die Behauptung.
	Man betrachte also $x_{vw}'>0$ und nach Lemma~\ref{lemma-nash-flow-waiting-queue-implies-active-edge} gilt $T\in\Theta_{vw}$.
	Für $\theta\leq l_v(T)$ ist $\tilde{F}_{vw}^+(\theta)=F_{vw}^+(\theta)\geq F_{vw}^-(\theta + \tau_{vw}) = \tilde{F}_{vw}^-(\theta + \tau_{vw})$, da $\theta+\tau_{vw} \leq l_v(T)+\tau_{vw}\leq l_w(T)$ gilt.
	Existiert ein $\gamma\leq\alpha$ mit $\theta=l_v(T) + \gamma l_v'$, so gilt:
	$$\tilde{F}_{vw}^+(\theta)=\tilde{F}_{vw}^+(l_v(T) + \gamma l_v')=F_{vw}^+(l_v(T))+\gamma x_{vw}' = F_{vw}^-(l_w(T))+ \gamma x_{vw}'= \tilde{F}_{vw}^-(l_w(T)+\gamma l_w').$$
	Daraus folgt die Aussage, da $\theta + \tau_{vw}\geq l_w(T) + \gamma l_w'$ nach (i) gilt.
	Für $\theta > l_v(T)+\alpha l_v'$ gilt $\tilde{F}_{vw}^+(\theta) = \tilde{F}_{vw}^+(l_v(T) + \alpha l_v') = \tilde{F}_{vw}^-(l_w(T) + \alpha l_w') \geq \tilde{F}_{vw}^-(\theta + \tau_{vw})$.
	
	Die Bedingung~\ref{def-feasible-flow-no-flow-at-node} folgt, da $x'(T)$ nach Bemerkung~\ref{remark-s-t-flow} ein statischer $s$-$t$-Fluss ist und Flusserhaltung erfüllt.
	
\end{proof}