
\documentclass[envcountsect, handout]{beamer}
\usetheme[headheight=0em, logo=none, themepath=beamercolorthemefibeamer, nofonts]{fibeamer}
\usepackage{beamercolorthemefibeamer}


\usepackage[utf8]{inputenc} % for input encoding
\usepackage[german]{babel} % for german localization
\usepackage{colonequals} % for :=
\usepackage{graphicx} % for \includegraphics
\usepackage{wrapfig} % for floating images to the right
\usepackage{tikz}
\usepackage{esvect}
\usepackage{tabto}
\usepackage{enumitem}
\usepackage{mathabx}
\usetikzlibrary{arrows, shapes}

%-------------------------------------------------------------------------------
% Hilfreiche Befehle
%-------------------------------------------------------------------------------
\newcommand{\betrag}[1]{\lvert #1 \rvert}	        % Betrag
\providecommand*{\Lfloor}{\left\lfloor}                 % gro\ss{}es Abrunden
\providecommand*{\Rfloor}{\right\rfloor}                % gro\ss{}es Abrunden
\providecommand*{\Floor}[1]{\Lfloor #1 \Rfloor}         % gro\ss{}es ganzes Abrunden
\providecommand*{\Ceil}[1]{\left\lceil #1 \right\rceil} % gro\ss{}es ganzes Aufrunden

\newcommand{\Z}{\mathbb{Z}}
\newcommand{\N}{\mathbb{N}}
\newcommand{\R}{\mathbb{R}}
\newcommand{\Q}{\mathbb{Q}}
\newcommand{\firstNumbers}[1]{[#1]}
\newcommand{\transpose}{^\intercal}
\newcommand{\subjectTo}{\textbf{s.t.}}
\newcommand{\MIPR}{MIP\textsuperscript{*}}
\newcommand{\MIPI}{MIP}
\newcommand{\oBdA}{oBdA.}
\newcommand{\rang}{\operatorname{rang}}
\newcommand{\norm}[1]{\left\lVert#1\right\rVert_\infty}
\newcommand{\zero}{0}
\newcommand{\todo}[1]{{\color{red}{\emph{TODO: }}#1}}
\newcommand{\one}{\mathbbm{1}}
\newcommand{\eq}[1]{{\operatorname{eq}(#1)}}
\newcommand{\co}[1]{\operatorname{co}(#1)}

\setbeamertemplate{theorems}[numbered]
\newtheorem{conjecture}[theorem]{Vermutung}
\newtheorem{korollar}[theorem]{Korollar}
\newtheorem{beispiel}[theorem]{Beispiel}
\newtheorem{proposition}[theorem]{Proposition}

\definecolor{darkblue}{HTML}{00446B}

\newcommand{\coloniff}{\vcentcolon\Longle ftrightarrow}



\makeatletter
\makeatother
\useoutertheme{infolines}
\newenvironment{noheadline}{
	\setbeamertemplate{headline}{}
}{}
\newenvironment{nofootline}{
\setbeamertemplate{footline}{}
}{}

\setbeamersize{text margin left=2.7em, text margin right=2.7em}
\setbeamertemplate{frametitle}{\insertframetitle}
\setbeamercolor{block body}{fg=black!90}

\setbeamerfont{title}{size=\huge}
\setbeamertemplate{institute}{\insertinstitute}

\AtBeginSection{
	\begin{noheadline}
		\begin{frame}
		\vfill
		\centering
		\begin{beamercolorbox}[sep=8pt,center]{title}
			\usebeamerfont{title}\insertsectionhead\par%
		\end{beamercolorbox}
		\vfill
	\end{frame}
	\end{noheadline}

}

\makeatletter
\setbeamertemplate{footline}{%
	\color{darkblue}% to color the progressbar
	\small
	\hfill{\insertframenumber\hspace{1.5em}}
	\vspace{1em}


	\hspace*{-\beamer@leftmargin}%
	\rule{\beamer@leftmargin}{1pt}%
	\rlap{\rule{\dimexpr\numexpr0\insertframenumber\dimexpr
			\textwidth\relax/\numexpr0\inserttotalframenumber}{1pt}}
	% next 'empty' line is mandatory!

	\vspace{0\baselineskip}
	{}
}


\setbeamertemplate{bibliography item}{\insertbiblabel}

\title{\LARGE Berechnung von Nash-Gleichgewichten in dynamischen Flüssen}
\subtitle{Bachelorkolloquium}
\author{~\\Michael Markl \\ 23. Januar 2020}
\date{23.01.2020}
\renewcommand{\[}{
	\setlength\abovedisplayskip{0.5ex}
	\setlength{\belowdisplayskip}{0.5ex}
	\setlength{\abovedisplayshortskip}{0.5ex}
	\setlength{\belowdisplayshortskip}{0.5ex}\begin{equation*}}

\institute{Insitut für Mathematik der Universität Augsburg\\Diskrete Mathematik, Optimierung und Operations Research}

\usefonttheme{professionalfonts}

\DeclareFontFamily{OML}{zavm}{\skewchar\font=127 }
\DeclareFontShape{OML}{zavm}{m}{it}{<-> s*[.85] zavmri7m}{}
\DeclareFontShape{OML}{zavm}{b}{it}{<-> s*[.85] zavmbi7m}{}
\DeclareFontShape{OML}{zavm}{m}{sl}{<->ssub * zavm/m/it}{}
\DeclareFontShape{OML}{zavm}{bx}{it}{<->ssub * zavm/b/it}{}
\DeclareFontShape{OML}{zavm}{b}{sl}{<->ssub * zavm/b/it}{}
\DeclareFontShape{OML}{zavm}{bx}{sl}{<->ssub * zavm/b/sl}{}
\DeclareMathAlphabet{\mathsf}{OML}{zavm}{m}{it} % not `n'

\DeclareFontFamily{OT1}{zavm}{\skewchar \font =127}
\DeclareFontShape{OT1}{zavm}{m}{n}{<-> s*[.85] zavmr7t}{}
\DeclareFontShape{OT1}{zavm}{b}{n}{<-> s*[.85] zavmb7t}{}
\DeclareFontShape{OT1}{zavm}{bx}{n}{<->ssub * zavm/b/n}{}

\DeclareFontFamily{OMS}{zavm}{}
\DeclareFontShape{OMS}{zavm}{m}{n}{<-> s*[.85] zavmr7y}{}

\SetSymbolFont{operators}   {normal}{OT1}{zavm}{m}{n}
\SetSymbolFont{letters}     {normal}{OML}{zavm}{m}{it}
\SetSymbolFont{symbols}     {normal}{OMS}{zavm}{m}{n}
\SetSymbolFont{largesymbols}{normal}{OMX}{iwona}{m}{n}

\DeclareMathAlphabet{\mathcal}{OMS}{cmsy}{m}{n}
\SetMathAlphabet{\mathcal}{bold}{OMS}{cmsy}{b}{n}


\newcommand*\diff{\mathop{}\!\mathrm{d}}
%\setlist[enumerate]{topsep=0.5ex,itemsep=0ex,partopsep=0ex,parsep=0.8ex}


\beamertemplatenavigationsymbolsempty

\begin{document}

	\setcounter{framenumber}{-1}

	\begin{nofootline}
		\frame{\titlepage}
	\end{nofootline}

	\begin{frame}{Gliederung}
		\tableofcontents
	\end{frame}

	\section{Dynamische Flüsse}
\subsection{Grundlegende Definitionen}

\begin{frame}{Grundlegende Definitionen}
	\begin{definition}[Netzwerk]
		Ein \emph{Netzwerk} $(G, s, t, u, \tau)$ ist ein gerichteter Graph mit
		\begin{itemize}[label=\color{darkblue}$\bullet$]
			\item einer Quelle $s\in V$, sodass alle Knoten von $s$ aus erreichbar sind,
			\item einer Senke $t\in V$,
			\item Kantenkapazitäten $u\in\R_+^E$,
			\item Verzögerungszeiten $\tau\in\R_{\geq 0}^E$, sodass Zyklen eine positive Gesamtverzögerung haben.
		\end{itemize}
	\end{definition}
	\pause
	\begin{definition}[Funktionenraum $\mathfrak{F}_0$]
		Der Funktionenraum $\mathfrak{F}_0$ sei die Menge
		\[ \{ g: \R\to\R_{\geq 0} \mid \text{$g$ lokal Lebesgue-integrierbar, $g(t) = 0$ für $t<0$} \} \]
	\end{definition}
\end{frame}


\begin{frame}{Grundlegende Definitionen}
	\begin{definition}[Dynamischer Fluss]
		Ein dynamischer Fluss ist ein Paar $f=(f^+, f^-)$ mit $f^+, f^-\in\mathfrak{F}_0^E$.
		
		Für eine Kante $e\in E$ und einen Zeitpunkt $\theta\in\R$ bezeichnet
		\begin{itemize}[label=$\color{darkblue}\bullet$]
			\pause\item $f_e^+(\theta)$ bzw. $f_e^-(\theta)$ die \emph{Zu- bzw. Abflussrate an $e$ zur Zeit $\theta$},
			\pause\item $F_e^+(\theta) \colonequals \int_0^\theta f^+_e(t) \diff t$ bzw. $F^-_e(\theta)\colonequals\int_0^\theta f^-_e(t)\diff t$ den Zu- bzw. Abfluss an $e$ bis zur Zeit $\theta$,
			\pause\item $z_e(\theta)\colonequals F_e^+(\theta) - F_e^-(\theta + \tau_e)$ bzw. $q_e(\theta)\colonequals z_e(\theta)/u_e$ die Warteschlange bzw. Wartezeit an $e$ zur Zeit $\theta$,
			\pause\item $T_e(\theta)\colonequals\theta + q_e(\theta) + \tau_e$ die Austrittszeit aus $e$ bei Eintrittszeit $\theta$.
		\end{itemize}	
	\end{definition}
\end{frame}


\begin{frame}{Grundlegende Definitionen}
	\begin{definition}[Zulässiger Dynamischer Fluss]
		Ein zulässiger dynamischer Fluss erfüllt folgende Voraussetzungen:
		\begin{enumerate}[label=(F\arabic*)]
			\pause\item\label{def-feasible-flow-capacity} Kapazitätsbedingung: $\forall e\in E, \theta\in\R: f_e^-(\theta)\leq u_e$.
			\pause\item\label{def-feasible-flow-no-negative-flow} Keine Flussentstehung in Kanten:\\ $\forall e\in E, \theta\in\R: F_e^-(\theta + \tau_e) \leq F_e^+(\theta).$
			\pause\item\label{def-feasible-flow-no-flow-at-node} Flusserhaltung in Knoten:
			\[
			\forall\theta\in\R: \sum_{e\in\delta^+(v)}f^+_e(\theta) - \sum_{e\in\delta^-(v)} f_e^-(\theta) \begin{cases}
				\geq 0, \text{ falls $v=s$,}\\
				\leq 0, \text{ falls $v=t$,}\\
				= 0, \text{ sonst.}
			\end{cases}\]
			\pause\item\label{def-feasible-flow-queue-with-capacity} Warteschlangenabbau:\\
			$\forall e\in E, \theta\in\R: z_e(\theta) > 0 \implies f_e^-(\theta + \tau_e) = u_e$.
		\end{enumerate}
	\end{definition}
\end{frame}

\subsection{Eigenschaften zulässiger Flüsse}
\begin{frame}{Eigenschaften zulässiger Flüsse}
	\begin{proposition}\label{prop-feasible-flow}
		Für eine Kante $e\in E$ gilt in einen zulässigen dynamischen Fluss $f$:
		\begin{enumerate}[label=(\roman*)]
			\item\label{prop-feasible-flow-T-mon-inc-cont} Die Funktion $\theta \mapsto \theta + q_e(\theta)$ ist monoton wachsend und stetig.
			\item\label{prop-feasible-flow-positive-queue} Für alle $\theta\in\R$ ist die Warteschlange $z_e$ auf $(\theta, \theta + q_e(\theta))$ positiv.
			\item\label{prop-feasible-flow-det-outflow} Zu jeder Zeit $\theta\in\R$ gilt $F_e^+(\theta) = F_e^-(T_e(\theta))$.
		\end{enumerate}
	\end{proposition}
\end{frame}



	\section{Kürzeste Wege}

\begin{frame}
	\begin{definition}[Kürzeste Wege]
		Für einen Fluss $f$ bezeichne:
		\begin{itemize}[label=\color{darkblue}$\bullet$]
			\item\pause $l^P(\theta)\colonequals T_{e_k}\circ\dots\circ T_{e_1}(\theta)$ die Ankunftszeit am Endknoten eines Pfades $P=(e_1,\dots,e_k)$ zur Startzeit $\theta$ am Startknoten,
			\item\pause $\mathcal{P}_w$ die Menge aller $s$-$w$-Pfade,
			\item\pause $l_w(\theta) \colonequals \min_{P\in\mathcal{P}_w} l^P(\theta)$ die früheste Ankunftszeit bei $w$ zur Startzeit $\theta$.
		\end{itemize}
	\pause Ein Pfad $P\in \mathcal{P}_w$ heißt \emph{kürzester $s$-$w$-Pfad zur Zeit $\theta$}, falls $l^P(\theta)=l_w(\theta)$.
	\end{definition}
	\pause\begin{lemma}[Dreiecksungleichung]
		Für alle Kanten $vw\in E$ gilt in einem zulässigen Fluss $T_{vw}(l_v(\theta))\geq l_w(\theta)$.
	\end{lemma}
\end{frame}

\begin{frame}
	\begin{definition}[Aktivität einer Kante]
		Eine Kante $vw\in E$ ist \emph{aktiv zum Zeitpunkt $\theta$}, falls $T_{vw}(l_v(\theta)) = l_w(\theta)$ gilt; sonst ist sie \emph{inaktiv zum Zeitpunkt $\theta$}.
		
		Die Menge $\Theta_e$ sei die abgeschlossene Menge aller Zeitpunkte, zu denen $e$ aktiv ist.
	\end{definition}
	\pause\begin{lemma}
		Für einen zulässigen Fluss ist der Teilgraph der zu $\theta$ aktiven Kanten ein azyklischer Graph, in dem $s$ jeden Knoten erreichen kann.
	\end{lemma}
	\pause\begin{proposition}
			Für einen zulässigen Fluss $f$ ist $(l_v(\theta))_{v\in V}$ die eindeutige Lösung von
		\[ \tilde{l}_w = \begin{cases}
		\theta, & \text{falls } w=s, \\
		\min\limits_{vw\in \delta^-(w)} T_{vw}(\tilde{l}_v), & \text{sonst}.
		\end{cases} \]
	\end{proposition}
\end{frame}
	
	\subsection{Dynamische Nash-Flüsse}

\begin{frame}{Dynamische Nash-Flüsse}
	\begin{definition}[Nash-Fluss]
		Ein zulässiger dynamischer Fluss $f$ heißt \emph{Nash-Fluss}, falls er nur entlang aktiver Kanten fließt, d. h. falls $f_{vw}^+(\theta) > 0$ für fast alle $\theta\in\R$ bereits $\theta\in l_v(\Theta_{vw})$ impliziert.
	\end{definition}
	\pause\begin{definition}
		Für einen zulässigen Fluss $f$ und einen Zeitpunkt $\theta$ bezeichne
		$x_{vw}^+(\theta)\coloneq  F_{vw}^+(l_v(\theta))$ bzw. $x_{vw}^-(\theta)\coloneq  F^-_{vw}(l_w(\theta))$ für $vw\in E$.
	\end{definition}
	\pause\begin{theorem}
	Ein zulässiger dynamischer Fluss $f$ ist genau dann ein Nash-Fluss,
	wenn $x_e^+(\theta) = x_e^-(\theta)$ für alle $e\in E$ und $\theta\in\R$ gilt.
	Ist dies der Fall, so definiere man $x_e(\theta)\coloneq x_e^+(\theta) = x_e^-(\theta)$ für $e\in E$.
	\end{theorem}
\end{frame}
	
	\section{Berechnung dynamischer Nash-Flüsse}

\subsection{Erweiterung dynamischer Nash-Flüsse}

\begin{frame}<presentation:0>[noframenumbering]
\begin{definition}[$s$-Fluss]
	Ist $x'$ ein statischer $b$-Fluss und $s$ ein Knoten, so nennt man $x'$ einen $s$-Fluss, falls $b_v\leq 0$ für alle $v\neq s$ gilt.
\end{definition}

\pause\begin{definition}[Schmaler Fluss mit Zurücksetzen]\label{def-thin-flow}
		Für einen statischen $s$-Fluss $x'$, $E_1\subseteq E$ und $l'_s\in\{ 0, 1 \}$ bezeichne
		\begin{itemize}[label=\color{darkblue}$\bullet$]
			\pause\item $\rho_{vw}: \R_{\geq 0}^2 \to \R_{\geq 0}$ die Funktion \[ \rho_{vw}(l'_v, x'_{vw}) \coloneq \begin{cases}
			\max(l'_v, x'_{vw} / u_{vw}), & \text{falls $vw\notin E_1$,}\\
			x'_{vw} / u_{vw}, & \text{falls $vw\in E_1$,}
			\end{cases}
			\]
			\item\pause $l'^P\coloneq \rho_{e_k}(\,\cdot\,, x'_{e_k}) \circ \cdots \circ \rho_{e_1}(\,\cdot\,, x'_{e_1}) (l'_s)$ die Auslastung des $s$-$v$-Pfads $(e_1, \cdots e_k)$ bzgl. $x'$,
			\item\pause $l'_v \coloneq \min_{P\in\mathcal{P}_v} l'^P(l_s')$ die Auslastung des Knoten $v$.
		\end{itemize}
		\pause Gilt $l_v' = l'^P$ für jeden $s$-$v$-Pfad $P$ mit positivem Fluss, so heißt $x'$ ein schmaler Fluss mit Zurücksetzen auf $E_1$.
		Ist $l_s' = 1$, so nennt man $x'$ normiert.
	\end{definition}
\end{frame}

\begin{frame}<presentation:0>[noframenumbering]
	\begin{definition}[Dynamischer Fluss mit Zeithorizont]
		Ein \emph{dynamischer Fluss $f$ mit Zeithorizont $T\geq0$} ist ein Fluss, für dessen Zufluss $d(\theta)= 0$ für $\theta\geq T$ gilt.
	\end{definition}

	\pause\begin{definition}[$\alpha$-Erweiterung]
		Seien ein dynamischer Nash-Fluss $f$ mit Zeithorizont $T$ und ein normierter schmaler $s$-$t$-Fluss $x'$ in $G_T$ mit Zurücksetzen auf $E_1 \coloneq \{ vw\in E \mid q_{vw}(l_v(\theta)) > 0 \}$ und ein $\alpha > 0$ gegeben.
		Ergänzt man $f$, sodass für zur Zeit $T$ aktive Kanten $vw$
		\begin{align*}
			\tilde{f}_{vw}^+(\theta)\coloneq x_{vw}'/l_v'& \text{~~~für $\theta\in [l_v(T), l_v(T)+\alpha l_v')$ und } \\
			\tilde{f}_{vw}^-(\theta)\coloneq x_{vw}'/l_w'& \text{~~~für $\theta\in [l_w(T), l_w(T)+\alpha l_w')$}
		\end{align*}
		gelten, erhält man eine \emph{$\alpha$-Erweiterung $\tilde{f}$ von $f$}.
	\end{definition}
\end{frame}

\begin{frame}<presentation:0>[noframenumbering]
	\begin{theorem}[Erweiterung eines Nash-Flusses]
		Jede $\alpha$-Erweiterung $\tilde{f}$ eines dynamischen Nash-Flusses $f$ mit Zeithorizont~$T$, welche die Bedingungen
		\begin{align*}
			l_w(T) - l_v(T) + \alpha(l_w' - l_v') \geq \tau_{vw}, &\text{ ~~~~falls $q_{vw}(l_v(T)) > 0$,}\\
			l_w(T) - l_v(T) + \alpha(l_w' - l_v') \leq \tau_{vw}, &\text{ ~~~~falls $T\in\Theta_{vw}^c $,}
		\end{align*}
		erfüllt, ist ein dynamischer Nash-Fluss.
	\end{theorem}
\end{frame}

	

	\begin{noheadline}
		\begin{frame}<presentation:0>[noframenumbering]
			\cite{Koch2011}
			\cite{Cominetti2011}
			\cite{Cominetti2015}
			\cite{Elstrodt2011Abs}
			\cite{Elstrodt2011Top}
		\end{frame}
	
		\begin{frame}[allowframebreaks]{Literatur}
			\scriptsize
			\bibliographystyle{alphadin}
			\bibliography{literature}
		\end{frame}
	\end{noheadline}

\end{document}
