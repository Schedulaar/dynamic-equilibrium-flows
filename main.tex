\newcommand*{\DocType}{scrartcl}
\renewcommand*\DocType{article} % Uncomment for screen optimization

\newcommand*\ClassList{scrartcl,article}
\documentclass[\DocType, paper=a4,fontsize=11pt,abstracton,headsepline,notitlepage]{generalclass}


\usepackage[utf8]{inputenc}
\usepackage[automark]{scrpage2}	        % Seiten-Stil für scrartcl
% Mathematische Zeichensätze und Umgebungen
\usepackage{amsfonts, amssymb}	        % Definition einer Liste mathematischer Fontbefehle und Symbole
\usepackage[intlimits,sumlimits]{amsmath} % Integral-/Summationsgrenzen über/unter Zeichen
\usepackage{mathabx}
% mathematische Verbesserungen
\usepackage{amsthm}	                    % spezielle theorem Stile
\usepackage{aliascnt} 
\usepackage{array}		                % erweiterte Tabellen
% Schriftzeichen, Format
\usepackage{latexsym}		            % Latex-Symbole
\usepackage[english, german, ngerman]{babel} % Mehrsprachenumgebung
% Layout
\usepackage{geometry}                   % Seitenränder
\usepackage{xcolor}                     % Farben
\usepackage{bbm}
% Tabellen und Listen
\usepackage{float}		                % Gleitobjekte 
\usepackage[flushright]{paralist}       % Bessere Behandlung der Auflistungen

% Bilder
\usepackage[final]{graphicx}            % Graphiken einbinden

\usepackage{caption}                    % Beschriftungen
\usepackage{subcaption}                 % Beschriftungen für Unterteilung

% Interaktive Referenzen, und PDF-Keys
\usepackage{xr-hyper}  
\usepackage[pagebackref,pdftex]{hyperref} % Rückreferenz im Literaturverzeichnis, Treiber für ps zu pdf ; für direkt nach pdf: pdftex

%%%%%%%%%%%%%%%%%%%%%%%%%%%%%%%%%%%%%%%%%%%%%%%%%%%%%%%%%%%%%%%%%%%%%%%%%%%%%%%%
% Zusammenfassung einiger nützlicher Pakete und Befehle
%-------------------------------------------------------------------------------
% Kopf-Zeilen
%-------------------------------------------------------------------------------

\pagestyle{scrheadings}		     % Kopfzeilen nach scr-Standard		
\ifx\chapter\undefined 		     % falls Kapitel nicht definiert sind
  \automark[subsection]{section} % Kopf- und Fusszeilen setzen
\else				             % Kapitel sind definiert
  \automark[section]{chapter}	 % Kopf- und Fusszeilen setzen
\fi

%-------------------------------------------------------------------------------
%   Maske für Überschrift 
%-------------------------------------------------------------------------------
% Belegung der notwendigen Kommandos für die Titelseite
\newcommand{\autor}{Markl, Michael} 		% bearbeitender Student
\newcommand{\veranstaltung}{Seminar zur Optimierung und Spieltheorie} 	% Titel des ganzen Seminars
\newcommand{\uni}{Institut für Mathematik der Universität Augsburg} % Universit\"at
\newcommand{\matrikelnummer}{1474802}
\newcommand{\lehrstuhl}{Diskrete Mathematik, Optimierung und Operations Research} % Lehrstuhl
\newcommand{\semester}{Sommersemester 2019}	% Winter-/Sommersemester mit Jahr
\newcommand{\datum}{27.06.2019} 			% Datumsangabe
\newcommand{\thema}{Nash Gleichgewichte in Dynamischen Flüssen}  		% Titel der Seminararbeit

\newcommand{\ownline}{\vspace{.7em}\hrule\vspace{.7em}} % horizontale Linie mit Abstand

\newcommand{\seminarkopf}{
	% Befehl zum Erzeugen der Titelseite 
 \textsc{\autor}  \hfill{\datum} \\ 
\textbf{\veranstaltung} \\ 
\uni \\ 
\lehrstuhl \\
\semester \hfill{Matrikelnummer: \matrikelnummer}
\ownline 

\begin{center}
{\LARGE \textbf{\thema}}
\end{center}

\ownline
}			% Befehle und Pakete für Titelseite


\DeclareMathOperator{\e}{ex}
\DeclareMathOperator{\ma}{mate}
\DeclareMathOperator{\Ex}{Ex}

%-------------------------------------------------------------------------------
%   Befehle für Nummerierung der Ergebnisse
%   fortlaufend innerhalb eines Abschnittes
%-------------------------------------------------------------------------------
\theoremstyle{plain}            % normaler Stil
\newtheorem{theorem}{Theorem}
\numberwithin{theorem}{section}
% Lemma
\newaliascnt{lemma}{theorem}
\newtheorem{lemma}[lemma]{Lemma}
\aliascntresetthe{lemma}
% Satz
\newaliascnt{satz}{theorem}
\newtheorem{satz}[satz]{Satz}
\aliascntresetthe{satz}
% Korollar
\newaliascnt{corollary}{theorem}
\newtheorem{corollary}[corollary]{Korollar}
\aliascntresetthe{corollary}
% Proposition
\newaliascnt{proposition}{theorem}
\newtheorem{proposition}[proposition]{Proposition}
\aliascntresetthe{proposition}
%-------------------------------------------------------------------------------
\theoremstyle{definition}	% Definitionsstil
% Definition
\newaliascnt{definition}{theorem}
\newtheorem{definition}[definition]{Definition}
\aliascntresetthe{definition}
% Beispiel
\newaliascnt{example}{theorem}
\newtheorem{example}[example]{Beispiel}
\aliascntresetthe{example}
% Problem
\newaliascnt{problem}{theorem}
\newtheorem{problem}[problem]{Problem}
\aliascntresetthe{problem}
% Algorithmus
\newaliascnt{algorithmus}{theorem}
\newtheorem{algorithmus}[algorithmus]{Algorithmus}
\aliascntresetthe{algorithmus}
%-------------------------------------------------------------------------------
\theoremstyle{remark}		% Bemerkungsstil
% Bemerkung
\newaliascnt{remark}{theorem}
\newtheorem{remark}[remark]{Bemerkung}
\aliascntresetthe{remark}
% Vermutung
\newaliascnt{conjecture}{theorem}
\newtheorem{conjecture}[conjecture]{Vermutung}
\aliascntresetthe{conjecture}
% Notation
\newaliascnt{notation}{theorem}
\newtheorem{notation}[notation]{Notation}
\aliascntresetthe{notation}

%-------------------------------------------------------------------------------
% automatische Referenzen mit interaktiven Text
%-------------------------------------------------------------------------------

% Texte
\renewcommand{\theoremautorefname}{Theorem}
\newcommand{\lemmaautorefname}{Lemma}
\newcommand{\satzautorefname}{Satz}
\newcommand{\korollarautorefname}{Korollar}
\newcommand{\propositionautorefname}{Proposition}

\newcommand{\definitionautorefname}{Definition}
\newcommand{\beispielautorefname}{Beispiel}
\newcommand{\problemautorefname}{Problem}
\newcommand{\algorithmusautorefname}{Algorithmus}

\newcommand{\bemerkungautorefname}{Bemerkung}
\newcommand{\vermutungautorefname}{Vermutung}
\newcommand{\notationautorefname}{Notation}

%-------------------------------------------------------------------------------
% Nummerierung der Gleichungen innerhalb der obersten Ebene
%-------------------------------------------------------------------------------
\ifx\chapter\undefined 			% Kapitel sind definiert
  \numberwithin{equation}{section}	% Gleichungsnummern in Section
\else					% Kapitel sind nicht definiert
  \numberwithin{equation}{chapter}	% Gleichungsnummern in Kapiteln
\fi





\makeatletter
\newcommand{\overleftsmallarrow}{\mathpalette{\overarrowsmall@\leftarrowfill@}}
\newcommand{\overrightsmallarrow}{\mathpalette{\overarrowsmall@\rightarrowfill@}}
\newcommand{\overleftrightsmallarrow}{\mathpalette{\overarrowsmall@\leftrightarrowfill@}}
\newcommand{\overarrowsmall@}[3]{%
	\vbox{%
		\ialign{%
			##\crcr
			#1{\smaller@style{#2}}\crcr
			\noalign{\nointerlineskip\vskip0.4pt}%
			$\m@th\hfil#2#3\hfil$\crcr
		}%
	}%
}
\def\smaller@style#1{%
	\ifx#1\displaystyle\scriptstyle\else
	\ifx#1\textstyle\scriptstyle\else
	\scriptscriptstyle
	\fi
	\fi
}
\makeatother
			% Mathematische Befehle und Pakete

% Literatur-Bibliothek
\bibliographystyle{alphadin}               % deutscher Bibliotheksstil

% Erweiterte Einstellungen zu hyperref

\hypersetup{
        breaklinks=true,        % zu lange Links unterbrechen
        colorlinks=true,        % Färben von Referenzen
        citecolor=black,        % Farbe der Zitate
        linkcolor=black,        % Farbe der Links
        extension=pdf,          % Externe Dokumente können eingebunden werden.
        ngerman,		
	pdfview=FitH,
	pdfstartview=FitH,		
	bookmarksnumbered=true,     % Anzeige der Abschnittsnummern	% pdf-Titel
	pdfauthor={\autor}          % pdf-Autor
}

% Namen für Referenzen 

\newcommand{\ownautorefnames}{
  \renewcommand{\sectionautorefname}{Kapitel}
  \renewcommand{\subsectionautorefname}{Unterkapitel}
  \renewcommand{\subsubsectionautorefname}{\subsectionautorefname}
  \renewcommand{\appendixautorefname}{Anhang}
  \renewcommand{\figureautorefname}{Abbildung}
}

% Rückreferenzentext zur Literatur
\def\bibandname{und}%
\renewcommand*{\backref}[1]{}
\renewcommand*{\backrefalt}[4]{%
\ifcase #1 %
 (Nicht zitiert, also Ergänzungsliteratur.)%
\or
 (Zitiert auf Seite #2.)%
\else
 (Zitiert auf den Seiten #2.)%
\fi
}
\renewcommand{\backreftwosep}{ und~} % seperate 2 pages
\renewcommand{\backreflastsep}{ und~} % seperate last of longer 

			% Befehle und Pakete für Referenzen

\geometry{a4paper, top=30mm, bottom=30mm, left=30mm, right=30mm}

\IfClass{article}{ % Optimize for screen
	\geometry{papersize={150mm,190mm},margin=5mm} 
}

\addtolength{\footskip}{-0.5cm}          % Seitenzahlen höher setzen
\renewcommand{\descriptionlabel}[1]{\hspace{\labelsep}\textit{#1}}


\newcommand{\todo}[1]{{\color{red}#1}}

\numberwithin{figure}{section}	% Abbildungsnummern in Section


%%%%%%%%%%%%%%%%%%%%%%%%%%%%%%%%%%%%%%%%%%%%%%%%%%%%%%%%%%%%%%%%%%%%%%%%%%%%%%%%
% Start des Dokuments
\begin{document}

\ownautorefnames		% Änderung einiger automatischen Texte von hyperref (wie in referenz.tex definiert)
\parindent0em 			% kein Einzug nach einer Leerzeile

%%%%%%%%%%%%%%%%%%%%%%%%%%%%%%%%%%%%%%%%%%%%%%%%%%%%%%%%%%%%%%%%%%%%%%%%%%%%%%%%
% Titelseite
\thispagestyle{empty}   % leerer Seitenstil, also keine Seitennummer
\begin{titlepage}
\seminarkopf            % Titelblatt (wie in kopf.tex definiert)
\begin{abstract}
Diese Arbeit gibt eine formale Definition von dynamischen Flüssen mit Warteschlangenmodell und darauf aufbauend eine Charakterisierung von Nash Gleichgewichten in diesem Kontext.
Dabei werden Partikel des Flusses als gleichwertige Spieler betrachtet, die von Zeit zu Zeit kontinuierlich bei einer Quelle entstehen und per Selfish Routing versuchen ihre Ankunftszeit bei einer Senke zu minimieren.
Eine spezielle Klasse von statischen Flüssen, die schmalen Flüsse mit Zurücksetzen, kann verwendet werden, um Nash Flüsse mit Zeithorizont zu erweitern.
\end{abstract}
\end{titlepage}

%%%%%%%%%%%%%%%%%%%%%%%%%%%%%%%%%%%%%%%%%%%%%%%%%%%%%%%%%%%%%%%%%%%%%%%%%%%%%%%%
\thispagestyle{empty}
\tableofcontents        % Inhaltsverzeichnis
%\listoffigures         % Abbildungsverzeichnis (eventuell einfügen)
%\listoftables          % Tabellenverzeichnis (eventuell einfügen)
\setcounter{page}{0}    % Eigentlicher Inhalt beginnt auf Seite 1
\clearpage              % neue Seite für eigentlichen Inhalt

%%%%%%%%%%%%%%%%%%%%%%%%%%%%%%%%%%%%%%%%%%%%%%%%%%%%%%%%%%%%%%%%%%%%%%%%%%%%%%%%
% Eigentlicher Inhalt der Seminararbeit; die einzelnen Teile werden hier (aus Gründen der Übersichtlichkeit) über \input{file} eingebunden

\section{Einführung}\label{introduction}

\newcommand{\R}{\mathbb{R}}

\begin{definition}[strategisches Spiel]
	Ein \emph{strategisches Spiel} $\mathcal{G}$ ist ein Tupel $(P, X, \theta)$, wobei $P$ eine Menge von Spielern, $X$ die Menge der Strategien  und $\theta=(\theta_p)_{p_\in P}$ die Familie der Auszahlungsfunktionen der Spieler ist, wobei $\theta_p$ eine Funktion $X^P\to\R$ für jeden Spieler $p\in P$ ist.
\end{definition}

Man beachte, dass in dieser Definition eines strategischen Spiels die Spieler eine gemeinsame Strategiemenge haben.

\begin{definition}[statisches Flussnetzwerk]
	Ein \emph{statisches Flussnetzwerk} $\mathcal{N}:=(G,s,t,\mu)$ ist ein Graph $G:=(V,E)$ mit Knoten $V$ und Kanten $E$, einer \emph{Quelle} $s\in V$, einer \emph{Senke} $t\in V$ zusammen mit einem \emph{$d$-wertigen statischen Fluss} $\mu: \mathcal{P} \to \R$, wobei $\mathcal{P}$ die Menge aller einfachen $s$-$t$-Pfade in G ist und $d=\sum_{p\in\mathcal{P}}\mu(p)$ gilt.
\end{definition}

\begin{definition}[statisches Routenplanungsspiel]
	Sei ein Graph $G:=(V,E)$ mit Knoten $V$ und Kanten $E$, einer Quelle $s\in V$, einer Senke $t\in V$ und einem Zufluss $d\in\R_+$ gegeben.
	Ein \emph{statisches Routenplanungsspiel} ist ein strategisches Spiel mit  Spielermenge $[0,d]\subseteq\R$, wobei ein Spieler als \emph{Flusspartikel} bezeichnet wird, und mit Strategiemenge $\mathcal{P}$ aller $s$-$t$-Pfade in $G$.
	Die Auszahlungsfunktion $\theta_x$ gibt die Kosten des Partikels $x$ in Abhängigkeit des Flusses an.
	
	Der Fluss $\mu$ heißt Nashfluss, falls $l_P(\mu)=min_{P'\in\mathcal{P}}l_{P'}(\mu)$ für alle $P\in\mathcal{P}$ mit $\mu_P > 0$ gilt.
\end{definition}

\section{Dynamische Flüsse mit Zeithorizont}


Betrachte Zeitraum $[0,T]$ mit $T\in\R_+$.
$d$ Spieler erscheinen an Quelle $s$ über Zeitraum von $0$ bis $T$

\begin{definition}[Netzwerk]
	Ein \emph{Netzwerk} ist ein gerichteter Graph $G=(V,E)$ mit endlicher Knotenmenge $V$ und Kantenmenge $E\subseteq V\times V$, einer \emph{Quelle} $s\in V$ und einer Senke $t\in V$.
	Jeder Kante $e\in E$ werden eine Kapazität $u_e > 0$ und eine Verzögerungszeit $\tau_e\geq 0$ zugeordnet, sodass alle Zykel $T$ eine positive Gesamtverzögerung $\sum_{e\in T}\tau_e > 0$ haben.
\end{definition}

Für ein Netzwerk $\mathcal{N}$ bezeichne $\mathcal{P}$ die Menge aller $s$-$t$-Pfade.

\begin{definition}[Fluss über Zeit]
	Ein Fluss über Zeit $f=(f^+, f^-)$ ist ein Paar zweier über die Kanten $E$ indizierten Familien von Lebesgue-integrierbaren Funktionen $f^+_e,f^-_e: \R \to \R_{\geq 0}$ an $e$ für alle $e\in E$.
	
	Dabei bezeichnen $f_e^+(\theta)$ die \emph{Einflussrate an $e$ zum Zeitpunkt $\theta\in\R$} und $f_e^-(\theta)$ die \emph{Ausflussrate aus $e$ zum Zeitpunkt $\theta\in\R$} für $e\in E$.
	
	Der (kumulative) \emph{Einfluss bzw. Ausfluss an einer Kante $e$ bis zum Zeitpunkt $\theta$} sei definiert durch $F^+_e(\theta):=\int_{[0,\theta)} f^+_e d\lambda$ bzw. $F^-_e(\theta):=\int_{[0,\theta)} f^-_e d\lambda$.
	
	Die \emph{(Länge der) Warteschlange an Kante $e$ zum Zeitpunkt $\theta\in\R$} sei definiert durch $z_e(\theta):= F_e^+(\theta) - F_e^-(\theta + \tau_e)$ und die \emph{Wartezeit an Kante $e$ zum Zeitpunkt $\theta\in\R$} sei definiert durch $q_e(\theta) = z_e / u_e$.
	
	Man beschreibe den \emph{Austrittszeitpunkt einer Kante $e$ bei Eintrittszeitpunkt $\theta$}, in dem ein Partikel eine Kante verlässt, die es zum Zeitpunkt $\theta$ betreten hat, als $T_e(\theta):=\theta + q_e(\theta) + \tau_e$.
\end{definition}

\begin{definition}[Zulässiger Fluss über Zeit]
	Sei ein Fluss über Zeit $f=(f^+, f^-)$ gegeben. $f$ heißt zulässig, falls
	\begin{enumerate}[(I)]
		\item keine Ausflussrate die Kapazität übersteigt, d.h. $\forall e\in E, \theta\in\R: f_e^-\leq u_e$, und
		\item\label{weak} Fluss eine Kante nur verlässt, falls er die Kante zuvor betreten hat, d.h. $\forall e\in E, \theta\in\R: F_e^+(\theta) \geq F_e^-(\theta + \tau_e)$, und
		\item\label{def-feasible-flow-no-flow-at-node} Fluss direkt nach der Ankunft an einem Knoten der nächsten Kante zugeordnet wird, d.h. $\forall v\in V\setminus \{ s, t \}, \theta\in\R: \sum_{e\in\delta^-(v)}f^-_e(\theta) = \sum_{e\in\delta^+(v)} f_e^+(\theta)$, und
		\item\label{queue-with-capacity} nicht-leere Warteschlangen mit der Kapazität der Kante abgebaut werden, d.h. $\forall e\in E, \theta\in\R: q_e(\theta) > 0 \implies f_e^-(\theta + \tau_e) = u_e$.
		\todo{
		 $F_e^+(\theta) = F_e^-(\theta + \tau_e + q_e(\theta))$ für alle $e\in E, \theta\in\R$, ist das das gleiche (f.ü.)?}
	\end{enumerate}
\end{definition}

\todo{source and sink nodes: Kein Fluss verschwindet bei Quelle, kein Fluss kommt hinzu bei Sinke}

\todo{FIFO Interpretation}

\begin{proposition}\label{prop-feasible-flow}
	Sei $e\in E$ eine Kante und $f$ ein zulässiger Fluss. Dann gilt:
	\begin{enumerate}[(i)]
		\item\label{prop-feasible-flow-T-mon-inc-cont} Die Funktion $\theta \mapsto \theta + q_e(\theta)$ ist monoton wachsend und stetig.
		\item\label{prop-feasible-flow-det-outflow} Für alle $\theta\in\R$ ist $F_e^+(\theta) = F_e^-(\theta+\tau_e+q_e(\theta))$.\todo{use $T_e(\theta)$ here}
		\item Für zwei Zeitpunkte $0\leq \theta_1 < \theta_2$ mit $\int_{\theta_1}^{\theta_2} f^+_e d\lambda = 0$ und $q_e(\theta_2)>0$ ist $\theta_1 + q_e(\theta_1) = \theta_2 + z_e(\theta_2)$.
	\end{enumerate}
\end{proposition}
\begin{proof}
	Für die Monotonie in~(\ref{prop-feasible-flow-T-mon-inc-cont}) seien $\theta_1 \leq \theta_2$ gegeben.
	Mit der Monotonie von $F_e^+$ und mit $F_e^-(\theta_1 + \tau_e) = F_e^-(\theta_2+\tau_e) - \int_{\theta_1+\tau}^{\theta_2+\tau} f_e^-(t)dt\leq F_e^-(\theta_2 + \tau_e) + (\theta_2 - \theta_1)u_e$ gilt: 
	$$
		\theta_1 + q_e(\theta_1)
		= \theta_1 + \frac{F_e^+(\theta_1) - F_e^-(\theta_1 + \tau_e)}{u_e}\\
		\leq \theta_2 + \frac{F_e^+(\theta_1) - F_e^-(\theta_2+\tau_e)}{u_e} \leq \theta_2 + q_e(\theta_2).
	$$
	Die Stetigkeit folgt aus der Stetigkeit von $F_e^+$ und $F_e^-$.
	
	Zu Aussage (ii): Nach Voraussetzung~(\ref{queue-with-capacity}) ist $\int_{\theta + \tau_e}^{\theta + \tau_e + q_e(\theta)} f_e^-(t) dt = \int_{\theta}^{\theta + q_e(\theta)}f_e^-(t + \tau_e) dt = q_e(\theta)  u_e = z_e(\theta)$ (\todo{für $\theta'$ ist $q_e(\theta')> 0$}). Damit ist $F_e^-(\theta + \tau_e + q_e(\theta)) = F_e^-(\theta+\tau_e) + \int_{\theta+\tau_e}^{\theta+\tau_e+q_e(\theta)}f_e^-(t)dt = F_e^+(\theta)$.
	
	Zu Aussage (iii): Für alle $\theta'\in [\theta_1, \theta_2]$ gilt $F_e^+(\theta_1) = F_e^+(\theta') = F_e^-(\theta_2)$.
	Also ist $F_e^+(\theta') - F_e^-(\theta' + \tau_e) \geq F_e^+(\theta_2)-F_e^-(\theta_2 + \tau_e) = z_e(\theta_2) > 0$ und es gilt $f_e^-(\theta' + \tau_e)=u_e$.
	Für die Warteschlangendifferenz gilt dann: 
	$z_e(\theta_1)-z_e(\theta_2)=-F^-(\theta_1 + \tau_e) + F^-(\theta_2 + \tau_e) = \int_{\theta_1 + \tau_e}^{\theta_2 + \tau_e} f^-_e(t) dt = (\theta_2 - \theta_1)u_e$.
	Man folgere $q_e(\theta_1) - q_e(\theta_2) = \theta_2 - \theta_1$.
\end{proof}

Insbesondere ist also die Kantenaustrittszeit $T_e$ monoton wachsend und stetig.

\todo{Stetigkeit von irgendwas um die "Wohldefiniertheit" dieser komischen Definition  von Nash flows zu zeigen}

\section{Charakterisierung von Nash-Flüssen über Zeit}

\begin{definition}
	Für einen Fluss $f$ und einen Pfad $P=(e_1,\dots,e_k)$ definiere $l^P(\theta):=T_{e_k}\circ\dots\circ T_{e_1}(\theta)$ den Zeitpunkt, an dem ein Partikel den Endknoten des Pfads erreicht, falls er den Pfad zum Zeitpunkt $\theta$ betritt.
	
	Für einen Knoten $w\in V$ beschreibe $\mathcal{P}_w$ die Menge aller $s$-$w$-Pfade.
	Dann ist die früheste Ankunft eines Partikels, das zum Zeitpunkt $\theta$ bei $s$ startet, gegeben durch $l_w(\theta):=\min_{P\in\mathcal{P}_w}l^P(\theta)$.
	Ein Pfad $P\in \mathcal{P}_w$ heißt \emph{kürzester $s$-$w$-Pfad zum Zeitpunkt $\theta$},  falls $l_w(\theta)=l^P(\theta)$.
\end{definition}

Für einen Knoten $v \in V$ und einen zulässigen Fluss ist die Funktion $l_v$ als Minimum von Kompositionen stetiger und monoton wachsender Funktionen $T_e$ (Proposition~\ref{prop-feasible-flow}~(\ref{prop-feasible-flow-T-mon-inc-cont})) ebenfalls stetig und monoton wachsend.


\begin{lemma}\label{lemma-dreicksungl}
	Für alle Kanten $vw\in E$ gilt in einem zulässigen Fluss 
	$T_{vw}(l_v(\theta)) \geq l_w(\theta)$.
\end{lemma}
\begin{proof}
	Sei ein kürzester $s$-$v$-Pfad $P$ zum Zeitpunkt $\theta$ gegeben.
	Hängt man an $P$ die Kante $vw$ an, erhält man einen $s$-$w$-Pfad, der zur Eintrittszeit $\theta$ die Ankunftszeit $T_{vw}(l_v(\theta))$ liefert.
	Da $l_w(\theta)$ das Minimum über die Ankunftszeit aller $s$-$w$-Pfade ist, gilt die Behauptung.
\end{proof}

\begin{definition}
	Man bezeichne eine Kante $uv\in E$ als \emph{aktiv zum Zeitpunkt $\theta$}, falls $T_{vw}(l_v(\theta)) = l_w(\theta)$, d.h. falls die Kante in einem kürzesten $s$-$w$-Pfad liegt.
\end{definition}

\begin{lemma}
	Für alle Knoten $v\in V$ ist in einem zulässigen Fluss die Funktion $l_v$ monoton wachsend und stetig.
\end{lemma}
\todo{
Mittels Belman-Ford-Algorithmus kann man $l_w$ auch berechnen, indem man die Lösung des folgenden Gleichungssystem löst:

$$ l_w(\theta) = \begin{cases}
	\theta & \text{falls } w=s \\
	\min_{vw\in E} T_{vw}(l_v(\theta)) & \text{sonst}
\end{cases} $$

Warum kann man das? Zykeln?
}

Die folgende Definition entspricht der Definition~2 aus~\cite{Koch2011}.

\begin{definition}
	Man sage, der \emph{Fluss $f$ fließe nur entlang aktiver Kanten}, falls für alle Kanten $uv\in E$ und für fast-alle Zeitpunkte $\theta\in\R$ Zeitpunkt gilt: $$ uv \text{ nicht aktiv zum Zeitpunkt } \theta \implies f_{uv}^+(\l_u(\theta)) = 0.$$
\end{definition}

\begin{lemma}
	Für einen zulässigen Fluss $f$ sind folgende Aussagen äquivalent:
	\begin{enumerate}[(i)]
		\item Der Fluss $f$ fließt nur entlang aktiver Kanten.
		\item Für jede Kante $uv\in E$ und für alle $\theta\in\R$ gilt $F_{uv}^-(T_{uv}(l_u(\theta)) = F_{uv}^-(l_v(\theta))$.
	\end{enumerate}
\end{lemma}
\begin{proof}
	$(i)\Rightarrow (ii)$: Seien $uv\in E$ und $\theta\in\R$ gegeben.
	Die Beziehung $F_{uv}^-(T_{uv}(l_u(\theta))) \geq F_{uv}^-(l_v(\theta))$ gilt bereits wegen der Monotonie von $F_{uv}^-$ und Lemma~\ref{lemma-dreicksungl}.
	Ist $uv$ aktiv zum Zeitpunkt $\theta$, so ist $T_{uv}(l_u(\theta))=l_v(\theta)$ und die Aussage gilt offensichtlich.
	Sonst ist $l_v(\theta) < T_{uv}(l_u(\theta))$. \todo{nur für fast alle theta}
	Sei $\tilde{\theta}:=\sup\{ \tilde{\theta}\in\R \mid l_v(\theta) \geq T_{uv}(l_u(\tilde{\theta})) \}$ der späteste Startzeitpunkt, sodass man spätestens zum Zeitpunkt $l_v(\theta)$ von $s$ über $uv$ zu $v$ gelangt.
	Man beachte, dass $\tilde{\theta}=-\infty$ gilt, falls solch ein Startzeitpunkt nicht existiert.
	Es ist $\tilde{\theta} \leq \theta$, da $T_{uv}\circ l_u$ monoton wachsend ist und für $\tilde{\theta} > \theta$ wäre $T_{uv}(l_u(\tilde{\theta})) > l_v(\theta)$.
	Wegen der Monotonie von $l_v$ und nach Definition von $\tilde{\theta}$ gilt $l_v(\theta')\leq l_v(\theta)< T_{uv}(l_u(\theta'))$ für $\theta'\in (\tilde{\theta}, \theta]$; insbesondere ist die Kante $uv$ in $\theta'$ nicht aktiv und nach Voraussetzung gilt $f_{uv}^+(l_u(\theta')) = 0$ für $\theta'\in (\tilde{\theta}, \theta]$. \todo{fast alle?}
	Also ist $F_{uv}^+(l_u(\theta)) = F_{uv}^+(l_u(\tilde{\theta}))$. Nach Proposition~\ref{prop-feasible-flow}~(\ref{prop-feasible-flow-det-outflow}) ist dann $F_{uv}^-(T_{uv}(l_u(\theta))) = F_{uv}^-(T_{uv}(l_u(\tilde{\theta}))\leq F_{uv}^-(l_v(\theta))$ wegen Montonie von $F_{uv}^-$ und der Definition von $\tilde{\theta}$.
	
	$(ii) \Rightarrow (i)$: Sei eine Kante $uv\in E$, die zu einem Zeitpunkt $\theta\in\R$ nicht aktiv ist, gegeben.
	Es gilt also $l_v(\theta) < T_{uv}(l_u(\theta))$ und wegen der Stetigkeit von $l_v$ und von $T_{uv}\circ l_u$ existiert ein $\varepsilon\in\R_+$ sodass $l_v(\theta + \varepsilon) < T_{uv}(l_u(\theta - \varepsilon))$ gilt.
	Dann ist 
	$$
	0
	\leq
	\int_{l_u(\theta-\varepsilon)}^{l_u(\theta + \varepsilon)}f_{uv}^+(t) dt
	=
	\int_{T_{uv}(l_u(\theta-\varepsilon))}^{T_{uv}(l_u(\theta+\varepsilon))} f_{uv}^-(t) dt
	\leq
	\int_{l_v(\theta + \varepsilon)}^{T_{uv}(l_u(\theta + \varepsilon))} f_{uv}^-(t) dt
	=
	0,
	$$
	wobei die letzte Gleichung aus der Voraussetzung $F_{uv}^-(l_v(\theta+\varepsilon)) = F_{uv}^-(T_{uv}(l_u(\theta + \varepsilon)))$ gefolgert wird.
	Also ist $f_{uv}^+$ in einer Umgebung um $l_u(\theta)$ fast überall gleich $0$.
	
	\todo{Da die Menge aller Zeitpunkte, in denen $uv$ nicht aktiv ist, wegen der Stetigkeit von $l_v$ und $T_{uv}\circ l_u$ eine Vereinigung (abzählbarer) Intervalle ist, genügt es zu zeigen, dass die Bedingung lokal fast immer erfüllt ist.}
\end{proof}


\todo{blablabla Definition FIFO bla}

\begin{definition}
	Für eine Kante $uv\in E$ bezeichne $x_{uv}^+(\theta):= F_{uv}^+(l_u(\theta))$ die Flussmenge, die die Kante $uv$ betreten hat, bevor Partikel, die zur Zeit $\theta$ in $s$ starten, $u$ erreichen können.
	
	Mit $x_{uv}^-(\theta):= F^-_{uv}(l_v(\theta))$ bezeichne man die Flussmenge, die die Kante $uv$ verlassen hat, bevor Partikel, die zur Zeit $\theta$ in $s$ starten, $v$ erreichen können.
	
	Für einen Knoten $v\in V$ sei $b_v(\theta):=\sum_{e\in\delta^+(v)} x_e^+(\theta) - \sum_{e\in\delta^-(v)} x_e^-(\theta)$ die Balance des Knoten $v$ zum Zeitpunkt $\theta$.
\end{definition}


Man bemerke, dass in einem zulässigen Fluss nach Proposition~\ref{prop-feasible-flow}~(\ref{prop-feasible-flow-det-outflow}) gilt $x_{uv}^-(\theta) = F_{uv}^-(l_v(\theta)) \leq F_{uv}^-(T_{uv}(l_u(\theta)))=F_{uv}^+(l_u(\theta)) = x_{uv}^+(\theta)$

\begin{lemma}\label{lemma-balance-0}
	Für einen zulässigen Fluss über Zeit $f$ gilt $b_v(\theta)=0$ für alle Knoten $v\in V\setminus\{ s,t \}$ und alle $\theta\in\R$.
\end{lemma}
\begin{proof}
	Unter Benutzung der Voraussetzung~(\ref{def-feasible-flow-no-flow-at-node}) folgere man für $v\in V\setminus \{ s, t\}, \theta\in\R$:
	$$\sum_{e\in\delta^-(v)} x_e^-(\theta) = \int_{0}^{l_v(\theta)} \sum_{e\in\delta^-(v)} f_e^-(t) dt = \int_{0}^{l_v(\theta)} \sum_{e\in\delta^+(v)} f_e^+(t) dt = \sum_{e\in\delta^+(v)}x_e^+(\theta)$$
\end{proof}

\begin{definition}
	Man sage, ein Fluss über Zeit $f$ \emph{fließe ohne Überholungen}, falls $b_s(\theta) = -b_t(\theta)$ für alle $\theta\in\R$.
\end{definition}

\todo{blabla definition blabla}

\begin{definition}
	Seien ein statischer Fluss, das heißt eine Kantenbewertung $f \in \R^E$, in einem Graphen $G=(V,E)$ und ein Balancevektor $b\in\R^V$ mit $\sum_{v\in V} b_v = 0$ sowie ein Kapazitätsvektor $u\in\R_+^E$ gegeben.
	Der Fluss $f$ heißt \emph{$b$-Fluss}, falls er Flusserhaltung bzgl. $b$ gewährt, d.h. $\forall v\in V: \sum_{e\in\delta^+(v)}f_e - \sum_{e\in\delta^-(v)}f_e = b_v$.
\end{definition}

\todo{blabla}

\newcommand{\newv}{\mathbf{v}}
\begin{lemma}
	Sei ein zulässiger Fluss über Zeit $f$ in einem Graphen $G=(V,E)$ gegeben.
	Der Graph $H$ entstehe aus $G$, indem man jede Kante $uv\in E$ aus $G$ durch einen neuen Knoten $\newv_{uv}$ und zwei Kanten $u\newv_{uv}$ und $\newv_{uv}v$ ersetze.
	Für $\theta\in\R$ sei der statische Fluss $g^\theta$ auf $H$ definiert durch
	$$g^\theta_{u\newv_{uv}} := x_{uv}^+(\theta) \text{ und } g^\theta_{\newv_{uv}v} := x_{uv}^-(\theta) \text{ für alle $uv\in E$}$$
	und der Balancevektor $b^\theta$ auf $H$ sei gegeben durch $b^\theta_v:= b_v(\theta)$ für $v\in V$ und $b^\theta_{\newv_e}:= x_e^-(\theta) - x_e^+(\theta)$ für $e\in E$.
	Dann gelten die folgenden Aussagen:
	
	\begin{enumerate}[(i)]
		\item  Für alle $\theta\in\R$ ist $g^\theta$ ein statischer $b^\theta$-Fluss.
		\item $\forall e\in E, \theta\in \R : x_e^+(\theta) = x_e^-(\theta)\iff \text{$f$ fließt ohne Überholungen}$.
	\end{enumerate}
\end{lemma} 
\begin{proof}
	$(i)$: Sei $\theta\in\R$ gegeben. Dann ist
		$$\sum_{v\in V}b^\theta_v + \sum_{e\in E} b^\theta_{\newv_e} = \sum_{e\in E}  (x_e^+(\theta) - x_e^-(\theta) + x_e^-(\theta) - x_e^+(\theta)) = 0.$$
		Es bleibt zu zeigen, dass $g^\theta$ bezüglich $b^\theta$ Flusserhaltung gewährt.
		Für die Knoten der Form $\newv_{uv}$ gilt dies, da $g^\theta_{\newv_{uv}v} - g^\theta_{u\newv_{uv}} = x_{uv}^-(\theta) - x_{uv}^+(\theta) = b^\theta_{\newv_e}$.
		Für $v\in V$ gilt nach Konstruktion $$b_v^\theta = b_v(\theta)=
		\sum_{e\in\delta^+_G(v)} x_{e}^+(\theta) - \sum_{e\in\delta^-_G(v)} x_{e}^-(\theta) =
	\sum_{e\in\delta_H^+(v)} g_e^\theta - \sum_{e\in\delta^-_H(v)}g_e^\theta
		.$$
	
	$(ii)$: Sei $\theta\in\R$ gegeben und es gelte $x_e^+(\theta) = x_e^-(\theta) $ für alle $\theta\in\R$.
	Dann sind auch alle $b_{\newv_e} = 0$ und nach Lemma~\ref{lemma-balance-0} gilt $b_s(\theta) + b_t(\theta) = 0$, da $g^\theta$ statischer $b^\theta$-Fluss ist.
	Das bedeutet $f$ fließt ohne Überholungen.
	
	Angenommen, $f$ fließe ohne Überholungen.
	Dann gilt $\sum_{e\in E} b_{\newv_e}^\theta = 0$ unter Verwendung von Lemma~\ref{lemma-balance-0} für alle $\theta\in\R$.
	Da $f$ zulässig ist, gilt $x_e^-(\theta)\leq x_e^+(\theta)$ und damit $b_{\newv_e}^\theta\leq 0$ für alle $e\in E$.
	Also sind bereits alle $b_{\newv_e}^\theta = 0$, was die Behauptung zeigt.
\end{proof}

\clearpage          % neue Seite für Literaturverzeichnis

%%%%%%%%%%%%%%%%%%%%%%%%%%%%%%%%%%%%%%%%%%%%%%%%%%%%%%%%%%%%%%%%%%%%%%%%%%%%%%%%
% Literaturverzeichnis
\nocite*  % Nicht zitierte Quellen werden auch ins Literaturverzeichnis aufgenommen
\thispagestyle{empty}
\bibliography{literature}  % Literaturverzeichnis liegt in der Datei literature

%%%%%%%%%%%%%%%%%%%%%%%%%%%%%%%%%%%%%%%%%%%%%%%%%%%%%%%%%%%%%%%%%%%%%%%%%%%%%%%%
%%%%%%%%%%%%%%%%%%%%%%%%%%%%%%%%%%%%%%%%%%%%%%%%%%%%%%%%%%%%%%%%%%%%%%%%%%%%%%%%
% Ende des Dokuments
\end{document}		
