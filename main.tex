\newcommand*{\DocType}{scrartcl}
%\renewcommand*\DocType{article} % Uncomment for screen optimization

\newcommand*\ClassList{scrartcl,article}
\documentclass[\DocType, paper=a4,fontsize=11pt,abstracton,headsepline,notitlepage]{generalclass}


\usepackage[utf8]{inputenc}
\usepackage[automark]{scrpage2}	        % Seiten-Stil für scrartcl
% Mathematische Zeichensätze und Umgebungen
\usepackage{amsfonts, amssymb}	        % Definition einer Liste mathematischer Fontbefehle und Symbole
\usepackage[intlimits,sumlimits]{amsmath} % Integral-/Summationsgrenzen über/unter Zeichen
\usepackage{mathabx}
% mathematische Verbesserungen
\usepackage{amsthm}	                    % spezielle theorem Stile
\usepackage{aliascnt} 
\usepackage{array}		                % erweiterte Tabellen
% Schriftzeichen, Format
\usepackage{latexsym}		            % Latex-Symbole
\usepackage[english, german, ngerman]{babel} % Mehrsprachenumgebung
% Layout
\usepackage{geometry}                   % Seitenränder
\usepackage{xcolor}                     % Farben
\usepackage{bbm}
% Tabellen und Listen
\usepackage{float}		                % Gleitobjekte 
\usepackage[flushright]{paralist}       % Bessere Behandlung der Auflistungen

% Bilder
\usepackage[final]{graphicx}            % Graphiken einbinden

\usepackage{caption}                    % Beschriftungen
\usepackage{subcaption}                 % Beschriftungen für Unterteilung

% Interaktive Referenzen, und PDF-Keys
\usepackage{xr-hyper}  
\usepackage[pagebackref,pdftex]{hyperref} % Rückreferenz im Literaturverzeichnis, Treiber für ps zu pdf ; für direkt nach pdf: pdftex

%%%%%%%%%%%%%%%%%%%%%%%%%%%%%%%%%%%%%%%%%%%%%%%%%%%%%%%%%%%%%%%%%%%%%%%%%%%%%%%%
% Zusammenfassung einiger nützlicher Pakete und Befehle
%-------------------------------------------------------------------------------
% Kopf-Zeilen
%-------------------------------------------------------------------------------

\pagestyle{scrheadings}		     % Kopfzeilen nach scr-Standard		
\ifx\chapter\undefined 		     % falls Kapitel nicht definiert sind
  \automark[subsection]{section} % Kopf- und Fusszeilen setzen
\else				             % Kapitel sind definiert
  \automark[section]{chapter}	 % Kopf- und Fusszeilen setzen
\fi

%-------------------------------------------------------------------------------
%   Maske für Überschrift 
%-------------------------------------------------------------------------------
% Belegung der notwendigen Kommandos für die Titelseite
\newcommand{\autor}{Markl, Michael} 		% bearbeitender Student
\newcommand{\veranstaltung}{Seminar zur Optimierung und Spieltheorie} 	% Titel des ganzen Seminars
\newcommand{\uni}{Institut für Mathematik der Universität Augsburg} % Universit\"at
\newcommand{\matrikelnummer}{1474802}
\newcommand{\lehrstuhl}{Diskrete Mathematik, Optimierung und Operations Research} % Lehrstuhl
\newcommand{\semester}{Sommersemester 2019}	% Winter-/Sommersemester mit Jahr
\newcommand{\datum}{27.06.2019} 			% Datumsangabe
\newcommand{\thema}{Nash Gleichgewichte in Dynamischen Flüssen}  		% Titel der Seminararbeit

\newcommand{\ownline}{\vspace{.7em}\hrule\vspace{.7em}} % horizontale Linie mit Abstand

\newcommand{\seminarkopf}{
	% Befehl zum Erzeugen der Titelseite 
 \textsc{\autor}  \hfill{\datum} \\ 
\textbf{\veranstaltung} \\ 
\uni \\ 
\lehrstuhl \\
\semester \hfill{Matrikelnummer: \matrikelnummer}
\ownline 

\begin{center}
{\LARGE \textbf{\thema}}
\end{center}

\ownline
}			% Befehle und Pakete für Titelseite


\DeclareMathOperator{\e}{ex}
\DeclareMathOperator{\ma}{mate}
\DeclareMathOperator{\Ex}{Ex}

%-------------------------------------------------------------------------------
%   Befehle für Nummerierung der Ergebnisse
%   fortlaufend innerhalb eines Abschnittes
%-------------------------------------------------------------------------------
\theoremstyle{plain}            % normaler Stil
\newtheorem{theorem}{Theorem}
\numberwithin{theorem}{section}
% Lemma
\newaliascnt{lemma}{theorem}
\newtheorem{lemma}[lemma]{Lemma}
\aliascntresetthe{lemma}
% Satz
\newaliascnt{satz}{theorem}
\newtheorem{satz}[satz]{Satz}
\aliascntresetthe{satz}
% Korollar
\newaliascnt{corollary}{theorem}
\newtheorem{corollary}[corollary]{Korollar}
\aliascntresetthe{corollary}
% Proposition
\newaliascnt{proposition}{theorem}
\newtheorem{proposition}[proposition]{Proposition}
\aliascntresetthe{proposition}
%-------------------------------------------------------------------------------
\theoremstyle{definition}	% Definitionsstil
% Definition
\newaliascnt{definition}{theorem}
\newtheorem{definition}[definition]{Definition}
\aliascntresetthe{definition}
% Beispiel
\newaliascnt{example}{theorem}
\newtheorem{example}[example]{Beispiel}
\aliascntresetthe{example}
% Problem
\newaliascnt{problem}{theorem}
\newtheorem{problem}[problem]{Problem}
\aliascntresetthe{problem}
% Algorithmus
\newaliascnt{algorithmus}{theorem}
\newtheorem{algorithmus}[algorithmus]{Algorithmus}
\aliascntresetthe{algorithmus}
%-------------------------------------------------------------------------------
\theoremstyle{remark}		% Bemerkungsstil
% Bemerkung
\newaliascnt{remark}{theorem}
\newtheorem{remark}[remark]{Bemerkung}
\aliascntresetthe{remark}
% Vermutung
\newaliascnt{conjecture}{theorem}
\newtheorem{conjecture}[conjecture]{Vermutung}
\aliascntresetthe{conjecture}
% Notation
\newaliascnt{notation}{theorem}
\newtheorem{notation}[notation]{Notation}
\aliascntresetthe{notation}

%-------------------------------------------------------------------------------
% automatische Referenzen mit interaktiven Text
%-------------------------------------------------------------------------------

% Texte
\renewcommand{\theoremautorefname}{Theorem}
\newcommand{\lemmaautorefname}{Lemma}
\newcommand{\satzautorefname}{Satz}
\newcommand{\korollarautorefname}{Korollar}
\newcommand{\propositionautorefname}{Proposition}

\newcommand{\definitionautorefname}{Definition}
\newcommand{\beispielautorefname}{Beispiel}
\newcommand{\problemautorefname}{Problem}
\newcommand{\algorithmusautorefname}{Algorithmus}

\newcommand{\bemerkungautorefname}{Bemerkung}
\newcommand{\vermutungautorefname}{Vermutung}
\newcommand{\notationautorefname}{Notation}

%-------------------------------------------------------------------------------
% Nummerierung der Gleichungen innerhalb der obersten Ebene
%-------------------------------------------------------------------------------
\ifx\chapter\undefined 			% Kapitel sind definiert
  \numberwithin{equation}{section}	% Gleichungsnummern in Section
\else					% Kapitel sind nicht definiert
  \numberwithin{equation}{chapter}	% Gleichungsnummern in Kapiteln
\fi





\makeatletter
\newcommand{\overleftsmallarrow}{\mathpalette{\overarrowsmall@\leftarrowfill@}}
\newcommand{\overrightsmallarrow}{\mathpalette{\overarrowsmall@\rightarrowfill@}}
\newcommand{\overleftrightsmallarrow}{\mathpalette{\overarrowsmall@\leftrightarrowfill@}}
\newcommand{\overarrowsmall@}[3]{%
	\vbox{%
		\ialign{%
			##\crcr
			#1{\smaller@style{#2}}\crcr
			\noalign{\nointerlineskip\vskip0.4pt}%
			$\m@th\hfil#2#3\hfil$\crcr
		}%
	}%
}
\def\smaller@style#1{%
	\ifx#1\displaystyle\scriptstyle\else
	\ifx#1\textstyle\scriptstyle\else
	\scriptscriptstyle
	\fi
	\fi
}
\makeatother
			% Mathematische Befehle und Pakete

% Literatur-Bibliothek
\bibliographystyle{alphadin}               % deutscher Bibliotheksstil

% Erweiterte Einstellungen zu hyperref

\hypersetup{
        breaklinks=true,        % zu lange Links unterbrechen
        colorlinks=true,        % Färben von Referenzen
        citecolor=black,        % Farbe der Zitate
        linkcolor=black,        % Farbe der Links
        extension=pdf,          % Externe Dokumente können eingebunden werden.
        ngerman,		
	pdfview=FitH,
	pdfstartview=FitH,		
	bookmarksnumbered=true,     % Anzeige der Abschnittsnummern	% pdf-Titel
	pdfauthor={\autor}          % pdf-Autor
}

% Namen für Referenzen 

\newcommand{\ownautorefnames}{
  \renewcommand{\sectionautorefname}{Kapitel}
  \renewcommand{\subsectionautorefname}{Unterkapitel}
  \renewcommand{\subsubsectionautorefname}{\subsectionautorefname}
  \renewcommand{\appendixautorefname}{Anhang}
  \renewcommand{\figureautorefname}{Abbildung}
}

% Rückreferenzentext zur Literatur
\def\bibandname{und}%
\renewcommand*{\backref}[1]{}
\renewcommand*{\backrefalt}[4]{%
\ifcase #1 %
 (Nicht zitiert, also Ergänzungsliteratur.)%
\or
 (Zitiert auf Seite #2.)%
\else
 (Zitiert auf den Seiten #2.)%
\fi
}
\renewcommand{\backreftwosep}{ und~} % seperate 2 pages
\renewcommand{\backreflastsep}{ und~} % seperate last of longer 

			% Befehle und Pakete für Referenzen

\geometry{a4paper, top=30mm, bottom=30mm, left=30mm, right=30mm}

\IfClass{article}{ % Optimize for screen
	\geometry{papersize={160mm,161.5mm},margin=5mm} 
}

\addtolength{\footskip}{-0.5cm}          % Seitenzahlen höher setzen
\renewcommand{\descriptionlabel}[1]{\hspace{\labelsep}\textit{#1}}

\makeatletter
\newcommand{\customlabel}[2]{%
	\protected@write \@auxout {}{\string \newlabel {#1}{{#2}{\thepage}{#2}{#1}{}} }%
	\hypertarget{#1}{#2}
}
\makeatother
\newcommand{\todo}[1]{{\color{red}#1}}

\numberwithin{figure}{section}	% Abbildungsnummern in Section

\newcommand{\R}{\mathbb{R}}
\newcommand{\Q}{\mathbb{Q}}
\newcommand{\Z}{\mathbb{Z}}
\newcommand{\N}{\mathbb{N}}
\newcommand*\diff{\mathop{}\!\mathrm{d}}

%%%%%%%%%%%%%%%%%%%%%%%%%%%%%%%%%%%%%%%%%%%%%%%%%%%%%%%%%%%%%%%%%%%%%%%%%%%%%%%%
% Start des Dokuments
\begin{document}

\ownautorefnames		% Änderung einiger automatischen Texte von hyperref (wie in referenz.tex definiert)
%\parindent0em 			% kein Einzug nach einer Leerzeile

%%%%%%%%%%%%%%%%%%%%%%%%%%%%%%%%%%%%%%%%%%%%%%%%%%%%%%%%%%%%%%%%%%%%%%%%%%%%%%%%
% Titelseite
\thispagestyle{empty}   % leerer Seitenstil, also keine Seitennummer
\begin{titlepage}
\seminarkopf            % Titelblatt (wie in kopf.tex definiert)
\begin{abstract}
Diese Arbeit gibt eine formale Definition von dynamischen Flüssen mit Warteschlangenmodell und darauf aufbauend eine Charakterisierung von Nash Gleichgewichten in diesem Kontext.
Dabei werden Partikel des Flusses als gleichwertige Spieler betrachtet, die von Zeit zu Zeit kontinuierlich bei einer Quelle entstehen und per Selfish Routing versuchen ihre Ankunftszeit bei einer Senke zu minimieren.
Eine spezielle Klasse von statischen Flüssen, die schmalen Flüsse mit Zurücksetzen, kann verwendet werden, um Nash Flüsse mit Zeithorizont zu erweitern.
\end{abstract}
\end{titlepage}

%%%%%%%%%%%%%%%%%%%%%%%%%%%%%%%%%%%%%%%%%%%%%%%%%%%%%%%%%%%%%%%%%%%%%%%%%%%%%%%%
\thispagestyle{empty}
\tableofcontents        % Inhaltsverzeichnis
%\listoffigures         % Abbildungsverzeichnis (eventuell einfügen)
%\listoftables          % Tabellenverzeichnis (eventuell einfügen)
\setcounter{page}{0}    % Eigentlicher Inhalt beginnt auf Seite 1
\clearpage              % neue Seite für eigentlichen Inhalt

%%%%%%%%%%%%%%%%%%%%%%%%%%%%%%%%%%%%%%%%%%%%%%%%%%%%%%%%%%%%%%%%%%%%%%%%%%%%%%%%
% Eigentlicher Inhalt der Seminararbeit; die einzelnen Teile werden hier (aus Gründen der Übersichtlichkeit) über \input{file} eingebunden

\section{Einführung}\label{introduction}

\newcommand{\R}{\mathbb{R}}

\begin{definition}[strategisches Spiel]
	Ein \emph{strategisches Spiel} $\mathcal{G}$ ist ein Tupel $(P, X, \theta)$, wobei $P$ eine Menge von Spielern, $X$ die Menge der Strategien  und $\theta=(\theta_p)_{p_\in P}$ die Familie der Auszahlungsfunktionen der Spieler ist, wobei $\theta_p$ eine Funktion $X^P\to\R$ für jeden Spieler $p\in P$ ist.
\end{definition}

Man beachte, dass in dieser Definition eines strategischen Spiels die Spieler eine gemeinsame Strategiemenge haben.

\begin{definition}[statisches Flussnetzwerk]
	Ein \emph{statisches Flussnetzwerk} $\mathcal{N}:=(G,s,t,\mu)$ ist ein Graph $G:=(V,E)$ mit Knoten $V$ und Kanten $E$, einer \emph{Quelle} $s\in V$, einer \emph{Senke} $t\in V$ zusammen mit einem \emph{$d$-wertigen statischen Fluss} $\mu: \mathcal{P} \to \R$, wobei $\mathcal{P}$ die Menge aller einfachen $s$-$t$-Pfade in G ist und $d=\sum_{p\in\mathcal{P}}\mu(p)$ gilt.
\end{definition}

\begin{definition}[statisches Routenplanungsspiel]
	Sei ein Graph $G:=(V,E)$ mit Knoten $V$ und Kanten $E$, einer Quelle $s\in V$, einer Senke $t\in V$ und einem Zufluss $d\in\R_+$ gegeben.
	Ein \emph{statisches Routenplanungsspiel} ist ein strategisches Spiel mit  Spielermenge $[0,d]\subseteq\R$, wobei ein Spieler als \emph{Flusspartikel} bezeichnet wird, und mit Strategiemenge $\mathcal{P}$ aller $s$-$t$-Pfade in $G$.
	Die Auszahlungsfunktion $\theta_x$ gibt die Kosten des Partikels $x$ in Abhängigkeit des Flusses an.
	
	Der Fluss $\mu$ heißt Nashfluss, falls $l_P(\mu)=min_{P'\in\mathcal{P}}l_{P'}(\mu)$ für alle $P\in\mathcal{P}$ mit $\mu_P > 0$ gilt.
\end{definition}

\section{Dynamische Flüsse mit Zeithorizont}


Betrachte Zeitraum $[0,T]$ mit $T\in\R_+$.
$d$ Spieler erscheinen an Quelle $s$ über Zeitraum von $0$ bis $T$

\begin{definition}[Netzwerk]
	Ein \emph{Netzwerk} ist ein gerichteter Graph $G=(V,E)$ mit endlicher Knotenmenge $V$ und Kantenmenge $E\subseteq V\times V$, einer \emph{Quelle} $s\in V$ und einer Senke $t\in V$.
	Jeder Kante $e\in E$ werden eine Kapazität $u_e > 0$ und eine Verzögerungszeit $\tau_e\geq 0$ zugeordnet, sodass alle Zykel $T$ eine positive Gesamtverzögerung $\sum_{e\in T}\tau_e > 0$ haben.
\end{definition}

Für ein Netzwerk $\mathcal{N}$ bezeichne $\mathcal{P}$ die Menge aller $s$-$t$-Pfade.

\begin{definition}[Fluss über Zeit]
	Ein Fluss über Zeit $f=(f^+, f^-)$ ist ein Paar zweier über die Kanten $E$ indizierten Familien von Lebesgue-integrierbaren Funktionen $f^+_e,f^-_e: \R \to \R_{\geq 0}$ an $e$ für alle $e\in E$.
	
	Dabei bezeichnen $f_e^+(\theta)$ die \emph{Einflussrate an $e$ zum Zeitpunkt $\theta\in\R$} und $f_e^-(\theta)$ die \emph{Ausflussrate aus $e$ zum Zeitpunkt $\theta\in\R$} für $e\in E$.
	
	Der (kumulative) \emph{Einfluss bzw. Ausfluss an einer Kante $e$ bis zum Zeitpunkt $\theta$} sei definiert durch $F^+_e(\theta):=\int_{[0,\theta)} f^+_e d\lambda$ bzw. $F^-_e(\theta):=\int_{[0,\theta)} f^-_e d\lambda$.
	
	Die \emph{(Länge der) Warteschlange an Kante $e$ zum Zeitpunkt $\theta\in\R$} sei definiert durch $z_e(\theta):= F_e^+(\theta) - F_e^-(\theta + \tau_e)$ und die \emph{Wartezeit an Kante $e$ zum Zeitpunkt $\theta\in\R$} sei definiert durch $q_e(\theta) = z_e / u_e$.
	
	Man beschreibe den \emph{Austrittszeitpunkt einer Kante $e$ bei Eintrittszeitpunkt $\theta$}, in dem ein Partikel eine Kante verlässt, die es zum Zeitpunkt $\theta$ betreten hat, als $T_e(\theta):=\theta + q_e(\theta) + \tau_e$.
\end{definition}

\begin{definition}[Zulässiger Fluss über Zeit]
	Sei ein Fluss über Zeit $f=(f^+, f^-)$ gegeben. $f$ heißt zulässig, falls
	\begin{enumerate}[(I)]
		\item keine Ausflussrate die Kapazität übersteigt, d.h. $\forall e\in E, \theta\in\R: f_e^-\leq u_e$, und
		\item\label{weak} Fluss eine Kante nur verlässt, falls er die Kante zuvor betreten hat, d.h. $\forall e\in E, \theta\in\R: F_e^+(\theta) \geq F_e^-(\theta + \tau_e)$, und
		\item\label{def-feasible-flow-no-flow-at-node} Fluss direkt nach der Ankunft an einem Knoten der nächsten Kante zugeordnet wird, d.h. $\forall v\in V\setminus \{ s, t \}, \theta\in\R: \sum_{e\in\delta^-(v)}f^-_e(\theta) = \sum_{e\in\delta^+(v)} f_e^+(\theta)$, und
		\item\label{queue-with-capacity} nicht-leere Warteschlangen mit der Kapazität der Kante abgebaut werden, d.h. $\forall e\in E, \theta\in\R: q_e(\theta) > 0 \implies f_e^-(\theta + \tau_e) = u_e$.
		\todo{
		 $F_e^+(\theta) = F_e^-(\theta + \tau_e + q_e(\theta))$ für alle $e\in E, \theta\in\R$, ist das das gleiche (f.ü.)?}
	\end{enumerate}
\end{definition}

\todo{source and sink nodes: Kein Fluss verschwindet bei Quelle, kein Fluss kommt hinzu bei Sinke}

\todo{FIFO Interpretation}

\begin{proposition}\label{prop-feasible-flow}
	Sei $e\in E$ eine Kante und $f$ ein zulässiger Fluss. Dann gilt:
	\begin{enumerate}[(i)]
		\item\label{prop-feasible-flow-T-mon-inc-cont} Die Funktion $\theta \mapsto \theta + q_e(\theta)$ ist monoton wachsend und stetig.
		\item\label{prop-feasible-flow-det-outflow} Für alle $\theta\in\R$ ist $F_e^+(\theta) = F_e^-(\theta+\tau_e+q_e(\theta))$.\todo{use $T_e(\theta)$ here}
		\item Für zwei Zeitpunkte $0\leq \theta_1 < \theta_2$ mit $\int_{\theta_1}^{\theta_2} f^+_e d\lambda = 0$ und $q_e(\theta_2)>0$ ist $\theta_1 + q_e(\theta_1) = \theta_2 + z_e(\theta_2)$.
	\end{enumerate}
\end{proposition}
\begin{proof}
	Für die Monotonie in~(\ref{prop-feasible-flow-T-mon-inc-cont}) seien $\theta_1 \leq \theta_2$ gegeben.
	Mit der Monotonie von $F_e^+$ und mit $F_e^-(\theta_1 + \tau_e) = F_e^-(\theta_2+\tau_e) - \int_{\theta_1+\tau}^{\theta_2+\tau} f_e^-(t)dt\leq F_e^-(\theta_2 + \tau_e) + (\theta_2 - \theta_1)u_e$ gilt: 
	$$
		\theta_1 + q_e(\theta_1)
		= \theta_1 + \frac{F_e^+(\theta_1) - F_e^-(\theta_1 + \tau_e)}{u_e}\\
		\leq \theta_2 + \frac{F_e^+(\theta_1) - F_e^-(\theta_2+\tau_e)}{u_e} \leq \theta_2 + q_e(\theta_2).
	$$
	Die Stetigkeit folgt aus der Stetigkeit von $F_e^+$ und $F_e^-$.
	
	Zu Aussage (ii): Nach Voraussetzung~(\ref{queue-with-capacity}) ist $\int_{\theta + \tau_e}^{\theta + \tau_e + q_e(\theta)} f_e^-(t) dt = \int_{\theta}^{\theta + q_e(\theta)}f_e^-(t + \tau_e) dt = q_e(\theta)  u_e = z_e(\theta)$ (\todo{für $\theta'$ ist $q_e(\theta')> 0$}). Damit ist $F_e^-(\theta + \tau_e + q_e(\theta)) = F_e^-(\theta+\tau_e) + \int_{\theta+\tau_e}^{\theta+\tau_e+q_e(\theta)}f_e^-(t)dt = F_e^+(\theta)$.
	
	Zu Aussage (iii): Für alle $\theta'\in [\theta_1, \theta_2]$ gilt $F_e^+(\theta_1) = F_e^+(\theta') = F_e^-(\theta_2)$.
	Also ist $F_e^+(\theta') - F_e^-(\theta' + \tau_e) \geq F_e^+(\theta_2)-F_e^-(\theta_2 + \tau_e) = z_e(\theta_2) > 0$ und es gilt $f_e^-(\theta' + \tau_e)=u_e$.
	Für die Warteschlangendifferenz gilt dann: 
	$z_e(\theta_1)-z_e(\theta_2)=-F^-(\theta_1 + \tau_e) + F^-(\theta_2 + \tau_e) = \int_{\theta_1 + \tau_e}^{\theta_2 + \tau_e} f^-_e(t) dt = (\theta_2 - \theta_1)u_e$.
	Man folgere $q_e(\theta_1) - q_e(\theta_2) = \theta_2 - \theta_1$.
\end{proof}

Insbesondere ist also die Kantenaustrittszeit $T_e$ monoton wachsend und stetig.

\todo{Stetigkeit von irgendwas um die "Wohldefiniertheit" dieser komischen Definition  von Nash flows zu zeigen}

\section{Charakterisierung von Nash-Flüssen über Zeit}

\begin{definition}
	Für einen Fluss $f$ und einen Pfad $P=(e_1,\dots,e_k)$ definiere $l^P(\theta):=T_{e_k}\circ\dots\circ T_{e_1}(\theta)$ den Zeitpunkt, an dem ein Partikel den Endknoten des Pfads erreicht, falls er den Pfad zum Zeitpunkt $\theta$ betritt.
	
	Für einen Knoten $w\in V$ beschreibe $\mathcal{P}_w$ die Menge aller $s$-$w$-Pfade.
	Dann ist die früheste Ankunft eines Partikels, das zum Zeitpunkt $\theta$ bei $s$ startet, gegeben durch $l_w(\theta):=\min_{P\in\mathcal{P}_w}l^P(\theta)$.
	Ein Pfad $P\in \mathcal{P}_w$ heißt \emph{kürzester $s$-$w$-Pfad zum Zeitpunkt $\theta$},  falls $l_w(\theta)=l^P(\theta)$.
\end{definition}

Für einen Knoten $v \in V$ und einen zulässigen Fluss ist die Funktion $l_v$ als Minimum von Kompositionen stetiger und monoton wachsender Funktionen $T_e$ (Proposition~\ref{prop-feasible-flow}~(\ref{prop-feasible-flow-T-mon-inc-cont})) ebenfalls stetig und monoton wachsend.


\begin{lemma}\label{lemma-dreicksungl}
	Für alle Kanten $vw\in E$ gilt in einem zulässigen Fluss 
	$T_{vw}(l_v(\theta)) \geq l_w(\theta)$.
\end{lemma}
\begin{proof}
	Sei ein kürzester $s$-$v$-Pfad $P$ zum Zeitpunkt $\theta$ gegeben.
	Hängt man an $P$ die Kante $vw$ an, erhält man einen $s$-$w$-Pfad, der zur Eintrittszeit $\theta$ die Ankunftszeit $T_{vw}(l_v(\theta))$ liefert.
	Da $l_w(\theta)$ das Minimum über die Ankunftszeit aller $s$-$w$-Pfade ist, gilt die Behauptung.
\end{proof}

\begin{definition}
	Man bezeichne eine Kante $uv\in E$ als \emph{aktiv zum Zeitpunkt $\theta$}, falls $T_{vw}(l_v(\theta)) = l_w(\theta)$, d.h. falls die Kante in einem kürzesten $s$-$w$-Pfad liegt.
\end{definition}

\begin{lemma}
	Für alle Knoten $v\in V$ ist in einem zulässigen Fluss die Funktion $l_v$ monoton wachsend und stetig.
\end{lemma}
\todo{
Mittels Belman-Ford-Algorithmus kann man $l_w$ auch berechnen, indem man die Lösung des folgenden Gleichungssystem löst:

$$ l_w(\theta) = \begin{cases}
	\theta & \text{falls } w=s \\
	\min_{vw\in E} T_{vw}(l_v(\theta)) & \text{sonst}
\end{cases} $$

Warum kann man das? Zykeln?
}

Die folgende Definition entspricht der Definition~2 aus~\cite{Koch2011}.

\begin{definition}
	Man sage, der \emph{Fluss $f$ fließe nur entlang aktiver Kanten}, falls für alle Kanten $uv\in E$ und für fast-alle Zeitpunkte $\theta\in\R$ Zeitpunkt gilt: $$ uv \text{ nicht aktiv zum Zeitpunkt } \theta \implies f_{uv}^+(\l_u(\theta)) = 0.$$
\end{definition}

\begin{lemma}
	Für einen zulässigen Fluss $f$ sind folgende Aussagen äquivalent:
	\begin{enumerate}[(i)]
		\item Der Fluss $f$ fließt nur entlang aktiver Kanten.
		\item Für jede Kante $uv\in E$ und für alle $\theta\in\R$ gilt $F_{uv}^-(T_{uv}(l_u(\theta)) = F_{uv}^-(l_v(\theta))$.
	\end{enumerate}
\end{lemma}
\begin{proof}
	$(i)\Rightarrow (ii)$: Seien $uv\in E$ und $\theta\in\R$ gegeben.
	Die Beziehung $F_{uv}^-(T_{uv}(l_u(\theta))) \geq F_{uv}^-(l_v(\theta))$ gilt bereits wegen der Monotonie von $F_{uv}^-$ und Lemma~\ref{lemma-dreicksungl}.
	Ist $uv$ aktiv zum Zeitpunkt $\theta$, so ist $T_{uv}(l_u(\theta))=l_v(\theta)$ und die Aussage gilt offensichtlich.
	Sonst ist $l_v(\theta) < T_{uv}(l_u(\theta))$. \todo{nur für fast alle theta}
	Sei $\tilde{\theta}:=\sup\{ \tilde{\theta}\in\R \mid l_v(\theta) \geq T_{uv}(l_u(\tilde{\theta})) \}$ der späteste Startzeitpunkt, sodass man spätestens zum Zeitpunkt $l_v(\theta)$ von $s$ über $uv$ zu $v$ gelangt.
	Man beachte, dass $\tilde{\theta}=-\infty$ gilt, falls solch ein Startzeitpunkt nicht existiert.
	Es ist $\tilde{\theta} \leq \theta$, da $T_{uv}\circ l_u$ monoton wachsend ist und für $\tilde{\theta} > \theta$ wäre $T_{uv}(l_u(\tilde{\theta})) > l_v(\theta)$.
	Wegen der Monotonie von $l_v$ und nach Definition von $\tilde{\theta}$ gilt $l_v(\theta')\leq l_v(\theta)< T_{uv}(l_u(\theta'))$ für $\theta'\in (\tilde{\theta}, \theta]$; insbesondere ist die Kante $uv$ in $\theta'$ nicht aktiv und nach Voraussetzung gilt $f_{uv}^+(l_u(\theta')) = 0$ für $\theta'\in (\tilde{\theta}, \theta]$. \todo{fast alle?}
	Also ist $F_{uv}^+(l_u(\theta)) = F_{uv}^+(l_u(\tilde{\theta}))$. Nach Proposition~\ref{prop-feasible-flow}~(\ref{prop-feasible-flow-det-outflow}) ist dann $F_{uv}^-(T_{uv}(l_u(\theta))) = F_{uv}^-(T_{uv}(l_u(\tilde{\theta}))\leq F_{uv}^-(l_v(\theta))$ wegen Montonie von $F_{uv}^-$ und der Definition von $\tilde{\theta}$.
	
	$(ii) \Rightarrow (i)$: Sei eine Kante $uv\in E$, die zu einem Zeitpunkt $\theta\in\R$ nicht aktiv ist, gegeben.
	Es gilt also $l_v(\theta) < T_{uv}(l_u(\theta))$ und wegen der Stetigkeit von $l_v$ und von $T_{uv}\circ l_u$ existiert ein $\varepsilon\in\R_+$ sodass $l_v(\theta + \varepsilon) < T_{uv}(l_u(\theta - \varepsilon))$ gilt.
	Dann ist 
	$$
	0
	\leq
	\int_{l_u(\theta-\varepsilon)}^{l_u(\theta + \varepsilon)}f_{uv}^+(t) dt
	=
	\int_{T_{uv}(l_u(\theta-\varepsilon))}^{T_{uv}(l_u(\theta+\varepsilon))} f_{uv}^-(t) dt
	\leq
	\int_{l_v(\theta + \varepsilon)}^{T_{uv}(l_u(\theta + \varepsilon))} f_{uv}^-(t) dt
	=
	0,
	$$
	wobei die letzte Gleichung aus der Voraussetzung $F_{uv}^-(l_v(\theta+\varepsilon)) = F_{uv}^-(T_{uv}(l_u(\theta + \varepsilon)))$ gefolgert wird.
	Also ist $f_{uv}^+$ in einer Umgebung um $l_u(\theta)$ fast überall gleich $0$.
	
	\todo{Da die Menge aller Zeitpunkte, in denen $uv$ nicht aktiv ist, wegen der Stetigkeit von $l_v$ und $T_{uv}\circ l_u$ eine Vereinigung (abzählbarer) Intervalle ist, genügt es zu zeigen, dass die Bedingung lokal fast immer erfüllt ist.}
\end{proof}


\todo{blablabla Definition FIFO bla}

\begin{definition}
	Für eine Kante $uv\in E$ bezeichne $x_{uv}^+(\theta):= F_{uv}^+(l_u(\theta))$ die Flussmenge, die die Kante $uv$ betreten hat, bevor Partikel, die zur Zeit $\theta$ in $s$ starten, $u$ erreichen können.
	
	Mit $x_{uv}^-(\theta):= F^-_{uv}(l_v(\theta))$ bezeichne man die Flussmenge, die die Kante $uv$ verlassen hat, bevor Partikel, die zur Zeit $\theta$ in $s$ starten, $v$ erreichen können.
	
	Für einen Knoten $v\in V$ sei $b_v(\theta):=\sum_{e\in\delta^+(v)} x_e^+(\theta) - \sum_{e\in\delta^-(v)} x_e^-(\theta)$ die Balance des Knoten $v$ zum Zeitpunkt $\theta$.
\end{definition}


Man bemerke, dass in einem zulässigen Fluss nach Proposition~\ref{prop-feasible-flow}~(\ref{prop-feasible-flow-det-outflow}) gilt $x_{uv}^-(\theta) = F_{uv}^-(l_v(\theta)) \leq F_{uv}^-(T_{uv}(l_u(\theta)))=F_{uv}^+(l_u(\theta)) = x_{uv}^+(\theta)$

\begin{lemma}\label{lemma-balance-0}
	Für einen zulässigen Fluss über Zeit $f$ gilt $b_v(\theta)=0$ für alle Knoten $v\in V\setminus\{ s,t \}$ und alle $\theta\in\R$.
\end{lemma}
\begin{proof}
	Unter Benutzung der Voraussetzung~(\ref{def-feasible-flow-no-flow-at-node}) folgere man für $v\in V\setminus \{ s, t\}, \theta\in\R$:
	$$\sum_{e\in\delta^-(v)} x_e^-(\theta) = \int_{0}^{l_v(\theta)} \sum_{e\in\delta^-(v)} f_e^-(t) dt = \int_{0}^{l_v(\theta)} \sum_{e\in\delta^+(v)} f_e^+(t) dt = \sum_{e\in\delta^+(v)}x_e^+(\theta)$$
\end{proof}

\begin{definition}
	Man sage, ein Fluss über Zeit $f$ \emph{fließe ohne Überholungen}, falls $b_s(\theta) = -b_t(\theta)$ für alle $\theta\in\R$.
\end{definition}

\todo{blabla definition blabla}

\begin{definition}
	Seien ein statischer Fluss, das heißt eine Kantenbewertung $f \in \R^E$, in einem Graphen $G=(V,E)$ und ein Balancevektor $b\in\R^V$ mit $\sum_{v\in V} b_v = 0$ sowie ein Kapazitätsvektor $u\in\R_+^E$ gegeben.
	Der Fluss $f$ heißt \emph{$b$-Fluss}, falls er Flusserhaltung bzgl. $b$ gewährt, d.h. $\forall v\in V: \sum_{e\in\delta^+(v)}f_e - \sum_{e\in\delta^-(v)}f_e = b_v$.
\end{definition}

\todo{blabla}

\newcommand{\newv}{\mathbf{v}}
\begin{lemma}
	Sei ein zulässiger Fluss über Zeit $f$ in einem Graphen $G=(V,E)$ gegeben.
	Der Graph $H$ entstehe aus $G$, indem man jede Kante $uv\in E$ aus $G$ durch einen neuen Knoten $\newv_{uv}$ und zwei Kanten $u\newv_{uv}$ und $\newv_{uv}v$ ersetze.
	Für $\theta\in\R$ sei der statische Fluss $g^\theta$ auf $H$ definiert durch
	$$g^\theta_{u\newv_{uv}} := x_{uv}^+(\theta) \text{ und } g^\theta_{\newv_{uv}v} := x_{uv}^-(\theta) \text{ für alle $uv\in E$}$$
	und der Balancevektor $b^\theta$ auf $H$ sei gegeben durch $b^\theta_v:= b_v(\theta)$ für $v\in V$ und $b^\theta_{\newv_e}:= x_e^-(\theta) - x_e^+(\theta)$ für $e\in E$.
	Dann gelten die folgenden Aussagen:
	
	\begin{enumerate}[(i)]
		\item  Für alle $\theta\in\R$ ist $g^\theta$ ein statischer $b^\theta$-Fluss.
		\item $\forall e\in E, \theta\in \R : x_e^+(\theta) = x_e^-(\theta)\iff \text{$f$ fließt ohne Überholungen}$.
	\end{enumerate}
\end{lemma} 
\begin{proof}
	$(i)$: Sei $\theta\in\R$ gegeben. Dann ist
		$$\sum_{v\in V}b^\theta_v + \sum_{e\in E} b^\theta_{\newv_e} = \sum_{e\in E}  (x_e^+(\theta) - x_e^-(\theta) + x_e^-(\theta) - x_e^+(\theta)) = 0.$$
		Es bleibt zu zeigen, dass $g^\theta$ bezüglich $b^\theta$ Flusserhaltung gewährt.
		Für die Knoten der Form $\newv_{uv}$ gilt dies, da $g^\theta_{\newv_{uv}v} - g^\theta_{u\newv_{uv}} = x_{uv}^-(\theta) - x_{uv}^+(\theta) = b^\theta_{\newv_e}$.
		Für $v\in V$ gilt nach Konstruktion $$b_v^\theta = b_v(\theta)=
		\sum_{e\in\delta^+_G(v)} x_{e}^+(\theta) - \sum_{e\in\delta^-_G(v)} x_{e}^-(\theta) =
	\sum_{e\in\delta_H^+(v)} g_e^\theta - \sum_{e\in\delta^-_H(v)}g_e^\theta
		.$$
	
	$(ii)$: Sei $\theta\in\R$ gegeben und es gelte $x_e^+(\theta) = x_e^-(\theta) $ für alle $\theta\in\R$.
	Dann sind auch alle $b_{\newv_e} = 0$ und nach Lemma~\ref{lemma-balance-0} gilt $b_s(\theta) + b_t(\theta) = 0$, da $g^\theta$ statischer $b^\theta$-Fluss ist.
	Das bedeutet $f$ fließt ohne Überholungen.
	
	Angenommen, $f$ fließe ohne Überholungen.
	Dann gilt $\sum_{e\in E} b_{\newv_e}^\theta = 0$ unter Verwendung von Lemma~\ref{lemma-balance-0} für alle $\theta\in\R$.
	Da $f$ zulässig ist, gilt $x_e^-(\theta)\leq x_e^+(\theta)$ und damit $b_{\newv_e}^\theta\leq 0$ für alle $e\in E$.
	Also sind bereits alle $b_{\newv_e}^\theta = 0$, was die Behauptung zeigt.
\end{proof}

\section{Dynamische Flüsse}


Betrachte Zeitraum $[0,T]$ mit $T\in\R_+$.
$d$ Spieler erscheinen an Quelle $s$ über Zeitraum von $0$ bis $T$

\begin{definition}[Netzwerk]
	Ein \emph{Netzwerk} ist ein gerichteter Graph $G=(V,E)$ mit endlicher Knotenmenge $V$ und Kantenmenge $E\subseteq V\times V$, einer \emph{Quelle} $s\in V$ und einer Senke $t\in V$.
	Jeder Kante $e\in E$ werden eine Kapazität $u_e > 0$ und eine Verzögerungszeit $\tau_e\geq 0$ zugeordnet, sodass alle Zykel $T$ eine positive Gesamtverzögerung $\sum_{e\in T}\tau_e > 0$ haben.
\end{definition}

Für ein Netzwerk $\mathcal{N}$ bezeichne $\mathcal{P}$ die Menge aller $s$-$t$-Pfade.

\begin{definition}[Fluss über Zeit]
	Ein Fluss über Zeit $f=(f^+, f^-)$ ist ein Paar zweier über die Kanten $E$ indizierten Familien von Lebesgue-integrierbaren Funktionen $f^+_e,f^-_e: \R \to \R_{\geq 0}$ an $e$ für alle $e\in E$.
	
	Dabei bezeichnen $f_e^+(\theta)$ die \emph{Einflussrate an $e$ zum Zeitpunkt $\theta\in\R$} und $f_e^-(\theta)$ die \emph{Ausflussrate aus $e$ zum Zeitpunkt $\theta\in\R$} für $e\in E$.
	
	Der (kumulative) \emph{Einfluss bzw. Ausfluss an einer Kante $e$ bis zum Zeitpunkt $\theta$} sei definiert durch $F^+_e(\theta):=\int_{[0,\theta)} f^+_e d\lambda$ bzw. $F^-_e(\theta):=\int_{[0,\theta)} f^-_e d\lambda$.
	
	Die \emph{(Länge der) Warteschlange an Kante $e$ zum Zeitpunkt $\theta\in\R$} sei definiert durch $z_e(\theta):= F_e^+(\theta) - F_e^-(\theta + \tau_e)$ und die \emph{Wartezeit an Kante $e$ zum Zeitpunkt $\theta\in\R$} sei definiert durch $q_e(\theta) = z_e / u_e$.
	
	Man beschreibe den \emph{Austrittszeitpunkt einer Kante $e$ bei Eintrittszeitpunkt $\theta$}, in dem ein Partikel eine Kante verlässt, die es zum Zeitpunkt $\theta$ betreten hat, als $T_e(\theta):=\theta + q_e(\theta) + \tau_e$.
\end{definition}

\begin{definition}[Zulässiger Fluss über Zeit]
	Sei ein Fluss über Zeit $f=(f^+, f^-)$ gegeben. $f$ heißt zulässig, falls
	\begin{enumerate}[(I)]
		\item\label{def-feasible-flow-capacity} keine Ausflussrate die Kapazität übersteigt, d.h. $\forall e\in E, \theta\in\R: f_e^-(\theta)\leq u_e$, und
		\item\label{weak} Fluss eine Kante nur verlässt, falls er die Kante zuvor betreten hat, d.h. $\forall e\in E, \theta\in\R: F_e^+(\theta) \geq F_e^-(\theta + \tau_e)$, und
		\item\label{def-feasible-flow-no-flow-at-node} Fluss direkt nach der Ankunft an einem Knoten der nächsten Kante zugeordnet wird, d.h. $\forall v\in V\setminus \{ s, t \}, \theta\in\R: \sum_{e\in\delta^-(v)}f^-_e(\theta) = \sum_{e\in\delta^+(v)} f_e^+(\theta)$, und
		\item\label{def-feasible-flow-queue-with-capacity} nicht-leere Warteschlangen mit der Kapazität der Kante abgebaut werden, d.h. $\forall e\in E, \theta\in\R: q_e(\theta) > 0 \implies f_e^-(\theta + \tau_e) = u_e$.
		\todo{
		 $F_e^+(\theta) = F_e^-(\theta + \tau_e + q_e(\theta))$ für alle $e\in E, \theta\in\R$, ist das das gleiche (f.ü.)? Vermutlich nein; da fehlt irgendwie noch was: $q_e(\theta) = 0 \implies f_e^-(\theta + \tau_e) = f_e^+(\theta)$ o. Ä.}
	\end{enumerate}
\end{definition}

\todo{$\theta < 0 \implies f(\theta) = 0$}

\todo{source and sink nodes: Kein Fluss verschwindet bei Quelle, kein Fluss kommt hinzu bei Sinke}

\todo{FIFO Interpretation}

\begin{proposition}\label{prop-feasible-flow}
	Sei $e\in E$ eine Kante und $f$ ein zulässiger Fluss. Dann gilt:
	\begin{enumerate}[(i)]
		\item\label{prop-feasible-flow-T-mon-inc-cont} Die Funktion $\theta \mapsto \theta + q_e(\theta)$ ist monoton wachsend und stetig.
		\item\label{prop-feasible-flow-positive-queue} Für alle $e\in E$ und $\theta\in\R$ ist die Länge der Warteschlange $z_e$ auf dem Intervall $(\theta, \theta + q_e(\theta))$ positiv.
		\item\label{prop-feasible-flow-det-outflow} Zu jeder Zeit $\theta\in\R$ ist $F_e^+(\theta) = F_e^-(T_e(\theta))$.
		\item\label{prop-feasible-flow-queue-delay} Für zwei Zeitpunkte $0\leq \theta_1 < \theta_2$ mit $\int_{\theta_1}^{\theta_2} f^+_e d\lambda = 0$ und $q_e(\theta_2)>0$ ist $\theta_1 + q_e(\theta_1) = \theta_2 + q_e(\theta_2)$.
	\end{enumerate}
\end{proposition}
\begin{proof}
	Für die Monotonie in~(\ref{prop-feasible-flow-T-mon-inc-cont}) seien $\theta_1 \leq \theta_2$ gegeben.
	Mit der Monotonie von $F_e^+$ und mit $F_e^-(\theta_1 + \tau_e) = F_e^-(\theta_2+\tau_e) - \int_{\theta_1+\tau}^{\theta_2+\tau} f_e^-(t)dt\leq F_e^-(\theta_2 + \tau_e) + (\theta_2 - \theta_1)u_e$ gilt: 
	$$
		\theta_1 + q_e(\theta_1)
		= \theta_1 + \frac{F_e^+(\theta_1) - F_e^-(\theta_1 + \tau_e)}{u_e}\\
		\leq \theta_2 + \frac{F_e^+(\theta_1) - F_e^-(\theta_2+\tau_e)}{u_e} \leq \theta_2 + q_e(\theta_2).
	$$
	Die Stetigkeit folgt aus der Stetigkeit von $F_e^+$ und $F_e^-$.
	
	Für $\theta'\in (\theta, \theta+q_e(\theta))$ gilt $\theta' + q_e(\theta') \geq \theta + q_e(\theta)$ nach~(\ref{prop-feasible-flow-T-mon-inc-cont}).
	Also ist $q_e(\theta') \geq (\theta - \theta') + q_e(\theta) > q_e(\theta)\geq 0$, was Aussage (ii) beweist.	
	
	Zu Aussage (iii): Nach Voraussetzung~(\ref{def-feasible-flow-queue-with-capacity}) und Aussage (ii) ist
	$\int_{\theta}^{\theta + q_e(\theta)}f_e^-(t + \tau_e) dt = q_e(\theta)  u_e = z_e(\theta)$.
	Damit ist $F_e^-(T_e(\theta)) = F_e^-(\theta+\tau_e) + \int_{\theta+\tau_e}^{\theta+\tau_e+q_e(\theta)}f_e^-(t)dt = F_e^+(\theta)$.
	
	Es bleibt noch Aussage (iv) zu zeigen:
	Für alle $\theta'\in [\theta_1, \theta_2]$ gilt $F_e^+(\theta_1) = F_e^+(\theta') = F_e^-(\theta_2)$.
	Also ist $F_e^+(\theta') - F_e^-(\theta' + \tau_e) \geq F_e^+(\theta_2)-F_e^-(\theta_2 + \tau_e) = z_e(\theta_2) > 0$ und es gilt $f_e^-(\theta' + \tau_e)=u_e$.
	Für die Warteschlangendifferenz gilt dann: 
	$z_e(\theta_1)-z_e(\theta_2)=-F^-(\theta_1 + \tau_e) + F^-(\theta_2 + \tau_e) = \int_{\theta_1 + \tau_e}^{\theta_2 + \tau_e} f^-_e(t) dt = (\theta_2 - \theta_1)u_e$.
	Man folgere $q_e(\theta_1) - q_e(\theta_2) = \theta_2 - \theta_1$.
\end{proof}

\todo{Insbesondere ist also die Kantenaustrittszeit $T_e$ monoton wachsend und stetig.}

\todo{Stetigkeit von irgendwas um die "Wohldefiniertheit" dieser komischen Definition  von Nash flows zu zeigen}


\section{Durchlaufzeiten}

\begin{definition}
	Für einen Fluss $f$ und einen Pfad $P=(e_1,\dots,e_k)$ definiere $l^P(\theta):=T_{e_k}\circ\dots\circ T_{e_1}(\theta)$ den Zeitpunkt, an dem ein Partikel den Endknoten des Pfads erreicht, falls er den Pfad zum Zeitpunkt $\theta$ betritt.
	
	Für einen Knoten $w\in V$ beschreibe $\mathcal{P}_w$ die Menge aller $s$-$w$-Pfade.
	Dann ist die früheste Ankunft eines Partikels, das zum Zeitpunkt $\theta$ bei $s$ startet, gegeben durch $l_w(\theta):=\min_{P\in\mathcal{P}_w}l^P(\theta)$.
	Ein Pfad $P\in \mathcal{P}_w$ heißt \emph{kürzester $s$-$w$-Pfad zum Zeitpunkt $\theta$},  falls $l_w(\theta)=l^P(\theta)$.
\end{definition}

\begin{proposition}
	Die Funktionen $T_e$, $l^P$ und $l_v$ sind für alle Kanten $e\in E$, Pfade $P$ in G und Knoten $v\in V$ monoton wachsend, lokal absolut stetig und surjektiv.
\end{proposition}
\begin{proof}
	\todo{TODO. Auch: Warum brauch ich nochmal absolute Stetigkeit?}
\end{proof}

\todo{$l_v$ wohldefiniert, d.h. minimum existiert, da Pfade mit Kreisen nie das Minimum sind $\implies$ endlich viele}

Für einen Knoten $v \in V$ und einen zulässigen Fluss ist die Funktion $l_v$ als Minimum von Kompositionen stetiger und monoton wachsender Funktionen $T_e$ (Proposition~\ref{prop-feasible-flow}~\ref{prop-feasible-flow-T-mon-inc-cont}) ebenfalls stetig und monoton wachsend.


\begin{lemma}\label{lemma-dreicksungl}
	Für alle Kanten $vw\in E$ gilt in einem zulässigen Fluss 
	$T_{vw}(l_v(\theta)) \geq l_w(\theta)$.
\end{lemma}
\begin{proof}
	Sei ein kürzester $s$-$v$-Pfad $P$ zum Zeitpunkt $\theta$ gegeben.
	Hängt man an $P$ die Kante $vw$ an, erhält man einen $s$-$w$-Pfad, der zur Eintrittszeit $\theta$ die Ankunftszeit $T_{vw}(l_v(\theta))$ liefert.
	Da $l_w(\theta)$ das Minimum über die Ankunftszeit aller $s$-$w$-Pfade ist, gilt die Behauptung.
\end{proof}

\begin{definition}
	Man bezeichne eine Kante $vw\in E$ als \emph{aktiv zum Zeitpunkt $\theta$}, falls $T_{vw}(l_v(\theta)) = l_w(\theta)$ gilt.
	Die Menge $\Theta_{vw}$ bezeichne alle Zeitpunkte, zu denen die Kante $vw$ aktiv ist.
\end{definition}

\begin{lemma}\label{lemma-shortest-path-using-active-edges}
	Zu jeder Zeit $\theta$ ist der durch die aktiven Kanten induzierte Teilgraph $G_\theta:=(V, E_\theta)$ azyklisch jeder Knoten ist von $s$ erreichbar.
\end{lemma}
\begin{proof}
	Angenommen es existiere ein Zyklus $C=(v_1, \dots, v_n)$ mit $v_1=v_n$ und ausschließlich aktiven Kanten.
	Es ist $l^C(\theta) > \theta$, da für Zykeln eine positive Gesamtverzögerung vorausgesetzt ist.
	Augrund der Aktivität von $C$ erzeugt $l_{v_1}(\theta) = T^C(l_{v_1}(\theta)) > l_{v_1}(\theta)$ einen Widerspruch.
	
	Für jeden Knoten $w\neq s$ existiert zumindest eine eingehende aktive Kante -- zum Beispiel die letzte Kante eines kürzesten $s$-$w$-Pfades, welcher existiert, da $w$ von $s$ aus erreichbar ist.
	Folgt man diesen aktiven Rückwärtskanten, so erreicht man schließlich $s$, da keine Zykeln auftreten.
%	Man induziere über die maximale Kantenzahl $k$ kürzester $s$-$w$-Pfade -- diese existiert, da für alle Zykeln eine positive Gesamtverzögerung vorausgesetzt wird.
%	Für $k>0$ seien ein kürzester $s$-$w$-Pfad $P$ mit maximaler Kantenzahl $k$ und seine letzte Kante $vw$ gegeben.
%	Wegen der Monotonie von $T_{vw}$ gilt $T_{vw}(l_v(\theta)) \leq T_{vw}(l^{P-vw}(\theta)) = l_w(\theta)$, wobei $P-vw$ der Pfad $P$ ohne die letzte Kante $vw$ ist.
%	Also ist $vw$ zur Zeit $\theta$ aktiv.
%	Die Erweiterung eines kürzesten $s$-$v$-Pfades $Q$ um $vw$ ist ein kürzester $s$-$w$-Pfad, weshalb die maximale Kantenzahl eines kürzesten $s$-$v$-Pfades kleiner als $k$ ist.
%	Nach Induktionsannahme existiert ein kürzester $s$-$v$-Pfad, der nur aktive Kanten verwendet.
%	Hängt man an diesen Pfad $vw$ an, so erhält man einen kürzesten $s$-$w$-Pfad mit ausschließlich aktiven Kanten.
\end{proof}

\todo{Move notation to first occurrence}
\begin{notation}
	Das Komplement einer Menge $M$ notiere man als $M^c:= \R\setminus M$.
\end{notation}

\todo{Bla: d.h. falls die Kante in einem kürzesten $s$-$w$-Pfad liegt}.

\begin{lemma}
	Für alle Knoten $v\in V$ ist in einem zulässigen Fluss die Funktion $l_v$ monoton wachsend und stetig.
\end{lemma}
\todo{Wo ist der Beweis hin?}

\begin{proposition}
	Der Vektor $(l_v(\theta))_{v\in V}$ lässt sich ermitteln, indem man folgendes Gleichungssystem eindeutig löst:
	$$ \tilde{l}_w = \begin{cases}
	\theta & \text{falls } w=s \\
	\min\limits_{vw\in \delta^-(w)} T_{vw}(\tilde{l}_v) & \text{sonst}
	\end{cases}.$$
\end{proposition}
\begin{proof}
	Offenbar löst $l_v(\theta)$ dieses System nach Lemma~\ref{lemma-shortest-path-using-active-edges}.
	
	Sei $(\tilde{l}_v)_{v\in V}$ eine Lösung dieses Gleichungssystems.
	Man zeige für jeden Knoten $w\in V$ gilt $l_w(\theta) = \tilde{l}_w$.
	Dazu führe man eine Induktion über die maximale Kantenzahl $k$ kürzester $s$-$w$-Pfade durch.
	Solch ein Maximum existiert für jeden Knoten $w$, da für Zyklen eine positive Gesamtverzögerung vorausgesetzt ist.
	Für $k=0$ gilt $l_s(\theta)=\theta = \tilde{l}_s$ für die Quelle $s$.
	Sei für $w\neq s$ ein kürzester $s$-$w$-Pfad $P$ zum Zeitpunkt $\theta$ mit maximaler Kantenzahl $k$ gegeben und sei $vw$ die letzte Kante von $P$.
	Offenbar ist $vw$ aktiv zur Zeit $\theta$.
	Der Pfad $P-vw$, also der Pfad $P$ ohne die letzte Kante $vw$, ist ein kürzester $s$-$v$-Pfad mit maximaler Kantenzahl zur Zeit $\theta$:
	Für einen kürzesten $s$-$v$-Pfad $Q$ ist $l^Q(\theta)\leq l^{P-vw}(\theta)$ und mit der Monotonie von $T_{vw}$ gilt $l^{Q+vw}(\theta)\leq l^P(\theta)$.
	Wegen der Maximalität der Kantenzahl von $P$ ist also auch die Kantenzahl von $P-vw$ maximal.
	Damit gilt nach Induktionsvoraussetzung $l_v(\theta) = \tilde{l}_v$
	und damit gelten $l_w(\theta) = T_{vw}(\tilde{l}_v)$ und $\tilde{l}_w \leq l_w(\theta)$.
\end{proof}
\todo{Belmann-Ford? Warum kann man das? Zykeln?}
\section{Charakterisierung dynamischer Nash-Flüsse}

\begin{definition}\label{def-flow-along-active-edges}
	Man sage, der Fluss $f$ \emph{fließe nur entlang aktiver Kanten}, falls $f_{vw}^+$ fast überall auf $l_v(\Theta_{vw}^c)$ verschwindet für alle Kanten $vw\in E$.
\end{definition}

\begin{remark}
	Diese Definition weicht von der Defintion von Koch und Skutella ab und entspricht derjenigen aus~\cite[Definition 1]{Cominetti2015}:
	Nach \cite[Definition 2]{Koch2011} sagt man, $f$ \emph{sende Fluss nur entlang aktuell kürzester Pfade}, falls $f_{vw}^+\circ l_v$ fast überall auf $\Theta_{vw}^c$ verschwindet für alle Kanten $vw$.
	
	Entspricht $f$ dieser Definition, so auch Definition~\ref{def-flow-along-active-edges}: 
	Da $l_v$ nach Proposition~\ref{prop-abs-cont-sur} absolut stetig ist, bildet es nach~\cite[Aufgabe 4.9]{Elstrodt2011} Nullmengen wieder auf Nullmengen ab, weshalb folgende Menge Nullmenge ist: $$l_v(\{ \theta \in \Theta_{vw}^c \mid f_{vw}^+ (l_v(\theta)) > 0 \}) = \{ \xi \in l_v(\Theta_{vw}^c) \mid f_{vw}^+ (\xi) > 0 \}.$$
	 
	Koch und Skutella zeigen im Beweis von~\cite[Lemma 1]{Koch2011} die entsprechende Äquivalenz von \ref{lemma-only-active-edges} (i) und \ref{lemma-only-active-edges} (iii) und
	verwenden im Teil (iii)$\Rightarrow$(i) das Argument, dass für jede Kante $vw\in E$ und alle $\theta\in \Theta_{vw}^c$ eine Umgebung $U$ von $\theta$ existiert, sodass $f_{vw}^+$ fast überall in $l_v(U)$ verschwindet.
	Dies reicht aber nicht aus, um zu zeigen, dass $f_{vw}^+(l_v(\theta))=0$ für fast alle $\theta\in\Theta_{vw}^c$ gilt:
	So kann $f_{vw}^+(l_v(\theta))$ für ein $\theta\in\Theta_{vw}^c$ positiv sein und $l_v$ konstant in einer Umgebung um $\theta$.
	Dann ist $f_{vw}^+ \circ l_v$ in einer Umgebung um $\theta$ positiv, was im Widerspruch zur Forderung ist.
	
	Dies wurde in~\cite[Example 2]{Cominetti2015} ausgenutzt, um einen Beispielfluss anzugeben, der beweist, dass die Forderung sogar echt stärker ist.
\end{remark}

\begin{lemma}\label{lemma-vanishes-intervals}
	Seien $g: \R \to \R_{\geq 0}$ eine lokal Lebesgue-integrierbare Funktion und $((a_i, b_i))_{i\in I}$ eine Familie offener Intervalle.
	Dann verschwindet $g$ fast überall auf $\Theta:=\bigcup_{i\in I} (a_i, b_i)$ genau dann, wenn es für alle $i\in I$ fast überall auf $(a_i, b_i)$ verschwindet.
\end{lemma}
\begin{proof}
	Verschwindet $g$ fast überall auf $\Theta$, so erst recht auf jedem Intervall $(a_i, b_i)$.
	Für die andere Richtung definiert die Funktion $\mu(A):= \int_A g \diff \lambda$ ein Maß auf den Borelmengen~$\mathfrak{B}$.
	Da jede offene Menge $O\subseteq\R$ $\sigma$-kompakt ist, also eine Darstellung als abzählbare Vereinigung kompakter Mengen -- hier $O=\bigcup_{n\in\N}(\{ x \in\R \mid d(x, O^c) \geq 1/n \} \cap [-n, n] )$ -- besitzt, ist jede offene Menge nach~\cite[1.2 Folgerungen (e)]{Elstrodt2011} innen regulär.
	Dies heißt, dass für offene Mengen $O\subseteq\R$ $$\mu(O)=\sup\{ \mu(K) \mid K\subseteq O \text{ kompakt} \}$$ gilt.
	Für ein kompaktes $K\subseteq \Theta$ existiert eine endliche Teilüberdeckung $\bigcup_{i=1}^n (a_k, b_i) \supseteq K$, für die $\mu(K) \leq \sum_{i=1}^{n} \mu((a_i, b_i)) = \sum_{k=1}^{n} \int_{a_i}^{b_i} g(t) \diff t = 0$ gilt.
	Also ist auch $\mu(\Theta)=0$.
\end{proof}

\begin{lemma}\label{lemma-only-active-edges}
	Für einen zulässigen Fluss $f$ sind folgende Aussagen äquivalent:
	\begin{enumerate}[label=(\roman*)]
		\item Der Fluss $f$ fließt nur entlang aktiver Kanten.
		\item Für jede Kante $vw\in E$ und für fast alle $\xi\in\R$ mit	$f_{vw}^+(\xi)>0$ gilt $\xi \in l_v(\Theta_{vw})$.
		\item Für jede Kante $vw\in E$ und für alle $\theta\in\R$ gilt $F_{vw}^+(l_v(\theta)) = F_{vw}^-(l_w(\theta))$.
	\end{enumerate}
\end{lemma}
\begin{proof}
	$(i) \Leftrightarrow (ii)$: Bedingung~(ii) gilt genau dann, wenn $f_{vw}^+$ fast überall auf $l_v(\Theta_{vw})^c$ verschwindet.
	Um die Äquivalenz zu beweisen, reicht es also zu zeigen, dass sich $l_v(\Theta_{vw})^c$ und $l_v(\Theta_{vw}^c)$ nur um eine Nullmenge voneinander unterscheiden.
	Für ein $\xi\in l_v(\Theta_{vw})^c$ existiert wegen der Surjektivität von $l_v$ nach Proposition~\ref{prop-abs-cont-sur} ein $\theta\in\R$ mit $l_v(\theta)=\xi$ und $\theta\notin\Theta_{vw}$, da $\xi\notin l_v(\Theta_{vw})$. 
	Also ist $\xi\in l_v(\Theta_{vw}^c)$ und es gilt $l_v(\Theta_{vw})^c\subseteq l_v(\Theta_{vw}^c)$.
	
	Des Weiteren ist $l_v(\Theta_{vw}^c)\setminus l_v(\Theta_{vw})^c = l_v(\Theta_{vw}^c)\cap l_v(\Theta_{vw})\subseteq l_v(\Q)$:
	Für ein $\xi$ aus der linken Menge existieren $\theta\in\Theta_{vw}^c$ und $\theta'\in\Theta_{vw}$ mit $l_v(\theta)=\xi=l_v(\theta')$.
	Da $\theta\neq\theta'$ ist, existiert ein $\theta_q\in\Q$ zwischen $\theta$ und $\theta'$.
	Wegen der Monotonie von $l_v$ gilt $l_v(\theta_q)=\xi$ und $\xi\in l_v(\Q)$.
	Also unterscheiden sich die beiden Mengen nur um eine abzählbare Menge.
	
	$(i)\Leftrightarrow (iii)$: Sei eine Kante $vw\in E$ gegeben.
	Die Beziehung $F_{vw}^-(T_{vw}(l_v(\theta))) \geq F_{vw}^-(l_w(\theta))$ gilt bereits wegen der Monotonie von $F_{vw}^-$ und Lemma~\ref{lemma-dreicksungl} für alle $\theta\in\R$.

	Für ein $\theta\in\R$ bezeichne man den spätesten Startzeitpunkt kleiner oder gleich $\theta$, sodass man unter Benutzung von $vw$ gerade zum Zeitpunkt $l_w(\theta)$ zu $w$ gelangt, als
	$$\omega_\theta:=\max\{ \omega\leq\theta \mid l_w(\theta) = T_{vw}(l_v(\omega)) \}.$$
	Dieses Maximum existiert, da die Menge nach oben beschränkt, abgeschlossen und wegen des Zwischenwertsatzes mit $T_{vw}(l_v(\theta))\geq l_w(\theta)$ und $\lim_{t\to -\infty}T_{vw}(l_v(t))=-\infty$ (\todo{ref}) nicht-leer ist.
	
	Es ist $\Theta_{vw}^c = \bigcup_{\theta\in\R} (\omega_\theta, \theta)$:
	Für $\theta\in\Theta_{vw}^c$ gilt $T_{vw}(l_v(\theta)) > l_w(\theta)$.
	Aufgrund der Stetigkeit von $T_{vw}\circ l_v$ und von $l_w$ existiert ein $\varepsilon>0$, sodass $T_{vw}(l_v(\theta')) > l_w(\theta+\varepsilon)$ für $\theta'\in[\theta,\theta+\varepsilon]$ gilt.
	Also ist $\theta\in(\omega_{\theta+\varepsilon}, \theta+\varepsilon)$.
	Ist umgekehrt $\theta'\in (\omega_\theta,\theta)$, so ist aufgrund der Monotonie $T_{vw}(l_v(\theta'))\geq T_{vw}(l_v(\omega_\theta)) = l_w(\theta)\geq l_w(\theta')$.
	Dies kann nicht mit Gleichheit erfüllt sein, da $w_\theta$ maximal mit der Eigenschaft $T_{vw}(l_v(\omega)) = l_w(\theta)$ ist.
	Also ist $vw$ zum Zeitpunkt $\theta'$ inaktiv.
	
	Insbesondere gilt $l_v(\Theta_{vw}^c) = \bigcup_{\theta\in\R}(l_v(\omega_\theta),l_v(\theta))$ wegen der Monotonie von $l_v$.
	Nach Lemma~\ref{lemma-vanishes-intervals} verschwindet $f_{vw}^+$ genau dann fast überall auf $l_v(\Theta_{vw})^c$, wenn es für alle $\theta\in\R$ fast überall auf $(l_v(\omega_\theta),l_v(\theta))$ verschwindet.
	Dies ist nach Proposition~\ref{prop-feasible-flow}~\ref{prop-feasible-flow-det-outflow} wiederum äquivalent zu
	$$F_{vw}^+(l_v(\theta))-F_{vw}^+(l_v(\omega_\theta)) = F_{vw}^+(l_v(\theta))-F_{vw}^-(l_w(\theta))=0~~\text{für alle $\theta\in\R$}.$$
\end{proof}


\todo{blablabla Definition FIFO bla}

\begin{definition}
	Für eine Kante $vw\in E$ bezeichne $x_{vw}^+(\theta):= F_{vw}^+(l_v(\theta))$ die Flussmenge, die die Kante $vw$ betreten hat, bevor Partikel, die zur Zeit $\theta$ in $s$ starten, $v$ erreichen können.
	
	Mit $x_{vw}^-(\theta):= F^-_{vw}(l_w(\theta))$ bezeichne man die Flussmenge, die die Kante $vw$ verlassen hat, bevor Partikel, die zur Zeit $\theta$ in $s$ starten, $w$ erreichen können.
	
	Für einen Knoten $v\in V$ sei $b_v(\theta):=\sum_{e\in\delta^+(v)} x_e^+(\theta) - \sum_{e\in\delta^-(v)} x_e^-(\theta)$ die Balance des Knoten $v$ zum Zeitpunkt $\theta$.
\end{definition}


\begin{remark}\label{remark-x^-leqx^+}
	In einem zulässigen Fluss ist $x_{vw}^-(\theta) = F_{vw}^-(l_w(\theta)) \leq F_{vw}^-(T_{vw}(l_v(\theta)))=F_{vw}^+(l_v(\theta)) = x_{vw}^+(\theta)$
	 nach Proposition~\ref{prop-feasible-flow}~\ref{prop-feasible-flow-det-outflow} und mit der Monotonie von $F_{vw}^+$.
\end{remark}

\begin{lemma}\label{lemma-balance-0}
	Für einen zulässigen dynamischen Fluss $f$ gilt $b_v(\theta)=0$ für alle Knoten $v\in V\setminus\{ s,t \}$ und alle $\theta\in\R$.
\end{lemma}
\begin{proof}
	Unter Benutzung der Voraussetzung~\ref{def-feasible-flow-no-flow-at-node} folgere man für $v\in V\setminus \{ s, t\}, \theta\in\R$:
	$$\sum_{e\in\delta^-(v)} x_e^-(\theta) = \int_{0}^{l_v(\theta)} \sum_{e\in\delta^-(v)} f_e^-(t) \diff t = \int_{0}^{l_v(\theta)} \sum_{e\in\delta^+(v)} f_e^+(t) \diff t = \sum_{e\in\delta^+(v)}x_e^+(\theta)$$
\end{proof}

\begin{definition}
	Man sage, ein zulässiger dynamischer Fluss $f$ \emph{fließe ohne Überholungen}, falls $b_s(\theta) = -b_t(\theta)$ für alle $\theta\in\R$.
\end{definition}

\todo{blabla definition blabla}

\begin{definition}
	Seien ein statischer Fluss $f \in \R^E$ in einem Graphen $G=(V,E)$ mit Kapazitäten $u\in \R_+^E$ und ein Balancevektor $b\in\R^V$ mit $\sum_{v\in V} b_v = 0$ gegeben.
	Der Fluss $f$ heißt \emph{$b$-Fluss}, falls er Flusserhaltung bzgl. $b$ gewährt, d.h. falls er für alle $v\in V$ die Bedingung $\sum_{e\in\delta^+(v)}f_e - \sum_{e\in\delta^-(v)}f_e = b_v$ erfüllt.
\end{definition}

\todo{blabla}

\newcommand{\newv}{\mathbf{v}}
\begin{lemma}\label{lemma-b-graph}
	Seien ein dynamischer Fluss $f$ in einem Graphen $G=(V,E)$ und ein Zeitpunkt $\theta\in\R$ gegeben.
	Der Graph $H$ entstehe aus $G$, indem man jede Kante $vw\in E$ aus $G$ durch einen neuen Knoten $\newv_{vw}$ und zwei Kanten $v\newv_{vw}$ und $\newv_{vw}w$ ersetze.
	Der statische Fluss $g$ auf $H$ sei definiert durch
	$$g_{v\newv_{vw}} := x_{vw}^+(\theta) \text{ und } g_{\newv_{vw}w} := x_{vw}^-(\theta) \text{ für alle $vw\in E$}$$
	und die Balance $b$ auf $H$ sei gegeben durch $b_v:= b_v(\theta)$ für $v\in V$ und $b_{\newv_e}:= x_e^-(\theta) - x_e^+(\theta)$ für $e\in E$.
	Dann gelten die folgenden Aussagen:
	
	\begin{enumerate}[label=(\roman*)]
		\item Der Fluss $g$ ist ein statischer $b$-Fluss.
		\item\label{lemma-b-graph-imp} Ist $f$ zulässig, so gilt $\forall e\in E : x_e^+(\theta) = x_e^-(\theta)\iff b_s(\theta) + b_t(\theta) = 0$.
	\end{enumerate}
\end{lemma} 
\begin{proof}
	$(i)$: Um zu zeigen, dass die Summe über die Balanceeinträge verschwindet, erkenne man, dass der Anteil einer Kante $e\in E$ in $\sum_{v_\in V} b_v$ gerade $x_e^+(\theta) - x_e^-(\theta)$ ist.
	Damit gilt:
		$$\sum_{v\in V}b_v + \sum_{e\in E} b_{\newv_e} = \sum_{e\in E}  (x_e^+(\theta) - x_e^-(\theta) + x_e^-(\theta) - x_e^+(\theta)) = 0.$$
		Es bleibt zu zeigen, dass $g$ bezüglich $b$ Flusserhaltung gewährt.
		Für die Knoten der Form $\newv_{vw}$ gilt dies, da $g_{\newv_{vw}w} - g_{v\newv_{vw}} = x_{vw}^-(\theta) - x_{vw}^+(\theta) = b_{\newv_{vw}}$.
		Für $v\in V$ gilt nach Konstruktion $$b_v =
		\sum_{e\in\delta^+_G(v)} x_{e}^+(\theta) - \sum_{e\in\delta^-_G(v)} x_{e}^-(\theta) =
	\sum_{e\in\delta_H^+(v)} g_e - \sum_{e\in\delta^-_H(v)}g_e
		.$$
	
	$(ii)$: Tatsächlich benötigt man aus $(i)$ nur die Eigenschaft, dass die Summe über die Einträge des Balancevektors verschwindet.
	Mit Lemma~\ref{lemma-balance-0} gilt wegen der Zulässigkeit von $f$ sogar $b_s(\theta)+b_t(\theta) + \sum_{e\in E} b_{\newv_e} = 0$.
	
	Angenommen, es gelte $x_e^+(\theta) = x_e^-(\theta)$ für alle $e\in E$.
	Dann sind auch alle $b_{\newv_e} = 0$ und es gilt $b_s(\theta) + b_t(\theta) = 0$.
	Setzt man $b_s(\theta) + b_t(\theta) = 0$ voraus, so ist $\sum_{e\in E} b_{\newv_e} = 0$ und, da $f$ zulässig ist, gilt $x_e^-(\theta)\leq x_e^+(\theta)$ nach Bemerkung~\ref{remark-x^-leqx^+}.
	Daher gilt $b_{\newv_e}\leq 0$ für alle $e\in E$, weshalb bereits alle $b_{\newv_e} = 0$ sein müssen.
\end{proof}


\begin{theorem}\label{thm-equivalencies-nash-flow}
	Für einen zulässigen dynamischen Fluss $f$ sind die folgenden Aussagen äquivalent:
	\begin{enumerate}[label=(\roman*)]
		\item Der Fluss $f$ fließt nur entlang aktiver Kanten
		\item Für alle Kanten $e\in E$ und zu allen Zeitpunkten $\theta\in\R$ ist $x_e^+(\theta) = x_e^-(\theta)$.
		\item Der Fluss $f$ fließt ohne Überholungen.
	\end{enumerate}
	Gilt eine dieser Aussagen, so nennt man $f$ einen \emph{dynamischen Nash-Fluss}.
\end{theorem}
\begin{proof}
	$(i) \Leftrightarrow (ii):$ Für eine Kante $vw\in E$ und einen Zeitpunkt $\theta\in\R$ gilt nach Proposition~\ref{prop-feasible-flow}~\ref{prop-feasible-flow-det-outflow}:
	$$x_{vw}^+(\theta) - x_{vw}^-(\theta) = F_{vw}^+(l_v(\theta)) - F_{vw}^-(l_w(\theta)) = F_{vw}^-(T_{vw}(l_v(\theta))) - F_{vw}^-(l_w(\theta)).$$
	Lemma~\ref{lemma-only-active-edges} schafft also die gewünschte Äquivalenz.
	
	$(ii) \Leftrightarrow (iii)$ ist eine direkte Folgerung von Lemma~\ref{lemma-b-graph}~\ref{lemma-b-graph-imp}.
\end{proof}

\begin{remark}\label{remark-s-t-flow}
	In einem Nash-Fluss ist der statische Fluss $x(\theta)$ mit $x_e(\theta):=x_e^+(\theta)=x_e^-(\theta)$ nach Lemma~\ref{lemma-balance-0} ein statischer $s$-$t$-Fluss für jedes $\theta\in\R$.
	Ist $x_e$ für alle $e\in E$ differenzierbar in $\theta$, so ist auch $x'(\theta)$ ein statischer $s$-$t$-Fluss, da Differenzieren die Flusserhaltung erhält.
\end{remark}

\begin{proposition}\label{prop-nash-flow-s-t-path-decomposable}
	Für einen dynamischen Nash-Fluss $f$ und zwei Zeitpunkte $\theta_1 \leq \theta_2$ ist der statische $s$-$t$-Fluss $x(\theta_2) - x(\theta_1)$ eine Komposition von $s$-$t$-Wegen.
\end{proposition}
\begin{proof}
	Sei $\theta$ das Infimum aller Zeitpunkte $\xi\geq\theta_1$, zu denen $x(\xi) - x(\theta_1)$ nicht $s$-$t$-Wege zerlegbar ist.
	Man nehme $\theta \leq \theta_2$ an.
	Da inaktive Kanten zum Zeitpunkt $\theta$ bereits kurz vor $\theta$ und noch kurz nach $\theta$ inaktiv sind, existiert ein Intervall $[\theta - \varepsilon, \theta + \varepsilon]$, in der keine inaktive Kante aktiv wird.
	Außerdem existiert für $\xi_0 := \max \{ \theta_1, \theta - \varepsilon \}$ eine $s$-$t$-Wegezerlegung von $x(\theta_1) - x(\xi_0)$.
	
	Für einen Pfad $P$ und einen statischen Fluss $g$ sei $g^P := \max_{e\in P} g_e$ der Fluss, der auf dem Pfad $P$ fließt.
	Sei $C$ ein Zyklus.
	Dann ist $x_C(\xi_0) = x_C(\xi)$ für $\xi\in [\xi_0, \theta+\varepsilon]$, da aufgrund der Azyklizität von $G_{\xi_0}$ eine Kante $vw\in C$ des Zyklus existiert, die zur Zeit $\xi_0$ und damit in ganz $[\xi_0, \theta+\varepsilon]$ inaktiv ist, wodurch
	$$F_{vw}^+(l_v(\theta + \varepsilon)) - F_{vw}^+(l_v(\xi_0)) = \int_{l_v(\xi_0)}^{l_v(\theta + \varepsilon)} f_{vw}^+(t) \diff t = 0$$
	folgt, da $f$ nur entlang aktiver Kanten fließt.
	
	Also hat der $s$-$t$-Fluss $x(\xi) - x(\xi_0)$ für $\xi\in [\xi_0, \theta + \varepsilon]$ keinen Zyklus mit positivem Fluss und besitzt daher eine $s$-$t$-Wegezerlegung.
	Addiert man diese zur $s$-$t$-Wegezerlegung von $x(\xi_0) - x(\theta_1)$, so erhält man eine $s$-$t$-Wegezerlegung von $x(\xi) - x(\theta_1)$, was für $\xi > \theta$ ein Widerspruch zur Definition von $\theta$ darstellt.
\end{proof}
\begin{corollary}
	Für einen dynamischen Nash-Fluss $f$ ist der statische $s$-$t$-Fluss $x(\theta)$ zu jeder Zeit eine Komposition von $s$-$t$-Wegen.
\end{corollary}
\begin{proof}
	Nach Proposition~\ref{prop-abs-cont-sur} existiert ein Zeitpunkt $\xi_0$ mit $l_v(\xi_0) \leq 0$ für alle Knoten $v\in V$.
	Für $\theta \leq \xi_0$ ist $x(\theta)$ der Nullfluss und offenbar $s$-$t$-Wege zerlegbar, da die Funktionen $f_e^+$ und $f_e^-$ links der $y$-Achse verschwinden.
	Sonst ist $x(\theta)=  x(\theta) - x(\xi_0)$ nach Proposition~\ref{prop-nash-flow-s-t-path-decomposable} in $s$-$t$-Wege zerlegbar.
\end{proof}
\section{A special class of static flows}

\begin{lemma}\label{lemma-no-inflow-until-l}
	Sei $f$ ein Nash-Fluss über Zeit.
	Dann gilt $F_{vw}^+(l_v(0)) = 0$ für jede Kante $vw\in E$.
\end{lemma}
\begin{proof}
	Für alle $\theta\in [0, l_v(0)]$ gilt nach Bedingung~\ref{def-feasible-flow-no-flow-at-node}, dass $f_{vw}^+(\theta) \leq \sum_{e\in\delta^-(v)} f_e^-(\theta)$.
	Also ist $F_{vw}^+(l_v(0)) \leq \sum_{uv\in\delta^-(v)} F_{uv}^-(l_v(0))\leq \sum_{uv\in\delta^-(v)} F_{uv}^+(l_u(0))$ nach Bemerkung~\ref{remark-x^-leqx^+}.
	\todo{TODO this doesn't work yet}
\end{proof}

\begin{lemma}\label{lemma-nash-flow-waiting-queue-implies-active-edge}
	Seien $f$ ein Nash-Fluss über Zeit, $uv\in E$ eine Kante und $\theta\in\R$ ein Zeitpunkt gegeben.
	Gilt eine der folgenden Aussagen, so ist $vw$ zum Zeitpunkt $\theta$ aktiv:
	\begin{enumerate}[label=(\roman*)]
		\item Es existiert ein $\varepsilon>0$ mit $x_{vw}(\theta')<x_{vw}(\theta)$ für alle $\theta'\in(\theta-\varepsilon, \theta)$.
		\item Die Ableitung $x_{vw}'(\theta)$ existiert und es gilt $x_{vw}'(\theta)> 0$.
		\item Die Wartezeit $q_{vw}(\theta)$ an der Kante $vw$ ist zur Zeit $\theta$ positiv.
	\end{enumerate}
\end{lemma}
\begin{proof}
	Zu Aussage (i): Für $\theta'$ gilt $F_{vw}^+(l_v(\theta')) > F_{vw}^+(l_v(\theta))$.
	Also existiert eine Teilmenge $Q_\varepsilon \subseteq (l_v(\theta-\varepsilon), l_v(\theta))=l_v((\theta - \varepsilon, \theta))$ mit positivem Maß und $f_{vw}^+\big|_{Q_\varepsilon} > 0$.
	Da $f$ ein Nash-Fluss ist, fließt $f$ nach Theorem~\ref{thm-equivalencies-nash-flow} nur entlang aktiver Kanten.
	Daher ist $\xi\in l_v(\Theta_{vw})$ für fast alle $\xi\in Q_\varepsilon$; insbesondere existiert ein $\theta_\varepsilon\in (\theta-\varepsilon, \theta)$ mit $\theta_\varepsilon\in\Theta_{vw}$.
	Für $\varepsilon\rightarrow0$ ist also auch $\theta\in\Theta_{vw}$ wegen der Abgeschlossenheit von $\Theta_{vw}$.
	
	Aussage (ii) folgt direkt aus (i) mittels Differenzenquotienten.
	\todo{don't argue with 0}Für Aussage (iii) ist wieder $T_{vw}(l_v(\theta)) \leq l_w(\theta)$ zu zeigen.
	Da $q_{vw}(l_v(\theta))$ positiv ist, gilt auch $F^+_{vw}(l_v(\theta)) > 0$.
	Nach Lemma~\ref{lemma-no-inflow-until-l} ist $F_{vw}(l_v(0))=0$ und wegen der Stetigkeit und Monotonie von $F_{vw}^+\circ l_v$ existiert
	ein frühester Zeitpunkt $\theta_1\in (0, \theta]$ mit $F_{vw}^+(l_v(\theta_1)) = F_{vw}^+(l_v(\theta))$.
	Nach Aussage (i) gilt also $T_{vw}(l_v(\theta_1)) = l_w(\theta_1)$.
	Nach Proposition~\ref{prop-feasible-flow}~\ref{prop-feasible-flow-queue-delay} ist $T_{vw}(l_v(\theta_1)) = T_{vw}(l_v(\theta))$ und mit der Monotonie von $l_w$ folgt $T_{vw}(l_v(\theta))\leq l_w(\theta)$.	
\end{proof}
\todo{Das ist glaub ich die Definition von schwachem Nash-flow}

\begin{definition}[Schmaler Fluss mit Zurücksetzen]
	Seien ein statischer $F$-wertiger Fluss $f$ in einem Graph $G=(V,E)$ mit Kantenkapazitäten $u\in\R^E$, eine Quelle $s\in V$ mit Zuflusskapazität $d\in\R$ und eine Senke $t\in V$ sowie eine Teilmenge $E_1\subseteq E$ von Kanten gegeben.
	Der Fluss $f$ heißt \emph{schmaler Fluss mit Zurücksetzen auf $E_1$}, falls eine Knotenbewertung $l\in\R^V$ existiert mit:
	$$\begin{array}{ll}
	(1)~~~	l_s = F/d& \\
	(2)~~~	l_v \leq l_u &\text{für $uv\in E \setminus E_1$ mit $f_{uv}=0$}\\
	(3)~~~	l_v = \max(l_u, f_{uv} / u_{uv} ) &\text{für $uv\in E\setminus E_1$ mit $f_{uv} > 0$}\\
	(4)~~~	l_v = f_{uv} / u_{uv} & \text{für $uv\in E_1$}
	\end{array}$$
\end{definition}

\begin{theorem}
	Seien ein Nash-Fluss über Zeit $f$ auf einem Graphen $G=(V,E)$ gegeben sowie ein Zeitpunkt $\theta$ gegeben.
	Der zum Fluss $f$ und dem Zeitpunkt $\theta$ zugeordnete statische Fluss sei $x(\theta)$ mit $x_{uv}(\theta) := x^+_{uv}(\theta) = x^-_{uv}(\theta)$.\todo{(muss vlt noch gezeigt werden)}
	Man betrachte den Teilgraph $G_\theta = (V, E_\theta)$, der nur zum Zeitpunkt $\theta$ aktive Kanten enthält, d.h. $E_\theta := \{ uv\in E \mid T_{uv}(l_u(\theta)) = l_v(\theta) \}$.
	
	Existieren die Ableitungen $\frac{dx_e}{d\theta}(\theta)$ und $\frac{dl_v}{d\theta}(\theta)$ für alle $e\in E$ und $v\in V$, so ist der statische Fluss $g\in\R^{E_\theta}$ mit $g_e:=\frac{dx_e}{d\theta}(\theta)$ ein schmaler $d$-wertiger Fluss mit Zurücksetzen auf den Kanten mit Warteschlange $E_1:=\{uv\in E \mid q_{uv}(l_u(\theta))>0 \}$ im Graphen $G_\theta$.
	Als Knotenbewertung dienen dazu die Ableitungen $(\frac{dl_v}{d\theta}(\theta))_{v\in V}$.
\end{theorem}
\begin{proof}
	zu zeigen: $\frac{dl_s}{d\theta}(\theta) = F/d$.
	Dabei ist $\frac{dl_s}{d\theta}(\theta) = 1$, da $l_s(\theta) = \theta$. Bleibt zu zeigen $F= d$.
	Der Wert von $g$ ist $\frac{d}{d\theta}\sum_{e\in\delta^+(s)}x_e(\theta) = \frac{d}{d\theta} d\cdot\theta = d$. \todo{Geht auch mit variablen Zufluss}.
	
	Sei nun eine Kante $uv\in E_\theta$, also eine aktive Kante zum Zeitpunkt $\theta$, gegeben. Wir prüfen die restlichen drei Bedingungen jeweils in den folgenden drei Fällen:
	
	\begin{description}
		\item[1. Fall:] $\exists \varepsilon > 0:\forall \theta'\in (\theta, \theta + \varepsilon ] : q_{uv}(l_u(\theta')) > 0$.
		
		Für $\theta'\in(\theta,\theta+\varepsilon]$ ist die Wartezeit $q_{uv}(l_u(\theta'))$ positiv und nach Proposition~\ref{prop-feasible-flow}~\ref{prop-feasible-flow-positive-queue} ist $q_{uv}$ positiv auf dem Intervall $[ l_u(\theta')  , l_u(\theta')+q_{uv}(l_u(\theta')) )$.
		Also ist $q_{uv}$ positiv auf dem Intervall $( l_u(\theta) , l_u(\theta + \varepsilon) + q_{uv}(l_u(\theta + \varepsilon) )
		\subseteq ( l_u(\theta) + q_{uv}(l_u(\theta)) , l_u(\theta + \varepsilon) + q_{uv}(l_u(\theta) + \varepsilon ) )$.
		
		Nach Voraussetzung ist die Kante ${uv}$ nach Lemma~\ref{lemma-nash-flow-waiting-queue-implies-active-edge} auch für $\theta'\in  (\theta, \theta + \varepsilon ]$ aktiv und es gilt $l_v(\theta') = T_{uv}(l_u(\theta')) = l_u(\theta') + q_{uv}(\theta') + \tau_{uv}$.
		Mit Bedingung~\ref{def-feasible-flow-queue-with-capacity} folgere man $x_{uv}(\theta + \varepsilon) - x_{uv}(\theta) = F_{uv}^-(l_v(\theta + \varepsilon)) - F_{uv}^-(l_v(\theta))
		= \int_{l_v(\theta)}^{l_v(\theta + \varepsilon)} f_{uv}^-(t) dt
		= \int_{l_u(\theta) + q_{uv}(\theta)}^{l_u(\theta + \varepsilon) + q_{uv}(\theta + \varepsilon)} f_{uv}^-(t + \tau_{uv}) dt
		= u_{uv} (l_v(\theta + \varepsilon) - l_v(\theta))$.
		Teilt man diese Gleichung durch $\varepsilon$, so erhält man für $\varepsilon\rightarrow 0$ die Bedingung $g_{uv} = \frac{dx_{uv}}{d\theta}(\theta) = u_{uv} \frac{dl_v}{d\theta}(\theta)$.
		Ist $uv\in E_1$, so ist also Bedingung (4) erfüllt.
		Für Bedingung (2) setze man $g_{uv}=\frac{dx_{uv}}{d\theta}(\theta)=0$ voraus.
		Also ist auch $\frac{dl_v}{d\theta}(\theta)=0$ und, da $l_u$ monoton wachsend ist, gilt $0 \leq \frac{dl_u}{d\theta}$.

		Ist $uv\notin E_1$, ist also die Warteschlange zum Zeitpunkt $l_u(\theta)$ leer, so gilt: $l_v(\theta+\varepsilon) - l_v(\theta) = l_u(\theta + \varepsilon) + q_{uv}(l_u(\theta + \varepsilon)) - l_u(\theta) \geq l_u(\theta + \varepsilon) - l_u(\theta)$.
		Teilt man wieder durch $\varepsilon$, so erhält man für $\varepsilon  \rightarrow 0$ Bedingung (3) mit $\frac{dl_v}{d\theta}(\theta) \leq \frac{dl_u}{d\theta}(\theta)$ und dem Resultat des letzten Absatzes.
		
		\item[2. Fall:] $\exists \varepsilon > 0: \forall \theta'\in(\theta, \theta + \varepsilon]: T_{uv}(l_u(\theta'))>l_v(\theta')$.
		
		Die Kante ${uv}$ ist also im Intervall $(\theta, \theta + \varepsilon]$ nicht aktiv.
		Nach Lemma~\ref{lemma-nash-flow-waiting-queue-implies-active-edge} ist $q_{uv} \circ l_u\big|_{(\theta, \theta+\varepsilon]}=0$ und wegen Stetigkeit ist auch $q_{uv}(l_u(\theta))=0$, was $uv\notin E_1$ impliziert.
		Da $f$ nur entlang aktiver Kanten fließt, ist außerdem $f_{uv}^+ \circ l_u \big|_{(\theta, \theta+\varepsilon)} = 0$, und es gilt $x_{uv}(\theta + \varepsilon) - x_{uv}(\theta)=0$.
		Wird durch $\varepsilon$ geteilt, so erhält man $\frac{dx_{uv}}{d\theta}(\theta) = 0$ für $\varepsilon\rightarrow0$.
		Es muss also nur Bedingung (2) geprüft werden:
		
		
		Da $uv$ zum Zeitpunkt $l_u(\theta)$ noch aktiv ist, folgt $l_v(\theta + \varepsilon) - l_v(\theta) < l_u(\theta+\varepsilon) - l_v(\theta) + \tau_{uv} = l_u(\theta + \varepsilon) - l_u(\theta)$.
		Teilt man diese Ungleichung  durch $\varepsilon$, so erhält man für $\varepsilon\rightarrow 0$ die Bedingung $\frac{dl_v}{d\theta}(\theta)\leq\frac{dl_u}{d\theta}(\theta)$.
		
		\item[3. Fall:] $\forall \varepsilon>0: \exists \theta_{\varepsilon}\in (\theta, \theta+\varepsilon]: T_{uv}(l_u(\theta_\varepsilon)) = l_v(\theta_\varepsilon)$.
		
		Dies ist die exakte Umkehrung der Bedingung von Fall 2.
		Zusätzlich betrachte man diesen Fall nur, falls Fall 1 nicht eintritt.
		Das heißt, für alle $\theta_\varepsilon$ existiert ein $\theta'\in(\theta, \theta_\varepsilon]$ mit $q_{uv}(l_u(\theta')) = 0$; insbesondere ist $q_{uv}(l_u(\theta))= 0$, wodurch $uv$ nicht in $E_1$ enthalten ist.
		Man wähle $\theta'_\varepsilon:=\max\{ \theta'\in (\theta, \theta_\varepsilon] \mid q_{uv}(l_u(\theta')) = 0 \}$ als das Maximum solcher Zeitpunkte $\theta'$, welches aufgrund der Stetigkeit von $q_{uv}\circ l_u$ existiert.
		Nach Konstruktion ist $q_{uv}\circ l_u$ im Intervall $(\theta_\varepsilon', \theta_\varepsilon)$ positiv und nach Lemma~\ref{lemma-nash-flow-waiting-queue-implies-active-edge} ist die Kante $uv$ in diesem Intervall aktiv.
		Ist dieses Intervall nicht-leer, so ist $uv$ auch zum Zeitpunkt $\theta_\varepsilon'$ wegen der Stetigkeit von $T_{uv}\circ l_u$ und $l_v$ aktiv.
		Ansonsten ist $\theta_\varepsilon'=\theta_\varepsilon$ trotzdem eine Zeit, zu der $uv$ aktiv ist.
		Also gilt: $l_v(\theta_\varepsilon') - l_v(\theta) = l_u(\theta_\varepsilon') - l_u(\theta)$.
		Bedingung~(2) wird erfüllt, indem man durch $\theta_\varepsilon'-\theta$ teil und $\frac{dl_v}{d\theta}(\theta) = \frac{dl_u}{d\theta}(\theta)$ für $\varepsilon\rightarrow0$ erhält.
		
		Für Bedingung~(3) bleibt zu zeigen, dass $\frac{dx_{uv}}{d\theta}(\theta) /u_{uv}\leq \frac{dl_v}{d\theta}(\theta)$ gilt.
		Wegen Bedingung~\ref{def-feasible-flow-capacity} ist $x_{uv}(\theta + \varepsilon)-x_{uv}(\theta) = \int_{l_v(\theta+\varepsilon)}^{l_v(\theta)} f_{uv}^-(t) dt\leq (l_v(\theta + \varepsilon) - l_v(\theta)) u_{uv}$ für beliebiges $\varepsilon > 0$.
		Durch Teilen mit $\varepsilon u_{uv}$ erhält man das gewünschte Resultat für $\varepsilon \rightarrow 0$.
	\end{description}
\end{proof}

\begin{definition}
	Ein \emph{Fluss über Zeit $f$ mit Zeithorizont $T>0$} ist ein Fluss, für dessen Zufluss $d(\theta)= 0$ für $\theta\geq T$ gilt.
\end{definition}

\begin{definition}
	Seien ein Nash-Fluss über Zeit $f$ mit Zeithorizont $T$ sowie dessen induzierter schmaler Fluss mit Zurücksetzen im Graphen $G_T$ zum Zeitpunkt $T$ und ein $\alpha > 0$ gegeben.
	
	\todo{Brauche vermutlich sowas wie $f_{vw}^+(\geq l_v(\theta))=0$}
	
	Man erhalte die \emph{$\alpha$-Erweiterung $\tilde{f}$ von $f$} aus $f$, wobei man für alle $vw\in E$ mit $x_{vw}'>0$ den Zu- bzw. Abfluss setze auf
	$$\tilde{f}_{vw}^+(\theta):= \frac{x_{vw}'}{l_v'} \text{ für $\theta\in [l_v(T), l_v(T)+\alpha l_v']$ und } \tilde{f}_{vw}^-(\theta):=\frac{x_{vw}'}{l_w'} \text{ für $\theta\in [l_w(T), l_w(T)+\alpha l_w']$.}$$
\end{definition}

\begin{lemma}
	Jede $\alpha$-Erweiterung $\tilde{f}$ eines Nash-Flusses über Zeit $f$ mit Zeithorizont $T$ ist ein zulässiger Fluss über Zeit mit Zeithorizont $T+\alpha$, falls für alle Kanten mit positiver Warteschlange zum Zeitpunkt $T$ gilt:
	$$l_w(T) - l_v(T) + \alpha(l_w' - l_v') \geq \tau_{vw}.$$
	Dann gelten die folgenden Aussagen:
	\begin{enumerate}[label=(\roman*)]
		\item Gilt $x_{vw}'(T) > 0$ für $vw\in E$, so ist $l_w(T) + (\theta - T)l_w' \geq l_v(T) + (\theta - T)l_v' + \tau_{vw}$
		\item Der Fluss $\tilde{f}$ ist zulässiger Fluss über Zeit.
		\item Die zu $\tilde{f}$ gehörigen frühesten Ankunftszeiten $\tilde{l}_v(\theta)$ sind für $\theta \leq T+\alpha$ gegeben durch:
		$$\tilde{l}_v(\theta) = \begin{cases}
		l_v(\theta) & \text{ für $\theta < T$} \\
		l_v(T) + (\theta - T) l_v' & \text{ für $\theta \in [T, T+\alpha]$}
		\end{cases}$$
		\item Der Fluss $\tilde{f}$ ist Nash-Fluss über Zeit.
	\end{enumerate}
	\todo{ Zusätzliche Voraussetzung $l_w(T) - l_v(T) + \alpha(l_w' - l_v') \leq t_{vw}$, falls $vw$ nicht aktiv zum Zeitpunkt $T$. }
\end{lemma}
\begin{proof}
	Zu $(i)$: Ist $vw\in E_1$ mit $l_w'<l_v'$, so gilt
	$l_w(T)-l_v(T) + (\theta - T)(l_w' - l_v') \geq l_w(T)-l_v(T)+\alpha(l_w'- l_v')\geq \tau_{vw}$  mit der zusätzlichen Voraussetzung an $\alpha$.
	Sonst gilt $l_w' \geq l_v'$ nach $(3)$ und mit $T\in \Theta_{vw}$ folgt $l_w(T)+(\theta-T)l_w'=l_v(T) + q_{vw}(l_v(T))+\tau_{vw}+(\theta - T)l_w' \geq l_v(T) + (\theta-T)l_v'+\tau_{vw}$.
	
	Zu $(ii)$: 
\end{proof}

\clearpage          % neue Seite für Literaturverzeichnis

%%%%%%%%%%%%%%%%%%%%%%%%%%%%%%%%%%%%%%%%%%%%%%%%%%%%%%%%%%%%%%%%%%%%%%%%%%%%%%%%
% Literaturverzeichnis
\nocite*  % Nicht zitierte Quellen werden auch ins Literaturverzeichnis aufgenommen
\thispagestyle{empty}
\bibliography{literature}  % Literaturverzeichnis liegt in der Datei literature

%%%%%%%%%%%%%%%%%%%%%%%%%%%%%%%%%%%%%%%%%%%%%%%%%%%%%%%%%%%%%%%%%%%%%%%%%%%%%%%%
%%%%%%%%%%%%%%%%%%%%%%%%%%%%%%%%%%%%%%%%%%%%%%%%%%%%%%%%%%%%%%%%%%%%%%%%%%%%%%%%
% Ende des Dokuments
\end{document}		
