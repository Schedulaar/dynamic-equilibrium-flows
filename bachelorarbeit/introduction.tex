\chapter{Einführung}\label{introduction}

Routenplanungsspiele haben in der Analyse und Optimierung von Verkehrs- und Kommunikationsnetzwerken eine große Bedeutung.
Bei der Modellierung solcher Netzwerke werden dabei häufig statische Flüsse zu Rate gezogen, welche jedoch nicht abbilden können, wie sich Veränderungen des Flusses im Laufe der Zeit innerhalb des Netzwerks, wie sie beispielsweise in Straßennetzen auftreten, auf das System auswirken.
Daher geht man zu dynamischen Flüssen (engl. flows over time) über, die den zeitlichen Verlauf der Kantenbelegung erfassen.

In dieser Arbeit unterwirft man dynamische Flüsse dem Modell der deterministischen Warteschlangen (engl. deterministic queuing model):
Sobald Partikel eine Kante durchlaufen möchten, reihen sich diese zunächst in eine Warteschlange der Kante ein, die mit einer gewissen Kapazität abgearbeitet wird, um dann nach einer zusätzlichen Verzögerung am Zielknoten anzukommen.
Dabei kann man sich die Kanten eines Netzwerks als Transportbänder vorstellen, die eine bestimmte Kapazität und eine bestimmte Verzögerung haben.
Die Kapazität entspricht der Breite des Bands, welche die Gütermenge, die das Band pro Zeiteinheit transportieren kann,  beschränkt.
Die Verzögerung entspricht der Transportzeit, die das Band benötigt, um Güter vom Startknoten zum Zielknoten zu tragen.
Des Weiteren gibt es im Netzwerk eine Quelle, also einen Knoten, an dem kontinuierlich Partikel entstehen, und eine Senke, in die Partikel aus dem Netzwerk abfließen.

Diesen Flüssen unterstellt man nun die folgende spieltheoretische Interpretation:
Jedes infinitesimal kleine Partikel, das in einem Fluss an der Quelle zu einem bestimmten Zeitpunkt entsteht, wird als Spieler in einem nicht-atomaren Auslastungsspiel aufgefasst:
Dabei verfolgt jeder Spieler das Ziel, in möglichst kurzer Zeit nach Erscheinen an der Quelle zur Senke des Netzwerks zu gelangen.
Jeder Spieler besitzt dabei bereits zur Zeit seines Entstehens über alle nötigen Informationen, die seine tatsächliche Ankunftszeit an der Senke beeinflussen -- das heißt, er weiß über den künftigen Verlauf der Warteschlangen an den Kanten Bescheid --, und kann diese nutzen, um einen zur aktuellen Zeit kürzesten Quelle-Senke-Pfad zu ermitteln.
Da es sich bei den Spielern um infinitesimal kleine Partikel handelt, verändert sich der dynamische Fluss durch die Entscheidung eines einzelnen Partikels nicht.

Ein Nash-Gleichgewicht wird für eine endliche Anzahl an Spielern für gewöhnlich als eine Strategiewahl charakterisiert, in der kein Spieler seine Strategie bei Beibehalten der Wahl der anderen Spieler ändern kann, um seine eigenen Kosten echt zu verringern.
Im Fall dynamischer Flüsse wird dieser Begriff etwas aufgeweicht:
So wird ein dynamischer Fluss ein Nash-Gleichgewicht genannt, falls eine Kante von $v$ nach $w$
fast nur zu Zeitpunkten genutzt wird, zu denen die Kante auf einem kürzesten  Pfad von der Quelle zu $w$ liegt.

Diese Arbeit gibt zunächst eine formale Definition von dynamischen Flüssen in Kapitel~\ref{chapter-dynamic-flows} sowie eine Charakterisierung von Nash-Gleichgewichten in Kapitel~\ref{chapter-nash-flows}.
Danach widmet sie sich der Berechnung ebensolcher Gleichgewichte für den Fall, dass der Netzwerkzufluss an der Quelle konstant ist.
Dazu werden jedoch einige Kenntnisse über zwei Klassen von statischen Flüssen benötigt:
So werden in Kapitel~\ref{chapter-min-con-flows} zunächst Resultate der Analyse von auslastungsminimalen Flüssen dargestellt:
Dies sind $b$-Flüsse, die die maximale Auslastung aller Kanten, d.h. das Verhältnis von Fluss und Kapazität einer Kante, minimieren.
Es wird gezeigt, dass das dazu duale Problem darin besteht, einen sogenannten dünnsten Schnitt zu finden, der das Verhältnis von Nettoangebot zur Kapazität der ausgehenden Kanten des Schnittes maximiert.
Außerdem wird ein Verfahren vorgestellt, mit dem man auslastungsminimale Flüsse in polynomieller Zeit exakt bestimmen kann.
Des Weiteren führt die Analyse der Ableitung dynamischer Nash-Flüsse zum Konzept sogenannter schmaler Flüsse, die in Kapitel~\ref{chapter-thin-flows} analysiert werden.
Dies sind $b$-Flüsse in einem Netzwerk mit einer einzigen Quelle $s$ und sogenannten zurücksetzenden Kanten, in denen alle $s$-$v$-Pfade mit positivem Fluss minimale Auslastung besitzen.
Nach einer Charakterisierung dieser Flüsse wird ein polynomieller Algorithmus erarbeitet, der schmale Flüsse in Netzwerken ohne zurücksetzende Kanten berechnet.
Außerdem wird diskutiert, ob das allgemeine Problem mit zurücksetzenden Kanten in der Komplexitätsklasse \PPAD\ liegt.
Schließlich lässt sich mithilfe der Existenz solcher schmaler Flüsse ein mögliches Verfahren ableiten, mit dem Nash-Flüsse bei konstantem Netzwerkzufluss berechnet werden sollen.
