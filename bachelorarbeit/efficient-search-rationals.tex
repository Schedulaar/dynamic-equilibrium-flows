\subsection{Effiziente Suche rationaler Zahlen}

Dieser Abschnitt ist ein Exkurs, der eine effiziente Suche von rationalen Zahlen einführen soll.
Dabei werden die Ergebnisse von Christos H. Papadimitriou aus~\cite{Papadimitriou1979} erarbeitet.
Die Zielsetzung lautet dabei wie folgt:
Eine nichtnegative, rationale Zahl $x$, deren Zähler und Nenner durch eine Konstante $M$ beschränkt ist, soll durch möglichst wenige Vergleiche der Form \glqq Ist $x\leq p/q$?\grqq\ bestimmt werden.
Dabei werden mindestens $\theta(\log M)$ Vergleiche benötigt, da bereits das Suchen von $x$ in der Menge $\{ 0/M, 1/M, \dots, M/M \}$ unter der Annahme, dass $x$ durch einen Bruch mit Nenner $M$ dargestellt werden kann, $\Omega(\log M)$ Vergleiche bedarf.
Der Wunsch ist also ein Verfahren zu finden, das nur $\bigO(\log M)$ viele solcher Vergleiche verwendet.

Dazu wird die sogenannte Farey-Folge eingeführt.

\begin{definition}[Farey-Folge]
	Für ein $n\geq 1$ sei $F_n$ die aufsteigend sortierte Folge aller irreduziblen, d.h. vollständig gekürzten, Brüche in $[0,1]$, deren Zähler  und Nenner ganze Zahlen in $\{0, \dots, n\}$ sind.
	Man nennt $F_n$ die \emph{Farey-Folge der Ordnung $n$}.
\end{definition}

\begin{example}
	Die Farey-Folge der Ordnung $5$ ist definiert als:
	\[
		F_5 = \left( \frac{0}{1}, \frac{1}{5}, \frac{1}{4}, \frac{1}{3}, \frac{2}{5}, \frac{1}{2}, \frac{3}{5}, \frac{2}{3}, \frac{3}{4}, \frac{4}{5}, \frac{1}{1} \right).
	\]
\end{example}

Es wird das bekannte Lemma von Bézout (siehe~\todo{REFERENZ EINFÜGEN}) wiederholt:
\begin{lemma}[Lemma von Bézout]\label{lemma-von-bezout}
	Für zwei ganze Zahlen $a, b\in\Z$ existieren zwei weitere ganze Zahlen $s,t\in\Z$, sodass $\ggT(a,b) = sa + tb$ gilt.
\end{lemma}

\begin{corollary}\label{cor-bezout-reverted}
	Für zwei ganze Zahlen $a,b\in\Z$ gilt $\ggT(a,b) = 1$ genau dann, wenn es zwei ganze Zahlen $s,t\in\Z$ gibt mit $1 = sa + tb$.
\end{corollary}
\begin{proof}
	Das Lemma von Bézout zeigt bereits die Hinrichtung.
	Seien also $s,t\in\Z$ mit $1= sa +tb$ gegeben und sei $d$ ein gemeinsamer Teiler von $a$ und $b$.
	Dann existieren $a',b'\in\Z$ mit $a=d a'$ und $b=d b'$.
	Insbesondere folgt $1 = sa'd + tb'd = (sa' + tb') d$, wodurch $d$ bereits eine Einheit, also $1$ oder $-1$, sein muss.
	Daher gilt $\ggT(a,b)=1$.
\end{proof}

Damit lässt sich eine wichtige Eigenschaft der Farey-Folgen zeigen:
\begin{lemma}\label{lemma-basic-property-farey-successor}
	Ist $h'/k'$ der Nachfolger von $h/k$ in $F_n$, so gilt \[
	h'k - hk' = 1 \text{~~~oder äquivalent~~~} \frac{h'}{k'} - \frac{h}{k} = \frac{1}{kk'}.
	\]
\end{lemma}
\begin{proof}
	Der Beweis gibt ausgehend von $h/k$ den Nachfolger in $F_n$ an.
	Weil ein Nachfolger existieren muss, kann man $h<k$ voraussetzen.
	Da $h/k$ irreduzibel ist, gilt $\ggT(h,k)= 1$ und nach Lemma~\ref{lemma-von-bezout} existieren $x_0, y_0\in\Z$, sodass $kx_0 -hy_0 = 1$ gilt.
	Für alle $r\in\Z$ ist dann auch $k(x_0 + rh) - h(y_0 + rk) = k x_0 + hy_0 = 1$.
	Sei $r$ so gewählt, dass die Ungleichungen $n-k<y_0 +rk \leq n$ erfüllt sind, so erhält man eine Lösung $(x,y)\in\Z^2$ von $kx - hy = 1$ mit $n-k<y\leq n$, die nach Korollar~\ref{cor-bezout-reverted} auch $\ggT(x,y)=1$ erfüllt.
	Aus $kx-hy=1$ lässt sich außerdem $x/y - h/k = 1/(ky)$ sowie $x = (1+hy)/k$ und damit $x \leq y(1+h)/k \leq y$ folgern.
	Somit erscheint $x/y$ nach $h/k$ in $F_n$.
	
	Angenommen, $x/y$ sei nicht der direkte Nachfolger $h'/k'$ von $h/k$ in $F_n$.
	Dann implizieren die Ungleichungen $h/k < h'/k' < x/y$ gerade \[
	\frac{x}{y} - \frac{h'}{k'} = \frac{k'x-h'y}{yk'} \geq \frac{1}{yk'}\text{~~~sowie~~~} \frac{h'}{k'} - \frac{h}{k} = \frac{h'k - hk'}{k'k} \geq  \frac{1}{kk'}. \]
	Benutzt man dies zusätzlich zu $y+k > n$, kann man folgenden Widerspruch schließen:
	\[
		\frac{1}{ky} = \frac{x}{y} - \frac{h}{k} \geq \frac{h'}{k'} + \frac{1}{yk'}- \frac{h}{k} \geq \frac{1}{kk'} + \frac{1}{yk'} = \frac{y+k}{kk'y} > \frac{n}{kk'y} \geq \frac{1}{ky}.
	\]
\end{proof}

\begin{lemma}\label{lemma-farey-middle-member}
	Seien drei Brüche $x_i/y_i < x_j / y_j < x_k / y_k$ in $F_n$ gegeben.
	Gilt die Gleichung $y_i x_k - y_k x_i = 1$, so existieren zwei ganze Zahlen $\alpha,\beta\in\Z$, die $x_j = \alpha x_i + \beta x_k$, $y_j = \alpha y_i + \beta y_k$ sowie $\ggT(\alpha, \beta)=1$ erfüllen.
\end{lemma}
\begin{proof}
	Man definiere $\alpha:= y_j x_k - y_k x_j\in\Z$ und $\beta:=y_i x_j - y_j x_i\in\Z$.
	Dann gilt  $\alpha = (x_k/y_k - x_j/y_j)y_k y_j > 0$ und analog $\beta > 0$.
	Es folgen die beiden Gleichungen:
	\begin{align*}
	\alpha x_i + \beta x_k &= x_j(x_k y_i - y_k x_i ) + y_j (x_k x_i - x_k x_i) = x_j, \\[1em]
	\alpha y_i + \beta y_k &= x_j (y_i y_k - y_k y_i) + y_j (x_k y_i - x_i y_k) = y_j.
	\end{align*}
	Nach Lemma~\ref{lemma-von-bezout} existieren wegen der Irreduzibilität von $x_j/y_j$ außerdem $s,t\in\Z$ mit $sx_j + ty_j = 1$.
	Betrachtet man $(sx_i + ty_i)\alpha + (sx_k + ty_k)\beta = s x_j + t x_j = 1$, folgt $\ggT(\alpha, \beta) = 1$ mit Korollar~\ref{cor-bezout-reverted}.
\end{proof}

\begin{lemma}\label{lemma-fraction-helpers}
	Seien $a,b,c,d\in\R_{>0}$ mit $a/b < c/d$ gegeben.
	Für alle $\alpha, \beta \in\R_{>0}$ gilt
	\[
		\frac{a}{b} < \frac{\alpha a + \beta c}{\alpha b + \beta d} < \frac{c}{d}.
	\]
	Außerdem ist der Bruch $(\alpha a + \beta c)/(\alpha b +  \beta d)$ als Funktion in $\alpha$ streng monoton fallend und als Funktion in $\beta$ streng monoton wachsend.
\end{lemma}
\begin{proof}
	Zunächst folgt aus $a/b < c/d$ sofort $ad<cb$.
	Der erste Teil der Aussage ist Resultat der folgenden Ungleichungen:
	\begin{align*}
		a(\alpha b + \beta d) - (\alpha a + \beta c)b = \beta(ad - cb) < 0,\\[1em]
		(\alpha a + \beta c)d - c(\alpha b + \beta d) = \alpha(ad - cb) < 0.
	\end{align*}
	Zudem kann man mit $0\leq\alpha_1< \alpha_2$ sowie $0\leq\beta_1< \beta_2$ die Monotonien des erweiterten Bruches zeigen:
	\begin{align*}
		(\alpha_1 a + \beta c)(\alpha_2 b + \beta d) - (\alpha_2a + \beta c)(\alpha_1 b + \beta d) &= \beta a d(\alpha_1 - \alpha_2) + \beta c b (\alpha_2 - \alpha_1) > 0, \\[1em]
		(\alpha a + \beta_1 c)(\alpha b + \beta_2 d) - (\alpha a + \beta_2c)(\alpha b + \beta_1 d) &= \alpha ad(\beta_2 - \beta_1) + \alpha cb(\beta_1 - \beta_2) < 0.
	\end{align*}
\end{proof}

Nun wird der Algorithmus erarbeitet.
Dazu betrachtet man zunächst den Fall $x\in(0,1)$.
Die Grundidee ist der binären Suche entnommen:
Beginnend bei $(0,1]$ wird ein Suchintervall verwaltet, in dem sich die gesuchte Zahl $x$ befindet.
Dazu wird in jeder Runde eine beste obere Schranke $x_r/y_r$ an $x$ in einer Farey-Folge steigender Ordnung ermittelt und der Suchraum durch das Intervall $(x_l/y_l, x_r/y_r]$ beschränkt, wobei $x_l/y_l$ der Vorgänger von $x_r/y_r$ in der Farey-Folge ist, bis man aufgrund der Endlichkeit des Suchraums die gesuchte Zahl erreicht.

Für die Bestimmung der Schranken werden einige Eigenschaften der Farey-Folgen diskutiert.
Zunächst wird der Suchraum, aus dem die Schranken bestimmt werden können, analysiert:

\begin{lemma}\label{lemma-candidates-next-farey-bound}
	Sei $x\in(0,1]$ eine rationale Zahl und sei $x_r/y_r$ die kleinste obere Schranke an $x$ in $F_n$ und sei $x_l/y_l$ der Vorgänger von $x_r/y_r$ in $F_n$.
	Dann sind alle Brüche der Farey-Folge von Ordnung $2n$ in $[x_l/y_l, x_r/y_r]$ gegeben durch
	\[
		M:= \{ x_l/y_l \} \cup \left\{ \frac{\alpha x_l + \beta x_r}{\alpha y_l + \beta y_r} ~\middle\vert~ \alpha,\beta\in\N, \alpha y_l + \beta y_r\leq 2n \right\} \cup \{ x_r/y_r \}.
	\]
	Außerdem gilt $\alpha y_l + \beta y_r > n$ für alle $\alpha,\beta \in\N$.
\end{lemma}
\begin{proof}
	Sei zunächst ein Bruch $x/y$ der Farey-Folge von Ordnung $2n$ gegeben.
	Ist $x/y$ bereits $x_l/y_l$ bzw. $x_r/y_r$, so wähle $\alpha = 1, \beta = 0$ bzw. $\alpha = 0, \beta = 1$.
	Ist $x/y$ jedoch echt zwischen $x_l/y_l$ und $x_r/y_r$ so sind $x_l/y_l$ und $x_r/y_r$ Nachbarn in der Farey-Folge von Ordnung $n$ und somit gilt nach Lemma~\ref{lemma-basic-property-farey-successor} bereits $x_ly_r - x_ry_l = 1$.
	Fässt man $x_l/y_l$ und $x_r/y_r$ nun als Folgenglieder der Farey-Folge von Ordnung $2n$ auf, existieren nach Lemma~\ref{lemma-farey-middle-member} für das Zwischenglied $x/y$ die Darstellungen $x = \alpha x_l + \beta x_r$ und $y= \alpha y_l + \beta y_l$ mit $\alpha, \beta \in\N$.
	
	Sei umgekehrt $x/y$ aus $M$.
	Ist der Bruch $x/y$ bereits $x_l/y_l$ oder $x_r/y_r$, so ist er in der Farey-Folge von Ordnung $n$, also erst recht in der von Ordnung $2n$.
	Sonst hat $x/y$ die Darstellung $x=\alpha x_l + \beta x_r$ und $y=\alpha y_l + \beta y_r$ mit $\alpha,\beta\in\N$ und $\alpha y_l + \beta y_r \leq 2n$.
	Angenommen der größte gemeinsame Teiler von $x$ und $y$ wäre $d\geq 2$.
	Dann gäbe es ganze Zahlen $z_1, z_2 \leq n$ mit $x=d z_1$ und $y=d z_2$.
	Insbesondere würde die irreduzible Darstellung von $x/y$ bereits in $F_n$ liegen.
	Dies widerspricht aber der Optimalität der Schranken, denn Lemma~\ref{lemma-fraction-helpers}~$(ii)$ impliziert $x_l / y_l < x/y < x_r/y_r$, und so könnte mindestens eine Schranke verbessert werden.
	Also ist $x/y$ in $F_{2n}$ enthalten.
	
	Zuletzt ist zu zeigen, dass für alle $\alpha,\beta\in\N$ der Ausdruck $\alpha y._l + \beta y_r$ größer als $n$ ist.
	Wäre dies nicht der Fall, so wäre $(\alpha x_l + \beta x_r) / (\alpha y_l + \beta y_r)$ nach Proposition~\ref{lemma-fraction-helpers}~$(ii)$ ein Bruch in $F_n$, der echt zwischen $x_l/y_l$ und $x_r/y_r$ liegt.
	Jedoch ist $x_l/y_l$ bereits der Vorgänger von $x_r/y_r$ in $F_n$.
\end{proof}

\begin{theorem}\label{theorem-find-fraction}
	Eine rationale Zahl $x\in[0,1)$, deren Zähler und Nenner durch $M$ beschränkt sind, kann mittels $\bigO(\log M)$ vielen Vergleichen der Form \glqq Ist $x\leq p/q$?\grqq\ mit $p,q\leq 2M$ und $\bigO(\log M)$ vielen arithmetischen Operationen auf ganzen Zahlen von Betrag maximal $2M$ bestimmt werden.
\end{theorem}
\begin{proof}
	Ist $x\leq 0/1$, so kann sofort $x=0$ geschlossen werden, sodass im Weiteren von $x\in(0,1)$ ausgegangen wird.
	
	Man betrachte zunächst den Fall, dass man über die beste obere Schranke $x_r/y_r$ an $x$ in $F_{2^k}$ sowie über dessen Vorgänger $x_l/y_l$ in $F_{2^k}$ verfügt und nun die beste obere Schranke $x_r'/y_r'$ in $F_{2^{k+1}}$ und dessen Vorgänger $x_l'/y_l'$ ermitteln möchte.
	Nach Lemma~\ref{lemma-candidates-next-farey-bound} kommen dabei als Kandidaten für $x_l'/y_l'$ und $x_r'/y_r'$ genau die Brüche der Form $(\alpha x_l + \beta x_r)/(\alpha y_l + \beta y_r)$ mit $\alpha,\beta\in\N$ und $\alpha y_l + \beta y_r \leq 2^{k+1}$ in Frage; für $x_l'/y_l'$ bzw. $x_r'/y_r'$ ist außerdem $x_l/y_l$ bzw. $x_r/y_r$ möglich.
	
	Unter diesen Brüchen ist für $x_r/y_r$ der kleinstmögliche gesucht, der $x\leq x_r/y_r$ erfüllt, und für $x_l/y_l$ der größtmögliche, für den $x_l/y_l < x_r/y_r$ gilt.
	Nach Proposition~\ref{lemma-farey-middle-member} ist $y_l + y_r > 2^k$.
	Daher ist $y_l$ oder $y_r$ größer als $2^{k-1}$.
	Aus Symmetriegründen sei im Folgenden ohne Einschränkung $y_l$ größer als $2^{k-1}$.
	Nun unterscheide man drei Fälle:
	
	\begin{description}
		\item[1. Fall:] $y_r \geq 2^{k-2}$.
		
		In diesem Fall sind alle Möglichkeiten an $(\alpha,\beta)\in\N^2$ mit $\alpha y_l + \beta y_r \leq 2^{k+1}$ enthalten in der Menge $\{ (1,1), (1,2),\dots,(1,5), (2,1), (2,2),(2,3),(3,1) \}$.
		Hier kann mit konstant vielen Vergleichen der Form \glqq Ist $x\leq p/q$?\grqq\ und arithmetischen Operationen die beste obere Schranke an $x$ in $F_{2^{k+1}}$ und den Vorgänger der Schranke finden.
		
		\item[2. Fall:] $x > \frac{x_l'}{y_l'} := \frac{\alpha x_l + \beta x_r}{\alpha y_l + \beta y_l}$ mit $\alpha := 1$ und $\beta := \lfloor \frac{2^{k+1} - y_l}{y_r}\rfloor$.
		
		Lemma~\ref{lemma-fraction-helpers}~$(ii)$ impliziert, dass $x_l'/y_l'$ der direkte Vorgänger von $x_r/y_r$ in $F_{2^{k+1}}$ ist, denn für $\alpha$ ist $1$ der minimale Wert und für $\beta$ ist der maximale Wert, der $y_l + \beta y_r \leq 2^{k+1}$ noch erfüllt, gerade $\lfloor (2^{k+1} - y_l)/y_r \rfloor$.
		Des Weiteren ist $\beta$ mindestens $\lfloor y_r/y_r  \rfloor = 1$.
		Dementsprechend wird als neue obere Schranke die vorherige Schranke übernommen, d.h. $x_r'/y_r' = x_r/y_r$, mit Vorgänger $x_l'/y_l'$ in $F_{2^{k+1}}$.
		
		\item[3. Fall:]~Weder Fall~1 noch Fall~2 treten ein.
		
			Man betrachte diesen Fall nur, falls Fall~1 und Fall~2 nicht eintreten.
			Das bedeutet, dass $y_r$ kleiner als $2^{k-2}$ ist und die gesuchte obere Schranke $x_r'/y_r'$ kleiner als die aktuelle Schranke $x_r/y_r$ sein muss.
			Also hat die gesuchte Schranke die Form $(\alpha x_l + \beta x_r)/(\alpha y_l + \beta y_r)$ mit $\alpha,\beta\in\N$ und $\alpha y_l + \beta y_r\leq 2^{k+1}$.
			Wegen $x_l > 2^{k-1}$ kann man die Suche dabei auf $\alpha \in \{ 1, 2, 3 \}$ beschränken.
			Für jedes $\alpha\in\{1,2,3\}$ bestimmt man nun das $\beta_\alpha$, sodass der entsprechende Bruch $x$ von oben möglichst gut beschränkt:
			\[
				\frac{\alpha x_l + (\beta_\alpha - 1) x_l}{\alpha y_l + (\beta_\alpha - 1) y_r} < x \leq \frac{\alpha x_l + \beta_\alpha x_l}{\alpha y_l + \beta_\alpha y_l}.
			\]
			
			Unter diesen Brüchen wird nun $x_r'/y_r'$ und $x_l'/y_l'$ bestimmt durch
			\[\frac{x_r'}{y_r'}:=\min_{\alpha\in[3]} \frac{\alpha x_l + \beta_\alpha x_r}{\alpha y_l + \beta_\alpha y_r}
			\text{~~~~~und~~~~~}
			\frac{x_l'}{y_l'}:=\max_{\alpha\in[3]} \begin{cases}
			\frac{\alpha x_l + (\beta_\alpha - 1)x_r}{\alpha y_l + (\beta_\alpha - 1)y_r}, &\text{falls $\beta_\alpha > 1$,}\\
			\frac{x_l}{y_l} &\text{sonst.}
			\end{cases}
		\]
			
			Die $\beta_\alpha$ können wegen der Monotonie nach Lemma~\ref{lemma-fraction-helpers} mittels binärer Suche auf $\beta \in \{1,\dots, \lfloor (2^{k+1} - \alpha y_l) / y_r \rfloor \}$ ermittelt werden.
			Dabei werden insgesamt maximal $3 \cdot \lfloor \log_2(2^{k+1} / y_r) + 1 \rfloor \leq 3 (k+2 - \log_2(y_r))$ Vergleiche mit $x$ benötigt.
			Man bemerke für die spätere Analyse, dass $y_r'$ in diesem Fall größer als $2^k$ ist.
 	\end{description}
 	
 	Der Algorithmus wird mit $k=0$, $x_l/y_l$ = $0/1$ sowie $x_r/y_r = 1/1$ initialisiert.
 	Dann wird das Suchintervall wie oben beschrieben in $n:=\lceil \log_2(M) \rceil$ Runden verkleinert.
 	In Runde $k$ soll also ausgehend von einer besten Schranke (und ihrem Vorgänger) in $F_{2^{k-1}}$ eine beste Schranke (und ihr Vorgänger) in $F_{2^k}$ ermittelt werden.
 	Schließlich erhält man eine niedrigste obere Schranke $x_r/y_r$ an $x$ in $F_{2^n}$.
 	Da eine Darstellung von $x$ selbst in $F_{2^n}$ existiert, muss $x_r/y_r$ also bereits mit $x$ übereinstimmen.
 	
 	Es bleibt die Analyse der Laufzeit.
 	Wie bereits argumentiert, werden $\bigO(\log M)$ Runden genutzt.
 	Wird in einer Runde nur Fall~1 oder Fall~2 betrachtet, so werden nur konstant viele Vergleiche und Operationen benötigt.
 	Jedoch braucht eine einzelne Ausführung von Fall~3 aufgrund der binären Suche unter Umständen $\bigO(\log M)$ viele Vergleiche.
 	Hier hilft eine genauere Analyse der akkumulierten Anzahl von Vergleichen, die durch Fall~3 ausgelöst werden.
 	
 	Dazu zeigt man, dass bis zur $k$-ten Runde des Algorithmus für den dritten Fall insgesamt nur maximal $12k$ Vergleiche mit $x$ unternommen werden.
 	Dies zeigt man per Induktion über die Anzahl $i$ der Ausführungen des dritten Falls, wobei der Fall $i=0$ klar ist.
 	Ist $i=1$ -- tritt also Fall 3 das erste Mal ein --, so werden maximal $3(k+2) \leq 12k$ Vergleiche benötigt.
 	Für $i\geq2$ sei $k'$ die letzte Runde, zu der Fall 3 angewandt wurde.
 	Bis zur Runde $k'$ wurden also nach Induktionsvoraussetzung höchstens $12k'$ Vergleiche benötigt.
 	Die obere Schranke $x_r^{k'} / y_r^{k'}$, die in Runde $k'$ im Fall $3$ ausgewählt wurde,  erfüllt außerdem $y_r^{k'} > 2^{k'-1}$.
 	Da der Nenner der oberen Schranke im Laufe der Runden nur wachsen kann, gilt $y_r^{k - 1} \geq 2^{k'-1}$ für die obere Schranke $x_r^{k-1}/y_r^{k-1}$, die in Runde $k-1$ ausgewählt wurde.
 	Daher ist die Anzahl an Vergleichen, die im dritten Fall in Runde $k$ gemacht werden, höchstens \[
 	3(k+2-\log_2(y_r^{k' - 1}))\leq 3(k - k' + 3).
 	\]
 	Insgesamt werden also bis Runde $k$ zur Ausführung von Fall 3 höchstens \[
 	12k' + 3(k-k'+3) = 3(3k' + k + 3) \leq 3(3k' + k + 3(k-k')) = 12 k
 	\]
 	Vergleiche mit $x$ benötigt, womit die Induktionsbehauptung gezeigt ist.
 	Damit werden von Fall~3 ebenfalls $\bigO(\log M)$ Vergleiche mit $x$ ausgelöst.
 	Darüber hinaus wird $x$ nur mit Brüchen verglichen, die im Zähler und Nenner durch $2^{\lceil \log_2(M) \rceil}\leq 2M$ beschränkt sind, wodurch das Theorem folgt.
 \end{proof}

\begin{theorem}
	Eine nichtnegative, rationale Zahl $x$, deren Zähler und Nenner durch $M$ beschränkt sind, kann mittels $\bigO(\log M)$ vielen Vergleichen der Form \glqq Ist $x\leq p/q$?\grqq\ mit $p,q\leq 2M$ und $\bigO(\log M)$ vielen arithmetischen Operationen auf ganzen Zahlen von Betrag maximal $8M$ bestimmt werden.
\end{theorem}
\begin{proof}
	\newcommand{\lowx}{\lfloor x \rfloor}
	Zunächst wird mittels binärer Suche auf der Menge $A:=\{0,\dots, M\}$ und $\bigO(\log M)$ Vergleichen $\lowx$ ermittelt.
	Es existiert ein irreduzibler Bruch $a/b\in[0,1)$ mit $x=\lowx + a/b$.
	Für den Fall $\lowx = 0$ wende man direkt Theorem~\ref{theorem-find-fraction} an.
	Sonst ist $(\lowx b + a)/b$ die irreduzible Darstellung von $x$, für die $b = (\lowx b+a)/x \leq M/\lowx$ gilt.
	Mit Theorem~\ref{theorem-find-fraction} kann nun die Zahl $y:=x-\lowx\in(0,1)$ gefunden werden, wobei Zähler und Nenner auf $\lceil M/\lowx \rceil$ beschränkt sind, indem man Vergleiche der Form \glqq Ist $y\leq p/q$?\grqq\ mit dem Resultat des Vergleichs \glqq Ist $x\leq (p + zq)/q$?\grqq\ beantwortet.
	Dabei ist der Zähler beschränkt durch \begin{align*}
	p+\lowx q &\leq 2 \left\lceil \frac{M}{\lowx} \right\rceil (1+\lowx) \leq 2 \left(\frac{M}{\lowx} + 1\right)(1+\lowx)
	= 2\left(\frac{M}{\lowx} + M + \lowx + 1\right) \leq 8M.
	\end{align*}
	Schließlich gelangt man durch Addition von $y$ und $\lowx$ zu $x$.
\end{proof}

Papadimitriou hat dieses Resultat verwendet, um zu zeigen, dass man lineare Optimierungsprobleme genau dann in polynomieller Zeit lösen kann, wenn man in polynomieller Zeit entscheiden kann, ob ein Polyeder zulässig ist.
Schließlich wurde das letztere Problem durch Leonid Khachiyan mit der Ellipsoidmethode gelöst.
Ein ähnliches Ergebnis wie hier vorgestellt konnten Stephen Kwek und Kurt Mehlhorn 2003 in~\cite{Kwek:2003} erzielen:
Diese haben ein Verfahren vorgestellt, das sogar nur maximal $2\log_2(M) + \bigO(1)$ Vergleiche zur Bestimmung einer rationalen Zahl, deren Zähler und Nenner durch $M$ beschränkt sind, benötigt.