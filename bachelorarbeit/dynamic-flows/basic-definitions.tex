\section{Grundlegende Definitionen}

Zunächst werden einige grundlegende Begriffe eingeführt.
In der gesamten Arbeit werden grundsätzlich nur gerichtete Graphen mit endlicher Knoten- und Kantenmenge betrachtet.
Dabei bezeichne $\tail(e)$ den Start- und $\head(e)$ den Zielknoten einer Kante $e$.
Außerdem sind parallele Kanten zwischen Knoten stets erlaubt, obwohl häufig die vereinfachte Schreibweise $vw\coloneq (v,w)\coloneq e$ mit $v=\tail(e)$ und $w=\head(e)$ benutzt wird.

Für eine Menge $X$ an Knoten bezeichne $\delta^+(X)\coloneq \{ e\in E \mid \tail(e) \in X \notni \head(e) \}$ die Menge der ausgehenden Kanten von~$X$ und analog bezeichne $\delta^-(X)$ die Menge der eingehenden Kanten in~$X$.
Ist der zugehörige Graph unklar, so schreibt man $\delta^+_M(X)$ bzw. $\delta^-_M(x)$ für ein Netzwerk, Graph oder eine Kantenmenge $M$.
Für die ausgehenden oder eingehenden Kanten eines einzelnen Knotens schreibt man verkürzend $\delta^+(v)$ bzw. $\delta^-(v)$.

Ist weiter $f: E \rightarrow \R$ eine Kantenbewertung und $b: V \rightarrow \R$ eine Knotenbewertung, so schreibt man meist $f_e$ statt $f(e)$ und $b_v$ statt $b(v)$ und für Teilmengen $E'\subseteq E$ und $V'\subseteq V$ abkürzend
\[ 
	f(E')\coloneq \sum_{e \in E'} f_e \text{~~~ und ~~~} b(V')\coloneq\sum_{v\in V'} b_v.
\]

Ein \emph{Pfad} $P=(e_1, \dots, e_k)$ ist eine Aneinanderreihung von Kanten, das heißt, es gilt $\head(e_i) = \tail(e_{i+1})$ für alle $i\in[k-1]$.
Dabei bezeichnet $[n]$ die Menge der ersten $n$ natürlichen Zahlen, also $\{ 1,\dots, n \}$.
Die Kanten eines Pfades werden in der Menge $E(P)$, die Knoten in der Menge $V(P)$ gesammelt.
Ein Pfad heißt \emph{Weg}, falls kein Knoten mehrmals besucht wird. 
Ein Pfad heißt \emph{Kreis}, falls $\tail(e_1) = \head(e_k)$ gilt, und \emph{Zyklus} oder \emph{elementarer Kreis}, falls zusätzlich $\tail(e_1)$ genau zweimal und sonst kein Knoten mehrmals besucht werden.
Ein Weg oder Zyklus $P$ wird oft als Vektor in $\R^E$ aufgefasst, der als Einträge $1$-en auf Kanten in $P$ und sonst $0$-en enthält.

\begin{definition}
	Ein \emph{Netzwerk $(V, E, u)$} ist ein gerichteter, endlicher Graph $(V, E, u)$ mit \emph{Kapazitäten} $u: E \to \R_{>0}$.
	Im Falle statischer Flüsse ist ein Netzwerk häufig mit Balancen $b:V\to\R$ ausgestattet.
	
	Ein \emph{Dynamisches Netzwerk} $(V, E, u, s, t, \tau)$ ist ein Netzwerk $(V, E, u)$, in dem $s$ alle Knoten in $V$ erreicht.
	Dabei heißen $s\in V$ die \emph{Quelle} und $t\in V$ die \emph{Senke} des Netzwerks.
	Jeder Kante $e\in E$ wird zusätzlich eine \emph{Verzögerungszeit} $\tau_e\geq 0$ zugeordnet, wobei alle Zyklen $C$ eine positive Gesamtverzögerung $\tau(E(C))$ haben.
\end{definition}
\begin{definition}[Statischer Fluss]
	Eine Kantenbewertung $f\in\R_{\geq 0}^E$ in einem gerichteten Graphen $(V, E)$ heißt \emph{statischer Fluss}.
	Sein \emph{Balancevektor $b\in \R^V$} ist gegeben durch
	\[ b_v \coloneq f(\delta^+(v)) - f(\delta^-(v)) \text{~~~ für $v\in V$}. \]
	Man nennt $f$ auch einen \emph{$b$-Fluss} oder sagt, \emph{$f$ gewähre Flusserhaltung bzgl. $b$}.
	Gibt es zwei Knoten $s$ und $t$, sodass $b_v$
	für alle Knoten $v\in V\setminus\{ s, t \}$ verschwindet, für $s$ nicht-negativ und für $t$ nicht-positiv ist, so ist $f$ ein \emph{statischer $s$-$t$-Fluss mit Wert $b_s$}.
	Ist $b$ der Nullvektor, so nennt man $f$ auch eine Strömung.
\end{definition}

Man bemerke, dass ein statischer Fluss keine Kapazitätsbedingung erfüllen muss.
Außerdem sind für eine Strömung nur nichtnegative Werte zugelassen.
Eine wichtige Aussage bei der Anaylse von statischen Flüssen liefert der Dekompositionssatz (siehe \cite[Satz 8.8]{Korte2012}):
\begin{theorem}[Dekompositionssatz]\label{thm-decomposition}
	Ein statischer $s$-$t$-Fluss $f$ besitzt eine Dekomposition in elementare $s$-$t$-Wege und Zyklen, das heißt es existieren $s$-$t$-Wege $P_1,\dots,P_k$ und Zyklen $C_1, \dots, C_l$ sowie $\lambda_1,\dots,\lambda_k,\mu_1,\dots,\mu_l > 0$ mit \[
		f = \sum_{i\in[k]}\lambda_i P_i + \sum_{i\in[l]} \mu_i C_i.
	\]
	Ist $f$ eine Strömung, so gibt es eine Dekomposition in Zyklen.
\end{theorem}

Die Einführung dynamischer Flüsse und kürzester Wege folgt im Wesentlichen der Darstellung von Cominetti, Correa und Larré aus~\cite{Cominetti2015} mit Ergänzungen von Ronald Koch und Martin Skutella aus~\cite{Koch2011}.
So haben Cominetti u. a. dynamische Flüsse beispielsweise statt Lebesgue-integrierbarer Funktionen mit lokal Lebesgue-integrierbaren Funktionen ausgestattet, die den Vorteil bieten, über den gesamten Zeitraum $[0, \infty)$ unendlich viel Fluss zu schicken.
Dazu wird der folgende Funktionenraum eingeführt:

\begin{definition}
	Der Raum $\mathfrak{F}_0$ sei die Menge der Funktionen $g: \R \to \R_{\geq 0}$, die lokal integrierbar bzgl. des Lebesgue-Maßes sind und auf der negativen Achse verschwinden, die also $\int_a^b |g(t)| \diff t< \infty$ für beliebige beschränkte Intervalle $(a,b)$ und $g(t)=0$ für $t<0$ erfüllen.
\end{definition}

\begin{definition}[Dynamischer Fluss]
	Ein \emph{dynamischer Fluss $f=(f^+, f^-)$} ist ein Paar zweier über die Kanten $E$ eines dynamischen Netzwerks indizierter Familien mit $f^+_e,f^-_e\in\mathfrak F_0$ für alle $e\in E$.
	Dabei bezeichnen $f_e^+(\theta)$ bzw. $f_e^-(\theta)$ die \emph{Zu- bzw. Abflussrate an Kante $e\in E$ zum Zeitpunkt $\theta\in\R$}.
	
	Der (kumulative) \emph{Zu- bzw. Abfluss an einer Kante $e$ bis zum Zeitpunkt $\theta$} sei definiert durch $F^+_e(\theta)\coloneq\int_0^\theta f^+_e(t) \diff t<\infty$ bzw. $F^-_e(\theta)\coloneq\int_0^\theta f^-_e(t) \diff t<\infty$.
	
	Die \emph{(Länge der) Warteschlange $z_e(\theta)$ und die Wartezeit $q_e(\theta)$ an einer Kante $e\in E$ zum Zeitpunkt $\theta\in\R$} seien gegeben durch $z_e(\theta)\coloneq F_e^+(\theta) - F_e^-(\theta + \tau_e)$ und $q_e(\theta) \coloneq z_e(\theta) / u_e$.
	
	Man beschreibe die \emph{Austrittszeit $T_e(\theta)$ aus einer Kante $e\in E$ bei Eintrittszeit $\theta$}, zu der ein Partikel eine Kante verlässt, die es zum Zeitpunkt $\theta$ betreten hat, als $T_e(\theta)\coloneq\theta + q_e(\theta) + \tau_e$.
\end{definition}

Der Definition der Austrittszeit kann man bereits entnehmen, wie sich Partikel, die an einer Kante $e$ ankommen, verhalten sollen:
Nachdem sie zur Zeit $\theta$ die Kante betreten haben, müssen sie sich zunächst in eine Warteschlange einreihen, welche mit der Kapazität $u_e$ nach dem FIFO-Prinzip (First-In-First-Out-Prinzip) abgebaut wird.
Nachdem diese Wartezeit $q_e(\theta)$ vorüber ist, vergeht eine weitere konstante Verzögerungszeit $\tau_e$, bevor sie wieder aus der Kante austreten.
Des Weiteren müssen sich die Partikel bereits sofort bei der Ankunft an einem Knoten entscheiden, in welche Kante sie eintreten wollen, und können nicht an einem Knoten verweilen.
Um das beschriebene Verhalten zu gewährleisten, führt man im Folgenden die Zulässigkeit dynamischer Flüsse ein:

\begin{definition}[Zulässiger dynamischer Fluss]
	Ein dynamischer Fluss $f$ heißt \emph{zulässig}, falls er die folgenden Eigenschaften erfüllt:
	\begin{enumerate}[label=(F\arabic*)]
		\item\label{def-feasible-flow-capacity} Keine Abflussrate übersteigt die Kapazität, d.h. $\forall e\in E, \theta\in\R: f_e^-(\theta)\leq u_e$.
		\item\label{def-feasible-flow-no-negative-flow} Fluss verlässt eine Kante nur, falls er sie zuvor betreten hat,\\ d.h. $\forall e\in E, \theta\in\R: F_e^+(\theta) \geq F_e^-(\theta + \tau_e).$
		\item\label{def-feasible-flow-no-flow-at-node} Bis auf Quelle und Senke erfüllt jeder Knoten $v$ Flusserhaltung,\\
		d.h. $\forall\theta\in\R: \sum_{e\in\delta^+(v)}f^+_e(\theta) - \sum_{e\in\delta^-(v)} f_e^-(\theta) = 0$.\\
		Für die Senke $t$ muss dieser Wert nicht-positiv und für die Quelle $s$ nicht-negativ sein. 
		Für $s$ bezeichnet er den \emph{Zufluss $d(\theta)$ in das Netzwerk}.
		\item\label{def-feasible-flow-queue-with-capacity} Warteschlangen werden mit der Kapazität der Kante abgebaut,\\ d.h. $\forall e\in E, \theta\in\R: z_e(\theta) > 0 \implies f_e^-(\theta + \tau_e) = u_e$.
	\end{enumerate}
\end{definition}

Die folgende Proposition beschreibt wichtige Folgerungen über zulässige dynamische Flüsse:

\begin{proposition}\label{prop-feasible-flow}
	Für eine Kante $e\in E$ und einen zulässigen dynamischen Fluss $f$ gilt:
	\begin{enumerate}[label=(\roman*)]
		\item\label{prop-feasible-flow-T-mon-inc-cont} Die Funktion $\theta \mapsto \theta + q_e(\theta)$ ist monoton wachsend und stetig.
		\item\label{prop-feasible-flow-positive-queue} Für alle $\theta\in\R$ ist die Warteschlange $z_e$ auf dem Intervall $(\theta, \theta + q_e(\theta))$ positiv.
		\item\label{prop-feasible-flow-det-outflow} Zu jeder Zeit $\theta\in\R$ gilt $F_e^+(\theta) = F_e^-(T_e(\theta))$.
		\item\label{prop-feasible-flow-queue-delay} Für alle $\theta_1 \leq \theta_2$ mit $\int_{\theta_1}^{\theta_2} f^+_e(t) \diff t = 0$ und $q_e(\theta_2)>0$ gilt $\theta_1 + q_e(\theta_1) = \theta_2 + q_e(\theta_2)$.
	\end{enumerate}
\end{proposition}
\begin{proof}
	In~\ref{prop-feasible-flow-T-mon-inc-cont} folgt die Stetigkeit bereits aus der Stetigkeit von $F_e^+$ und $F_e^-$.
	Um zu zeigen, dass die Funktion monoton wachsend ist, seien $\theta_1 \leq \theta_2$ gegeben.
	Mit $F_e^-(\theta_2 + \tau_e) = F_e^-(\theta_1+\tau_e) + \int_{\theta_1+\tau_e}^{\theta_2+\tau_e} f_e^-(t)\diff t\leq F_e^-(\theta_1 + \tau_e) + (\theta_2 - \theta_1)u_e$ und mit der Monotonie von $F_e^+$ folgt: 
	\[
		\theta_1 + q_e(\theta_1)
		= \theta_1 + \frac{F_e^+(\theta_1) - F_e^-(\theta_1 + \tau_e)}{u_e}
		\leq \theta_1 + \frac{F_e^+(\theta_2) - F_e^-(\theta_1+\tau_e)}{u_e} \leq \theta_2 + q_e(\theta_2).
	\]
	
	Für $\theta'\in (\theta, \theta+q_e(\theta))$ gilt also $\theta' + q_e(\theta') \geq \theta + q_e(\theta)$, womit $q_e(\theta') \geq \theta + q_e(\theta) - \theta' > 0$ gerade Aussage (ii) beweist.
	
	Aussage (iii) folgt dann mit~\ref{def-feasible-flow-queue-with-capacity} und Aussage~(ii), weil daraus
	\[ 
	\int_{\theta}^{\theta + q_e(\theta)}f_e^-(t + \tau_e) \diff t = q_e(\theta)  u_e = z_e(\theta)
	\]
	folgt, weshalb $F_e^-(T_e(\theta)) = F_e^-(\theta+\tau_e) + \int_{\theta+\tau_e}^{\theta+\tau_e+q_e(\theta)}f_e^-(t)\diff t = F_e^+(\theta)$ gilt.
	
	Zu Aussage (iv):
	Für alle $\theta'\in [\theta_1, \theta_2]$ gilt $F_e^+(\theta') = F_e^+(\theta_2)$.
	Also ist die Warteschlange $z_e(\theta') = F_e^+(\theta_2) - F_e^-(\theta' + \tau_e) \geq z_e(\theta_2)$ positiv und nach~\ref{def-feasible-flow-queue-with-capacity} gilt $f_e^-(\theta' + \tau_e)=u_e$.
	Die Differenz der Warteschlangen erfüllt
	\[
	z_e(\theta_1)-z_e(\theta_2)=-F^-_e(\theta_1 + \tau_e) + F^-_e(\theta_2 + \tau_e) = (\theta_2 - \theta_1)u_e,
	\]
	was $q_e(\theta_1) - q_e(\theta_2) = \theta_2 - \theta_1$ impliziert.
\end{proof}