\chapter{Auslastungsminimale $b$-Flüsse}

Ein effiziente Berechnung von Nash-Gleichgewichten erfordert zunächst Verständnis von speziellen Klassen von statischen Flüssen.
Eine wichtige Klasse sind die sogenannten auslastungsminimalen $b$-Flüsse:

\begin{definition}[Auslastungsminimaler $b$-Fluss]
	Sei ein $b$-Fluss $f$ auf einem Netzwerk mit Kapazitäten $u\in\R_{>0}^E$ gegeben.
	Dann heißt $f_e/u_e$ die \emph{Auslastung der Kante $e$ durch $f$} und die maximale Kantenauslastung $c(f):=\max_{e\in E} f_e/u_e$ bezeichnet die \emph{Auslastung des Flusses $f$}.
	Eine Kante mit Auslastung $c(f)$ wird auch als \emph{Flaschenhalskante} (engl. bottleneck edge) bezeichnet.
	Ein $b$-Fluss mit minimaler Auslastung wird dann \emph{auslastungsminimaler $b$-Fluss} genannt.
\end{definition}

In diesem Kapitel wird nun ein Optimalitätskriterium und ein effizienter Algorithmus zur Berechnung solcher auslastungsminimaler $b$-Flüsse vorgestellt.

\section{Optimalitätskriterium auslastungsminimaler $b$-Flüsse}

In diesem Abschnitt wird ein hinreichendes sowie notwendiges Kriterium auslastungsminimaler $b$-Flüsse erarbeitet.

Ähnlich zum wohlbekannten Max-Flow-Min-Cut-Theorem von Ford und Fulkerson betrachtet man dabei ein duales Problem, das die Auslastung von Schnitten involviert.

\begin{definition}[Schnitt]
	In einem gerichteten Netzwerk $(V, E, u)$ heißt eine Teilmenge $X\subseteq V$ \emph{Schnitt}, wobei die aus $X$ ausgehenden Kanten mit $\delta^+(X)$ und die in $X$ eingehenden Kanten mit $\delta^-(X)$ bezeichnet werden.
	
	Sind zusätzlich Knotenbalancen $b\in\R^V$ mit $\sum_{v\in V} b_v = 0$ gegeben und ist $\delta^+(X)$ nichtleer, so bezeichne $b(X) / u(\delta^+(X))$ die \emph{Auslastung des Schnittes $X$}.
	Dabei ist $b(X)$ bzw. $u(E')$ eine Kurzschreibweise für $\sum_{v\in X} b_v$ bzw. $\sum_{e\in E'} u_e$.
	Existiert ein Schnitt, dessen Auslastung maximal ist, so nennt man ihn einen \emph{dünnsten Schnitt}.
\end{definition}
Man bemerke, dass ein dünnster Schnitt existiert, wenn die Kantenmenge $E$ nichtleer ist.
\begin{definition}[Doppelgraph]
	Der Doppelgraph $G^\leftrightarrow$ eines Graphen $G=(V,E)$ ist das Paar $(V, \overrightsmallarrow{E}\cup \overleftsmallarrow{E})$, wobei $\overrightsmallarrow{E}:=\{ \overrightsmallarrow{e} \mid e\in E \}$ die Menge der Vorwärtskanten, die die Richtung der Ursprungskante beibehalten, und $\overleftsmallarrow{E}:=\{ \overleftsmallarrow{e} \mid e\in E \}$ die Menge der Rückswärtskanten, die die Richtung der Ursprungskante umkehren, sind.
\end{definition}
\begin{definition}[Residualgraph eines $b$-Flusses]
	Sei ein $b$-Fluss $f$ auf einem Netzwerk $(V, E)$ mit Kapazitäten $u\in\R^E_{>0}$ gegeben.
	Der \emph{Residualgraph von $G$ bezüglich $f$} ist definiert durch $G_f := (V, E_f)$ mit \[
	E_f := \{ \overrightsmallarrow{e}\in \overrightsmallarrow{E} \mid f_e/u_e < c(f) \} \cup \{ \overleftsmallarrow{e} \in \overleftsmallarrow{E} \mid f_e/u_e > 0 \}.
	\]
\end{definition}

\begin{lemma}\label{lemma-min-flow-criterion}
	Ist $f$ ein $b$-Fluss, dessen Residualgraph keine gerichteten Kreise besitzt, die eine Flaschenhalskante als Rückwärtskante benutzen, so ist $f$ auslastungsminimal.
\end{lemma}

Um diese Aussage zu zeigen, benötigt man folgende Hilfsproposition:

\begin{proposition}\label{prop-difference-b-flows-stream}
	Für zwei $b$-Flüsse $f$, $f'$ mit $c(f) \geq c(f')$ ist $f'\Delta f := g\in \R_{\geq0}^{E^\leftrightarrow}$, definiert durch
	\begin{align*}
	g_{\overrightsmallarrow{e}} := \max\{ 0, f_e' - f_e \} \text{~~~und~~~}
	g_{\overleftsmallarrow{e}} := \max\{ 0, f_e - f_e' \} \text{~~~für $e\in E$},
	\end{align*}
	eine Strömung auf $G^\leftrightarrow$, die auf $E^\leftrightarrow \setminus E_f$ verschwindet.
\end{proposition}
\begin{proof}
	Eine Strömung ist eine Kantenbewertung, die in jedem Knoten Flusserhaltung erhält.
	Man zeige also $g(\delta^+_{G^\leftrightarrow}(v)) - g(\delta^-_{G^\leftrightarrow}(v)) = 0$ für alle Knoten $v\in V$.
	Durch Fallunterscheidung erkenne man $g_{\overrightsmallarrow{e}} - g_{\overleftsmallarrow{e}} = f_e' - f_e$ für alle $e\in E$ und man folgere:
	\begin{align*}
	g(\delta^+_{G^\leftrightarrow}(v)) - g(\delta^-_{G^\leftrightarrow}(v))
	&= \left( \sum_{e\in\delta^+_G(v)} g_{\overrightsmallarrow{e}} +  \sum_{e\in\delta^-_G(v)} g_{\overleftsmallarrow{e}} \right)
	- \left(\sum_{e\in\delta^-_G(v)} g_{\overrightsmallarrow{e}} + \sum_{e\in\delta^+_G(v)} g_{\overleftsmallarrow{e}} \right) \\
	&= \sum_{e\in\delta^+_G(v)} (g_{\overrightsmallarrow{e}} - g_{\overleftsmallarrow{e}}) - \sum_{e\in\delta^-_G(v)} (g_{\overrightsmallarrow{e}} - g_{\overleftsmallarrow{e}})\\
	&= \sum_{e\in\delta^+_G(v)} (f_e' - f_e) - \sum_{e\in\delta^-_G(v)} (f_e' - f_e) = b_v - b_v = 0.
	\end{align*}
	Es bleibt also zu zeigen, dass $g$ auf $E^\leftrightarrow \setminus E_f$ verschwindet.
	Sei zunächst eine Vorwärtskante $\overrightsmallarrow{e}$ mit $f_{e}/u_{e} = c(f)$ gegeben.
	Dann gilt nach Voraussetzung auch $f_{e}/u_{e} = c(f) \geq c(f') \geq f'_{e}/u_{e}$, wodurch $g_{\overrightsmallarrow{e}}= 0$ folgt.
	Für eine Rückwärtskante $\overleftsmallarrow{e}$ mit $f_e = 0$ folgt $g_{\overleftsmallarrow{e}} = 0$ direkt.
\end{proof}

Insbesondere existiert für solche Strömungen $f'\Delta f$ eine Dekomposition in Zyklen.
Diese Eigenschaft wird im folgenden Beweis ausgenutzt:

\begin{proof}[Beweis von Lemma~\ref{lemma-min-flow-criterion}]
	Sei $f$ ein $b$-Fluss, dessen Residualgraph keine gerichteten Kreise mit einer Flaschenhalskante besitzt, und sei $e$ eine beliebige Flaschenhalskante.
	Angenommen, es existiere ein $b$-Fluss $f'$ mit geringerer Auslastung; es gilt also insbesondere $f'_e/u_e < f_e/u_e$.
	Die Strömung $g:= f'\Delta f$ besitzt nach Proposition~\ref{prop-difference-b-flows-stream} eine Dekomposition in Zyklen $g = \lambda_1 \cdot C_1 +\dots + \lambda_k \cdot C_k$ mit $\lambda_i > 0$ für $i\in[k]$.
	Da $g_{\overleftsmallarrow{e}}$ positiv ist, gibt es einen Zyklus $C_i$, der ${\overleftsmallarrow{e}}$ enthält.
	Da $g$ außerdem nur auf $E_f$ verläuft, enthält $E_f$ also den Zyklus $C_i$, der die Flaschenhalskante $e$ benutzt, was im Widerspruch zur Voraussetzung steht.
\end{proof}


\begin{lemma}\label{lemma-no-circle-in-res-graph-inclus-min}
	Sei $q^*$ die minimale Auslastung eines $b$-Flusses im Netzwerk $(V, E, u)$.
	Ist $E'\subseteq E$ inklusionsminimal mit der Eigenschaft, dass ein auslastungsminimaler $b$-Fluss $f$ mit Flaschenhalskanten $E'$ existiert, so enthält der Residualgraph $G_f$
	für solche Flüsse $f$ keine gerichteten Kreise, die eine Flaschenhalskante als Rückwärtskante benutzen.
\end{lemma}
\begin{proof}
	\newcommand{\VK}{\text{VK}}
	\newcommand{\RK}{\text{RK}}
	Angenommen, es existiere ein einfacher Kreis $C$, der eine Flaschenhalskante $e\in E$ als Rückwärtskante enthält.
	Es seien $C_\VK$ die Menge der Vorwärtskanten und $C_\RK$ die Menge der Rückwärtskanten in $C$.
	Man setze $\gamma_\VK := \min_{e\in C_\VK} q^*u_e - f_e > 0$ als die minimale Flussmenge, die man jeder Kante in $C_\VK$ zufügen müsste, sodass mindestens eine Kante darin mindestens Auslastung $q^*$ erhält.
	Weiter sei $\gamma_\RK := \min_{e^\leftarrow\in C_\RK} f_e > 0$ die minimale Flussmenge der Rückwärtskanten in $C$.
	Wählt man nun $0 < \gamma < \min\{ \gamma_\VK, \gamma_\RK  \}$, so erhält man durch Augmentierung von $f$ entlang $Q$ mit $\gamma$ einen $b$-Fluss $\tilde{f}$.
	Bezüglich $\tilde{f}$ haben dann sowohl alle Vorwärtskanten als auch alle Rückwärtskanten von $C$ eine geringere Auslastung als $q^*$.
	Demnach ist auch $\tilde{f}$ ein auslastungsminimaler $b$-Fluss, dessen Flaschenhalskanten, also Kanten mit Auslastung $q^*$, eine echte Teilmenge von $E'$ sind, da mindestens eine Rückwärtskante in $E'$ enthalten war.
	Dies ist jedoch ein Widerspruch zur Inklusionsminimalität von $E'$.
\end{proof}

Weiter lässt sich nun das folgende fundamentale Theorem zeigen, das den Zusammenhang zwischen der Auslastung von $b$-Flüssen und der von Schnitten darlegt:

\begin{theorem}\label{thm-strong-duality-sparsest-cut-min-flow}
	In einem gerichteten Netzwerk $(V, E, u)$ mit Balancevektor $b\in\R^V$ mit $\sum_{v\in V}b_v = 0$ ist die Auslastung eines $b$-Flusses $f$ mindestens so groß wie die Auslastung eines Schnittes $X$ mit ausgehenden Kanten; das heißt \[\max_{e\in E} \frac{f_e}{u_e} \geq \frac{b(X)}{u(\delta^+(X))}.\]
	Existiert ein $b$-Fluss und ist $E$ nichtleer, so ist die minimale Auslastung eines $b$-Flusses gerade die Auslastung eines dünnsten Schnittes; das bedeutet
	\[
	\min_{\text{$f$ $b$-Fluss}}~\max_{e\in E}\frac{f_e}{u_e} = \max_{\substack{X\subseteq V\\ \delta^+(X)\neq\emptyset}} ~ \frac{b(X)}{u(\delta^+(X))}.
	\]
\end{theorem}
\begin{proof}
	Sei zunächst $f$ ein $b$-Fluss und $X$ ein Schnitt mit $\delta^+(X)\neq\emptyset$.
	Sei $e^*$ eine Kante mit maximaler Auslastung, also eine Kante mit $f_{e^*}/u_{e^*}=\max_{e\in E} f_e / u_e$.
	Für $e^*$ gilt somit $0\geq f_e u_{e^*} - f_{e^*}u_e$ für alle $e\in E$ und man folgere
	\[
	0\geq \frac{\sum_{e\in \delta^+(X)}(f_e u_{e^*} - f_{e^*}u_e)}{u(\delta^+(X)) u_{e^*}} = \frac{f(\delta^+(X))}{u(\delta^+(X))} - \frac{f_{e^*}}{u_{e^*}} \geq \frac{b(X)}{u(\delta^+(X))} - \frac{f_{e^*}}{u_{e^*}},
	\]
	wobei in der letzten Ungleichung $b(X) = f(\delta^+(X)) - f(\delta^-(X))$ eingeht.
	
	Sei nun ein $b$-Fluss $f$ mit minimaler Auslastung $q^*:=\max_{e\in E} f_e/u_e$ gegeben.
	Gesucht ist nun ein Schnitt mit Auslastung $q^*$.
	Ist $b$ der Nullvektor, so ist $q^*=0$ und jeder Schnitt mit ausgehenden Kanten ist ein dünnster Schnitt.
	
	Sonst sei $f$ ein $b$-Fluss minimaler Auslastung $q^* > 0$ mit möglichst wenig maximal ausgelasteten Kanten.
	Sei $e^*=vw$ eine solche maximal ausgelastete Kante und sei $X\subseteq V$ die Menge aller Knoten, die im Residualgraph $G_f$ vom Knoten $v$ erreicht werden können.
	Eine ausgehende Kante $xy\in\delta^+(X)$ muss dann Auslastung $q^*$ haben, da $x$ von $v$ in $G_f$ erreichbar ist und daher $xy$ nicht als Vorwärtskante in $G_f$ erscheinen kann; sonst wäre $y$ auch in $X$.
	Die Auslastung einer eingehenden Kante $xy\in\delta^-(X)$ muss jedoch verschwinden, da $xy$ nicht als Rückwärtskante in $G_f$ auftaucht.
	Außerdem ist $e^*\in \delta^+(X)$, da sonst $w$ von $v$ aus in $G_f$ erreichbar wäre und mit $e^{*\leftarrow}$ einen Kreis in $G_f$ bilden würde, der eine Rückwärtskante maximaler Auslastung benutzt, welchen es nach Lemma~\ref{lemma-no-circle-in-res-graph-inclus-min} nicht geben kann.
	Also ist $\delta^+(X)\neq \emptyset$ und es gilt:
	\[
	\frac{b(X)}{u(\delta^+(X))} = \frac{f(\delta^+(X))}{u(\delta^+(X))} = \sum_{e\in\delta^+(X)} \frac{u_e}{u(\delta^+(X))} q^* = q^*.
	\]
\end{proof}

\begin{corollary}\label{cor-easy-characterization-sparsest-cut}
	Seien $f$ ein $b$-Fluss mit Auslastung $q$ und $X$ ein Schnitt mit $\delta^+(X)\neq \emptyset$ in einem Netzwerk $(V, E, u)$.
	Dann sind die beiden Aussagen äquivalent:
	\begin{enumerate}[label=(\roman*)]
		\item Es sind $X$ ein dünnster Schnitt und $f$ ein $b$-Fluss minimaler Auslastung.
		\item Die Auslastung bezüglich $f$ aller ausgehenden Kanten von $X$ beträgt $q$ und $f$ verschwindet auf allen eingehenden Kanten von $X$.
	\end{enumerate}
\end{corollary}
\begin{proof}
	Aufgrund der Flusserhaltung gilt $b(X) = f(\delta^+(X)) - f(\delta^-(X))$.
	In \[
	\frac{b(X)}{u(\delta^+(X))} \leq \frac{f(\delta^+(X))}{u(\delta^+(X))} \leq \frac{u(\delta^+(X)) q}{u(\delta^+(X))} = q
	\]
	gilt Gleichheit genau dann, wenn $f(\delta^-(X))=0$ und $f_{e}/u_{e} = q$ für alle $e\in\delta^+(X)$ gelten.
	Theorem~\ref{thm-strong-duality-sparsest-cut-min-flow} liefert nun die Behauptung.
\end{proof}

\todo{Insbesondere sind alle inklusionsminimalen dünnsten Schnitte -- gegeben eines minimalen $b$-Flusses -- durch die Methode Residualgraph berechenbar (in sehr geringer polynomieller Zeit, nämlich vermutlich sogar $O(n+m)$ -- Allerdings muss man hier bisschen aufpassen, findet man ausgehend von einer maximal ausgelasteten Kante eine weitere, so muss man (für inklusionsminimalität das Vorgehen von hier zurücksetzen)}

\todo{Removing circles as naive approach of finding minimal b flow does not terminate.}