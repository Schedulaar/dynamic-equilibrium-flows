\section{Schmale Flüsse mit Zurücksetzen}\label{sec-thin-flows}
\newcommand*{\PlH}{\makebox[1ex]{\textbf{$\cdot$}}}

Der Abschnitt beginnt mit der Einführung einer neuen Klasse statischer Flüsse:

\begin{definition}[$s$-Fluss]
	Ein $b$-Fluss $f$ in einem Graphen $(V, E)$ heißt \emph{$s$-Fluss} für ein $s\in V$, falls $b_v\leq 0$ für alle $v\in V\setminus \{ s \}$ gilt.
	Dabei bezeichnet $b_s$ den \emph{Wert von $f$}. 
\end{definition}

\begin{definition}[Netzwerk mit Zurücksetzen]
	Ein \emph{Netzwerk mit Zurücksetzen auf $E_1$ und Ursprung $s$} ist ein Tupel $(V, E, u, s, E_1)$, wobei $(V, E, u)$ ein azyklisches Netzwerk ist, in dem der Knoten $s\in V$ jeden anderen Knoten erreicht, und $E_1\subseteq E$ die Menge der \emph{zurücksetzenden Kanten} ist.
\end{definition}

\begin{definition}[Auslastung]
	Sei ein $s$-Fluss $f$ in einem Netzwerk mit Zurücksetzen auf $E_1$ und Ursprung $s$ gegeben.
	Dann ist die \emph{Auslastung einer Kante} gegeben durch $f_e/u_e$.
	Die \emph{Auslastungsübertragung einer Kante $vw$} sei gegeben durch \[ \rho_{vw}(l_v, f_{vw}) := \begin{cases}
		\max\{ l_v, f_{vw} / u_{vw} \}, & \text{falls $vw\notin E_1$,}\\
		f_{vw} / u_{vw}, & \text{falls $vw\in E_1$.}
	\end{cases}
	\]
	Die \emph{Auslastung eines $s$-$v$-Pfades $P=(e_1,\dots,e_k)$} ist dann gegeben durch die Verkettung $(\rho_{e_k}(\PlH, f_{e_k})\circ \dots \circ \rho_{e_1}(\PlH, f_{e_1}))(0)$.
	Die zu $f$ \emph{zugehörige Knotenauslastung $l\in\R_{\geq 0}^V$} sei dann für einen Knoten $v\in V$ festgelegt durch die minimale Auslastung eines $s$-$v$-Pfades.
\end{definition}

Betrachtet man einen $s$-$v$-Pfad $P=(e_1, \dots, e_k)$, der Kanten aus $E_1$ enthält, so ist die Auslastung von $P$ gerade $\max_{j=i}^k f_{e_j}/u_{e_j}$, wobei $e_i$ die letzte Kante des Pfades in $E_1$ ist.
Daher nennt man die Kanten in $E_1$ die zurücksetzenden Kanten, da sie die Auslastung eines vorangegangen Pfades auf ihre eigene zurücksetzen.

\begin{proposition}\label{prop-congestion-labels-dijkstra}
	Die Knotenauslastungen eines $s$-Flusses $f$ sind gegeben durch die eindeutige Lösung $(\tilde{l}_v)_{v\in V}$ des Gleichungssystems
	\[
	\tilde{l}_w = \begin{cases}
		0, & \text{falls $w=s$,}\\
		\min_{vw\in \delta^-(w)} \rho_{vw}(\tilde{l}_v, f_{vw}), & \text{sonst.}
	\end{cases}
	\]
\end{proposition}
\begin{proof}
	Die Existenz einer Lösung folgt aus der Azyklizität des Netzwerks und wegen der Erreichbarkeit jedes Knotens von $s$ aus.
	
	Sei also $\tilde{l}\in\R_{\geq 0}^V$ die Lösung des Gleichungssystems.
	Der Teilgraph $(V, E')$ mit \[
	E':= \{ e\in E \mid \rho_{vw}(\tilde{l}_v, f_{vw}) = \min_{uw\in\delta^-(w)}  \rho_{uw}(\tilde{l}_u, f_{uw}) \}
	\]
	ebenfalls azyklisch und jeder Knoten ist von $s$ aus erreichbar.
	Man zeige $l_w = \tilde{l}_w$ durch eine Induktion über die Distanz von $w$ zu $s$ bezüglich der Anzahl an Kanten.
	Für $w=s$ gilt offenbar $l_s = 0$.
	Sei nun ein $s$-$w$-Pfad $P$ gegeben, seien $vw$ die letzte Kante und $Q$ das restliche Anfangsstück dieses Pfades.
	Nach Induktionsvoraussetzung ist die Auslastung von $Q$ mindestens $\tilde{l}_v$.
	Daher ist die Auslastung von $P$ mindestens $\rho_{vw}(\tilde{l}_v, f_{vw})$ aufgrund der Monotonie von $\rho_{vw}(\PlH, f_{vw})$, wodurch $l_v \geq \tilde{l}_v$ folgt.
	Außerdem hat ein $s$-$v$-Pfad $P$, der in $(V, E')$ verläuft, hat Auslastung $\tilde{l}_v$.
	Da solch ein Pfad existiert, gilt also $l_v \leq \tilde{l}_v$.
\end{proof}

\begin{definition}[Schmaler Fluss]
	Seien ein $s$-Fluss $f$ in einem Netzwerk mit Zurück\-setzen auf $E_1$ und Ursprung $s$ sowie die durch $f$ induzierten Knotenauslastungen $l$ gegeben.
	Der Fluss $f$ heißt \emph{schmaler Fluss mit Zurücksetzen auf $E_1$}, falls die Auslastung jedes $s$-$v$-Pfades mit positivem Fluss $l_v$ beträgt. 
\end{definition}
\begin{remark}
	Ronald Koch betrachtet in~\cite{Koch2012} allgemeinere schmale Flüsse: Dort ist das zugrundeliegende Netzwerk nicht als azyklisch vorausgesetzt.
	Diese Einschränkung wird hier im Sinne von~\cite{Cominetti2015} beibehalten.
\end{remark}

\begin{lemma}\label{lemma-thin-flow-t-def}
	Ein $s$-Fluss $f$ in einem Netzwerk mit Zurücksetzen auf $E_1$ und Ursprung $s$ ist genau dann ein schmaler Fluss mit Zurücksetzen auf $E_1$, wenn eine Knotenbewertung $l\in\R_{\geq 0}^V$ existiert, die die folgenden Bedingungen erfüllt:
	\begin{enumerate}[label=(T\arabic*)]
		\item\label{def-thin-flow-source} $l_s = 0$,
		\item\label{def-thin-flow-x-zero} $l_w \leq l_v$, \tabto{4.5cm} für $vw\in E \setminus E_1$ mit $f_{vw}=0$,
		\item\label{def-thin-flow-x-positive} $l_w = \max\{ l_v, f_{vw} / u_{vw} \}$,  \tabto{4.5cm} für $vw\in E\setminus E_1$ mit $f_{vw} > 0$,
		\item\label{def-thin-flow-resetting-edge} $l_w = f_{vw} / u_{vw}$,  \tabto{4.5cm} für $vw\in E_1$,
		\item\label{def-thin-flow-no-resetting-edge} $l_w \geq \min_{vw\in \delta^-(w)} l_v$, \tabto{4.5cm} falls $\delta^-(w)\cap E_1 = \emptyset$.
	\end{enumerate}
	Diese Knotenbewertung stimmt dann mit der Knotenauslastung von $f$ überein.
	Außerdem gelten die Bedingungen~\ref{def-thin-flow-source}, \ref{def-thin-flow-x-zero} und~\ref{def-thin-flow-no-resetting-edge} bereits für die Knotenauslastungen von $f$, falls $f$ nur ein $s$-Fluss ist.
\end{lemma}
\begin{proof}
	Sei $f$ ein schmaler Fluss mit Zurücksetzen auf $E_1$ und sei $l\in\R_{\geq 0}^V$ die Knotenauslastung von $f$.
	Man zeige, dass $l$ die Bedingungen~\ref{def-thin-flow-source}-\ref{def-thin-flow-no-resetting-edge} erfüllt, und benutze dabei die Darstellung aus Proposition~\ref{prop-congestion-labels-dijkstra}.
	Wegen $l_s = 0$ gilt bereits~\ref{def-thin-flow-source}.
	Für eine Kante $vw\in E\setminus E_1$ mit $f_{vw}=0$ gilt $l_w\leq \rho_{vw}(l_v, f_{vw}) = l_v$, wodurch auch~\ref{def-thin-flow-x-zero} folgt.
	Ist $vw\in E$ mit $f_{vw} > 0$, so existiert ein $s$-$w$-Pfad mit positivem Fluss, der die Kante $vw$ benutzt, sodass die Auslastung dieses Pfades $l_w=\rho_{vw}(l_v, f_{vw})$ ist, da der $s$-$v$-Teilpfad die Auslastung $l_v$ besitzt.
	Für $vw\notin E_1$ bedeutet das gerade $l_w = \max\{ l_v, f_{vw}/u_{vw} \}$ also~\ref{def-thin-flow-x-positive}, und für $vw\in E_1$ zeigt das $l_w = f_{vw} / u_{vw}$.
	Für~\ref{def-thin-flow-resetting-edge} bleibt der Fall $f_{vw} = 0$ zu prüfen:
	Dann ist aber $l_w = \min_{uw\in\delta^-(w)} \rho_{uw}(l_u, f_{uw}) = f_{vw} / u_{vw} = 0$.
	Zuletzt betrachte man den Fall, dass $w$ keine eingehende zurücksetzende Kante hat.
	Dann folgt aber bereits $l_w = \min_{vw\in \delta^-(w)} \max\{ l_v, f_{vw} / u_{vw} \} \geq \min_{vw\in\delta^-(w)} l_v$, was auch~\ref{def-thin-flow-no-resetting-edge} impliziert.
	
	Sei nun umgekehrt $l\in\R_{\geq 0}^V$ eine Knotenbewertung, die~\ref{def-thin-flow-source}-\ref{def-thin-flow-no-resetting-edge} erfüllt.
	Man verwende eine Induktion über die Distanz eines Knotens $w$ zu $s$ bezüglich der Kantenzahl, um zu zeigen, dass $l_w$ die Knotenauslastung von  $f$ ist.
	Für $w=s$ gilt $l_s=0$ bereits nach~\ref{def-thin-flow-source}.
	Für $w\neq s$ ist $l_w$ nach~\ref{def-thin-flow-x-zero},~\ref{def-thin-flow-x-positive} und~\ref{def-thin-flow-resetting-edge} eine untere Schranke an $\rho_{vw}(l_v, f_{vw})$ für $vw\in\delta^-(w)$.
	Es bleibt zu zeigen, dass für eine eingehende Kante $vw$ der Wert $\rho_{vw}(l_v, f_{vw})$ auch $l_w$ annimmt:
	Falls eine eingehende Kante $vw\in E_1$ oder eine eingehende Kante $vw$ mit $f_{vw} > 0$ existiert, ist $\rho_{vw}(l_v, f_{vw}) = l_w$ nach~\ref{def-thin-flow-resetting-edge} und~\ref{def-thin-flow-x-positive}.
	Sonst ist $l_w\geq \min_{vw\in \delta^-(w)} l_v = \min_{vw\in \delta^-(w)} \rho_{vw}(l_v', x_{vw}')$ nach~\ref{def-thin-flow-no-resetting-edge}, was die Behauptung zeigt.
	Um zu sehen, dass $f$ nun ein schmaler Fluss mit Zurücksetzen auf $E_1$ ist, genügen die Bedingungen~\ref{def-thin-flow-source}, \ref{def-thin-flow-x-positive} und~\ref{def-thin-flow-resetting-edge}, da dadurch jeder $s$-$v$-Teilpfad eines Pfades mit positivem Fluss gerade Auslastung $l_v$ hat.
\end{proof}

\begin{remark}\label{remark-thin-flow}
	Hier ist, wie in~\cite[Definition~4]{Cominetti2011}, im Vergleich zu~\cite[Definition~6]{Koch2011} die Bedingung~\ref{def-thin-flow-no-resetting-edge} zusätzlich eingeführt worden.
	Diese ist nötig, um im Beweis von Theorem~\ref{thm-alpha-extension-is-nash-flow} zu zeigen, dass die erweiterten Ankunftszeiten tatsächlich mit den angegebenen übereinstimmen.
	Ohne diese Bedingung würde dies nämlich nicht gelten, wie man in Abbildung~\ref{figure-labels} erkennen kann.
\end{remark}
\todo{
	Introduce network inflow with normalized thin-flows
	\begin{definition}[Schmaler Fluss]\label{def-thin-flow}
		Der Fluss $x'$ heißt \emph{normiert}, falls $F=d$ gilt.
		Ist $E_1$ nichtleer, so heißt $x'$ \emph{schmaler Fluss mit Zurücksetzen auf $E_1$}, sonst heißt $x'$ \emph{schmaler Fluss ohne Zurücksetzen}.
	\end{definition}
}

\begin{lemma}\label{lemma-equivalent-thin-flow}
	Ein $s$-Fluss $f$ ist genau dann ein schmaler Fluss mit Zurück\-setzen auf $E_1$, wenn $f_{vw}= 0$ für alle $vw\in E$ mit $l_w < \rho_{vw}(l_v, f_{vw})$ gilt.
	Dabei sei $l\in\R_{\geq 0}^V$ die Knotenauslastung von $f$.
\end{lemma}
\begin{proof}
	Angenommen, $f$ sei ein schmaler Fluss mit Zurücksetzen auf $E_1$.
	Für eine Kante $vw\in E$ mit $l_w < \rho_{vw}(l_v, f_{vw})$ ist $vw\notin E_1$ und $f_{vw}=0$ nach~\ref{def-thin-flow-resetting-edge} und~\ref{def-thin-flow-x-positive}.

	Nun nehme man an, es gelte $f_{vw}=0$ für alle $vw\in E$ mit $l_w < \rho_{vw}(l_v, f_{vw})$, und man zeige die Eigenschaften~\ref{def-thin-flow-x-positive} und \ref{def-thin-flow-resetting-edge}.
	Der Rest folgt dann mit Lemma~\ref{lemma-thin-flow-t-def}.
	Ist $vw\notin E_1$ mit $f_{vw}>0$, so gilt $l_w = \rho_{vw}(l_v, f_{vw})$, wodurch Bedingung~\ref{def-thin-flow-x-positive} folgt.
	Für~\ref{def-thin-flow-resetting-edge} betrachte man zunächst $vw\in E_1$ mit $f_{vw}>0$: Hier folgt wieder $l_w = \rho_{vw}(l_v, f_{vw})$.
	Falls $f_{vw}=0$ gilt, ist $l_w=\min_{\tilde{v}w\in \delta^-(w)} \rho_{\tilde{v}w}(l_{\tilde{v}}, f_{\tilde{v}w}) = 0$.
\end{proof}

\todo{Vlt im folgenden Korollar auch/oder Gleichheit zu $\max f_e/u_e$ zeigen}
\begin{corollary}\label{cor-thin-flows-max-node-label-equals-max-edge-label}
	Für einen schmalen $s$-Fluss $f$ in einem Netzwerk mit nichtleerer Kantenmenge und zugehöriger Knotenauslastung $l$ gilt
	\[
		\max_{vw\in E} \rho_{vw}(l_v, f_{vw}) = \max_{v\in V} l_v
	\]
\end{corollary}
\begin{proof}
	Für $w\neq s$ ist $l_w = \min_{vw\in\delta^-(w)} \rho_{vw}(l_v, x_{vw})$.
	Daher gilt bereits \glqq$\geq$\grqq.
	Angenommen es gebe eine Kante $vw\in E$ mit $\rho_{vw}(l_v, f_{vw}) > \max_{u\in V} l_u$
	Dann ist aber bereits $f_{vw}/u_{vw} > \max_{u\in V} l_u \geq 0$, womit $l_w \geq \rho_{vw}(l_v, f_{vw})$ nach Lemma~\ref{lemma-equivalent-thin-flow} gelten müsste.
\end{proof}

\begin{lemma}[Eindeutigkeit der Knotenbewertung]
	Seien ein azyklisches Netzwerk mit Ursprung $s$ und Versorgungsrate $d\in\R_{\geq 0}$ sowie $E_1\subseteq E$ gegeben.
	Dann sind die Knotenbewertungen aller normierter schmaler Flüsse mit Zurücksetzen auf $E_1$ identisch.
\end{lemma}
\begin{proof}
	Angenommen, es existieren zwei normierte schmale Flüsse $x'$ und $y'$ mit Zurück\-setzen auf $E_1$ mit unterschiedlichen Knotenbewertungen $l'\neq h'$.
	Dann ist oBdA. die Menge $S:=\{ v\in V \mid l_v' > h_v' \}$ nichtleer.
	Der Balancevektor $(b_v)_{v\in V}$ mit $$b_v:=\sum_{e\in\delta^+(v)} x_e' - \sum_{e\in\delta^-(s)} x_e' = \sum_{e\in\delta^+(v)} y_e' - \sum_{e\in\delta^-(v)} y_e'$$ ist für $x'$ und $y'$ identisch, da beide Flusserhaltung in $v\notin\{ s, t\}$, und damit $b_v=0$ erfüllen und Wert $d$ besitzen.
	Betrachtet man Kanten, die innerhalb $S$ verlaufen, erhält man 
	\begin{equation}\label{proof-uniqueness-labels-eq}
	\sum_{e\in \delta^+(S)} x'_e - \sum_{e\in\delta^-(S)} x'_e = \sum_{v\in S} b_v = \sum_{e\in\delta^+(S)} y_e' - \sum_{e\in\delta^-(S)} y_e'.
	\end{equation}
	Für Kanten $vw\in \delta^+(S)$ gilt $x_e' \leq y_e'$, da für $x_e' > y_e'$ Lemma~\ref{lemma-equivalent-thin-flow} und $v\in S$ die Ungleichung $l_w' = \rho_{vw}(l_v', x_{vw}') > \rho_{vw}(h_v', y_{vw}')\geq h_w'$ implizieren würden, welche jedoch im Widerspruch zu $w\notin S$ steht.
	Ebenso gilt für Kanten $vw\in\delta^-(S)$ die Ungleichung $x_e' \geq y_e'$, weil sonst $l_w' \leq \rho_{vw}(l_v', x_{vw}') \leq \rho_{vw}(h_v', y_{vw}') = h_w'$ wegen $v\notin S$ gelten würde, was $w\in S$ widerspricht.
	Gleichung~\ref{proof-uniqueness-labels-eq} impliziert dann sogar $x_e = y_e$ für alle $e\in \delta(S):=\delta^-(S) \cup \delta^+(S)$.
	Für $vw\in \delta^-(S)$ ist dann $y_e'=x_e'=0$, da für $y_e'=x_e'>0$ wieder mit $v\notin S$ die Ungleichung $l_w' = \rho_{vw}(l_v, x_{vw}')\leq \rho_{vw}(h_v, y_{vw}')=h_w'$ der Bedingung $w\in S$ widerspricht.
	
	Aufgrund der Azyklizität existiert ein Knoten $w\in S$ mit $\delta^-(w)\subseteq \delta^-(S)$.
	Eingehende Kanten von $w$ sind daher nicht in $E_1$ enthalten, da für solch eine Kante $\rho_{vw}(l_v', x_e') = 0 = \rho_{vw}(l_v', y_e')$ gelten würde, sodass $l_w' = 0 = h_w'$ der Voraussetzung $w\in S$ widerspricht.
	Daher ist $l_w' = \min_{vw\in \delta^-(w)} l_v'$ und $h_w' = \min_{vw\in\delta^-(w)} h_v'$, was den Widerspruch $l_w' \leq h_w'$ impliziert.
\end{proof}

\begin{theorem}[Existenz eines normierten schmalen Flusses]\label{thm-existence-thin-flow}
	Seien ein azyklisches Netzwerk $(G,u,s,t)$ mit Ursprung $s$ und Versorgungsrate $d\in\R_{\geq 0}$ sowie $E_1\subseteq E$ gegeben.
	Es existiert ein normierter schmaler Fluss mit Zurücksetzen auf $E_1$.
\end{theorem}

Um die Existenz zu beweisen, benötigen wir zunächst den Fixpunktsatz von Kakutani.
Dazu führen wir die folgenden Begriffe ein:

\begin{definition}[Korrespondenz, Fixpunkt]
	Eine \emph{Korrespondenz} von einer Menge $A$ in eine Menge $B$ ist eine Abbildung $f: A \to \mathcal{P}(B)\setminus \{ \emptyset \}$ von $A$ in die Potenzmenge von $B$ ohne die leere Menge.
	Sind $A$ und $B$ topologische Räume, so nennt man $f$ \emph{abgeschlossen}, wenn die zugehörige Relation $R_f := \{ (a,b) \in A\times B \mid b\in f(a)  \}$ in der Produkttopologie abgeschlossen ist.

	Ein \emph{Fixpunkt} einer Korrespondenz $f: X \to \mathcal{P}$ ist ein Punkt $x\in X$ mit $x\in f(x)$.
\end{definition}

Nun lautet der Fixpunktsatz von Kakutani (siehe~\cite{Heuser1991Fix}):

\begin{satz}[Fixpunktsatz von Kakutani]\label{satz-kakutani}
	Seien $C\subseteq E$ eine nichtleere, konvexe und kompakte Teilmenge eines normierten Raumes $E$ und $f: C \to \mathcal{P}(C)$ eine abgeschlossene, konvexwertige Korrespondenz.
	Dann besitzt $f$ einen Fixpunkt.
\end{satz}

\begin{proof}[Beweis von Theorem~\ref{thm-existence-thin-flow}]
	Man betrachte die Menge $C$ aller $s$-$t$-Flüsse mit Wert $d$ als Teilmenge des metrischen Raums $\R^E$.
	Diese ist nichtleer, weil $t$ von $s$ aus erreichbar ist, konvex, da für $x', y'\in C, \lambda \in [0,1]$ der Fluss $\lambda x' + (1-\lambda)y'$ wieder die Flusserhaltung erfüllt, und kompakt, da jeder Fluss in $C$ von Wert $d$ ist und aufgrund der Azyklizität in $s$-$t$-Wege zerlegbar ist, wodurch jede Kante maximal Fluss $d$ besitzen kann, sodass $C\subseteq [0, d]^E$ folgt.
	Auf $C$ definiere man die Korrespondenz
	\[
	\Gamma: C\to \mathcal{P}(C)\setminus\{ \emptyset \}, ~ x'\mapsto \{ y'\in C \mid \forall vw\in E: l_w' < \rho_{vw}(l_v', x_{vw}') \implies y_{vw}' = 0 \}.
	\]
	Dabei sei $l'\in\R_{\geq 0}^V$ die zu $x'$ gehörige Knotenauslastung.
	Diese Korrespondenz ist wohldefiniert, da $\Gamma(x')$ für jedes $x'\in C$ nichtleer ist:
	Der Knoten $s$ kann im Graphen $G':= (V, E')$ mit $E':= \{ vw\in E \mid \rho_{vw}(l_v', x_{vw}')=l_w' \}$ jeden Knoten erreichen, da $G'$ azyklisch ist und jeder Knoten $w\neq s$ hat mindestens eine eingehende Kante mit $l_w' = \rho_{vw}(l_v', x_{vw}')$ besitzt. 
	Daher existiert ein $s$-$t$-Weg $P$ in $G'$ und der Fluss $d\cdot P$ ist in $\Gamma(x')$ enthalten.
	
	Für jedes $x'\in C$ ist $\Gamma(x')$ konvex: Ist für $y', z'\in \Gamma(x'), \lambda\in [0,1]$ und eine Kante $vw\in E$ die Bedingung $l_w' < \rho_{vw}(l_w', x_{vw}')$ erfüllt, so gilt auch $\lambda y_{vw}' + (1-\lambda) z_{vw}' = 0$, womit $\lambda y' + (1-\lambda)z'$ im Bild $\Gamma(x)$ liegt.
	Des Weiteren ist $\Gamma$ abgeschlossen: 
	Sei eine konvergente Folge $((x^{n}, y^{n}))_{n\in\N}$ in $R_\Gamma$ mit Grenzwert $(x, y)$ gegeben.
	Die zu $x$ bzw. $x^n$ gehörigen Knotenauslastungen seien gegeben durch $l$ bzw. $l^n$. 
	Angenommen, es gelte $l_w < \rho_{vw}(l_v, x_{vw})$.
	Da die Zuordnung $x'\mapsto l'$ eines $s$-Flusses auf seine Knotenauslastung nach Lemma~\ref{prop-congestion-labels-dijkstra} stetig ist, existiert ein $N\in\N$ mit $l_w^n<\rho_{vw}(l_v^n, x_{vw}^n)$ für alle $n\geq N$.
	Damit ist auch $y_{vw} = \lim_{n\to\infty} y_{vw}^n = 0$.
	
	Daher existiert nach dem Fixpunktsatz von Kakutani~\ref{satz-kakutani} ein Fixpunkt von $\Gamma$, welcher nach Lemma~\ref{lemma-equivalent-thin-flow} ein normierter schmaler Fluss mit Zurücksetzen auf $E_1$ ist.
\end{proof}

\section{Effiziente Berechnung schmaler Flüsse}

Die Existenz schmaler Flüsse mit Zurücksetzen aus Theorem~\ref{thm-existence-thin-flow} suggeriert bereits einen exponentiellen Algorithmus, um einen solchen Fluss mit Wert $d$ zu berechnen:
Man rate die Menge $E'\subseteq E$ aller Kanten $vw$, die $l_w' = \rho_{vw}(l_v', x_{vw}')$ erfüllen, sowie für alle $vw\in E'\setminus E_1$ eine Binärzahl $\sigma_{vw}\in\{0,1\}$, die $\max\{ l_v', x_{vw}/u_{vw}\} = \sigma_{vw} l_v' + (1-\sigma_{vw}) x_{vw}/u_{vw}$ erfüllt.
Dann suche man einen zulässigen Punkt in folgendem Polyeder:
\begin{align*}
&	(x', l') \in \R_{\geq 0}^{E'\times V} \\
	\text{u.d.N.}\quad &
	x'(\delta^+(v)) - x'(\delta^-(v)) =
		\begin{cases}
			d, & \text{falls $v=s$},\\
			0, & \text{sonst,} 
		\end{cases} & \text{für $v\in V\setminus\{t \}$,} \\
	& \begin{array}{@{}l}
		l_w' = \sigma_{vw} l_v' + (1-\sigma_{vw}) x'_{vw}/u_{vw}\\
		\sigma_{vw} l_v' + (1-\sigma_{vw}) x'_{vw}/u_{vw} \geq x'_{vw}/u_{vw}\\
		\sigma_{vw} l_v' + (1-\sigma_{vw}) x'_{vw}/u_{vw} \geq l_v'
	\end{array}
	& \text{für $vw\in E'\setminus E_1$},\\
	& l_w' = x'_{vw}/u_{vw} & \text{für $vw\in E_1$.}
\end{align*}

\newcommand{\TFNP}{\mathbf{TFNP}}

Existiert solch ein Punkt $(x', l')$, so ist $x'$ nach Lemma~\ref{lemma-equivalent-thin-flow} ein schmaler Fluss mit Zurücksetzen auf $E_1$ mit zugehöriger Knotenbewertung $l'$.
Außerdem muss aufgrund der Existenz eines schmalen Flusses mindestens eine Wahl von $E'$ und $\sigma$ existieren, sodass der zugehörige Polyeder nichtleer ist.
\todo{Habe ich hier nicht das Problem, dass ich das nur für $\Q$ in polynomieller Zeit berechnen kann?}
Die Zulässigkeit eines solchen Polyeders kann in polynomieller Zeit beispielsweise durch die Ellipsoidmethode ermittelt werden. 
Daher liegt das Problem, zu einem Netzwerk $(V, E, u, s, t)$ einen schmalen $d$-Fluss mit Zurücksetzen auf $E_1\subseteq E$ zu finden, in der Komplexitätsklasse $\TFNP$:

\begin{definition}[Komplexitätsklasse $\TFNP$]
	Eine Relation $P(x,y)$ liegt in der Komplexitätsklasse $\TFNP$ genau dann, wenn es einen deterministischen, polynomiellen Algorithmus gibt, der entscheidet, ob $P(x,y)$ für zwei Kandidaten $x$ und $y$ gilt, und wenn für alle $x$ ein Kandidat $y$ existiert, dessen Kodierung polynomielle Größe in der Kodierungslänge von $x$ besitzt, sodass $P(x, y)$ gilt.
\end{definition}

In diesem Abschnitt wird ein polynomieller Algorithmus zur Berechnung von schmalen Flüssen ohne Zurücksetzen vorgestellt.
Dazu werden zunächst einige Begriffe eingeführt:

\begin{definition}[Schnitt]
	In einem gerichteten Netzwerk $(V, E, u)$ heißt eine Teilmenge $X\subseteq V$ \emph{Schnitt}, wobei die aus $X$ ausgehenden Kanten mit $\delta^+(X)$ und die in $X$ eingehenden Kanten mit $\delta^-(X)$ bezeichnet werden.
	
	Sind zusätzlich Knotenbalancen $b\in\R^V$ mit $\sum_{v\in V} b_v = 0$ gegeben und ist $\delta^+(X)$ nichtleer, so bezeichne $b(X) / u(\delta^+(X))$ die \emph{Auslastung des Schnittes $X$}.
	Dabei ist $b(X)$ bzw. $u(E')$ eine Kurzschreibweise für $\sum_{v\in X} b_v$ bzw. $\sum_{e\in E'} u_e$.
	Existiert ein Schnitt, dessen Auslastung maximal ist, so nennt man ihn einen \emph{dünnsten Schnitt}.
\end{definition}
Man bemerke, dass ein dünnster Schnitt existiert, wenn die Kantenmenge $E$ nichtleer ist.
\begin{definition}[Auslastung eines $b$-Flusses]
	In einem Netzwerk $(V, E, u)$ nennt man die maximale Kantenauslastung $\max_{e\in E} f_e / u_e$ die \emph{Auslastung des $b$-Flusses $f$}.
\end{definition}

\begin{lemma}
	\todo{Ein kantenminimaler b-Fluss kann in $O(xyz)$ Zeit gefunden werden}
\end{lemma}

\begin{lemma}\label{lemma-no-circle-in-res-graph-inclus-min}
	Sei $q^*$ die minimale Auslastung eines $b$-Flusses im Netzwerk $(V, E, u)$.
	Ist $E'\subseteq E$ inklusionsminimal mit der Eigenschaft, dass ein $b$-Fluss $f$ minimaler Auslastung existiert, sodass $E'$ die durch $f$ maximal ausgelasteten Kanten sind, so enthält der Residualgraph $G_f := (V, E_f)$, definiert durch \[
		E_f := \{ e \mid e\in E, f_e/u_e < q^* \} \cup \{ e^\leftarrow \mid e\in E, f_e/u_e > 0 \},
	\]
	für solche Flüsse $f$ keine gerichteten Kreise, die eine Rückwärtskante einer maximal ausgelasteten Kante benutzen.
	
	Kanten $e\in E_f$ mit $f_e/u_e < q^*$ nennt man Vorwärtskanten und Kanten $e^\leftarrow\in E_f$ mit $f_e/u_e > 0$ Rückwärtskanten.
\end{lemma}
\begin{proof}
	\newcommand{\VK}{\text{VK}}
	\newcommand{\RK}{\text{RK}}
	Angenommen, es existiere ein einfacher Kreis $C$, der eine Rückwärtskante $e^\leftarrow$ mit $e\in E'$ enthält.
	Es seien $C_\VK$ die Menge der Vorwärtskanten und $C_\RK$ die Menge der Rückwärtskanten in $C$.
	Man setze $\gamma_\VK := \min_{e\in C_\VK} q^*u_e - f_e > 0$ als die minimale Flussmenge, die man jeder Kante in $C_\VK$ zufügen müsste, sodass mindestens eine Kante darin mindestens Auslastung $q^*$ erhält.
	Weiter sei $\gamma_\RK := \min_{e^\leftarrow\in C_\RK} f_e > 0$ die minimale Flussmenge der Rückwärtskanten in $C$.
	Wählt man nun $0 < \gamma < \min\{ \gamma_\VK, \gamma_\RK  \}$, so erhält man durch Augmentierung von $f$ entlang $Q$ mit $\gamma$ einen $b$-Fluss $\tilde{f}$.
	Bezüglich $\tilde{f}$ haben dann sowohl alle Vorwärtskanten als auch alle Rückwärtskanten von $C$ eine geringere Auslastung als $q^*$.
	Demnach ist auch $\tilde{f}$ ein minimaler $b$-Fluss, dessen maximal ausgelastete Kanten, also Kanten mit Auslastung $q^*$, eine echte Teilmenge von $E'$ sind, da mindestens eine Rückwärtskante in $E'$ enthalten war.
	Dies ist jedoch ein Widerspruch zur Inklusionsminimalität von $E'$.
\end{proof}

Weiter wird das folgende fundamentale Theorem benötigt, das den Zusammenhang zwischen der Auslastung von $b$-Flüssen und der von Schnitten aufzeigt:

\begin{theorem}\label{thm-strong-duality-sparsest-cut-min-flow}
	In einem gerichteten Netzwerk $(V, E, u)$ mit Balancevektor $b\in\R^V$ mit $\sum_{v\in V}b_v = 0$ ist die Auslastung eines $b$-Flusses $f$ mindestens so groß wie die Auslastung eines Schnittes $X$ mit ausgehenden Kanten; das heißt \[\max_{e\in E} \frac{f_e}{u_e} \geq \frac{b(X)}{u(\delta^+(X))}.\]
	Existiert darüber hinaus ein $b$-Fluss und ist $E$ nichtleer, so ist die minimale Auslastung eines $b$-Flusses gerade die Auslastung eines dünnsten Schnittes; das bedeutet
	\[
		\min_{\text{$f$ $b$-Fluss}}~\max_{e\in E}\frac{f_e}{u_e} = \max_{\substack{X\subseteq V\\ \delta^+(X)\neq\emptyset}} ~ \frac{b(X)}{u(\delta^+(X))}.
	\]
\end{theorem}
\begin{proof}
	Sei zunächst $f$ ein $b$-Fluss und $X$ ein Schnitt mit $\delta^+(X)\neq\emptyset$.
	Sei $e^*$ eine Kante mit maximaler Auslastung, also eine Kante mit $f_{e^*}/u_{e^*}=\max_{e\in E} f_e / u_e$.
	Für $e^*$ gilt somit $0\geq f_e u_{e^*} - f_{e^*}u_e$ für alle $e\in E$ und man folgere
	\[
		0\geq \frac{\sum_{e\in \delta^+(X)}(f_e u_{e^*} - f_{e^*}u_e)}{u(\delta^+(X)) u_{e^*}} = \frac{f(\delta^+(X))}{u(\delta^+(X))} - \frac{f_{e^*}}{u_{e^*}} \geq \frac{b(X)}{u(\delta^+(X))} - \frac{f_{e^*}}{u_{e^*}},
	\]
	wobei in der letzten Ungleichung $b(X) = f(\delta^+(X)) - f(\delta^-(X))$ eingeht.
	
	Sei nun ein $b$-Fluss $f$ mit minimaler Auslastung $q^*:=\max_{e\in E} f_e/u_e$ gegeben.
	Gesucht ist nun ein Schnitt mit Auslastung $q^*$.
	Ist $b$ der Nullvektor, so ist $q^*=0$ und jeder Schnitt mit ausgehenden Kanten ist ein dünnster Schnitt.
	
	Sonst sei $f$ ein $b$-Fluss minimaler Auslastung $q^* > 0$ mit möglichst wenig maximal ausgelasteten Kanten.
	Sei $e^*=vw$ eine solche maximal ausgelastete Kante und sei $X\subseteq V$ die Menge aller Knoten, die im Residualgraph $G_f$ vom Knoten $v$ erreicht werden können.
	Eine ausgehende Kante $xy\in\delta^+(X)$ muss dann Auslastung $q^*$ haben, da $x$ von $v$ in $G_f$ erreichbar ist und daher $xy$ nicht als Vorwärtskante in $G_f$ erscheinen kann; sonst wäre $y$ auch in $X$.
	Die Auslastung einer eingehenden Kante $xy\in\delta^-(X)$ muss jedoch verschwinden, da $xy$ nicht als Rückwärtskante in $G_f$ auftaucht.
	Außerdem ist $e^*\in \delta^+(X)$, da sonst $w$ von $v$ aus in $G_f$ erreichbar wäre und mit $e^{*\leftarrow}$ einen Kreis in $G_f$ bilden würde, der eine Rückwärtskante maximaler Auslastung benutzt, welchen es nach Lemma~\ref{lemma-no-circle-in-res-graph-inclus-min} nicht geben kann.
	Also ist $\delta^+(X)\neq \emptyset$ und es gilt:
	\[
		\frac{b(X)}{u(\delta^+(X))} = \frac{f(\delta^+(X))}{u(\delta^+(X))} = \sum_{e\in\delta^+(X)} \frac{u_e}{u(\delta^+(X))} q^* = q^*.
	\]
\end{proof}

\begin{corollary}\label{prop-easy-characterization-sparsest-cut}
	Seien $f$ ein $b$-Fluss mit Auslastung $q$ und $X$ ein Schnitt mit $\delta^+(X)\neq \emptyset$ in einem Netzwerk $(V, E, u)$.
	Dann sind die beiden Aussagen äquivalent:
	\begin{enumerate}[label=(\roman*)]
		\item Es sind $X$ ein dünnster Schnitt und $f$ ein $b$-Fluss minimaler Auslastung.
		\item Die Auslastung bezüglich $f$ aller ausgehenden Kanten von $X$ beträgt $q$ und $f$ verschwindet auf allen eingehende Kanten von $X$.
	\end{enumerate}
\end{corollary}
\begin{proof}
	Aufgrund der Flusserhaltung gilt $b(X) = f(\delta^+(X)) - f(\delta^-(X))$.
	In \[
	\frac{b(X)}{u(\delta^+(X))} \leq \frac{f(\delta^+(X))}{u(\delta^+(X))} \leq \frac{u(\delta^+(X)) q}{u(\delta^+(X))} = q
	\]
	gilt Gleichheit genau dann, wenn $f(\delta^-(X))=0$ und $f_{e}/u_{e} = q$ für alle $e\in\delta^+(X)$ gelten.
	Theorem~\ref{thm-strong-duality-sparsest-cut-min-flow} liefert nun die Behauptung.
\end{proof}

\todo{Insbesondere sind alle inklusionsminimalen dünnsten Schnitte -- gegeben eines minimalen $b$-Flusses -- durch die Methode Residualgraph berechenbar (in sehr geringer polynomieller Zeit, nämlich vermutlich sogar $O(n+m)$}

\begin{lemma}
	Sei ein schmaler $b$-Fluss $f$ ohne Zurücksetzen mit zugehöriger Knotenauslastung $l$ gegeben.
	Dann ist $f$ ein $b$-Fluss minimaler Auslastung $q$ und für $E\neq\emptyset$ ist $X:=\{ v\in V \mid l_v < q \}$ ein dünnster Schnitt.
\end{lemma}
\begin{proof}
	Ist $b$ der Nullvektor -- wie im Fall $E=\emptyset$ --, ist $f$ der Nullfluss, da das Netzwerk azyklisch ist.
	
	Sei also $q>0$ die Auslastung von $f$ im Falle von $b\neq 0$.
	Nach Korollar~\ref{cor-thin-flows-max-node-label-equals-max-edge-label} genügt es zu
	zeigen, dass die Auslastung jeder ausgehenden Kante von $X$ bezüglich $f$ gerade $q$ beträgt und auf eingehenden Kanten von $X$ kein Fluss fließt.
	Wegen $l_s=0$ ist $s$ im Schnitt $X$ enthalten.
	Sei $vw\in\delta^+(X)$ eine ausgehende Kante.
	Dann gilt $l_v < l_w = q $ und nach Proposition~\ref{prop-congestion-labels-dijkstra} auch $q = l_w \leq \rho_{vw}(l_v, f_{vw}) = f_{vw} / u_{vw}$.
	Da $q$ eine obere Schranke an $f_{vw}/u_{vw}$ ist, gilt also $f_{vw}/u_{vw} = q$.
	Für eine eingehende Kante $vw\in\delta^-(X)$ gilt $\rho_{vw}(l_v, f_{vw})\geq l_v > l_w$ und somit muss nach Lemma~\ref{lemma-equivalent-thin-flow} der Fluss $f$ auf der Kante $vw$ verschwinden.
\end{proof}

\begin{lemma}
	Seien ein schmaler $s$-Fluss $f$  sowie ein $s$-Schnitt $X\subseteq V$ mit $f(\delta^-(X)) = 0$ gegeben.
	Dann ist $f$ auch ein schmaler $s$-Fluss auf dem von $X$ induzierten Teilnetz.
\end{lemma}
\begin{proof}
	Seien $l$ bzw. $\tilde{l}$ die Knotenauslastungen von $f$ 
\end{proof}

