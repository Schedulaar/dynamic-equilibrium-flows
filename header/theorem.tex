

\DeclareMathOperator{\e}{ex}
\DeclareMathOperator{\ma}{mate}
\DeclareMathOperator{\Ex}{Ex}

%-------------------------------------------------------------------------------
%   Befehle für Nummerierung der Ergebnisse
%   fortlaufend innerhalb eines Abschnittes
%-------------------------------------------------------------------------------
\theoremstyle{plain}            % normaler Stil
\newtheorem{theorem}{Theorem}[section]
% Lemma
\newaliascnt{lemma}{theorem}
\newtheorem{lemma}[lemma]{Lemma}
\aliascntresetthe{lemma}
% Satz
\newaliascnt{satz}{theorem}
\newtheorem{satz}[satz]{Satz}
\aliascntresetthe{satz}
% Korollar
\newaliascnt{corollary}{theorem}
\newtheorem{corollary}[corollary]{Korollar}
\aliascntresetthe{corollary}
% Proposition
\newaliascnt{proposition}{theorem}
\newtheorem{proposition}[proposition]{Proposition}
\aliascntresetthe{proposition}
%-------------------------------------------------------------------------------
\theoremstyle{definition}	% Definitionsstil
% Definition
\newaliascnt{definition}{theorem}
\newtheorem{definition}[definition]{Definition}
\aliascntresetthe{definition}
% Beispiel
\newaliascnt{example}{theorem}
\newtheorem{example}[example]{Beispiel}
\aliascntresetthe{example}
% Problem
\newaliascnt{problem}{theorem}
\newtheorem{problem}[problem]{Problem}
\aliascntresetthe{problem}
% Algorithmus
\newaliascnt{algorithmus}{theorem}
\newtheorem{algorithmus}[algorithmus]{Algorithmus}
\aliascntresetthe{algorithmus}
%-------------------------------------------------------------------------------
\theoremstyle{remark}		% Bemerkungsstil
% Bemerkung
\newaliascnt{remark}{theorem}
\newtheorem{remark}[remark]{Bemerkung}
\aliascntresetthe{remark}
% Vermutung
\newaliascnt{conjecture}{theorem}
\newtheorem{conjecture}[conjecture]{Vermutung}
\aliascntresetthe{conjecture}
% Notation
\newaliascnt{notation}{theorem}
\newtheorem{notation}[notation]{Notation}
\aliascntresetthe{notation}

%-------------------------------------------------------------------------------
% automatische Referenzen mit interaktiven Text
%-------------------------------------------------------------------------------

% Texte
\renewcommand{\theoremautorefname}{Theorem}
\newcommand{\lemmaautorefname}{Lemma}
\newcommand{\satzautorefname}{Satz}
\newcommand{\korollarautorefname}{Korollar}
\newcommand{\propositionautorefname}{Proposition}

\newcommand{\definitionautorefname}{Definition}
\newcommand{\beispielautorefname}{Beispiel}
\newcommand{\problemautorefname}{Problem}
\newcommand{\algorithmusautorefname}{Algorithmus}

\newcommand{\bemerkungautorefname}{Bemerkung}
\newcommand{\vermutungautorefname}{Vermutung}
\newcommand{\notationautorefname}{Notation}

%-------------------------------------------------------------------------------
% Nummerierung der Gleichungen innerhalb der obersten Ebene
%-------------------------------------------------------------------------------
\ifx\chapter\undefined 			% Kapitel sind definiert
  \numberwithin{equation}{section}	% Gleichungsnummern in Section
\else					% Kapitel sind nicht definiert
  \numberwithin{equation}{chapter}	% Gleichungsnummern in Kapiteln
\fi
