%-------------------------------------------------------------------------------
% Kopf-Zeilen
%-------------------------------------------------------------------------------

\pagestyle{scrheadings}		     % Kopfzeilen nach scr-Standard		
\ifx\chapter\undefined 		     % falls Kapitel nicht definiert sind
  \automark[subsection]{section} % Kopf- und Fusszeilen setzen
\else				             % Kapitel sind definiert
  \automark[section]{chapter}	 % Kopf- und Fusszeilen setzen
\fi

%-------------------------------------------------------------------------------
%   Maske für Überschrift 
%-------------------------------------------------------------------------------
% Belegung der notwendigen Kommandos für die Titelseite
\newcommand{\autor}{Markl, Michael} 		% bearbeitender Student
\newcommand{\veranstaltung}{Seminar zur Optimierung und Spieltheorie} 	% Titel des ganzen Seminars
\newcommand{\uni}{Institut für Mathematik der Universität Augsburg} % Universit\"at
\newcommand{\matrikelnummer}{1474802}
\newcommand{\lehrstuhl}{Diskrete Mathematik, Optimierung und Operations Research} % Lehrstuhl
\newcommand{\semester}{Sommersemester 2019}	% Winter-/Sommersemester mit Jahr
\newcommand{\datum}{27.06.2019} 			% Datumsangabe
\newcommand{\thema}{TODO}  		% Titel der Seminararbeit

\newcommand{\ownline}{\vspace{.7em}\hrule\vspace{.7em}} % horizontale Linie mit Abstand

\newcommand{\seminarkopf}{
	% Befehl zum Erzeugen der Titelseite 
 \textsc{\autor}  \hfill{\datum} \\ 
\textbf{\veranstaltung} \\ 
\uni \\ 
\lehrstuhl \\
\semester \hfill{Matrikelnummer: \matrikelnummer}
\ownline 

\begin{center}
{\LARGE \textbf{\thema}}
\end{center}

\ownline
}