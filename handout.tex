\documentclass[paper=a4, 	% Seitenformat
fontsize=11pt, 		% Schriftgr\"o\ss{}e
abstracton, 	% mit Abstrakt
headsepline, 	% Trennlinie f\"ur die Kopfzeile
notitlepage	% keine extra Titelseite
]{scrartcl}

\usepackage[utf8]{inputenc}
\usepackage[automark]{scrpage2}	        % Seiten-Stil für scrartcl
% Mathematische Zeichensätze und Umgebungen
\usepackage{amsfonts, amssymb}	        % Definition einer Liste mathematischer Fontbefehle und Symbole
\usepackage[intlimits,sumlimits]{amsmath} % Integral-/Summationsgrenzen über/unter Zeichen
\usepackage{tabto}
\usepackage{mathabx}
% mathematische Verbesserungen
\usepackage{amsthm}	                    % spezielle theorem Stile
\usepackage{aliascnt} 
\usepackage{array}		                % erweiterte Tabellen
% Schriftzeichen, Format
\usepackage{latexsym}		            % Latex-Symbole
\usepackage[english, german, ngerman]{babel} % Mehrsprachenumgebung
% Layout
\usepackage{geometry}                   % Seitenränder
\usepackage{xcolor}                     % Farben
\usepackage{bbm}
% Tabellen und Listen
\usepackage{float}		                % Gleitobjekte 
\usepackage[flushright]{paralist}       % Bessere Behandlung der Auflistungen
\usepackage{datetime}
% Bilder
\usepackage[final]{graphicx}            % Graphiken einbinden
\usepackage{caption}                    % Beschriftungen
\usepackage{subcaption}                 % Beschriftungen für Unterteilung

% Interaktive Referenzen, und PDF-Keys
\usepackage{xr-hyper}  
\usepackage[pagebackref,pdftex, plainpages=false]{hyperref} % Rückreferenz im Literaturverzeichnis, Treiber für ps zu pdf ; für direkt nach pdf: pdftex
\usepackage{enumitem}

%%%%%%%%%%%%%%%%%%%%%%%%%%%%%%%%%%%%%%%%%%%%%%%%%%%%%%%%%%%%%%%%%%%%%%%%%%%%%%%%
% Zusammenfassung einiger nützlicher Pakete und Befehle
%-------------------------------------------------------------------------------
% Kopf-Zeilen
%-------------------------------------------------------------------------------

\pagestyle{scrheadings}		     % Kopfzeilen nach scr-Standard		
\ifx\chapter\undefined 		     % falls Kapitel nicht definiert sind
  \automark[subsection]{section} % Kopf- und Fusszeilen setzen
\else				             % Kapitel sind definiert
  \automark[section]{chapter}	 % Kopf- und Fusszeilen setzen
\fi

%-------------------------------------------------------------------------------
%   Maske für Überschrift 
%-------------------------------------------------------------------------------
% Belegung der notwendigen Kommandos für die Titelseite
\newcommand{\autor}{Markl, Michael} 		% bearbeitender Student
\newcommand{\veranstaltung}{Seminar zur Optimierung und Spieltheorie} 	% Titel des ganzen Seminars
\newcommand{\uni}{Institut für Mathematik der Universität Augsburg} % Universit\"at
\newcommand{\matrikelnummer}{1474802}
\newcommand{\lehrstuhl}{Diskrete Mathematik, Optimierung und Operations Research} % Lehrstuhl
\newcommand{\semester}{Sommersemester 2019}	% Winter-/Sommersemester mit Jahr
\newcommand{\datum}{27.06.2019} 			% Datumsangabe
\newcommand{\thema}{Nash Gleichgewichte in Dynamischen Flüssen}  		% Titel der Seminararbeit

\newcommand{\ownline}{\vspace{.7em}\hrule\vspace{.7em}} % horizontale Linie mit Abstand

\newcommand{\seminarkopf}{
	% Befehl zum Erzeugen der Titelseite 
 \textsc{\autor}  \hfill{\datum} \\ 
\textbf{\veranstaltung} \\ 
\uni \\ 
\lehrstuhl \\
\semester
\ownline 

\begin{center}
{\LARGE \textbf{\thema}}
\end{center}

\ownline
}			% Befehle und Pakete für Titelseite


\DeclareMathOperator{\e}{ex}
\DeclareMathOperator{\ma}{mate}
\DeclareMathOperator{\Ex}{Ex}

%-------------------------------------------------------------------------------
%   Befehle für Nummerierung der Ergebnisse
%   fortlaufend innerhalb eines Abschnittes
%-------------------------------------------------------------------------------
\theoremstyle{plain}            % normaler Stil
\newtheorem{theorem}{Theorem}
\numberwithin{theorem}{section}
% Lemma
\newaliascnt{lemma}{theorem}
\newtheorem{lemma}[lemma]{Lemma}
\aliascntresetthe{lemma}
% Satz
\newaliascnt{satz}{theorem}
\newtheorem{satz}[satz]{Satz}
\aliascntresetthe{satz}
% Korollar
\newaliascnt{corollary}{theorem}
\newtheorem{corollary}[corollary]{Korollar}
\aliascntresetthe{corollary}
% Proposition
\newaliascnt{proposition}{theorem}
\newtheorem{proposition}[proposition]{Proposition}
\aliascntresetthe{proposition}
%-------------------------------------------------------------------------------
\theoremstyle{definition}	% Definitionsstil
% Definition
\newaliascnt{definition}{theorem}
\newtheorem{definition}[definition]{Definition}
\aliascntresetthe{definition}
% Beispiel
\newaliascnt{example}{theorem}
\newtheorem{example}[example]{Beispiel}
\aliascntresetthe{example}
% Problem
\newaliascnt{problem}{theorem}
\newtheorem{problem}[problem]{Problem}
\aliascntresetthe{problem}
% Algorithmus
\newaliascnt{algorithmus}{theorem}
\newtheorem{algorithmus}[algorithmus]{Algorithmus}
\aliascntresetthe{algorithmus}
%-------------------------------------------------------------------------------
\theoremstyle{remark}		% Bemerkungsstil
% Bemerkung
\newaliascnt{remark}{theorem}
\newtheorem{remark}[remark]{Bemerkung}
\aliascntresetthe{remark}
% Vermutung
\newaliascnt{conjecture}{theorem}
\newtheorem{conjecture}[conjecture]{Vermutung}
\aliascntresetthe{conjecture}
% Notation
\newaliascnt{notation}{theorem}
\newtheorem{notation}[notation]{Notation}
\aliascntresetthe{notation}

%-------------------------------------------------------------------------------
% automatische Referenzen mit interaktiven Text
%-------------------------------------------------------------------------------

% Texte
\renewcommand{\theoremautorefname}{Theorem}
\newcommand{\lemmaautorefname}{Lemma}
\newcommand{\satzautorefname}{Satz}
\newcommand{\korollarautorefname}{Korollar}
\newcommand{\propositionautorefname}{Proposition}

\newcommand{\definitionautorefname}{Definition}
\newcommand{\beispielautorefname}{Beispiel}
\newcommand{\problemautorefname}{Problem}
\newcommand{\algorithmusautorefname}{Algorithmus}

\newcommand{\bemerkungautorefname}{Bemerkung}
\newcommand{\vermutungautorefname}{Vermutung}
\newcommand{\notationautorefname}{Notation}

%-------------------------------------------------------------------------------
% Nummerierung der Gleichungen innerhalb der obersten Ebene
%-------------------------------------------------------------------------------
\ifx\chapter\undefined 			% Kapitel sind definiert
  \numberwithin{equation}{section}	% Gleichungsnummern in Section
\else					% Kapitel sind nicht definiert
  \numberwithin{equation}{chapter}	% Gleichungsnummern in Kapiteln
\fi





\makeatletter
\newcommand{\overleftsmallarrow}{\mathpalette{\overarrowsmall@\leftarrowfill@}}
\newcommand{\overrightsmallarrow}{\mathpalette{\overarrowsmall@\rightarrowfill@}}
\newcommand{\overleftrightsmallarrow}{\mathpalette{\overarrowsmall@\leftrightarrowfill@}}
\newcommand{\overarrowsmall@}[3]{%
	\vbox{%
		\ialign{%
			##\crcr
			#1{\smaller@style{#2}}\crcr
			\noalign{\nointerlineskip\vskip0.4pt}%
			$\m@th\hfil#2#3\hfil$\crcr
		}%
	}%
}
\def\smaller@style#1{%
	\ifx#1\displaystyle\scriptstyle\else
	\ifx#1\textstyle\scriptstyle\else
	\scriptscriptstyle
	\fi
	\fi
}
\makeatother
			% Mathematische Befehle und Pakete

% Literatur-Bibliothek
\bibliographystyle{alphadin}               % deutscher Bibliotheksstil

% Erweiterte Einstellungen zu hyperref

\hypersetup{
        breaklinks=true,        % zu lange Links unterbrechen
        colorlinks=true,        % Färben von Referenzen
        citecolor=black,        % Farbe der Zitate
        linkcolor=black,        % Farbe der Links
        extension=pdf,          % Externe Dokumente können eingebunden werden.
        ngerman,		
	pdfview=FitH,
	pdfstartview=FitH,		
	bookmarksnumbered=true,     % Anzeige der Abschnittsnummern	% pdf-Titel
	pdfauthor={\autor}          % pdf-Autor
}

% Namen für Referenzen 

\newcommand{\ownautorefnames}{
  \renewcommand{\sectionautorefname}{Kapitel}
  \renewcommand{\subsectionautorefname}{Unterkapitel}
  \renewcommand{\subsubsectionautorefname}{\subsectionautorefname}
  \renewcommand{\appendixautorefname}{Anhang}
  \renewcommand{\figureautorefname}{Abbildung}
}

% Rückreferenzentext zur Literatur
\def\bibandname{und}%
\renewcommand*{\backref}[1]{}
\renewcommand*{\backrefalt}[4]{%

}
\renewcommand{\backreftwosep}{ und~} % seperate 2 pages
\renewcommand{\backreflastsep}{ und~} % seperate last of longer 

			% Befehle und Pakete für Referenzen

\geometry{a4paper, top=30mm, bottom=30mm, left=28mm, right=28mm}

\parindent0em

\addtolength{\footskip}{-0.5cm}          % Seitenzahlen höher setzen
\renewcommand{\descriptionlabel}[1]{\hspace{\labelsep}\textit{#1}}

\makeatletter
\newcommand{\customlabel}[2]{%
	\protected@write \@auxout {}{\string \newlabel {#1}{{#2}{\thepage}{#2}{#1}{}} }%
	\hypertarget{#1}{#2}
}
\makeatother
\newcommand{\todo}[1]{{\color{red}#1}}

\numberwithin{figure}{section}	% Abbildungsnummern in Section

\newcommand{\R}{\mathbb{R}}
\newcommand{\Q}{\mathbb{Q}}
\newcommand{\Z}{\mathbb{Z}}
\newcommand{\N}{\mathbb{N}}
\newcommand*\diff{\mathop{}\!\mathrm{d}}
\setlist[enumerate]{topsep=0.5ex,itemsep=0ex,partopsep=0ex,parsep=0.8ex}
\setlist[itemize]{topsep=0.5ex,itemsep=0ex,partopsep=0ex,parsep=0.8ex}


\renewcommand{\[}{
	\setlength\abovedisplayskip{1.1ex}
	\setlength{\belowdisplayskip}{1.1ex}
	\setlength{\abovedisplayshortskip}{1.1ex}
	\setlength{\belowdisplayshortskip}{1.1ex}\begin{equation*}}

\renewcommand{\]}{\end{equation*}}

% Start des Dokuments
\begin{document}
	\ownautorefnames		% Änderung einiger automatischen Texte von hyperref (wie in referenz.tex definiert)
	\parindent0em 			% kein Einzug nach einer Leerzeile
	
	%%%%%%%%%%%%%%%%%%%%%%%%%%%%%%%%%%%%%%%%%%%%%%%%%%%%%%%%%%%%%%%%%%%%%%%%%%%%%%%%
	% Titelseite
	\thispagestyle{empty}		% leerer Seitenstil, also keine Seitennummer
	\seminarkopf 			% Titelblatt (wie in kopf.tex definiert)
	
	%%%%%%%%%%%%%%%%%%%%%%%%%%%%%%%%%%%%%%%%%%%%%%%%%%%%%%%%%%%%%%%%%%%%%%%%%%%%%%%%
	% Eigentlicher Inhalt des Handouts; die einzelnen Teile werden hier (aus Gründen der Übersichtlichkeit) über \input{file} eingebunden
	
	\section{Dynamische Flüsse}
	\begin{definition}[Netzwerk]
		Ein \emph{Netzwerk} $(G, u, s, t, \tau)$ ist ein gerichteter Graph mit
		einer Quelle $s\in V$ und einer Senke $t\in V$, sodass alle Knoten von $s$ aus erreichbar sind,
			sowie mit Kantenkapazitäten $u\in\R_+^E$,
			und Verzögerungszeiten $\tau\in\R_{\geq 0}^E$, sodass Zyklen eine positive Gesamtverzögerung haben.
	\end{definition}

	\begin{definition}
		Der Funktionenraum $\mathfrak{F}_0$ sei die Menge aller lokal Lebesgue-integrierbaren Funktionen $g:\R \to \R_{\geq 0}$, die auf der negativen Achse verschwinden.
	\end{definition}

	\begin{definition}[Dynamischer Fluss]
		Ein dynamischer Fluss ist ein Paar $f=(f^+, f^-)$ mit $f^+, f^-\in\mathfrak{F}_0^E$.		
		Für eine Kante $e\in E$ und einen Zeitpunkt $\theta\in\R$ bezeichnet
		\begin{itemize}[label=$\bullet$]
			\item $f_e^+(\theta)$ bzw. $f_e^-(\theta)$ die \emph{Zu- bzw. Abflussrate an $e$ zur Zeit $\theta$},
			\item $F_e^+(\theta) := \int_0^\theta f^+_e(t) \diff t$ bzw. $F^-_e(\theta):=\int_0^\theta f^-_e(t)\diff t$ den kumulativen Zu- bzw. Abfluss an $e$ bis zur Zeit $\theta$,
			\item $z_e(\theta):= F_e^+(\theta) - F_e^-(\theta + \tau_e)$ bzw. $q_e(\theta):= z_e(\theta)/u_e$ die Warteschlange bzw. Wartezeit an $e$ zur Zeit $\theta$,
			\item $T_e(\theta):=\theta + q_e(\theta) + \tau_e$ die Austrittszeit aus $e$ bei Eintrittszeit $\theta$.
		\end{itemize}	
	\end{definition}

	\begin{definition}[Zulässiger Dynamischer Fluss]
		Ein zulässiger dynamischer Fluss erfüllt folgende Voraussetzungen:
		\begin{enumerate}[label=(F\arabic*)]
			\item\label{def-feasible-flow-capacity} Kapazitätsbedingung: $\forall e\in E, \theta\in\R: f_e^-(\theta)\leq u_e$.
			\item\label{def-feasible-flow-no-negative-flow} Keine Flussentstehung in Kanten: $\forall e\in E, \theta\in\R: F_e^-(\theta + \tau_e) \leq F_e^+(\theta).$
			\item\label{def-feasible-flow-no-flow-at-node} Flusserhaltung in Knoten:
			\[
			\forall\theta\in\R: \sum_{e\in\delta^+(v)}f^+_e(\theta) - \sum_{e\in\delta^-(v)} f_e^-(\theta) \begin{cases}
				\geq 0, \text{ falls $v=s$,}\\
				\leq 0, \text{ falls $v=t$,}\\
				= 0, \text{ sonst.}
			\end{cases}\]
			\item\label{def-feasible-flow-queue-with-capacity} Warteschlangenabbau:
			$\forall e\in E, \theta\in\R: z_e(\theta) > 0 \implies f_e^-(\theta + \tau_e) = u_e$.
		\end{enumerate}
	\end{definition}
	\begin{proposition}\label{prop-feasible-flow}
		Für eine Kante $e\in E$ gilt in einen zulässigen dynamischen Fluss $f$:
		\begin{enumerate}[label=(\roman*)]
			\item\label{prop-feasible-flow-T-mon-inc-cont} Die Funktion $\theta \mapsto \theta + q_e(\theta)$ ist monoton wachsend und stetig.
			\item\label{prop-feasible-flow-positive-queue} Für alle $\theta\in\R$ ist die Warteschlange $z_e$ auf $(\theta, \theta + q_e(\theta))$ positiv.
			\item\label{prop-feasible-flow-det-outflow} Zu jeder Zeit $\theta\in\R$ gilt $F_e^+(\theta) = F_e^-(T_e(\theta))$.
			\item\label{prop-feasible-flow-queue-delay} Für alle $\theta_1 \leq \theta_2$ mit $\int_{\theta_1}^{\theta_2} f^+_e(t) \diff t = 0$ und $q_e(\theta_2)>0$ gilt $\theta_1 + q_e(\theta_1) = \theta_2 + q_e(\theta_2)$.
		\end{enumerate}
	\end{proposition}

\section{Kürzeste Wege}

	\begin{definition}[Kürzeste Wege]
		Für einen Fluss $f$ bezeichne:
		\begin{itemize}[label=$\bullet$]
			\item $l^P(\theta) := T_{e_k}\circ\dots\circ T_{e_1}(\theta)$ die Ankunftszeit am Endknoten eines Pfades $P=(e_1,\dots,e_k)$ zur Startzeit $\theta$ am Startknoten,
			\item $\mathcal{P}_w$ die Menge aller $s$-$w$-Pfade,
			\item $l_w(\theta) := \min_{P\in\mathcal{P}_w} l^P(\theta)$ die früheste Ankunftszeit bei $w$ zur Startzeit $\theta$.
		\end{itemize}
	\end{definition}
	\begin{lemma}[Dreiecksungl.]
		In einem zulässigen Fluss gilt $T_{vw}(l_v(\theta))\geq l_w(\theta)$ für $vw\in E$.
	\end{lemma}

	\begin{definition}[Aktivität einer Kante]
		Eine Kante $vw\in E$ ist \emph{aktiv zum Zeitpunkt $\theta$}, falls $T_{vw}(l_v(\theta)) = l_w(\theta)$ gilt; sonst ist sie \emph{inaktiv zum Zeitpunkt $\theta$}.
		
		Die Menge $\Theta_e$ sei die abgeschlossene Menge aller Zeitpunkte, zu denen $e$ aktiv ist.
	\end{definition}
	\begin{lemma}
		Für einen zulässigen Fluss und einem $\theta\in\R$ ist der Teilgraph der zur Zeit $\theta$ aktiven Kanten $G_\theta:=(V, E_\theta)$ ein azyklischer Graph, in dem $s$ jeden Knoten erreichen kann.
	\end{lemma}
	\begin{proposition}
		Für einen zulässigen Fluss $f$ ist $(l_v(\theta))_{v\in V}$ die eindeutige Lösung von
		\[ \tilde{l}_w = \begin{cases}
		\theta, & \text{falls } w=s, \\
		\min\limits_{vw\in \delta^-(w)} T_{vw}(\tilde{l}_v), & \text{sonst}.
		\end{cases} \]
	\end{proposition}

\section{Dynamische Nash-Flüsse}	
	\begin{definition}
		Für einen zulässigen Fluss $f$ und einen Zeitpunkt $\theta$ bezeichne
		\begin{itemize}[label=$\bullet$]
			\item $x_{vw}^+(\theta):= F_{vw}^+(l_v(\theta))$ bzw. $x_{vw}^-(\theta):= F^-_{vw}(l_w(\theta))$ für $vw\in E$,
			\item $b_v(\theta) := \sum_{e\in\delta^+(v)} x_e^+(\theta) - \sum_{e\in\delta^-(v)} x_e^-(\theta)$ für $v\in V$.
		\end{itemize}
	\end{definition}
	
	\begin{definition}\label{def-flow-along-active-edges}
		Man sage, der Fluss $f$ \emph{fließe nur entlang aktiver Kanten}, falls $f_{vw}^+$ fast überall auf $l_v(\Theta_{vw}^c)$ verschwindet für alle Kanten $vw\in E$.
	\end{definition}
	
	\begin{definition}
		Man sage, der Fluss $f$ \emph{fließe ohne Überholungen}, falls $b_s(\theta) = -b_t(\theta)$ für alle $\theta\in\R$.
	\end{definition}

	\begin{theorem}[Charakterisierung dynamischer Nash-Flüsse]\label{thm-equivalencies-nash-flow}
		Für einen zulässigen dynamischen Fluss $f$ sind die folgenden Aussagen äquivalent:
		\begin{enumerate}[label=(\roman*)]
			\item Der Fluss $f$ fließt nur entlang aktiver Kanten
			\item Für alle Kanten $e\in E$ und zu jeder Zeit $\theta\in\R$ gilt $x_e^+(\theta) = x_e^-(\theta)$.
			\item Der Fluss $f$ fließt ohne Überholungen.
		\end{enumerate}
		Gilt eine dieser Aussagen, so nennt man $f$ einen \emph{dynamischen Nash-Fluss}.
	\end{theorem}
	
	%%%%%%%%%%%%%%%%%%%%%%%%%%%%%%%%%%%%%%%%%%%%%%%%%%%%%%%%%%%%%%%%%%%%%%%%%%%%%%%%
	% Literaturverzeichnis
	\nocite{Cominetti2015}
	\nocite{Koch2011}  % Nicht zitierte Quellen werden auch ins Literaturverzeichnis aufgenommen
	\thispagestyle{empty}
	\scriptsize
	\bibliography{literature}  % Literaturverzeichnis liegt in der Datei seminararbeit
	
	%%%%%%%%%%%%%%%%%%%%%%%%%%%%%%%%%%%%%%%%%%%%%%%%%%%%%%%%%%%%%%%%%%%%%%%%%%%%%%%%
	%%%%%%%%%%%%%%%%%%%%%%%%%%%%%%%%%%%%%%%%%%%%%%%%%%%%%%%%%%%%%%%%%%%%%%%%%%%%%%%%
	% Ende des Dokuments
\end{document}		
